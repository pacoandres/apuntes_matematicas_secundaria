\documentclass[a4paper,9pt,answers]{exam}

\usepackage{graphicx}
\usepackage{pstricks}
\usepackage[utf8]{inputenc}
\usepackage[spanish]{babel}
\usepackage[T1]{fontenc}
%textcomp es para el símbolo del euro
\usepackage{lmodern, textcomp}

\usepackage[left=1in, right=1in, top=1in, bottom=1in]{geometry}
%\usepackage{mathexam}
\usepackage{amsmath}
\usepackage{amssymb}
\usepackage{multicol}
\usepackage{longtable}
%para la última página
%\usepackage{lastpage}

%Para padding en celdas
\usepackage{cellspace}
\setlength\cellspacetoplimit{1mm}
\setlength\cellspacebottomlimit{1mm}

%Para hacer tachados
\usepackage[makeroom]{cancel}

%Creative commons
%\usepackage{ccicons}
\usepackage[type={CC}, modifier={by-nc-sa}, version={4.0}, %imagemodifier={-eu-80x25},
lang={spanish}]{doclicense}

%Para las gráficas:
\usepackage{tikz}
\usepackage{pgfplots}
\pgfplotsset{compat = newest}
\pgfplotsset{compat=1.12}
\usetikzlibrary{babel} %Si no da errores con algunas cosas al compilar los gráficos.
\usetikzlibrary{arrows,shapes,positioning}
\usetikzlibrary{matrix}
\usepgfplotslibrary{fillbetween}
\usetikzlibrary{arrows.meta}
\usetikzlibrary{fit}
\usetikzlibrary{quotes,angles}
\usepackage{nicematrix}

\usepackage{color,colortbl}
\definecolor{Gray}{gray}{0.9}
\newcolumntype{g}{>{\columncolor{Gray}}c}
\usepackage{arydshln} %Este pone la línea punteada en la matriz ampliada. TIENE QUE ESTAR DESPUÉS DEL colortbl porque si no casca.
%\pagestyle{headandfoot}
\pagestyle{headandfoot}
\newcommand\ExamNameLine{
\par
\vspace{\baselineskip}
Nombre:\hrulefill\relax
\par}

\renewcommand{\solutiontitle}{\noindent\textbf{Solución:}\par\noindent}

\everymath{\displaystyle}
\newcommand\ddfrac[2]{\frac{\displaystyle #1}{\displaystyle #2}}

\def \autor{Paco Andrés}
\def \titulo{Tablas para trigonometría.}
\def \titulofichas {\textbf {\titulo}}
\def \cursofichas {}
\def \fechaexamen {}
%\firstpageheader{\cursofichas}{\titulofichas}{\fechaexamen}
\header{\cursofichas}{\begin{small}
\titulofichas
\end{small}}{\fechaexamen}
%\header{\cursofichas}{\titulofichas}{\fechaexamen}
%\firtspagefooter{}{\thepage}{}
%Por alguna razón no sale lo del cc en el pie
\firstpagefootrule
\footrule
\footer{\autor}{\thepage}{\doclicenseIcon}
\pointpoints{punto}{puntos}

\shadedsolutions
%\definecolor{SolutionColor}{rgb}{0.99,0.99,.99}
\renewcommand{\baselinestretch}{1.3}

%Use * instead of \cdot
\mathcode`\*="8000
{\catcode`\*\active\gdef*{\cdot}} 
\newcommand{\Card}{\,\mathrm{Card}}

%For e number
\newcommand{\e}{\,\mathrm{e}}
\newcommand{\asen}{\,\mathrm{asen}\,}
\newcommand{\acos}{\,\mathrm{acos}\,}
\newcommand{\atg}{\,\mathrm{atg}\,}

%Para el diferencial y la integral:
\newcommand\dif[1]{\mathrm{d}#1}
\newcommand\integral[2]{\int #1\,\dif{#2}}
\newcommand\integrald[4]{\int_{#3}^{#4} #1\,\dif{#2}}
\newcommand\adjunto[1]{#1^\text{*}}
\newcommand\rango[1]{\mathrm{rg}(#1)}
\newcommand\vectort[3]{#1*\vec i + #2*\vec j + #3*\vec k}
%Para escribir explicaciones encima del igual:
%\newcommand\igexpl[1]{{\mathrel{\overset{\makebox{\mbox{\normalfont\tiny\sffamily $#1$}}}{=}}}}
%Parece que mejor con stackrel
\usepackage{multirow}
\usepackage{xcolor}

\begin{document}


%\author{Paco Andrés}
\title{\titulo}
\date{}
\author{\autor}
\maketitle

\begin{center}
\doclicenseLongText\\
\vspace{.25cm}
\doclicenseImage
\end{center}
\section{Razones trigonométricas de ángulos ``famosos''.}
\begin{tabular}{SlSlSlSlSl}
	\rowcolor[HTML]{9B9B9B} 
	\multicolumn{5}{c}{\cellcolor[HTML]{9B9B9B}Cuadrante I}                                                                                                                                                                                                                                                                             \\
	\rowcolor[HTML]{C0C0C0} 
	\multicolumn{2}{|Sc|}{\cellcolor[HTML]{C0C0C0}Ángulo}                                                         & \multicolumn{1}{Sc|}{\cellcolor[HTML]{C0C0C0}}                       & \multicolumn{1}{Sc|}{\cellcolor[HTML]{C0C0C0}}                         & \multicolumn{1}{Sc|}{\cellcolor[HTML]{C0C0C0}}                           \\
	\rowcolor[HTML]{C0C0C0} 
	\multicolumn{1}{|Sc|}{\cellcolor[HTML]{C0C0C0}Grados} & \multicolumn{1}{c|}{\cellcolor[HTML]{C0C0C0}Radianes} & \multicolumn{1}{Sc|}{\multirow{-2}{*}{\cellcolor[HTML]{C0C0C0}Seno}} & \multicolumn{1}{Sc|}{\multirow{-2}{*}{\cellcolor[HTML]{C0C0C0}Coseno}} & \multicolumn{1}{Sc|}{\multirow{-2}{*}{\cellcolor[HTML]{C0C0C0}Tángente}} \\ \hline
	\multicolumn{1}{|Sr|}{$0^\circ$}                              & \multicolumn{1}{Sr|}{0}                                & \multicolumn{1}{Sr|}{0}                                              & \multicolumn{1}{Sr|}{1}                                                & \multicolumn{1}{Sr|}{0}                                                  \\ \hline
	\multicolumn{1}{|Sr|}{$30^\circ$}                     & \multicolumn{1}{Sr|}{$\frac{\pi}{6}$}                  & \multicolumn{1}{Sr|}{$\frac{1}{2}$}                                  & \multicolumn{1}{Sr|}{$\frac{\sqrt{3}}{2}$}                             & \multicolumn{1}{Sr|}{$\frac{\sqrt{3}}{3}$}                               \\ \hline
	\multicolumn{1}{|Sr|}{$45^\circ$}                     & \multicolumn{1}{Sr|}{$\frac{\pi}{4}$}                  & \multicolumn{1}{Sr|}{$\frac{\sqrt{2}}{2}$}                           & \multicolumn{1}{Sr|}{$\frac{\sqrt{2}}{2}$}                             & \multicolumn{1}{Sr|}{1}                                                  \\ \hline
	\multicolumn{1}{|Sr|}{$60^\circ$}                     & \multicolumn{1}{Sr|}{$\frac{\pi}{3}$}                  & \multicolumn{1}{Sr|}{$\frac{\sqrt{3}}{2}$}                           & \multicolumn{1}{Sr|}{$\frac{1}{2}$}                                    & \multicolumn{1}{Sr|}{$\sqrt{3}$}                                         \\ \hline
	\multicolumn{1}{|Sr|}{$90^\circ$}                     & \multicolumn{1}{Sr|}{$\frac{\pi}{2}$}                  & \multicolumn{1}{Sr|}{1}                                              & \multicolumn{1}{Sr|}{0}                                                & \multicolumn{1}{Sr|}{$\nexists$}                                         \\ \hline
\end{tabular}
\begin{tabular}{SlSlSlSlSl}
	\rowcolor[HTML]{9B9B9B} 
	\multicolumn{5}{c}{\cellcolor[HTML]{9B9B9B}Cuadrante II}                                                                                                                                                                                                                                                                             \\
	\rowcolor[HTML]{C0C0C0} 
	\multicolumn{2}{|Sc|}{\cellcolor[HTML]{C0C0C0}Ángulo}                                                         & \multicolumn{1}{Sc|}{\cellcolor[HTML]{C0C0C0}}                       & \multicolumn{1}{Sc|}{\cellcolor[HTML]{C0C0C0}}                         & \multicolumn{1}{Sc|}{\cellcolor[HTML]{C0C0C0}}                           \\
	\rowcolor[HTML]{C0C0C0} 
	\multicolumn{1}{|Sc|}{\cellcolor[HTML]{C0C0C0}Grados} & \multicolumn{1}{c|}{\cellcolor[HTML]{C0C0C0}Radianes} & \multicolumn{1}{Sc|}{\multirow{-2}{*}{\cellcolor[HTML]{C0C0C0}Seno}} & \multicolumn{1}{Sc|}{\multirow{-2}{*}{\cellcolor[HTML]{C0C0C0}Coseno}} & \multicolumn{1}{Sc|}{\multirow{-2}{*}{\cellcolor[HTML]{C0C0C0}Tángente}} \\ \hline
	\multicolumn{1}{|Sr|}{$90^\circ$}                              & \multicolumn{1}{Sr|}{$\frac{\pi}{2}$}                                & \multicolumn{1}{Sr|}{1}                                              & \multicolumn{1}{Sr|}{0}                                                & \multicolumn{1}{Sr|}{$\nexists$}                                                  \\ \hline
	\multicolumn{1}{|Sr|}{$120^\circ$}                     & \multicolumn{1}{Sr|}{$\frac{2\pi}{3}$}                  & \multicolumn{1}{Sr|}{$\frac{\sqrt{3}}{2}$}                                  & \multicolumn{1}{Sr|}{$-\frac{1}{2}$}                             & \multicolumn{1}{Sr|}{$\sqrt{3}$}                               \\ \hline
	\multicolumn{1}{|Sr|}{$135^\circ$}                     & \multicolumn{1}{Sr|}{$\frac{3\pi}{4}$}                  & \multicolumn{1}{Sr|}{$\frac{\sqrt{2}}{2}$}                           & \multicolumn{1}{Sr|}{$\frac{-\sqrt{2}}{2}$}                             & \multicolumn{1}{Sr|}{$-1$}                                                  \\ \hline
	\multicolumn{1}{|Sr|}{$150^\circ$}                     & \multicolumn{1}{Sr|}{$\frac{5\pi}{6}$}                  & \multicolumn{1}{Sr|}{$\frac{1}{2}$}                           & \multicolumn{1}{Sr|}{-$\frac{\sqrt{3}}{2}$}                                    & \multicolumn{1}{Sr|}{$-\frac{\sqrt{3}}{3}$}                                         \\ \hline
	\multicolumn{1}{|Sr|}{$180^\circ$}                     & \multicolumn{1}{Sr|}{$\pi$}                  & \multicolumn{1}{Sr|}{0}                                              & \multicolumn{1}{Sr|}{$-1$}                                                & \multicolumn{1}{Sr|}{0}                                         \\ \hline
\end{tabular}\\

\begin{tabular}{SlSlSlSlSl}
	\rowcolor[HTML]{9B9B9B} 
	\multicolumn{5}{c}{\cellcolor[HTML]{9B9B9B}Cuadrante III}                                                                                                                                                                                                                                                                             \\
	\rowcolor[HTML]{C0C0C0} 
	\multicolumn{2}{|Sc|}{\cellcolor[HTML]{C0C0C0}Ángulo}                                                         & \multicolumn{1}{Sc|}{\cellcolor[HTML]{C0C0C0}}                       & \multicolumn{1}{Sc|}{\cellcolor[HTML]{C0C0C0}}                         & \multicolumn{1}{Sc|}{\cellcolor[HTML]{C0C0C0}}                           \\
	\rowcolor[HTML]{C0C0C0} 
	\multicolumn{1}{|Sc|}{\cellcolor[HTML]{C0C0C0}Grados} & \multicolumn{1}{c|}{\cellcolor[HTML]{C0C0C0}Radianes} & \multicolumn{1}{Sc|}{\multirow{-2}{*}{\cellcolor[HTML]{C0C0C0}Seno}} & \multicolumn{1}{Sc|}{\multirow{-2}{*}{\cellcolor[HTML]{C0C0C0}Coseno}} & \multicolumn{1}{Sc|}{\multirow{-2}{*}{\cellcolor[HTML]{C0C0C0}Tángente}} \\ \hline
	\multicolumn{1}{|Sr|}{$180^\circ$}                              & \multicolumn{1}{Sr|}{$\pi$}                                & \multicolumn{1}{Sr|}{0}                                              & \multicolumn{1}{Sr|}{-1}                                                & \multicolumn{1}{Sr|}{0}                                                  \\ \hline
	\multicolumn{1}{|Sr|}{$210^\circ$}                     & \multicolumn{1}{Sr|}{$\frac{7\pi}{6}$}                  & \multicolumn{1}{Sr|}{-$\frac{1}{2}$}                                  & \multicolumn{1}{Sr|}{-$\frac{\sqrt{3}}{2}$}                             & \multicolumn{1}{Sr|}{$\frac{\sqrt{3}}{3}$}                               \\ \hline
	\multicolumn{1}{|Sr|}{$225^\circ$}                     & \multicolumn{1}{Sr|}{$\frac{5\pi}{4}$}                  & \multicolumn{1}{Sr|}{$-\frac{\sqrt{2}}{2}$}                           & \multicolumn{1}{Sr|}{$-\frac{\sqrt{2}}{2}$}                             & \multicolumn{1}{Sr|}{1}                                                  \\ \hline
	\multicolumn{1}{|Sr|}{$240^\circ$}                     & \multicolumn{1}{Sr|}{$\frac{4\pi}{3}$}                  & \multicolumn{1}{Sr|}{$-\frac{\sqrt{3}}{2}$}                           & \multicolumn{1}{Sr|}{$-\frac{1}{2}$}                                    & \multicolumn{1}{Sr|}{$\sqrt{3}$}                                         \\ \hline
	\multicolumn{1}{|Sr|}{$270^\circ$}                     & \multicolumn{1}{Sr|}{$\frac{3\pi}{2}$}                  & \multicolumn{1}{Sr|}{-1}                                              & \multicolumn{1}{Sr|}{0}                                                & \multicolumn{1}{Sr|}{$\nexists$}                                         \\ \hline
\end{tabular}
\begin{tabular}{SlSlSlSlSl}
	\rowcolor[HTML]{9B9B9B} 
	\multicolumn{5}{c}{\cellcolor[HTML]{9B9B9B}Cuadrante IV}                                                                                                                                                                                                                                                                             \\
	\rowcolor[HTML]{C0C0C0} 
	\multicolumn{2}{|Sc|}{\cellcolor[HTML]{C0C0C0}Ángulo}                                                         & \multicolumn{1}{Sc|}{\cellcolor[HTML]{C0C0C0}}                       & \multicolumn{1}{Sc|}{\cellcolor[HTML]{C0C0C0}}                         & \multicolumn{1}{Sc|}{\cellcolor[HTML]{C0C0C0}}                           \\
	\rowcolor[HTML]{C0C0C0} 
	\multicolumn{1}{|Sc|}{\cellcolor[HTML]{C0C0C0}Grados} & \multicolumn{1}{c|}{\cellcolor[HTML]{C0C0C0}Radianes} & \multicolumn{1}{Sc|}{\multirow{-2}{*}{\cellcolor[HTML]{C0C0C0}Seno}} & \multicolumn{1}{Sc|}{\multirow{-2}{*}{\cellcolor[HTML]{C0C0C0}Coseno}} & \multicolumn{1}{Sc|}{\multirow{-2}{*}{\cellcolor[HTML]{C0C0C0}Tángente}} \\ \hline
	\multicolumn{1}{|Sr|}{$270^\circ$}                              & \multicolumn{1}{Sr|}{$\frac{3\pi}{2}$}                                & \multicolumn{1}{Sr|}{-1}                                              & \multicolumn{1}{Sr|}{0}                                                & \multicolumn{1}{Sr|}{$\nexists$}                                                  \\ \hline
	\multicolumn{1}{|Sr|}{$300^\circ$}                     & \multicolumn{1}{Sr|}{$\frac{5\pi}{3}$}                  & \multicolumn{1}{Sr|}{$-\frac{\sqrt{3}}{2}$}                                  & \multicolumn{1}{Sr|}{$\frac{1}{2}$}                             & \multicolumn{1}{Sr|}{$-\sqrt{3}$}                               \\ \hline
	\multicolumn{1}{|Sr|}{$315^\circ$}                     & \multicolumn{1}{Sr|}{$\frac{7\pi}{4}$}                  & \multicolumn{1}{Sr|}{$-\frac{\sqrt{2}}{2}$}                           & \multicolumn{1}{Sr|}{$\frac{\sqrt{2}}{2}$}                             & \multicolumn{1}{Sr|}{$-1$}                                                  \\ \hline
	\multicolumn{1}{|Sr|}{$330^\circ$}                     & \multicolumn{1}{Sr|}{$\frac{11\pi}{6}$}                  & \multicolumn{1}{Sr|}{$-\frac{1}{2}$}                           & \multicolumn{1}{Sr|}{$\frac{\sqrt{3}}{2}$}                                    & \multicolumn{1}{Sr|}{$-\frac{\sqrt{3}}{3}$}                                         \\ \hline
	\multicolumn{1}{|Sr|}{$360^\circ$}                     & \multicolumn{1}{Sr|}{$2\pi$}                  & \multicolumn{1}{Sr|}{0}                                              & \multicolumn{1}{Sr|}{$1$}                                                & \multicolumn{1}{Sr|}{0}                                         \\ \hline
\end{tabular}

\section{Razones inversas.}
A la hora de calcular un ángulo a partir de la razón la calculadora solo nos da un resultado. Pero, tal y como se deduce de las definiciones de las razones y los cuadrantes, hay más resultados posibles y en ocasiones tendremos que razonar cual es el que buscamos realmente.\\
Los posibles resultados para cada razón son:
\begin{itemize}
	\item $\alpha = \asen x \Rightarrow \left\lbrace\begin{array}{l}
		\alpha + 2\pi k\\
		\pi - \alpha + 2\pi k
	\end{array}\right.$
	\item $\alpha = \acos x \Rightarrow \left\lbrace\begin{array}{l}
		\alpha + 2\pi k\\
		2\pi - \alpha + 2\pi k
	\end{array}\right.$
	\item $\alpha = \atg x \Rightarrow \alpha + \pi k$
\end{itemize}

\section{Razones de la suma y la resta de ángulos.}
\begin{itemize}
	\item $\sen (\alpha \pm \beta) = \sen \alpha *\cos \beta \pm \cos \alpha\sen\beta$
	\item $\cos (\alpha \pm \beta) = \cos \alpha \cos \beta \mp \sen \alpha\sen \beta$
	\item $\tg (\alpha \pm \beta) = \frac{\tg \alpha \pm \tg \beta}{1\mp \tg \alpha*\tg \beta}$
\end{itemize}

\section{Razones del ángulo doble.}
\begin{itemize}
	\item $\sen (2*\alpha) = 2\sen \alpha\cos \alpha$
	\item $\cos (2*\alpha) = \cos^2 \alpha - \sen^2 \alpha$
	\item $\tg (2*\alpha) = \frac{2\tg \alpha}{1 - \tg^2 \alpha}$
\end{itemize}
\section{Razones del ángulo mitad.}
Aquí tendremos que elgir el $+$ o el $-$ dependiendo del contexto.
\begin{itemize}
	\item $\sen \frac{\alpha}{2} = \pm \sqrt{\frac{1 - \cos \alpha}{2}}$
	\item $\cos \frac{\alpha}{2} = \pm \sqrt{\frac{1 + \cos \alpha}{2}}$
	\item $\tg \frac{\alpha}{2} = \pm \sqrt{\frac{1 - \cos \alpha}{1 + \cos \alpha}} = \pm \frac{\sen \alpha}{1 + \cos \alpha}$
\end{itemize}
\section{Transformaciones de sumas en productos.}
Y viceversa.
\begin{itemize}
	\item $\sen A + \sen B = 2 \sen \frac{A + B}{2} \cos\frac{A -B}{2}$
	\item $\sen A - \sen B = 2 \sen \frac{A-B}{2} \cos \frac{A+B}{2}$
	\item $\cos A + \cos B = 2 \cos \frac{A + B}{2} \cos \frac{A - B}{2}$
	\item $\cos A - \cos B = -2\sen \frac{A + B}{2} \sen \frac{A - B}{2}$
\end{itemize}
\end{document}
