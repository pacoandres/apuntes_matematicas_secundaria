\documentclass[a4paper,11pt,answers]{exam}

\usepackage{hyperref}
\usepackage{graphicx}
%\usepackage{pstricks}
\usepackage[utf8]{inputenc}
\usepackage[spanish]{babel}
\usepackage[T1]{fontenc}
%textcomp es para el símbolo del euro
\usepackage{lmodern, textcomp}

\usepackage[left=1in, right=1in, top=1in, bottom=1in]{geometry}
%\usepackage{mathexam}
\usepackage{amsmath}
\usepackage{amssymb}
\usepackage{multicol}
\usepackage{longtable}
%para la última página
%\usepackage{lastpage}

%Para padding en celdas
\usepackage{cellspace}
\setlength\cellspacetoplimit{1mm}
\setlength\cellspacebottomlimit{1mm}

%Para hacer tachados
\usepackage[makeroom]{cancel}

%Creative commons
%\usepackage{ccicons}
\usepackage[type={CC}, modifier={by-nc-sa}, version={4.0}, %imagemodifier={-eu-80x25},
lang={spanish}]{doclicense}

%Para las gráficas:
\usepackage{tikz}
\usepackage{pgfplots}
\pgfplotsset{compat = newest}
\usetikzlibrary{babel} %Si no da errores con algunas cosas al compilar los gráficos.
\usetikzlibrary{arrows.meta,shapes,positioning}
\usetikzlibrary{matrix}
\usepgfplotslibrary{fillbetween}
\usetikzlibrary{arrows.meta}
\usetikzlibrary{fit}
\usetikzlibrary{quotes,angles}
%\usepackage{nicematrix}

\usepackage{color,colortbl}
\definecolor{Gray}{gray}{0.9}
\newcolumntype{g}{>{\columncolor{Gray}}c}
\usepackage{arydshln} %Este pone la línea punteada en la matriz ampliada. TIENE QUE ESTAR DESPUÉS DEL colortbl porque si no casca.
%\pagestyle{headandfoot}
\pagestyle{headandfoot}
\newcommand\ExamNameLine{
\par
\vspace{\baselineskip}
Nombre:\hrulefill\relax
\par}

\renewcommand{\solutiontitle}{\noindent\textbf{Solución:}\par\noindent}

\everymath{\displaystyle}
\newcommand\ddfrac[2]{\frac{\displaystyle #1}{\displaystyle #2}}

\def \autor{Paco Andrés}
\def \titulo{Apuntes de geometría analítica en el plano I.\\Coordenadas y vectores.}
\def \titulofichas {\textbf {\titulo}}
\def \cursofichas {}
\def \fechaexamen {}
%\firstpageheader{\cursofichas}{\titulofichas}{\fechaexamen}
\header{\cursofichas}{\begin{small}
\titulofichas
\end{small}}{\fechaexamen}
%\header{\cursofichas}{\titulofichas}{\fechaexamen}
%\firtspagefooter{}{\thepage}{}
%Por alguna razón no sale lo del cc en el pie
\firstpagefootrule
\footrule
\footer{\autor}{\thepage}{\doclicenseIcon}
\pointpoints{punto}{puntos}

\shadedsolutions
%\definecolor{SolutionColor}{rgb}{0.99,0.99,.99}
\renewcommand{\baselinestretch}{1.3}

%Use * instead of \cdot
\mathcode`\*="8000
{\catcode`\*\active\gdef*{\cdot}} 
\newcommand{\Card}{\,\mathrm{Card}}

%For e number
\newcommand{\e}{\,\mathrm{e}}

%Para trigonometría
\newcommand{\asen}{\,\mathrm{asen}\,}
\newcommand{\acos}{\,\mathrm{acos}\,}
\newcommand{\atg}{\,\mathrm{atg}\,}
\newcommand{\degree}{^\circ}
%Para el diferencial y la integral:
\newcommand\dif[1]{\mathrm{d}#1}
\newcommand\integral[2]{\int #1\,\dif{#2}}
\newcommand\integrald[4]{\int_{#3}^{#4} #1\,\dif{#2}}
\newcommand\adjunto[1]{#1^\text{*}}
\newcommand\rango[1]{\mathrm{rg}(#1)}

%Geometría:
\newcommand\vectort[3]{#1*\vec i + #2*\vec j + #3*\vec k}
\newcommand\distancia[2]{\text{d}(#1,#2)}
%Para escribir explicaciones encima del igual:
%\newcommand\igexpl[1]{{\mathrel{\overset{\makebox{\mbox{\normalfont\tiny\sffamily $#1$}}}{=}}}}
%Parece que mejor con stackrel

%Para las unidades:
\newcommand{\unidad}[1]{\,\text{#1}}


\renewcommand{\questionlabel}{\textbf{Ejemplo \thequestion:}}

%Colores
\definecolor{gridgray}{gray}{0.7}
\pgfplotsset{grid style={color=gridgray}}
\begin{document}


%\author{Paco Andrés}
\title{\titulo}
\date{}
\author{\autor}
\maketitle

\begin{center}
\doclicenseLongText\\
\vspace{.25cm}
\doclicenseImage
\end{center}
\tableofcontents
\newpage

\section{Coordenadas en el plano.}

En matemáticas cada \textbf{posición} en el plano se define con \textbf{una pareja de valores}
que indican la \textbf{distancia horizontal y la altura con respecto} a un punto determinado
llamado \textbf{origen}.\\

Esta pareja de valores se escribe entre paréntesis y separada por una coma o un punto y coma
(esto último es obligatorio cuando trabajemos con decimales).\\

El primer valor siempre es la distancia horizontal y el segundo la altura, con el siguiente
criterio de signos:
\begin{itemize}
\item En horizontal: positivo indica a la derecha del origen, negativo a la izquierda.
\item En vertical: positivo indica por encima del origen, negativo por debajo.
\end{itemize}

De esta manera $(-1,2)$ indica una posición que está $1$ unidad a la izquierda del origen y
$2$ unidades por encima. Gráficamente:
\begin{center}
\begin{tikzpicture}
	\begin{axis}[xmin=-3, xmax=3, ymin=-3, ymax=3, xtick={-3, -2, - 1, 0, 1, 2, 3}, ytick={-3, -2, - 1, 0, 1, 2, 3}, axis x line=center, axis y line=center, grid=both]
		\addplot[mark=*, only marks] coordinates {(-1, 2)};
	\end{axis}
\end{tikzpicture}
\end{center}

Para hacer más sencilla la comprensión gráfica se añaden unos ejes en los que se indican las
distancias horizontales y verticales al origen (que es el punto en el que se cortan los ejes)
y, a veces, se añade una cuadrícula para mayor
simplicidad. A lo largo de este primer cuadernillo de apuntes se intentará representar todo de
esta manera salvo en situaciones en las que consideremos que el dibujo de los ejes o la cuadrícula
pueda dificultar la comprensión de lo que pretendemos explicar.

\section{El punto.}
El punto es el elemento geométrico más elemental, todo se define a partir de él y esto hace que
su definición precisa no sea posible sin hacer referencia a otros objetos que dependen de él.
Aún así vamos a dar una definición algo cogida por los pelos.\\

El punto no tiene dimensiones, ni longitud, ni superficie ni volumen. Se corresponde con una
posición en el espacio, de manera que se indica de la misma manera que ésta pero añadiéndole un
nombre que siempre es una letra mayúscula.\\

De esta manera los puntos $A(3, 0)$, $B(-1, -1)$ y $C\left(2, -\frac{3}{2}\right)$ son:
\begin{center}
  \begin{tikzpicture}
    \begin{axis}[xmin=-4, xmax=4, ymin=-4, ymax=4, xtick={-3, -2, - 1, 0, 1, 2, 3}, ytick={-3, -2, - 1, 0, 1, 2, 3}, axis x line=center, axis y line=center, grid=both]
      % \addplot[mark=*, only marks] coordinates {(3,0) (-1,-1) (2,.5)};
      \addplot[mark=*, only marks] coordinates {(3,0)} node[above] {$A$};
      \addplot[mark=*, only marks] coordinates {(-1,-1)} node[left] {$B$};
      \addplot[mark=*, only marks] coordinates {(2,-1.5)} node[below] {$C$};
    \end{axis}
  \end{tikzpicture}
\end{center}

Dependiendo del contexto los puntos se nombran a partir de la $A$ o de la $P$, o de otra letra
si el contexto lo hace necesario. Lo importante es que es una letra mayúscula.
\section{El vector.}
En matemáticas existen múltiples definiciones de vectores, más sencillas o más complejas,
cada una de ellas prestando más o menos atención a determinadas características (que veremos más
adelante).

La definición que vamos a utilizar aquí es una definición intuitiva, basada en el movimiento:
\begin{center}
  \emph{\textbf{Un vector es el desplazamiento que hay que realizar para ir de un punto a otro.}}  
\end{center}
Y la ventaja de esta definición, además de ser fácilmente comprensible, es que lleva implícito
el mecanismo para calcular un vector, porque \textbf{¿cómo se calcula el desplazamiento de un lugar a otro?}.

Pensemos en cómo calculamos el desplazamiento para ir del 2º al 5º piso, \textbf{lo que hacemos es
  restar} $5-2 = 3$, el desplazamiento es de 3 pisos.

De esta manera el vector que nos lleva desde el punto $A$ al punto $B$ (que se escribe $\overrightarrow{AB}$, con una flecha encima) será:
\[\overrightarrow{AB} = B -A\quad\quad\text{(el desplazamiento siempre es destino menos origen)}\]
\begin{center}
  \small{(En algunos textos en lugar de la flecha lo que se hace es escribir el vector en
    negrita $\overrightarrow{AB} = \boldsymbol{AB}$, pero no es lo normal)}
\end{center}

Pero, ¿cómo podemos restar un punto a otro si tiene dos valores entre paréntesis?.\\
Lo que tenemos que hacer es no mezclar la parte horizontal con la vertical, tenemos que operarlas
siempre por separado.\\
De esta manera, si $A(A_x, A_y)$ y $B(B_x,B_y)$ el vector $\overrightarrow{AB}$ será:
\[\overrightarrow{AB} = B - A = (B_x,B_y) - (A_x,A_y) = (B_x - A_x, B_y - A_y)\]
Es decir, la parte horizontal del vector $\overrightarrow{AB}$, que se llama \textit{componente
  horizontal}, es la resta de las coordenadas horizontales de los puntos. Y lo mismo ocurre para la
componente vertical del vector.\\

A partir de esta definición es fácil darse cuenta de que $\overrightarrow{AB}$ no es el mismo vector
que $\overrightarrow{BA}$, el primero es el que va a $A$ a $B$ y el segundo al revés, de manera que
son vectores opuestos:
\begin{itemize}
\item $\overrightarrow{AB} = B-A$.
\item $\overrightarrow{BA} = A - B$.
\end{itemize}

Vamos a ver unos ejemplos y así vemos cómo se interpreta geométricamente el vector.

\begin{questions}
\question Calcula y dibuja el vector que va del punto $A(3,-2)$ al punto $B(-1, 1)$.
  \begin{solution}
    El cálculo del vector es sencillo, teniendo cuidado con los signos:
    \[\overrightarrow{AB} = (-1, 1) - (3, -2) = (-1-3, 1-(-2)) = (-4, 3)\]
    
    Y la representación de la situación:
    \begin{center}
      \begin{tikzpicture}
        \begin{axis}[xmin=-4, xmax=4, ymin=-4, ymax=4, xtick={-3, -2, - 1, 0, 1, 2, 3}, ytick={-3, -2, - 1, 0, 1, 2, 3}, axis x line=center, axis y line=center, grid=both]
          % \addplot[mark=*, only marks] coordinates {(3,0) (-1,-1) (2,.5)};
          \addplot[mark=*, only marks] coordinates {(3,-2)} node[above] {$A$};
          \addplot[mark=*, only marks] coordinates {(-1,1)} node[left] {$B$};
          \draw[arrows={-Latex[length=3mm]}] (3,-2) -- (-1,1);
        \end{axis}
      \end{tikzpicture}
    \end{center}
  \end{solution}
Ahora vamos a interpretar lo que significa $\overrightarrow{AB} = (-4,3)$ en el dibujo.\\
Si nos fijamos en la parte horizontal el vector recorre cuatro unidades (cuadritos)
hacia la izquierda,
por eso la componente horizontal es $-4$; mientras que la componente vertical recorre tres
unidades hacia arriba y esa componente es $3$ positivo.

A modo de regla tenemos que:
\begin{itemize}
\item Si la componente horizontal es positiva significa que el desplazamiento es hacia la
  derecha, mientras que si es negativo es a la izquierda.
\item Si la componente vertical es positiva significa que el desplazamiento es hacia arriba,
  mientras que si es negativa el desplazamiento es hacia abajo.
\end{itemize}

\question Calcula a qué punto se desplaza el punto $A(3,-2)$ mediante el vector
  $\overrightarrow{AB} = (-4, -1)$.
  \begin{solution}
    Si hemos visto que el vector se calcula como el destino menos el origen
    \[\overrightarrow{AB} = B-A\]
    para calcular el destino solo tenemos que despejar:
    \[B = \overrightarrow{AB} + A\]
    \[B = (-4, -1) + (3, -2)\]
    Y sumaremos como antes, horizontal con horizontal y vertical con vertical:
    \[B = (-4+ 3, -1 + (-2)) = (-1, -3)\]

    De manera gráfica para poder interpretarlo:
    \begin{center}
      \begin{tikzpicture}
        \begin{axis}[xmin=-4, xmax=4, ymin=-4, ymax=4, xtick={-3, -2, - 1, 0, 1, 2, 3}, ytick={-3, -2, - 1, 0, 1, 2, 3}, axis x line=center, axis y line=center, grid=both]
          % \addplot[mark=*, only marks] coordinates {(3,0) (-1,-1) (2,.5)};
          \addplot[mark=*, only marks] coordinates {(3,-2)} node[above] {$A$};
          \addplot[mark=*, only marks] coordinates {(-1,-3)} node[left] {$B$};
          \draw[arrows={-Latex[length=3mm]}] (3,-2) -- (-1,-3);
        \end{axis}
      \end{tikzpicture}
    \end{center}
  \end{solution}
  La interpretación gráfica es que hemos dibujado el punto $A$ en las coordenadas correspondientes,
  después hemos dibujado $\overrightarrow{AB}$ desde $A$, $4$ unidades a la izquierda y una
  hacia abajo como indican sus componentes, y en el punto en el que termina
  hemos situado $B$.
  Y vemos que el resultado gráfico de $B$ coincide con el calculado.
\end{questions}

\subsection{Propiedades de los vectores.} \label{propiedades_vectores}
Al estar compuestos de varios valores y tener el significado que tienen hace que los vectores
tengan unas propiedades fundamentales que no tiene ningún otro objeto matemático que hayamos
visto hasta ahora.\\

Estas propiedades son las siguientes:
\begin{itemize}
\item \textbf{Módulo}: que es el tamaño del vector. Es fácil ver que, geométricamente, el vector y sus componentes forman un triángulo rectángulo en el que el vector es la hipotenusa.
  \begin{center}
    \begin{tikzpicture}
      \coordinate (O) at (0,0);
      \coordinate (A) at (3,2);
      \coordinate (B) at (3,0);
      \draw[-Latex] (O)--(A) node[midway, above, sloped] {$\vec{v}$};
      \draw[dashed] (O)--(B) node[midway, below] {$v_x$};
      \draw[dashed] (B)--(A) node[midway, right] {$v_y$}; 
    \end{tikzpicture}
  \end{center}
  Entonces podemos utilizar el teorema de Pitágoras para calcular el módulo del vector, que
  se escribe poniéndolo entre dos barras verticales:
  \[|\vec{v}| = \sqrt{v_x^2 + v_y^2}\]
  Es importante darse cuenta de que \textbf{el módulo es siempre positivo}, ya que es un tamaño
  y no hay tamaños negativos.
\item \textbf{Dirección}: la dirección de un vector es la recta que lo contiene. Para entenderlo
  mejor podemos imaginar que la dirección de un vector es la inclinación que tiene este con
  respecto a la horizontal.\\
  Muchas veces se utiliza como ``sinónimo'' \textbf{la pendiente}, que es \textbf{la tangente del
    ángulo que forma el vector con la horizontal positiva} y, teniendo en cuenta que el vector con sus
  componentes forma un triángulo rectángulo, la pendiente se calcula como:
  \begin{center}
    Pendiente$=\tg \alpha = \frac{v_y}{v_x}$
    \small{(Aprovechamos esta propiedad para recordar que es necesario saber
      trigonometría básica para poder aprender geometría analítica)}
  \end{center}
  De aquí se saca la conclusión de que \textbf{dos vectores que tienen la misma dirección} forman el
  mismo ángulo con la horizontal y por tanto \textbf{son paralelos}.
\item \textbf{Sentido}: el sentido del vector es hacia que lado de la dirección apunta. El sentido
  viene indicado por el signo de las componentes.
\end{itemize}

\textbf{Para que dos vectores sean iguales tienen que tener el mismo módulo, la misma dirección
  y el mismo sentido.}\\
Por eso en la definición de vectores hemos dicho que $\overrightarrow{AB}$ y $\overrightarrow{BA}$
no son el mismo vector ya que tienen igual módulo y dirección pero distinto sentido, por lo que
hemos dicho que son opuestos.

\begin{questions}
\question Calcula el módulo y el ángulo que forma con la horizontal positiva el vector $\vec{v} = (-1,3)$.
  \begin{solution}
    Por lo que hemos visto el módulo se calcula utilizando el teorema de Pitágoras:
    \[|\vec{v}| = \sqrt{(-1)^2 + 3^2} = \sqrt{10}\]

    Y el ángulo que forma con la horizontal positiva es el arco de tangente de su pendiente:
    \[\tg \alpha = \frac{3}{-1} = -3\]
    \[\alpha = \atg (-3) \simeq -71.57\degree = 288.43\degree\]
    El ángulo lo podemos dejar en negativo o positivo según nos convenga o nos diga el contexto.\\
    Más adelante veremos que a veces hay que aplicar correcciones al resultado que nos
    da la calculadora, como ya vimos en trigonometría.
  \end{solution}
\end{questions}
\subsection{Vectores unitarios.}\label{vector_unitario}
Se dice que que un vector es unitario cuando su módulo es $1$.\\
Más adelante veremos como obtener un vector unitario en la dirección y sentido de otro vector.

\subsection{Vectores ligados y vectores libres.}
Esta es una definición que depende completamente del contexto en el que estemos trabajando y no
afecta a la manera en que realicemos las operaciones.\\

Se dice que un vector es ligado cuando está asociado a un punto de origen. En el ejemplo que hemos
utilizado para la definición de vector hemos usado un vector ligado y por eso en el nombre
hemos puesto el nombre de los puntos ($\overrightarrow{AB}$).\\
El \textbf{punto en el que empieza} el vector se llama \textbf{punto de aplicación}.\\
A los vectores ligados a veces también se les llama vectores fijos.\\

Se dice que un vector es libre cuando no está asociado a un punto en concreto. Este caso sería el
que hemos utilizado para definir las propiedades básicas de los vectores y por eso lo hemos
llamado con una letra minúscula ($\vec{v}$). Cuando un vector es libre siempre se nombra con
una letra minúscula.

\section{Operaciones con vectores.}
Todas las operaciones con vectores tienen dos formas de hacerse, la analítica y la geométrica.\\

La analítica es más sencilla ya que consiste en sumas, restas, \dots Es decir, operaciones
aritméticas básicas. Pero esta manera no nos proporciona información sobre lo que está
sucediendo.\\

La geométrica es más compleja ya que es necesario dibujar todo, pero nos permite saber qué está
sucediendo y así poder interpretar el resultado o saber que operación tenemos que realizar
para conseguir un objetivo.\\

Es evidente que hay que saber y entender las dos maneras ya que el procedimiento a utilizar va a
ser el siguiente:
\begin{enumerate}
\item Dibujar la situación inicial.
\item Pensar en las operaciones geométricas que tenemos que hacer para conseguir el
  objetivo.
\item Realizar de manera analítica las operaciones del punto anterior para llegar al resultado.
\end{enumerate}

Y es por esto que en cada operación se va a explicar el cálculo analítico y la interpretación
geométrica de la operación, tanto en la explicación como en los ejemplos.

\subsection{Opuesto de un vector.}
Dos vectores son opuestos si tienen el mismo módulo y la misma dirección pero distinto sentido.\\
Para calcular el opuesto de un vector solo hay que cambiar de signo a las componentes.\\
Si tenemos $\vec{v} =(v_x, v_y)$ el opuesto será:
\[-\vec{v} = (-v_x, -v_y)\]

La interpretación geométrica es:
\begin{center}
  \begin{tikzpicture}
    \draw[-Latex] (0,0)--(2,1) node[midway, sloped, above] {$\vec{v}$};
    \draw[Latex-] (0,-1)--(2,0) node[midway, sloped, above] {$-\vec{v}$};
  \end{tikzpicture}
\end{center}
Como se puede ver ambos tienen el mismo módulo y la misma dirección, pero distinto sentido.

\begin{questions}
\question Escribe un vector $\vec{w}$ opuesto a $\vec{v} = (-2,3)$.
  \begin{solution}
    Este ejemplo es muy sencillo:
    \[\vec{w} = -\vec{v} = (2,-3)\]
  \end{solution}
\end{questions}
\subsection{Suma de punto y vector.} \label{suma_punto_vector}
Esta operación ya se ha explicado en la definición de vector, ya que está íntimamente ligada
a esta definición, pero no está de más volverla a contar, tanto de manera analítica como
geométrica.\\

Si tenemos un punto $A(A_x, A_y)$ y un vector $\vec{v}= (v_x, v_y)$ el resultado de su suma
($A + \vec{v}$) es un punto cuyas coordenadas son:
\[B(B_x,B_y) = (A_x + v_x, A_y + v_y)\]

Geométricamente es el punto que en el que acaba el vector cuando se aplica en el punto $A$:
\begin{center}
  \begin{tikzpicture}[baseline=(current bounding box.center)]
    \begin{axis}[xmin=-2, xmax=2, ymin=-3, ymax=3, axis lines =none]
      \addplot[mark=*, only marks] coordinates {(-1,-2)} node[left] {$A$};
      \addplot[draw=none] coordinates {(1,2)};
      \node at (-.5,1.5) {\textbf{1.}};
    \end{axis}
  \end{tikzpicture}
  \quad
  $\longrightarrow$
  \quad
  \begin{tikzpicture}[baseline=(current bounding box.center)]
    \begin{axis}[xmin=-2, xmax=2, ymin=-3, ymax=3, axis lines =none]
      \addplot[mark=*, only marks] coordinates {(-1,-2)} node[left] {$A$};
      \draw[arrows={-Latex[length=3mm]}] (-1,-2) -- (1,2) node[midway, above, sloped]
      {$\vec{v}$};
      \node at (-.5,1.5) {\textbf{2.}};
    \end{axis}
  \end{tikzpicture}
  \quad
  $\longrightarrow$
  \quad
  \begin{tikzpicture}[baseline=(current bounding box.center)]
    \begin{axis}[xmin=-2, xmax=2, ymin=-3, ymax=3, axis lines =none]
      \addplot[mark=*, only marks] coordinates {(-1,-2)} node[left] {$A$};
      \addplot[mark=*, only marks] coordinates {(1,2)} node[right] {$B$};
      \draw[arrows={-Latex[length=3mm]}] (-1,-2) -- (1,2) node[midway, above, sloped]
      {$\vec{v}$};
      \node at (-.5,1.5) {\textbf{3.}};
    \end{axis}
  \end{tikzpicture}
\end{center}
Tal y como indica el dibujo la operación geométrica es
\begin{enumerate}
\item Dibujamos el punto de partida.
\item Aplicamos el vector en el punto de partida avanzando en horizontal y vertical los valores
  indicados en las componentes.
\item El resultado es el punto en el que finaliza el vector.
\end{enumerate}
\subsection{Suma de vectores.} \label{suma_vectores}
\textbf{La suma de vectores es otro vector cuya componente horizontal es la suma de las componentes horizontales y cuya componente vertical es la suma de componentes verticales.}\\
Es decir, si $\boldsymbol{\vec{u} = (u_x, u_y)}$ y $\boldsymbol{\vec{v} = (v_x, v_y)}$ entonces:
\[\boldsymbol{\vec{u} + \vec{v} = (u_x + v_x, u_y + v_y)}\]

Geométricamente la suma de vectores es como poner un vector a continuación del otro, el vector
resultante tiene el mismo origen que el primero y el mismo destino que el segundo.
\begin{center}
  \begin{multicols}{2}
    \begin{tikzpicture}
      % \begin{axis}
      \draw[-latex] (0,0) -- (2,3) node[midway,above,sloped] {$\vec u$};
      \draw[-latex] (2,3) -- (4,1) node[midway,above,sloped] {$\vec v$};
      \draw[ultra thick, -latex] (0,0) -- (4,1) node[midway,above,sloped] {$\vec u + \vec{v}$};
      % \end{axis}
    \end{tikzpicture}
    
    \begin{tikzpicture}
      
      \draw[-latex] (0,0) -- (2,-2) node[midway,above,sloped] {$\vec v$};
      \draw[-latex] (2,-2) -- (4,1) node[midway,above,sloped] {$\vec u$};
      \draw[ultra thick, -latex] (0,0) -- (4,1) node[midway,above,sloped] {$\vec u + \vec{v}$};
      
    \end{tikzpicture}
  \end{multicols}
\end{center}

En la representación geométrica hemos puesto la operación en dos ordenes distintos y es fácil ver
que el resultado es el mismo, lo que quiere decir que la suma de vectores es conmutativa.\\

El igual que ocurre con los enteros, la resta de vectores es equivalente a la suma con el
opuesto del segundo.

Vamos a ver un par de ejemplos de manera analítica y gráfica.
\begin{questions}
\question Dados los vectores $\vec{u} = (3,2)$ y $\vec{v} = (-1, 2)$ calcula y dibuja
  $\vec{w} = \vec{u} + \vec{v}$.
  \begin{solution}
    El cálculo es sencillo:
    \[\vec{w} = (3,2) + (-1,2) = (2,4)\]

    Vamos a hacer ahora el cálculo geométrico, primero empezando por $\vec{u}$ y luego
    empezando por $\vec{v}$ para comprobar que sale lo mismo (cada cuadradito representa
    una unidad):

    \begin{center}
      \begin{tikzpicture}[baseline=(current bounding box.center), scale=.9]
        \begin{axis}[axis equal, xmin=-1, xmax=4, ymin=-1, ymax=5, axis lines = middle,
          axis line style={draw=gridgray}, xticklabel=\empty, yticklabel=\empty,
          tick label style={major tick length=0pt}, grid=major, ytick={-1, 0, 1, 2, 3, 4, 5}]
          \draw[-Latex] (0,0) --(3,2) node[midway, sloped, above] {$\vec{u}=(3,2)$};
          \node at (-1,4) {\textbf{1.}};
        \end{axis}
      \end{tikzpicture}
      \quad
      $\longrightarrow$
      \quad
      \begin{tikzpicture}[baseline=(current bounding box.center), scale=.9]
        \begin{axis}[axis equal, xmin=-1, xmax=4, ymin=-1, ymax=5, axis lines = middle,
          axis line style={draw=gridgray}, xticklabel=\empty, yticklabel=\empty,
          tick label style={major tick length=0pt}, grid=major, ytick={-1, 0, 1, 2, 3, 4, 5}]
          \draw[-Latex] (0,0) --(3,2) node[midway, sloped, below] {$\vec{u}=(3,2)$};
          \draw[-Latex] (3,2) -- (2,4) node[midway, sloped, above] {$\vec{v}=(-1,2)$};
          \draw[ultra thick, -Latex] (0,0) -- (2,4) node[midway, sloped, above] {$\vec{w}$};
          \node at (-1,4) {\textbf{2.}};
        \end{axis}
      \end{tikzpicture}
    \end{center}

    Hemos dibujado primero $\vec{u}$ (tres unidades a la derecha y dos hacia arriba) y a
    continuación $\vec{v}$ (una unidad a la izquierda y dos hacia arriba). El resultado es
    el vector que une el origen de $\vec{u}$ y el destino de $\vec{v}$ y recorre dos unidades a la
    derecha y cuatro hacia arriba, que es lo que nos ha salido en el cálculo.\\

    Vamos a hacerlo ahora al revés y veremos que el resultado es el mismo:
    \begin{center}
      \begin{tikzpicture}[baseline=(current bounding box.center), scale=.9]
        \begin{axis}[axis equal, xmin=-2, xmax=3, ymin=-1, ymax=5, axis lines = middle,
          axis line style={draw=gridgray}, xticklabel=\empty, yticklabel=\empty,
          tick label style={major tick length=0pt}, grid=major, ytick={-1, 0, 1, 2, 3, 4, 5}]
          \draw[-Latex] (0,0) --(-1,2) node[midway, sloped, above] {$\vec{v}=(-1,2)$};
          \node at (-2,4) {\textbf{1.}};
        \end{axis}
      \end{tikzpicture}
      \quad
      $\longrightarrow$
      \quad
      \begin{tikzpicture}[baseline=(current bounding box.center), scale=.9]
        \begin{axis}[axis equal, xmin=-2, xmax=3, ymin=-1, ymax=5, axis lines = middle,
          axis line style={draw=gridgray}, xticklabel=\empty, yticklabel=\empty,
          tick label style={major tick length=0pt}, grid=major, ytick={-1, 0, 1, 2, 3, 4, 5}]
          \draw[-Latex] (0,0) --(-1,2) node[midway, sloped, below] {$\vec{v}=(-1,2)$};
          \draw[-Latex] (-1,2) -- (2,4) node[midway, sloped, above] {$\vec{u}=(3,2)$};
          \draw[ultra thick, -Latex] (0,0) -- (2,4) node[midway, sloped, below] {$\vec{w}$};
          \node at (-2,4) {\textbf{2.}};
        \end{axis}
      \end{tikzpicture}
    \end{center}

    Hemos empezado dibujando $\vec{v}$ y a continuación $\vec{u}$, y el resultado sigue teniendo
    dos unidades a la derecha y cuatro hacia arriba.\\
    Esto nos quiere decir que la suma de vectores es conmutativa.
  \end{solution}
\question Dados los mismos vectores que en el enunciado anterior calcula $\vec{a} = \vec{u} -
  \vec{v}$.
  \begin{solution}
    Recordemos los vectores que nos daban $\vec{u} = (3,2)$ y $\vec{v} = (-1, 2)$.\\

    El cálculo sigue siendo sencillo:
    \[\vec{a} = (3,2) - (-1, 2) = (3-(-1), 2-2) = (4,0)\]

    Para hacer la operación geométricamente vamos a tener en cuenta que la resta es lo mismo
    que sumar con el opuesto. Entonces
    \[\vec{a} = \vec{u} -\vec{v} = \vec{u} + (-\vec{v})\]

    Con lo que para el cálculo geométrico vamos a dibujar $\vec{u}$ y a continuación
    $-\vec{v} = (1, -2)$:

    \begin{center}
      \begin{tikzpicture}[baseline=(current bounding box.center), scale=.9]
        \begin{axis}[axis equal, xmin=0, xmax=5, ymin=-1, ymax=4, axis lines = middle,
          axis line style={draw=gridgray}, xticklabel=\empty, yticklabel=\empty,
          tick label style={major tick length=0pt}, grid=major, ytick={-1, 0, 1, 2, 3, 4}]
          \draw[-Latex] (0,0) --(3,2) node[midway, sloped, above] {$\vec{u}=(3,2)$};
          \node at (0,3) {\textbf{1.}};
        \end{axis}
      \end{tikzpicture}
      \quad
      $\longrightarrow$
      \quad
      \begin{tikzpicture}[baseline=(current bounding box.center), scale=.9]
        \begin{axis}[axis equal, xmin=0, xmax=5, ymin=-1, ymax=4, axis lines = middle,
          axis line style={draw=gridgray}, xticklabel=\empty, yticklabel=\empty,
          tick label style={major tick length=0pt}, grid=major, ytick={-1, 0, 1, 2, 3, 4}]
          \draw[-Latex] (0,0) --(3,2) node[midway, sloped, above] {$\vec{u}=(3,2)$};
          \draw[-Latex] (3,2) -- (4,0) node[midway, sloped, above] {$-\vec{v}=(1,-2)$};
          \draw[ultra thick, -Latex] (0,0) -- (4,0) node[midway, sloped, below] {$\vec{a}$};
          \node at (0,3) {\textbf{2.}};
        \end{axis}
      \end{tikzpicture}
    \end{center}
    Vemos que dibujando $\vec{u}$ y a continuación $-\vec{v}$ el vector que obtenemos tiene las
    mismas componentes que habíamos calculado.\\
    Y esta sería la interpretación geométrica de la resta de vectores.
  \end{solution}
\end{questions}
\subsubsection{Regla del paralelogramo.} \label{regla_paralelogramo}
Para entender bien esta regla vamos a utilizar los vectores de los ejemplos de la suma de
vectores.\\

Estos vectores han sido $\vec{u} = (3,2)$ y $\vec{v} = (-1, 2)$, y lo que vamos a hacer es dibujar
el paralelogramo que tiene de lados esos dos vectores:
\begin{center}
  \begin{tikzpicture}
    \begin{axis}[axis equal, xmin=-1, xmax=4, ymin=-1, ymax=5, axis lines = middle,
      axis line style={draw=lightgray}, xticklabel=\empty, yticklabel=\empty,
      tick label style={major tick length=0pt}, grid=major, ytick={-1, 0, 1, 2, 3, 4, 5}]
      \draw[thick, -Latex] (0,0) --(3,2) node[midway, sloped, below] {$\vec{u}$};
      \draw[thick, -Latex] (0,0) --(-1,2) node[midway, sloped, below] {$\vec{v}$};
      \draw[thick,-Latex] (-1,2) --(2,4) node[midway, sloped, above] {$\vec{u}$};
      \draw[thick,-Latex] (3,2) --(2,4) node[midway, sloped, above] {$\vec{u}$};
      \addplot[mark=*, only marks] coordinates {(0, 0)};
    \end{axis}
  \end{tikzpicture}
\end{center}

Y resulta que \textbf{el vector suma $\boldsymbol{\vec{u} + \vec{v}}$ es la diagonal que va del
  punto del que salen los dos vectores al punto al que llegan los dos}:
\begin{center}
  \begin{tikzpicture}
    \begin{axis}[axis equal, xmin=-1, xmax=4, ymin=-1, ymax=5, axis lines = middle,
      axis line style={draw=lightgray}, xticklabel=\empty, yticklabel=\empty,
      tick label style={major tick length=0pt}, grid=major, ytick={-1, 0, 1, 2, 3, 4, 5}]
      \draw[dashed, thick,-Latex] (0,0) --(3,2) node[midway, sloped, below] {$\vec{u}$};
      \draw[dashed,thick,-Latex] (0,0) --(-1,2) node[midway, sloped, below] {$\vec{v}$};
      \draw[dashed,thick,-Latex] (-1,2) --(2,4) node[midway, sloped, above] {$\vec{u}$};
      \draw[dashed,thick,-Latex] (3,2) --(2,4) node[midway, sloped, above] {$\vec{u}$};
      \addplot[mark=*, only marks] coordinates {(0, 0)};
      \draw[ultra thick, -Latex] (0,0) --(2,4) node[midway, sloped, above] {$\vec{u}+\vec{v}$};
    \end{axis}
  \end{tikzpicture}
\end{center}

\textbf{El vector resta $\boldsymbol{\vec{u} - \vec{v}}$ es la diagonal que va desde el destino de
  $\boldsymbol{\vec{v}}$ al destino de $\boldsymbol{\vec{u}}$}:
\begin{center}
  \begin{tikzpicture}
    \begin{axis}[axis equal, xmin=-1, xmax=4, ymin=-1, ymax=5, axis lines = middle,
      axis line style={draw=lightgray}, xticklabel=\empty, yticklabel=\empty,
      tick label style={major tick length=0pt}, grid=major, ytick={-1, 0, 1, 2, 3, 4, 5}]
      \draw[dashed,thick, -Latex] (0,0) --(3,2) node[midway, sloped, below] {$\vec{u}$};
      \draw[dashed,thick, -Latex] (0,0) --(-1,2) node[midway, sloped, below] {$\vec{v}$};
      \draw[dashed,thick,-Latex] (-1,2) --(2,4) node[midway, sloped, above] {$\vec{u}$};
      \draw[dashed,thick,-Latex] (3,2) --(2,4) node[midway, sloped, above] {$\vec{u}$};
      \addplot[mark=*, only marks] coordinates {(0, 0)};
      \draw[ultra thick, -Latex] (-1,2)--(3,2) node[midway, sloped, above] {$\vec{u} - \vec{v}$};
    \end{axis}
  \end{tikzpicture}
\end{center}

Y si juntamos mentalmente las dos representaciones tenemos que la suma y la recta de dos
vectores son las diagonales del paralelogramo cuyos lados son los dos vectores.

\subsubsection{Propiedades de la suma de vectores.}
Es importante que conozcamos las propiedades de la suma para saber lo que podemos y no podemos
hacer. Y además nos sirven para repasar los nombres de las propiedades de las operaciones
con números.
\begin{itemize}
\item \textbf{Conmutativa}. La suma de vectores es conmutativa.
  \[\boldsymbol{\vec{a} + \vec{b} = \vec{b} + \vec{a}}\]
\item \textbf{Asociativa}.
  \[\boldsymbol{(\vec{a} + \vec{b}) + \vec{c} = \vec{a} + (\vec{b} + \vec{c})}\]
\item \textbf{Elemento neutro}. Existe el vector nulo: $\vec{o} = (0,0)$
  \[\boldsymbol{\vec{a} + \vec{o} = \vec{o} + \vec{a} = \vec{a}}\]
\item \textbf{Elemento opuesto o simétrico}. Dado $\vec{a} = (a_x, a_y)$ su opuesto es
  $-\vec{a} = (-a_x, -a_y)$, ya que:
  \[\boldsymbol{\vec{a} + (-\vec{a}) = (a_x + a_y) + (-a_x, -a_y) = (0,0) = \vec{o}}\]
\end{itemize}
\subsection{Producto de un vector por un número.} \label{producto_por_escalar}
Vamos a aprovechar este punto para introducir un concepto, que es el concepto de \emph{escalar}.
En el contexto en el que estamos, matemáticas de secundaria, \textbf{un escalar es un número}.\\
Esta definición se hace necesaria porque en muchos textos se habla de ``producto de un escalar
por un vector'', y también hablaremos de escalares en estos apuntes.\\
Dicho esto vamos a ver cómo se hace esta operación:\\

Para multiplicar un número por un vector lo tenemos que multiplicar por sus componentes, de tal
manera que el resultado es otro vector.\\
De manera simbólica, si $\vec{v} = (v_x, v_y)$ y $a \in \mathbb{R}$ entonces:
\[\boldsymbol {a*\vec{v} = (a*v_x, a*v_y)}\]

Vamos a ver la interpretación gráfica con un ejemplo, que en este caso va a hacer que se entienda mejor.\\
\textbf{Multiplica $\vec{v} = (2, 1)$ por $3$ y por $-2$.}
\begin{solution}
  Por lo que hemos dicho:
  \[3*\vec{v} = (6,3)\]
  Y
  \[-2*\vec{v} = (-4,-2)\]
  Y si lo dibujamos tenemos:
  \begin{center}
    \begin{tikzpicture}
      \begin{axis}[axis equal, xmin=-1, xmax=7, ymin=-1, ymax=5, axis lines = middle,
        axis line style={draw=lightgray}, xticklabel=\empty, yticklabel=\empty,
        tick label style={major tick length=0pt}, grid=major, ytick={-1, 0, 1, 2, 3, 4, 5},
        xtick={-1, 0, 1, 2, 3, 4, 5, 6, 7}]
        \draw[thick, -Latex] (0,1) --(2,2) node[midway, sloped, above] {$\vec{v}$};
        \draw[thick, -Latex] (0,2) --(6,5) node[midway, sloped, above] {$3*\vec{v}$};
        \draw[thick, -Latex] (4,2) --(0,0) node[midway, sloped, above] {$-2*\vec{v}$};
      \end{axis}
    \end{tikzpicture}
  \end{center}
  \begin{center}\small{(Los hemos dibujado saliendo de distintos puntos porque son vectores
      libres. Lo importante son sus propiedades)}\end{center}
\end{solution}
Vemos que el vector resultado de $3*\vec{v}$ tiene la misma dirección y el mismo sentido que el original, lo que cambia es el módulo. Y el resultado de $-2*\vec{v}$ tiene la misma dirección, distinto
sentido y también ha cambiado el módulo.\\
¿Y en \textbf{cuánto cambia el módulo}?. Pues vamos a deducirlo:\\

Tenemos $a \in \mathbb{R}$ y $\vec{v} = (v_x, v_y)$, y por la definición:
\[a*\vec{v} = (a*v_x, a*v_y)\]
Entonces el módulo es:
\[|a*\vec{v}| = \sqrt{(a*v_x)^2 + (a*v_y)^2}\]
Por las propiedades de las potencias:
\[|a*\vec{v}| = \sqrt{a^2*v_x^2 + a^2*v_y^2}\]
Sacamos factor común en el radicando:
\[|a*\vec{v}| = \sqrt{a^2*(v_x^2 + v_y^2)}\]
Y finalmente extraemos $a$ del radical:
\[|a*\vec{v}| = a*\sqrt{v_x^2 + v_y^2}\]
Nos queda por ajustar algo, ya que puede ocurrir que $a<0$, y en \ref{propiedades_vectores} (página
\pageref{propiedades_vectores}) hemos visto que el módulo tiene que ser positivo. Entonces:
\[|a*\vec{v}| = |a|*\sqrt{v_x^2 + v_y^2}\]
Con lo cual:
\[\boldsymbol{|a*\vec{v}| = |a|*|\vec{v}|}\]

Con \textbf{respecto al sentido} tenemos que $3*\vec{v}$ ha mantenido el sentido mientras que $-2*\vec{v}$ ha cambiado.\\

De todo esto podemos concluir que \textbf{el resultado de multiplicar un escalar por un vector es
  otro vector que:
  \begin{itemize}
  \item Tiene la misma dirección que el original.
  \item Su módulo es el módulo del original multiplicado por el valor absoluto del número.
  \item Su sentido se mantiene si el número es positivo y cambia si el número es negativo.
  \end{itemize}
}

Con lo que la interpretación de los ejemplos sería:
\begin{itemize}
\item $3*\vec{v}$ es un vector que:
  \begin{itemize}
  \item Tiene la misma dirección que $\vec{v}$.
  \item Su modulo es el triple del módulo de $\vec{v}$.
  \item Al ser $3$ positivo tiene el mismo sentido que $\vec{v}$.
  \end{itemize}
\item $-2*\vec{v}$ es un vector que:
  \begin{itemize}
  \item Tiene la misma dirección que $\vec{v}$.
  \item Su módulo es el doble del de $\vec{v}$.
  \item Tiene sentido contrario a $\vec{v}$ por ser $-2$ negativo.
  \end{itemize}
\end{itemize}

Y como conclusión final podemos decir que \textbf{el producto de un escalar por un vector
  siempre nos da un vector paralelo}.

\subsubsection{Propiedades del producto por un escalar.}
Al igual que con la suma, tenemos que saber las propiedades para poder operar correctamente.\\

Tenemos unos vectores $\vec{v}$, $\vec{w}$, \dots y varios números $a,b,\dots \in \mathbb{R}$:
\begin{itemize}
\item \textbf{Conmutativa.}
  \[\boldsymbol{a*\vec{v} = \vec{v}*a}\]
\item \textbf{Asociativa.}
  \[\boldsymbol{a*(b*\vec{v}) = (a*b)*\vec{v}}\]
\item \textbf{Distributiva respecto a la suma de vectores.}
  \[\boldsymbol{a*(\vec{v} + \vec{w}) = a*\vec{v} + a*\vec{w}}\]
\item \textbf{Distributiva respecto a la suma de números (o escalares).}
  \[\boldsymbol{(a+b)*\vec{v} = a*\vec{v} + b*\vec{v}}\]
\end{itemize}
\subsection{División de un vector entre un escalar.}
Es exactamente lo mismo que la multiplicación pero dividiendo:\\
Si $\vec{v} = (v_x, v_y)$ y $a \in \mathbb{R}$ y distinto de cero:
\[\frac{\vec{v}}{a} = \left(\frac{v_x}{a}, \frac{v_y}{a}\right)\]

Realmente la división no difiere de la multiplicación, ya que a efectos de resultado
dividir entre $2$ es lo mismo que multiplicar por $\frac{1}{2}$.\\
Por tanto, la interpretación geométrica va a ser la misma que en la multiplicación
(nos da un vector paralelo) salvo en el apartado del módulo, que en lugar de multiplicarse
por el número se va a dividir entre él.\\

Hay que decir que \emph{no se habla nunca de esta operación, ya que es más sencillo hacer la
multiplicación por la fracción y utilizar las propiedades de la multiplicación}.

\section{Combinaciones lineales. Bases.}
\subsection{Combinación lineal de vectores.}
Dados los vectores \{$\vec{v_1}, \vec{v_2}, \vec{v_3},\dots,\vec{v_n}$\} y los escalares
\{$a_1, a_2, a_3, \dots, a_n$\}, se llama combinación lineal a la operación
\[a_1*\vec{v_1} + a_2*\vec{v_2}+\dots+a_n*\vec{v_n}\]
El resultado de esta operación es otro vector.

\subsubsection{Dependencia e independencia lineal.}
Se dice que un vector $\vec{w}$ es linealmente dependiente de un conjunto de vectores
\{$\vec{v_1}, \vec{v_2}, \vec{v_3},\dots,\vec{v_n}$\} si se puede escribir como combinación lineal
de ellos.\\

En caso contrario se dice que es linealmente independiente.\\

\begin{questions}
\question Escribe el vector $\vec{w} = (3, 5)$ como combinación lineal de los vectores
  $\vec{u} = (1,2)$ y $\vec{v} = (-1, 0)$.
  \begin{solution}
    Por lo visto en este apartado de combinaciones lineales, para que un vector sea combinación
    lineal de otros dos tenemos que encontrar dos números $a$ y $b$ que hagan
    \[\vec{w} = a*\vec{u} + b*\vec{v}\]
    \[(3,5) = a*(1, 2) + b*(-1, 0)\]
    \[(3,5) = (a, 2a) + (-b, 0)\]

    Como sabemos que una componente es independiente de la otra podemos separarlo en dos ecuaciones:
    \[
      \begin{cases}
        3 & = a+(-b)\\
        5 &= 2a + 0
      \end{cases}
    \]
    Con lo que el ejercicio se reduce a resolver un sistema de ecuaciones.\\

    De la segunda obtenemos que $a =\frac{5}{2}$, sustituimos en la primera y
    \[3 = a - \frac{5}{2}\]
    \[a = 3 + \frac{5}{2}\]
    \[a = \frac{11}{2}\]

    Con lo que la combinación lineal es:
    \[(3, 5) = \frac{11}{2}*(1, 2) + \frac{5}{2}*(-1, 0)\]
  \end{solution}
\question Indica si el vector $\vec{w} = (2, -3)$ es linealmente dependiente de los vectores
  $\vec{u} = (-2, 4)$ y $\vec{v}= (3, -6)$.
  \begin{solution}
    Por la definición de linealmente dependiente tiene que ocurrir que podamos escribir $\vec{w}$
    como combinación lineal de los otros dos. Se reduce a resolver un problema como el anterior:
    \[(2, -3) = a*(-2, 4) + b*(3, -6)\]
    \[
      \begin{cases}
        2 & = -2a + 3b\\
        -3 &= 4a -6b
      \end{cases}
    \]
    Si multiplicamos la primera por 2 nos queda:
    \[
      \begin{cases}
        4 & = -4a + 6b\\
        -3 &= 4a -6b
      \end{cases}
    \]
    Y al reducir nos queda la ecuación
    \[1 = 0\]
    Y sabemos que esto quiere decir que \textbf{no hay solución}.\\
    Y \textbf{si no hay solución es porque $\vec{w}$ no puede escribirse como combinación lineal de
      $\vec{u}$ y $\vec{v}$}.\\
    Y si no se puede escribir como combinación lineal es porque $\vec{w}$
    \textbf{es linealmente independiente} de $\vec{u}$ y $\vec{v}$.
  \end{solution}
\end{questions}

\subsection{Base de un espacio vectorial.}
Se dice que \textbf{un conjunto de vectores linealmente independientes
  \{$\vec{v_1}, \vec{v_2}, \vec{v_3},\dots,\vec{v_n}$\} es base
  si cualquier vector se puede construir como combinación lineal de ellos}. A los escalares que
multiplican a los vectores de la base se les llama componentes o coordenadas.\\

En el caso del \textbf{plano} (que a veces también llamaremos $\mathbb{R}^2$) se llama
\textbf{base canónica} al conjunto formado por los vectores $\boldsymbol{\vec{u_1} = (1, 0)}$ y
$\boldsymbol{\vec{u_2} = (0,1)}$.\\
\emph{(Los vectores de la base canónica reciben muchos nombres que puedes encontrar en otros
  apuntes o libros: \{$\vec{e_1}, \vec{e_2}$\} ó \{$\vec{i},\vec{j}$\}. Estos últimos nombres los
  usaremos mucho en bachillerato)}\\

Es fácil ver cómo construir cualquier vector como combinación lineal de la base canónica.
\begin{questions}
\question Indica la combinación lineal de la base canónica para construir el vector
  $\vec{v} = (-2, 5)$.
  \begin{solution}
    Es bastante sencillo:
    \[(-2, 5) = -2*(1,0) + 5*(0,1)\]
    O también
    \[(-2, 5) = -2*\vec{u_1} + 5*\vec{u_2}\]
  \end{solution}
\question Dada la base $B=\{(1,2), (2,1)\}$, escribe las componentes del vector $\vec{v} = (1,1)$
  en esa base.
  \begin{solution}
    Por lo que hemos dicho las componentes son los escalares que multiplican a los vectores
    de la base en la combinación lineal, con lo que en este ejercicio tenemos que hacer lo mismo
    que en los del apartado anterior. Escribimos la ecuación vectorial:
    \[(1, 1) = a*(1, 2) + b*(2, 1)\]
    Que en forma de sistema es:
    \[
      \begin{cases}
        1 &=a + 2b\\
        1 &=2a +b
      \end{cases}
    \]
    Que vamos a resolver por sustitución. Despejamos $a$ en la primera
    \[a = 1- 2b\]
    Sustituimos en la segunda:
    \[1 = 2(1-2b) + b\]
    De donde se obtiene $b = \frac{1}{3}$.\\
    Sustituimos $a = 1- 2*\frac{1}{3} = \frac{1}{3}$, con lo que las componentes de $\vec{v}$ en
    la base del enunciado son:
    \[a= \frac{1}{3},\ b=\frac{1}{3}\]
  \end{solution}
\end{questions}
\section{Coordenadas rectangulares y polares.}
Las \textbf{coordenadas rectangulares}, o cartesianas, son las que hemos utilizado en todo lo que hemos
visto hasta ahora: \textbf{dos valores que indican el desplazamiento horizontal y vertical}, y se
escribe poniendo esos dos valores entre paréntesis primero el horizontal y luego el vertical.
\[\vec{v} = (v_x, v_y)\]

Las \textbf{coordenadas polares} también son una pareja de valores, pero en este caso
\textbf{son el módulo del vector y el ángulo que forma con la horizontal}. En este caso se escriben
poniendo el ángulo como subíndice del módulo:
\Large{\[\vec{v} = |\vec{v}|_{\,\alpha}\]}

Si representamos el vector con sus distintas coordenadas tenemos:
\begin{center}
  \begin{tikzpicture}[scale=1.3]
    \draw[-Latex] (0,0) coordinate (O) -- (3, 2) coordinate (A) node[midway, sloped, above]
    {$\vec{v}$};
    \draw[dashed, latex-latex] (3,0) -- (A) node[midway, right] {$v_y$};
    \draw[dashed, latex-latex] (0,-.2)-- (3, -.2) node[midway, below] {$v_x$};
    \draw[dashed, latex-latex] (0.2,0)-- (3, 1.9) node[midway, sloped, below] {$|\vec{v}|$};
    \draw[color=gridgray] (-1, 0) -- (4,0) coordinate (B);
    \pic["$\scriptstyle{\alpha}$", draw=black, angle eccentricity=1.4,
        angle radius=.8cm] {angle=B--O--A};
  \end{tikzpicture}
\end{center}
Es decir, el vector con sus componentes forman un triángulo rectángulo cuyo cateto vertical ($v_y$)
es el cateto opuesto al ángulo que forma el vector con la horizontal, que es el ángulo que estamos
utilizando.\\
Echando mano de lo que recordamos de trigonometría y el teorema de Pitágoras tenemos las siguientes
relaciones (algunas de las cuales ya las conocíamos):
\begin{itemize}
\item $|\vec{v}| = \sqrt{v_x^2 + v_y^2}$
\item $\tg \alpha = \frac{v_y}{v_x}$
\item $\cos \alpha = \frac{v_x}{|\vec{v}|}$
\item $\sen \alpha = \frac{v_y}{|\vec{v}|}$
\end{itemize}

Y con esto ya podemos ver cómo convertir de coordenadas rectangulares y viceversa.\\

Vamos a empezar por la conversión de polares a rectangulares que es la más sencilla.
\subsection{De polares a rectangulares}
Para ello vamos a utilizar las relaciones del seno y el coseno, que en trigonometría
utilizábamos para calcular catetos.\\
Como hemos visto que las componentes son los catetos se reduce a hacer lo mismo que hacíamos para
calcular catetos:
\begin{itemize}
\item $\cos \alpha = \frac{v_x}{|\vec{v}|} \longrightarrow \boldsymbol{v_x = |\vec{v}|*\cos \alpha}$
\item $\sen \alpha = \frac{v_y}{|\vec{v}|} \longrightarrow \boldsymbol{v_y = |\vec{v}|*\sen \alpha}$
\end{itemize}
Entonces:
\[\boldsymbol{\vec{v} = |\vec{v}|_\alpha = \left(|\vec{v}|\cos \alpha, |\vec{v}|\sen \alpha\right)}\]

\textbf{Hacemos un pequeño ejemplo:}\\
Dado el vector $\vec{v} = 3_{\,40\degree}$ escríbelo en coordenadas cartesianas.
\begin{solution}
  \begin{itemize}
  \item Calculamos la componente horizontal $v_x = 3*\cos 40\degree = 2.298$
  \item Calculamos la componente vertical $v_y =3*\sen 40\degree = 1.928$
  \end{itemize}
  Por tanto tenemos que
  \[\vec{v} = (2.298;1.928)\quad\text{\small{(Como hay decimales hemos puesto ';' para
      separar las componentes)}}\]
\end{solution}

\subsection{De rectangulares a polares.}
Para hacer la conversión de rectangulares a polares vamos a utilizar el cálculo del módulo y la
pendiente:
\begin{itemize}
\item $\boldsymbol{|\vec{v}| = \sqrt{v_x^2 + v_y^2}}$
\item $\tg \alpha = \frac{v_y}{v_x} \longrightarrow \boldsymbol{\alpha = \atg \frac{v_y}{v_x}}$
\end{itemize}

La complicación de esta conversión está en que para calcular el ángulo vamos a tener que utilizar
el arco de tangente, y en trigonometría hemos visto que la calculadora no entiende de contexto y
para el arco de tangente solo nos da resultados de $-90\degree$ a $90\degree$, es decir del primer
y del cuarto cuadrante. Pero un vector puede estar en cualquier cuadrante, con lo que tendremos que
tener en cuenta la situación para corregir el resultado de la calculadora según el criterio
que vimos:
\begin{itemize}
\item Si el contexto nos dice que está en el primer o cuarto cuadrante no hay que corregir
  el resultado de la calculadora.
\item Si el contexto nos dice que está en el segundo o tercer cuadrante hay que sumar
  $180\degree$ al resultado obtenido.
\end{itemize}

Como está conversión es más complicada vamos a hacer unos cuantos ejemplos.
\begin{questions}
\question Convierte el vector $\vec{u} = (4,-2)$ a coordenadas polares.
  \begin{solution}
    El cálculo del módulo es sencillo:
    \[|\vec{v}| = \sqrt{4^2 + (-2)^2} = \sqrt{20} = 2\sqrt{5}\quad\text{\small{(recuerda como
          se sacan factores de una raíz)}}\]
    Calculamos ahora el ángulo, para ello utilizamos la calculadora:
    \[\alpha = \atg \frac{-2}{4} = -26.57\degree\]
    Este \textbf{ángulo es del cuarto cuadrante}, \emph{¿se corresponde con el ángulo que forma
      el vector con el eje horizontal positivo?}\\
    Representamos la situación:
    \begin{center}
      \begin{tikzpicture}
	\begin{axis}[xmin=-1, xmax=5, ymin=-3, ymax=3, xticklabel=\empty, yticklabel=\empty, axis x line=center, axis y line=center, ytick={-3,-2,-1,0,1,2,3}, grid=both]
          \draw[ultra thick, -Latex] (0,0) -- (4,-2) node[midway, sloped, below] {$\vec{u}$};
	\end{axis}
      \end{tikzpicture}
    \end{center}
    Vemos que si lo dibujamos empezando desde el origen \textbf{el vector se encuentra en el
      cuarto cuadrante},con lo que \textbf{no necesitamos corregir nada} y por tanto el vector
    en coordenadas polares es:
    \[\vec{u} =2\sqrt{5}_{\,-26.57\degree}\]
    Ó también se podría escribir:
    \[\vec{u} =2\sqrt{5}_{\,333.43\degree}\]
  \end{solution}
\question Escribe el vector $\vec{a} = (-2, -1)$ en coordenadas polares.
  \begin{solution}
    Hacemos lo mismo que en el anterior, primero calculamos el módulo:
    \[|\vec{a}| = \sqrt{2^2 + 1^2} = \sqrt{5}\]

    Y el ángulo:
    \[\alpha = \atg \frac{-1}{-2} = 26.565\degree\]
    Que es un ángulo del primer cuadrante, mientras que nuestro vector:
    \begin{center}
      \begin{tikzpicture}
	\begin{axis}[xmin=-3, xmax=3, ymin=-3, ymax=3, xticklabel=\empty, yticklabel=\empty, axis x line=center, axis y line=center, ytick={-3,-2,-1,0,1,2,3}, grid=both]
          \draw[ultra thick, -Latex] (0,0) -- (-2,-1) node[midway, sloped, below] {$\vec{a}$};
	\end{axis}
      \end{tikzpicture}
    \end{center}
    está en el tercer cuadrante con lo que tenemos que corregir, $\alpha = 26.656\degree +
    180\degree = 206.565\degree$.\\

    Con esto el vector en coordenadas polares queda:
    \[\vec{a} = \sqrt{5}_{\,206.565\degree}\]
  \end{solution}
\end{questions}
A pesar de que en los dos ejemplos anteriores hemos representado el vector para ver en qué
cuadrantes está, no es necesario hacerlo para pasar de rectangulares a polares. Solo debemos
recordar los signos del coseno y el seno en los distintos cuadrantes y aplicarlos a las componentes
del vector, obteniendo así el siguiente criterio:
\begin{itemize}
\item Horizontal y vertical positivas $(+,+)$: primer cuadrante.
\item Horizontal negativa y vertical positiva $(-,+)$: segundo cuadrante.
\item Horizontal y vertical negativas $(-,-)$: tercer cuadrante.
\item Horizontal positiva y vertical negativa $(+,-)$: cuarto cuadrante.
\end{itemize}

\section{Ejercicios básicos.}
A continuación se presentan una serie de ejemplos básicos para comprender como utilizar las
operaciones y las propiedades de los vectores para resolver una serie de problemas básicos.

\subsection{Aplicaciones de las propiedades de los vectores.}
En el apartado \ref{propiedades_vectores} (página \pageref{propiedades_vectores}) hemos visto que
las propiedades de los vectores eran tres:
\begin{itemize}
\item Módulo.
\item Dirección.
\item Sentido.
\end{itemize}
Vamos a utilizar esas propiedades para resolver algunos problemas geométricos.\\

\begin{questions}
\question Calcula la distancia entre los puntos $A(-1, 3)$, $B(2, 4)$.
  \begin{solution}
    Vamos a representar primero la situación aproximada para ver qué es lo que tenemos:
    \begin{center}
      \begin{tikzpicture}
        \draw[-Latex] (0,0) coordinate (A) -- (3,2) coordinate (B) node[midway, sloped, above]
        {$\overrightarrow{AB}$};
        \draw[fill] (A) circle(2pt) node[left] {$A$};
        \draw[fill] (B) circle(2pt) node[right] {$B$};
      \end{tikzpicture}
    \end{center}
    Al representar la situación vemos que \textbf{la distancia entre los puntos} $A$ y $B$ es el
    tamaño del vector $\overrightarrow{AB}$.\\
    Y ¿qué propiedad de los vectores nos dice cual es su tamaño? \textbf{El módulo del vector}.\\

    Es decir, \textbf{el ejercicio se reduce a calcular el módulo del vector $\boldsymbol{
        \overrightarrow{AB}}$.}.\\

    Pues lo hacemos:
    \[\overrightarrow{AB} = (2, 4) - (-1, 3) = (3, 1)\]
    \[\distancia{A}{B} = |\overrightarrow{AB}| = \sqrt{3^2 + 1^2}\]
    \[\distancia{A}{B} = \sqrt{10}\]
    \begin{center}
      \small{($\distancia{A}{B}$ es la distancia entre $A$ y $B$. Lo usaremos a menudo)}
    \end{center}

    Con lo cual la distancia entre los puntos $A$ y $B$ es de $\sqrt{10}$ unidades.
  \end{solution}
\question Indica si los vectores $\vec{u} = (3, -1)$ y $\vec{v} = (2, -2)$ son paralelos.
  \begin{solution}
    Por que lo hemos visto en el apartado \ref{propiedades_vectores}
    (página \pageref{propiedades_vectores}) para que dos vectores sean paralelos tienen que tener
    la misma dirección, formar el mismo ángulo con la horizontal, tener la misma pendiente:
    \begin{center}
      \begin{tikzpicture}
        \coordinate (A) at (5, 0);
        \draw[-Latex] (0,0) coordinate (O1) -- (3,2) coordinate (B1) node[midway, sloped, below]
        {$\vec{u}$};
        \draw[-Latex] (-1, 0) coordinate (O2) -- (3.5,3) coordinate (B2)
        node[midway, sloped, above] {$\vec{v}$};
        \pic["$\scriptstyle{\alpha}$", draw=black, angle eccentricity=1.3, angle radius=.6cm]
        {angle=A--O1--B1};
        \pic["$\scriptstyle{\alpha}$", draw=black, angle eccentricity=1.3, angle radius=.6cm]
        {angle=A--O2--B2};
        \draw[color=darkgray, dashed] (-2, 0)--(A);
      \end{tikzpicture}
    \end{center}

    
    Y la pendiente es lo que vamos a utilizar, teniendo en cuenta que  es la tangente del
    ángulo que forma el vector con la horizontal, para tener la misma dirección esa tangente debe
    ser la misma.\\
    Recordemos de trigonometría que \emph{la tangente de un ángulo es el cociente del cateto opuesto
      entre el cateto adyacente}, y en el caso del ángulo que forma el vector $\vec{v}$
    con la horizontal su tangente es:
    \[\tg \alpha = \frac{v_y}{v_x}\]

    Por tanto si $\vec{u} = (u_x, u_y)$ y $\vec{v}= (v_x, v_y)$ son paralelos (se puede
    escribir $\vec{u} \parallel \vec{v}$. El símbolo $\boldsymbol{\parallel}$ significa paralelos) entonces sus pendientes:
    \[\frac{u_y}{u_x} = \frac{v_y}{v_x}\quad\text{\small{(las pendientes tienen que ser iguales)}}\]
    Si despejamos para tener una componente a cada lado en vez de un vector nos queda:
    \[\frac{v_x}{u_x} = \frac{v_y}{u_y}\]
    Es decir, los vectores tienen que ser proporcionales, las componentes de uno de ellos tienen
    que ser las del otro multiplicadas por el mismo número.\\
    
    Hacemos lo que acabamos de ver con los datos del enunciado, el cociente de las componentes
    de $\vec{u} = (3, -1)$ y $\vec{v} = (2, -2)$:
    \begin{itemize}
    \item $\frac{u_x}{v_x} = \frac{3}{2}$
    \item $\frac{u_y}{v_y} = \frac{-1}{-2}$
    \end{itemize}
    No obtenemos el mismo resultado, \textbf{no son proporcionales}, con lo que
    $\vec{u}$ y $\vec{v}$ \textbf{no son paralelos}.
  \end{solution}
\question Calcula cuanto tiene que valer $\boldsymbol{k}$ para que el vector $\vec{u} = (-2, 3+2k)$
  sea paralelo a $\vec{v} = (6, -4)$.
  \begin{solution}
    Por todo lo anterior, para que sean paralelos tiene que ocurrir que
    \[\frac{-2}{6} = \frac{3+2k}{-4}\]
    Así que resolvemos la ecuación.\\
    Primero simplificamos:
    \[-\frac{1}{3} = -\frac{3+2k}{4}\]
    Simplificamos un poco más y hacemos denominador común:
    \[\frac{4}{12} = \frac{9+6k}{12}\]
    Con lo que:
    \[4 = 9 + 6k\]
    \[k = -\frac{5}{6}\]

    \textbf{Para que sean paralelos tiene que ocurrir que $\boldsymbol{k = -\frac{5}{6}}$}.
  \end{solution}
\question Discute si los puntos $A(1,2)$, $B(3, -1)$ y $C(7, -7)$ están alineados.
  \begin{solution}
    Para ver como hacer este ejercicio vamos a ver gráficamente cual es la situación en caso
    de que los puntos no estén alineados y en caso de que lo estén:
    \newpage %He tenido que hacer una trampa porque no lo hace solo.
    \begin{center}
      \begin{tikzpicture}[baseline=(current bounding box.center), scale=.9]
        \begin{axis}[axis equal, xmin=0, xmax=5, ymin=-1, ymax=4, axis lines = middle,
          axis line style={draw=gridgray}, xticklabel=\empty, yticklabel=\empty,
          tick label style={major tick length=0pt}, grid=major, ytick={-1, 0, 1, 2, 3, 4},
          title={\textbf{No alineados}}]
            \addplot[mark=*, only marks] coordinates {(0, 1) (1, 3) (3, -1)};
            \node[left] at (0,1) {$A$};
            \node[above] at (1,3) {$B$};
            \node[above right] at (3,-1) {$C$};
            \draw[-Latex] (0,1)--(1,3) node[midway, sloped, above] {$\overrightarrow{AB}$};
            \draw[-Latex] (1,3)--(3,-1) node[midway, sloped, above] {$\overrightarrow{BC}$};
            \draw[-Latex] (0,1)--(3,-1) node[midway, sloped, below] {$\overrightarrow{AC}$};
        \end{axis}
      \end{tikzpicture}
      \quad
      \begin{tikzpicture}[baseline=(current bounding box.center), scale=.9]
        \begin{axis}[axis equal, xmin=0, xmax=5, ymin=-1, ymax=4, axis lines = middle,
          axis line style={draw=gridgray}, xticklabel=\empty, yticklabel=\empty,
          tick label style={major tick length=0pt}, grid=major, ytick={-1, 0, 1, 2, 3, 4},
          title={\textbf{Alineados}}]
            \addplot[mark=*, only marks] coordinates {(0, 0) (2, 1) (4, 2)};
            \node[left] at (0,0) {$A$};
            \node[above] at (2,1) {$B$};
            \node[above right] at (4,2) {$C$};
            \draw[-Latex] (0,0)--(2,1) node[midway, sloped, above] {$\overrightarrow{AB}$};
            \draw[-Latex] (2,1)--(4,2) node[midway, sloped, below] {$\overrightarrow{BC}$};
        \end{axis}
      \end{tikzpicture}
    \end{center}

    En el caso de \textbf{no alineados} vemos que los vectores $\overrightarrow{AB}$,
    $\overrightarrow{AC}$ y $\overrightarrow{BC}$ \textbf{hay ninguna relación de paralelismo entre
    ellos}.\\
    En el caso de \textbf{alineados} tenemos que tanto $\overrightarrow{AB}$, como
    $\overrightarrow{AC}$ y $\overrightarrow{BC}$ \textbf{son paralelos} (en realidad están
    superpuestos, pero a la hora de hacer cálculos es lo mismo que si fuesen paralelos).\\

    A la vista de esto lo que tenemos que hacer es calcular dos vectores con los tres puntos y ver
    si son paralelos.\\
    Vamos a calcular $\overrightarrow{AB}$ y $\overrightarrow{BC}$ (también podríamos hacerlo
    con $\overrightarrow{AB}$ y $\overrightarrow{AC}$):
    \begin{itemize}
    \item $\overrightarrow{AB} = (3,-1) - (1,2) = (2,-3)$
    \item $\overrightarrow{BC} = (7,-7) - (3,-1) = (4,-6)$
    \end{itemize}
    Si hacemos el cociente de las componentes vemos que dan el mismo resultado:
    \[\frac{4}{2} = \frac{-6}{-3}\]
    Por tanto $\overrightarrow{AB} \parallel \overrightarrow{BC}$, y por lo que hemos visto esto
    quiere decir que $A$, $B$ y $C$ están alineados.
  \end{solution}
\question Discute si los puntos $P(-2, 3)$, $Q(5, 2)$ y $R(-1, 1)$ pueden ser los vértices de un
  triángulo.
  \begin{solution}
    Si nos fijamos en la representación del ejemplo anterior la única manera en la que pueden ser
    los vértices de un triángulo es no estando alineados, con lo que lo que tenemos que hacer es
    lo mismo que en el ejercicio anterior.\\

    Calculamos dos vectores:
    \begin{itemize}
    \item $\overrightarrow{PQ} = (7, -1)$
    \item $\overrightarrow{PR} = (1, -2)$
    \end{itemize}
    Comprobamos su paralelismo:
    \[\frac{7}{1} \neq \frac{-1}{-2}\]
    Y resulta que \textbf{no son paralelos}, con lo que \textbf{no están alineados} y por lo tanto
    \textbf{pueden ser los vértices de un triángulo}.
  \end{solution}
\question Calcula un vector perpendicular al vector $\vec{v} = (3, -2)$ (\textbf{\emph{Este
      ejercicio es muy importante. Hay que aprenderse bien el razonamiento y el mecanismo}}).
  \begin{solution}
    \emph{¿Qué tiene que ocurrir para que dos vectores sean perpendiculares?} Que formen un ángulo
    recto.\\
    Como en casi todos los ejemplos anteriores, vamos a empezar por representar la situación:
    \begin{center}
      \begin{tikzpicture}
        \coordinate (A) at (5, 0);
        \draw[-Latex] (0,0) coordinate (O) -- (2,4) coordinate (B) node[midway, sloped, below]
        {$\vec{v}$};
        \draw[-Latex] (O) -- (-4,2) coordinate (C) node[midway, sloped, below]
        {$\vec{w}$};
        \pic["$\scriptstyle{\alpha}$", draw=black, angle eccentricity=1.3, angle radius=1cm]
        {angle=A--O--B};
        \pic["$\scriptstyle{90\degree + \alpha}$"{xshift=-5mm}, draw=black, angle eccentricity=1.4,
        angle radius=.8cm] {angle=A--O--C};
        \pic[draw=black, angle eccentricity=.5, angle radius=.4cm, pic text=.] {right angle=B--O--C};
        \draw[color=darkgray, dashed] (-5, 0)--(A);
      \end{tikzpicture}
    \end{center}
    Se ve claramente que si $\vec{v} \perp \vec{w}$ (el símbolo $\boldsymbol{\perp}$ significa
    que son perpendiculares) el ángulo que forma uno de ellos con la horizontal es $90\degree$
    más el ángulo que forma el otro.\\
    Calculamos las pendientes de esos ángulos:
    \begin{itemize}
    \item $\tg \alpha = \frac{v_y}{v_x}$
    \item $\tg (90\degree + \alpha) = \frac{w_y}{w_x}$
    \end{itemize}
    Por lo que sabemos de trigonometría (relaciones entre $\alpha$ y $90\degree + \alpha$)
    tenemos que:
    \[\tg (90\degree + \alpha) = -\frac{1}{\tg \alpha}\]
    \[\frac{w_y}{w_x} = -\ddfrac{1}{\ddfrac{v_y}{v_x}}\]
    \[\frac{w_y}{w_x} = -\frac{v_x}{v_y}\]
    Y para que se mantenga esa relación basta con que hagamos $w_y = -v_x$ y $w_x = v_y$ (o también
    $w_y = v_x$ y $w_x = -v_y$, da igual cual cambie de signo).\\

    Entonces tenemos que un vector perpendicular a $\vec{v} = (3, -2)$ puede ser
    \begin{itemize}
    \item $\vec{w} = (2, 3)$\\
      ó
    \item $\vec{w} = (-2, -3)$
    \end{itemize}
    Ó cualquiera proporcional a ellos.\\

    Y por dar una respuesta concreta elegimos el primero, $\vec{w} = (2, 3)$.
  \end{solution}
\end{questions}
\subsection{Aplicaciones del producto por un escalar.}
En el apartado \ref{producto_por_escalar} hemos visto que el producto por un escalar nos da un
vector paralelo que multiplica su tamaño por el número (también hemos visto que algo parecido
sucede con la división).

Y lo que vamos a hacer en este apartado es aprovechar esto junto con las propiedades de los vectores para resolver algunos ejercicios básicos.\\


\begin{questions}

\question Escribe un vector unitario con la dirección y el sentido de $\vec{v} = (1,-2)$.
  \begin{solution}
    Hemos visto que si dividimos $\vec{v}$ entre un número $a>0$ vamos a obtener un vector
    con la misma dirección y el mismo sentido que $\vec{v}$ y con un módulo que será
    \[|\vec{u}| = \left|\frac{\vec{v}}{a}\right| = \frac{|\vec{v}|}{a}\]
    \begin{center}
      \small{(No hemos puesto el valor absoluto a $a$ porque hemos dicho que $a>0$)}
    \end{center}
    Si queremos que $|\vec{u}| = 1$ para que sea unitario entonces:
    \[\frac{|\vec{v}|}{a} = 1\]
    Con lo que
    \[a = |\vec{v}|\]
    Es decir, tenemos que dividir el vector entre su módulo.\\

    Calculamos el módulo del vector del enunciado:
    \[|\vec{v}| = \sqrt{1^2 + (-2)^2} =\sqrt{5}\]
    Y por lo que hemos visto
    \[\vec{u} = \frac{\vec{v}}{\sqrt{5}}\]
    Va a ser el vector unitario que nos piden.\\

    Operamos y simplificamos:
    \[\vec{u} = \left(\frac{3}{\sqrt{5}},-\frac{2}{\sqrt{5}}\right)\]
    \[\vec{u} = \left(\frac{3\sqrt{5}}{5},-\frac{2\sqrt{5}}{5}\right)\]
  \end{solution}
\question Calcula el punto medio del segmento que tiene de extremos $A(2, -1)$ y $B(1, 1)$.
  \begin{solution}
    Vamos a representar la situación de manera aproximada llamando $M$ al punto medio:
    \begin{center}
      \begin{tikzpicture}
        \coordinate (A) at (0,0);
        \coordinate (M) at (3,2);
        \coordinate (B) at (6,4);
        \draw[fill] (A) circle(2pt) node [left] {$A$};
        \draw[fill] (B) circle(2pt) node[right] {$B$};
        \draw[fill] (M) circle(2pt) node[below right] {$M$};
        \draw[-Latex] (A)--(B);
        \draw[-Latex] (A)--(M);
      \end{tikzpicture}
    \end{center}
    Claramente el vector $\overrightarrow{AM}$ tiene la misma dirección y sentido que el vector
    $\overrightarrow{AB}$ pero su módulo será la mitad.\\
    Es decir
    \[\overrightarrow{AM} = \frac{\overrightarrow{AB}}{2}\]

    Definamos las coordenadas de los puntos simbólicamente:
    \begin{itemize}
    \item $A(A_x,A_y)$
    \item $B(B_x,B_y)$
    \item $M(M_x,M_y)$
    \end{itemize}
    De esta manera tenemos que
    \[\overrightarrow{AM} = \frac{\overrightarrow{AB}}{2} = \left(\frac{B_x-A_x}{2},
        \frac{B_y-A_y}{2}\right)\]
    Y como $\overrightarrow{AM} = M - A$ tenemos que
    \[M = A + \overrightarrow{AM}\]
    \[M = (A_x,A_y) + \left(\frac{B_x-A_x}{2},\frac{B_y-A_y}{2}\right)\]
    \[M = \left(A_x + \frac{B_x-A_x}{2}, A_y + \frac{B_y-A_y}{2}\right)\]
    Hacemos la resta de fracciones y resulta:
    \[\boldsymbol{M = \left(\frac{B_x+A_x}{2},\frac{B_y+A_y}{2}\right)}
      \quad\text{\small{(esta fórmula hay que aprendérsela)}}\]
    Y con esto ya es fácil calcular el punto medio pedido:
    \[M = \left(\frac{2+1}{2}, \frac{1-1}{2}\right)\]
    \[M = \left(\frac{3}{2}, 0\right)\]
  \end{solution}
\question Calcula el simétrico de $A(-2, 3)$ respecto de $B(-1,1)$.
  \begin{solution}
    Antes de empezar a resolver vamos a repasar qué es eso del ``simétrico''.\\

    El simétrico es el reflejo, como en un espejo. Y el punto con respecto al cual hacemos
    el simétrico es el que actúa de espejo.\\
    
    Vamos a ver la representación que así se entiende mejor.\\
    Tenemos el punto $A$, el punto $B$ que actúa de espejo, y el punto $A'$ que es el simétrico
    que queremos calcular:
    \begin{center}
      \begin{tikzpicture}
        \draw[-Latex] (-3, -2) coordinate (A) -- (0,0) coordinate (B) node[midway, sloped,
        above] {$\overrightarrow{AB}$};
        \draw[Latex-] (3, 2) coordinate (A1) -- (0,0) coordinate (B) node[midway, sloped,
        above] {$\overrightarrow{BA'}$};
        \draw[fill] (A) circle(2pt) node[left] {$A$};
        \draw[fill] (B) circle(2pt) node[below right] {$B$};
        \draw[fill] (A1) circle(2pt) node[right] {$A'$};
      \end{tikzpicture}
    \end{center}
    Y a partir de aquí vamos a sacar conclusiones.\\

    La primera es que los vectores $\overrightarrow{AB}$ y $\overrightarrow{BA'}$ son paralelos,
    ya que en otro caso lo que tendríamos no sería el reflejo.\\
    La segunda es que para que sea el reflejo tiene que ser todo igual pero al otro lado, y ese
    ``todo igual'' incluye el tamaño.\\
    Por lo tanto, si los vectores $\overrightarrow{AB}$ y $\overrightarrow{BA'}$ tienen que tener
    la misma dirección, el mismo módulo y, a la vista de la representación, el mismo sentido;
    entonces tienen que ser iguales:
    \[\overrightarrow{AB} = \overrightarrow{BA'}\]

    Entonces podemos resolver el ejercicio de \emph{dos maneras}:
    \begin{itemize}
    \item El vector que va de $B$ al simétrico de $A$ es $\overrightarrow{AB}$ y por tanto:
      \[A' = B + \overrightarrow{AB} = (-1,1) + (1,-2) = (0, -1)\]
    \item El vector que va de $A$ a su simétrico $A'$ es el doble del vector $\overrightarrow{AB}$,
      con lo que:
      \[A' = A + 2*\overrightarrow{AB} = (-2, 3) + 2*(1, -2) = (0, -1)\]
    \end{itemize}

    \textbf{Otra interpretación}\\
    También se puede interpretar de la siguiente manera: el punto que actúa de espejo es el
    punto medio del segmento que une $A$ con su simétrico.\\
    De esta manera se entiende que el vector que va de $A$ a su simétrico tiene que ser el
    doble del que va de $A$ a $B$ por lo visto en el ejercicio anterior.
    Si lo hacemos con esta interpretación tendríamos
    \[A' = A + 2*\overrightarrow{AB}\]
    Que es una operación que habíamos obtenido también con la interpretación anterior.
  \end{solution}
\end{questions}
\subsection{Aplicaciones de la suma de vectores y la regla del paralelogramo.}
En el apartado \ref{suma_vectores} hemos visto que el significado geométrico de la suma de
vectores es el poner uno a continuación del otro.\\
También hemos visto la \emph{regla del paralelogramo} (página \pageref{regla_paralelogramo}), que
también nos va a servir para resolver situaciones.\\
Y por supuesto también hay que acordarse del significado de la suma de un punto y un vector (
página \pageref{suma_punto_vector}).\\

\begin{questions}
\question Calcula el simétrico de $A(-2, 3)$ respecto de $B(-1,1)$.
  \begin{solution}
    Este ejercicio lo hemos hecho en el apartado anterior, pero también se puede hacer utilizando
    la interpretación geométrica de la suma de vectores.\\

    Si observamos la representación anterior tenemos que para ir de $A$ a $A'$ tenemos que
    poner dos veces el vector $AB$, una a continuación de otra.\\
    De esta manera tenemos:
    \[A' = A + \overrightarrow{AB} + \overrightarrow{AB}\]
    que es equivalente a
    \[A' = A + 2*\overrightarrow{AB}\]
    que hemos utilizado en el apartado anterior.
  \end{solution}
\question Los puntos $A(0,2)$, $B(-1,1)$ y $D(2, 0)$ son los vértices de un paralelogramo. Calcula
  el vértice que falta.
  \begin{solution}
    Como siempre, lo primero es hacer una representación de lo que tenemos y en este caso tenemos
    que recordad que los vértices de un polígono se nombran en sentido contrario a las agujas del
    reloj por orden alfabético.
    \begin{center}
      \begin{tikzpicture}
        \draw[-Latex] (0,4) coordinate (A) -- (-2, 2) coordinate (B) node[midway, sloped, above]
        {$\overrightarrow{AB}$};
        \draw[-Latex] (A) -- (4, 0) coordinate (D) node[midway, sloped, above]
        {$\overrightarrow{AD}$};
        \draw[dashed] (D) -- (2, -2) coordinate (C);
        \draw[dashed] (B) -- (C);
        \draw[fill] (A) circle(2pt) node[above] {$A$};
        \draw[fill] (B) circle(2pt) node[left] {$B$};
        \draw[fill] (C) circle(2pt) node[below] {$C$};
        \draw[fill] (D) circle(2pt) node[right] {$D$};
      \end{tikzpicture}
    \end{center}
    Y se ve claramente que es la misma situación que tenemos en la \emph{regla del paralelogramo}
    (página \pageref{regla_paralelogramo}), con lo que
    \[\overrightarrow{AC} = \overrightarrow{AB} + \overrightarrow{AC}\]
    \[\overrightarrow{AC} = (-1,-1) + (2, -2) = (1, -3)\]
    
    Y por tanto el punto $C$ es:
    \[\boldsymbol{C = A + \overrightarrow{AC} = (0,2) + (1, -3) = (1,-1)}\]

    \vspace{1cm}\textbf{Otra manera de interpretarlo es la siguiente}:
    Para ir desde $A$ hasta $C$ hay que aplicar primero el vector $\overrightarrow{AD}$ y
    después el vector $\overrightarrow{DC}$.
    \[C = A + \overrightarrow{AD} + \overrightarrow{DC}\]
      
    El vector $\overrightarrow{DC}$ tiene el mismo módulo, dirección y sentido que
    $\overrightarrow{AB}$ (por la definición de paralelogramo). Sustituyendo en la expresión que
    hemos obtenido antes:
    \[C = A + \overrightarrow{AD} + \overrightarrow{AB}\]
    Que es lo mismo que hemos obtenido con la regla del paralelogramo.\\
    
    \vspace{5mm}Si hacemos el mismo razonamiento pasando por $B$ llegaremos al mismo resultado
    cambiado de orden, pero al ser la suma conmutativa es exactamente lo mismo.
  \end{solution}
\question Un triángulo tiene como lados a los vectores $\overrightarrow{AB} = (2, 1)$ y
  $\overrightarrow{AC} = (-1, 3)$. ¿Cuánto mide el lado que falta?. Si nos dicen que $A(0,3)$
  calcula los otros dos vértices.
  \begin{solution}
    Representamos la situación:
    \begin{center}
      \begin{tikzpicture}
        \draw[-Latex] (0,0) coordinate (A) -- (2, 1) coordinate (B) node[midway, sloped, below]
        {$\scriptstyle{\overrightarrow{AB}}$};
        \draw[-Latex] (A) -- (-1, 3) coordinate (C) node[midway, sloped, below]
        {$\scriptstyle{\overrightarrow{AC}}$};
        \draw[dashed] (B) -- (C);
        \draw[fill] (A) circle(2pt) node[left] {$A$};
        \draw[fill] (B) circle(2pt) node[right] {$B$};
        \draw[fill] (C) circle(2pt) node[above] {$C$};
      \end{tikzpicture}
    \end{center}

    No es difícil ver que estamos en la interpretación de la resta en la \emph{regla del
      paralelogramo} (página \pageref{regla_paralelogramo}), lo que:
    \[\overrightarrow{BC} = \overrightarrow{AC} - \overrightarrow{AB}\]
    \[\overrightarrow{BC} = (-3, 2)\]

    Y como nos piden cuánto mide tenemos que calcular el módulo.
    \[|\overrightarrow{BC}| = \sqrt{(-3)^2 + 2^2} = \sqrt{13}\]
    Con lo que el lado que falta mide $\sqrt{13}$\,unidades.

    \vspace{1cm}La segunda parte es aplicar el desplazamiento de un punto por un vector:
    \[B = A + \overrightarrow{AB} = (0,3) + (2, 1) = (2,4)\]
    \[B = A + \overrightarrow{AC} = (0,3) + (-1, 3) = (-1,6)\]
    Y con esto ya tenemos los tres puntos del triángulo:
    \begin{itemize}
    \item $A(0, 3)$
    \item $B(2, 4)$
    \item $C(-1, 6)$
    \end{itemize}
  \end{solution}
\question Dado el vector $\vec{v} = (-2, 1)$ y el punto $P(1, -1)$ calcula y representa los puntos
  que se obtienen al aplicar $\vec{v}$, $-\vec{v}$, $2*\vec{v}$ y $-2\vec{v}$ sobre $P$. ¿Qué
  se obtiene si se unen todos los puntos?
  \begin{solution}
    Este ejercicio es sencillo pero tiene su importancia ya que la idea que contiene es
    la que nos lleva a la ecuación de la recta que veremos más adelante.\\

    En primer lugar vamos a calcular los puntos que pide y a darles un nombre:
    \begin{itemize}
    \item $P_1 = P + \vec{v} = (-1, 0)$.
    \item $P_2 = P + (-\vec{v}) = P - \vec{v} = (3, -2)$.
    \item $P_3 = P + 2\vec{v} = (-3,1)$.
    \item $P_4 = P - 2\vec{v} = (5, -3)$.
    \end{itemize}

    Y una vez calculados podemos representarlos:
    \begin{center}
      \begin{tikzpicture}
        \begin{axis}[axis equal, xmin=-4, xmax=5, ymin=-4, ymax=5, axis lines = middle,
          grid=major, xtick distance=1, ytick distance=1,
          tick label style={font=\scriptsize}]
          \draw[dashed, color=gray] (-4.5, 1.75) -- (5.5, -3.25);
          \draw[fill] (1, -1) circle(2pt) node[below] {\small{$P$}};
          \draw[fill] (-1, 0) circle(2pt) node[above] {\small{$P_1$}};
          \draw[fill] (3,-2) circle(2pt) node[below] {\small{$P_2$}};
          \draw[fill] (-3, 1) circle(2pt) node[above] {\small{$P_3$}};
          \draw[fill] (5, -3) circle(2pt) node[below] {\small{$P_4$}};
        \end{axis}
      \end{tikzpicture}
    \end{center}
    Es fácil ver que si unimos todos los puntos vamos a obtener una línea recta.\\
    \vspace{7mm}Es decir, todos los puntos que obtengamos al hacer la operación 
    \[P + a*\vec{v}\]
    con $a \in \mathbb{R}$ van a estar en la misma línea recta.\vspace{5mm}\\
    Con esto surgen varias preguntas:
    ¿cualquier punto que esté en esa línea recta va se puede obtener con esa
    operación? ¿Y si el punto no está en la recta se puede obtener con esa operación?\\
    Vamos a contestar a esto con el próximo ejercicio.
  \end{solution}
\question Dados el punto y el vector del ejercicio anterior calcula $a$ para que la operación $P+a*
  \vec{v}$ dé como resultado:
  \begin{parts}
  \part El punto $Q\left(2, -\frac{3}{2}\right)$ que está alineado con los obtenidos en el ejercicio
    anterior.
  \part El punto $R(-1, -1)$ que no está alineado con los del ejercicio anterior.
  \end{parts}
  \begin{solution}
    \begin{parts}
    \part Nos pide que calculemos $a$ para que
      \[Q = P + a*\vec{v}\]
      Entonces sustituimos:
      \[\left(2, -\frac{3}{2}\right) = (1, -1) + a*(-2,1)\]
      Y separando por componentes obtenemos el siguiente sistema:
      \[
        \begin{cases}
          2 &= 1 - 2a\\
          -\frac{3}{2} &= -1 + a\\
        \end{cases}
      \]
      y es un sistema de ecuaciones con una incógnita que a lo mejor no hemos visto cómo resolverlo.\\
      La idea es la misma que en todos los sistemas, todas las ecuaciones tienen que cumplirse
      con los mismos valores de las incógnitas. Es decir:
      \begin{itemize}
      \item Las dos ecuaciones tienen la misma
        solución, esa es la solución.
      \item Cada ecuación tiene una solución, entonces no hay solución.
      \end{itemize}
      
      
      Pues resolvemos el sistema y obtenemos que para ambas ecuaciones la solución es $a = -\frac{1}{2}$.\\
      El sistema tiene solución y es $a = -\frac{1}{2}$.
    \part Tenemos que hacer lo mismo que en el apartado anterior.\\
      \[R = P + a\vec{v}\]
      \[(-1, -1) = (1, -1) + a*(-2, 1)\]
      Lo convertimos en sistema:
      \[
        \begin{cases}
          -1 &= 1 - 2a\\
          -1 &= -1 + a
        \end{cases}
      \]
      Y al resolverlo obtenemos que:
      \begin{itemize}
      \item La solución de la primera es $a = 1$.
      \item La solución de la segunda es $a = 0$.
      \end{itemize}
      Como las soluciones no coinciden este apartado no tiene solución.\\
      
      Es decir, con esa operación no podemos construir puntos que no estén alineados.\\
      Más adelante
      utilizaremos algo parecido a esto para saber si un punto está contenido en una recta o no.
    \end{parts}
  \end{solution}
\end{questions}
\end{document}


% LocalWords:  escríbelo º geométricamente cuadritos cuadradito v x R
% LocalWords:  w b n j AB k C AC BC P Q PQ PR M AM BA DC
% LocalWords:  overrightarrow
