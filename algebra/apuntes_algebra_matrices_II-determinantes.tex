\documentclass[a4paper,11pt,answers]{exam}
\usepackage{graphicx}
\usepackage{pstricks}
\usepackage[utf8]{inputenc}
\usepackage[spanish]{babel}
\usepackage[T1]{fontenc}
%textcomp es para el símbolo del euro
\usepackage{lmodern, textcomp}

\usepackage[left=1in, right=1in, top=1in, bottom=1in]{geometry}
%\usepackage{mathexam}
\usepackage{amsmath}
\usepackage{amssymb}
\usepackage{multicol}
\usepackage{longtable}
%para la última página
%\usepackage{lastpage}

%Para padding en celdas
\usepackage{cellspace}
\setlength\cellspacetoplimit{1mm}
\setlength\cellspacebottomlimit{1mm}

%Para hacer tachados
\usepackage[makeroom]{cancel}

%Creative commons
%\usepackage{ccicons}
\usepackage[type={CC}, modifier={by-nc-sa}, version={4.0}, %imagemodifier={-eu-80x25},
lang={spanish}]{doclicense}

%Para las gráficas:
\usepackage{tikz}
\usepackage{pgfplots}
\pgfplotsset{compat = newest}
\pgfplotsset{compat=1.12}
\usetikzlibrary{babel} %Si no da errores con algunas cosas al compilar los gráficos.
\usetikzlibrary{arrows,shapes,positioning}
\usetikzlibrary{matrix}
\usepgfplotslibrary{fillbetween}
\usetikzlibrary{arrows.meta}
\usetikzlibrary{fit}
\usepackage{nicematrix}

\usepackage{color,colortbl}
\definecolor{Gray}{gray}{0.9}
\newcolumntype{g}{>{\columncolor{Gray}}c}
%\pagestyle{headandfoot}
\pagestyle{headandfoot}
\newcommand\ExamNameLine{
\par
\vspace{\baselineskip}
Nombre:\hrulefill\relax
\par}

\renewcommand{\solutiontitle}{\noindent\textbf{Solución:}\par\noindent}

\everymath{\displaystyle}
\newcommand\ddfrac[2]{\frac{\displaystyle #1}{\displaystyle #2}}

\def \autor{Paco Andrés}
\def \titulo{Apuntes de álgebra de matrices II. Determinantes y aplicaciones.}
\def \titulofichas {\textbf {\titulo}}
\def \cursofichas {}
\def \fechaexamen {}
%\firstpageheader{\cursofichas}{\titulofichas}{\fechaexamen}
\header{\cursofichas}{\begin{small}
\titulofichas
\end{small}}{\fechaexamen}
%\header{\cursofichas}{\titulofichas}{\fechaexamen}
%\firtspagefooter{}{\thepage}{}
%Por alguna razón no sale lo del cc en el pie
\firstpagefootrule
\footrule
\footer{\autor}{\thepage}{\doclicenseIcon}
\pointpoints{punto}{puntos}

\shadedsolutions
%\definecolor{SolutionColor}{rgb}{0.99,0.99,.99}
\renewcommand{\baselinestretch}{1.3}

%Use * instead of \cdot
\mathcode`\*="8000
{\catcode`\*\active\gdef*{\cdot}} 
\newcommand{\Card}{\,\mathrm{Card}}

%For e number
\newcommand{\e}{\,\mathrm{e}}
\newcommand{\asen}{\,\mathrm{asen}\,}
\newcommand{\acos}{\,\mathrm{acos}\,}
\newcommand{\atg}{\,\mathrm{atg}\,}

%Para el diferencial y la integral:
\newcommand\dif[1]{\mathrm{d}#1}
\newcommand\integral[2]{\int #1\,\dif{#2}}
\newcommand\integrald[4]{\int_{#3}^{#4} #1\,\dif{#2}}
\newcommand\adjunto[1]{#1^\text{*}}

%Para escribir explicaciones encima del igual:
%\newcommand\igexpl[1]{{\mathrel{\overset{\makebox{\mbox{\normalfont\tiny\sffamily $#1$}}}{=}}}}
%Parece que mejor con stackrel
\begin{document}


%\author{Paco Andrés}
\title{\titulo}
\date{}
\author{\autor}
\maketitle

\begin{center}
\doclicenseLongText\\
\vspace{.25cm}
\doclicenseImage
\end{center}
\section{Determinante de una matriz cuadrada.}
El determinante es una operación que se realiza sobre una matriz cuadrada y que nos da como resultado un número real. No existe el determinante de
matrices no cuadradas.\\

La definición formal de determinante no es sencilla y en algunos aspectos supera el nivel de bachillerato, pero las operaciones que hay que realizar son
sencillas y lo que vamos a ver es como se calcula con matrices $2\times 2$, $3 \times 3$ y sus propiedades para aplicarlos a matrices de dimensiones superiores.\\

Antes de empezar el cálculo recordemos dos definiciones que en esta parte van a salir mucho:
\begin{itemize}
	\item \textbf{Diagonal principal}: la que va desde el extremo superior izquierdo al extremo inferior derecho (\textbackslash).
	\item \textbf{Diagonal secundaria}: la que va desde el extremo inferior izquierdo al superior derecho (/).
\end{itemize}


Ahora ya podemos empezar con el \textbf{determinante de una matriz $2 \times 2$}. El determinante se indica con dos barras verticales, y si la matriz $A$ es:
\[A= \left(\begin{array}{rr}
	3&5\\
	2&-1
\end{array}\right)\]
El determinante de $A$ es el producto de la diagonal principal menos el de la secundaria:
\[|A| = \left| \begin{array}{rr}
	3&5\\
	2&-1
\end{array}\right| = 3*(-1) - 5*2 = -13\]

El cálculo para calcular el \textbf{determinante de una matriz $3 \times 3$} es algo más complejo. El mecanismo que se utiliza se llama \emph{regla de Sarrus} y consiste en lo siguiente:\\
Si tenemos una matriz
\[A = \left(\begin{array}{rrr}
	a_{1\,1}&a_{1\,2}&a_{1\,3}\\
	a_{2\,1}&a_{2\,2}&a_{2\,3}\\
	a_{3\,1}&a_{3\,2}&a_{3\,3}
\end{array}\right)\]
\begin{enumerate}
	\item Copiamos las dos primeras filas debajo, de manera que tengamos 5 filas:
	\[\left(\begin{array}{rrr}
		a_{1\,1}&a_{1\,2}&a_{1\,3}\\
		a_{2\,1}&a_{2\,2}&a_{2\,3}\\
		a_{3\,1}&a_{3\,2}&a_{3\,3}\\
		a_{1\,1}&a_{1\,2}&a_{1\,3}\\
		a_{2\,1}&a_{2\,2}&a_{2\,3}
	\end{array}\right)\]
	\item El producto de las paralelas a la diagonal principal que tienen tres elementos suma.
	\[\begin{pNiceMatrix}
		a_{1\,1}&a_{1\,2}&a_{1\,3}\\
		a_{2\,1}&a_{2\,2}&a_{2\,3}\\
		a_{3\,1}&a_{3\,2}&a_{3\,3}\\
		a_{1\,1}&a_{1\,2}&a_{1\,3}\\
		a_{2\,1}&a_{2\,2}&a_{2\,3}
		\CodeAfter
		\line{1-1}{2-2}
		\line{2-2}{3-3}
		\line{2-1}{3-2}
		\line{3-2}{4-3}
		\line{3-1}{4-2}
		\line{4-2}{5-3}
	\end{pNiceMatrix}\longrightarrow a_{1\,1}*a_{2\,2}*a_{3\,3}+a_{2\,1}*a_{3\,2}*a_{1\,3} + a_{3\,1}*a_{1\,2}*a_{2\,3}\]
	\item Y el producto de las paralelas a la diagonal secundaria que tienen tres elementos resta.
		\[\begin{pNiceMatrix}
		a_{1\,1}&a_{1\,2}&a_{1\,3}\\
		a_{2\,1}&a_{2\,2}&a_{2\,3}\\
		a_{3\,1}&a_{3\,2}&a_{3\,3}\\
		a_{1\,1}&a_{1\,2}&a_{1\,3}\\
		a_{2\,1}&a_{2\,2}&a_{2\,3}
		\CodeAfter
		\line{3-1}{2-2}
		\line{2-2}{1-3}
		\line{4-1}{3-2}
		\line{3-2}{2-3}
		\line{5-1}{4-2}
		\line{4-2}{3-3}
	\end{pNiceMatrix}\longrightarrow -a_{3\,1}*a_{2\,2}*a_{1\,3}-a_{1\,1}*a_{3\,2}*a_{2\,3} - a_{2\,1}*a_{1\,2}*a_{3\,3}\]
	\item Y juntando las dos partes tenemos el determinante:
	\begin{flalign*}|A| = \left|\begin{array}{rrr}
		a_{1\,1}&a_{1\,2}&a_{1\,3}\\
		a_{2\,1}&a_{2\,2}&a_{2\,3}\\
		a_{3\,1}&a_{3\,2}&a_{3\,3}\\
	\end{array}\right|=& a_{1\,1}*a_{2\,2}*a_{3\,3}+a_{2\,1}*a_{3\,2}*a_{1\,3} + a_{3\,1}*a_{1\,2}*a_{2\,3}-a_{3\,1}*a_{2\,2}*a_{1\,3}\\
	&-a_{1\,1}*a_{3\,2}*a_{2\,3} - a_{2\,1}*a_{1\,2}*a_{3\,3}
	\end{flalign*}
\end{enumerate}
Lógicamente, los pasos 1, 2 y 3 se hacen en sucio y únicamente se pasa a limpio el paso 4. Además, con la práctica, terminaremos aprendiéndonos
de memoria qué productos suman y qué productos restan de manera que escribiremos directamente la operación del paso 4 con cada término calculado
mentalmente.\\

Vamos a verlo con un \textbf{ejemplo}: calcular el determinante de $A=\left(\begin{array}{rrr}
	3&-2& 0\\
	-2&-4&2\\
	5&1&-1
\end{array} \right)$
\begin{solution}
	Vamos a hacerlo por pasos como en la explicación.
	\begin{enumerate}
		\item Copiamos las dos primeras filas debajo, de manera que tengamos 5 filas:
		\[\left(\begin{array}{rrr}
			3&-2& 0\\
			-2&-4&2\\
			5&1&-1\\
			3&-2& 0\\
			-2&-4&2
		\end{array}\right)\]
		\item El producto de las paralelas a la diagonal principal que tienen tres elementos suma.
		\[\begin{pNiceMatrix}
			3&-2& 0\\
			-2&-4&2\\
			5&1&-1\\
			3&-2& 0\\
			-2&-4&2
			\CodeAfter
			\line{1-1}{2-2}
			\line{2-2}{3-3}
			\line{2-1}{3-2}
			\line{3-2}{4-3}
			\line{3-1}{4-2}
			\line{4-2}{5-3}
		\end{pNiceMatrix}\longrightarrow 3*(-4)*(-1) + (-2)*1*0 + 5*(-2)*2\]
		\item Y el producto de las paralelas a la diagonal secundaria que tienen tres elementos resta.
		\[\begin{pNiceMatrix}
			3&-2& 0\\
			-2&-4&2\\
			5&1&-1\\
			3&-2& 0\\
			-2&-4&2
			\CodeAfter
			\line{3-1}{2-2}
			\line{2-2}{1-3}
			\line{4-1}{3-2}
			\line{3-2}{2-3}
			\line{5-1}{4-2}
			\line{4-2}{3-3}
		\end{pNiceMatrix}\longrightarrow -5*(-4)*0 - 3*1*2 - (-2)*(-2)*(-1)\]
		\item Y juntando las dos partes tenemos el determinante:
		\begin{flalign*}|A| = \left|\begin{array}{rrr}
				3&-2& 0\\
				-2&-4&2\\
				5&1&-1
			\end{array}\right|=&3*(-4)*(-1) + (-2)*1*0 + 5*(-2)*2\\
			&-5*(-4)*0 - 3*1*2 - (-2)*(-2)*(-1) = -10
		\end{flalign*}
	\end{enumerate}
\end{solution}

Para calcular el \textbf{determinante de una matriz $4\times4$} no existe una regla sencilla (aunque parezca que Sarrus es complicado), de manera que lo
calcularemos utilizando las propiedades de los determinantes que veremos más adelante.
\subsection{Menores y adjuntos.}
Dada una matriz cuadrada $A$ se define el \textbf{menor} $\boldsymbol{M_{i\,j}}$ como el \textbf{determinante de la matriz que se obtiene al suprimir la fila $i$ y la columna $i$}.\\

Y el \textbf{adjunto es} $\boldsymbol{\adjunto{a_{i\,j}}= (-1)^{i+j}*M_{i\,j}}$.\\
Al adjunto a veces se le llama cofactor.\\

\section{Propiedades de los determinantes.}
\subsection{Determinante de una matriz triangular.}
El \textbf{determinante de una matriz triangular} (superior o inferior, da lo mismo) \textbf{es el producto de los elementos de su diagonal principal}.\\
Esto es fácil de comprobar, ya que en este caso el único término del determinante en el que no aparece algún factor que sea 0 es el de la diagonal principal.
Esta propiedad se utiliza en el cálculo de determinantes por el método de Gauss.\\

De esta propiedad también se saca una conclusión importante: \textbf{el determinante de la matriz identidad vale 1}, independientemente de la dimensión de la matriz.
\[|I_n| = 1\]
\subsection{Cambios de filas o columnas.}
\textbf{Si intercambiamos entre sí dos filas o dos columnas de la matriz, el determinante cambia de signo.}

\subsection{Filas o Columnas de ceros}
Si una matriz tiene \textbf{una fila o una columna de ceros su determinante vale 0}.\\

Si una matriz tiene una fila o columna con \textbf{todo ceros excepto un elemento el determinante es igual a ese elemento por su adjunto}. Por ejemplo:
\[\left|\begin{array}{rrr}
	0&2&0\\
	4&-1&3\\
	1&-1&-5
\end{array}\right| = 2 * (-1)^{1 + 2} *\left|\begin{array}{rr}
4&3\\
1&-5
\end{array}\right| = 2* (-1) * (4*(-5) - 3*1) = 46\]

\subsection{Combinaciones lineales de filas y columnas.}
Vamos a empezar recordando lo que es una \emph{combinación lineal} con vectores, que es con lo que se ve en 4º de E.S.O.\\
Dados dos vectores $\vec{v}$ y $\vec{w}$, una combinación lineal es $a*\vec{v} + b*\vec{w}$ donde $a, b \in \mathbb{R}$\\

Cuando hablamos de filas o columnas de una matriz podemos decir que son vectores, ya que la suma de matrices se realiza por la posición del elemento como en los vectores.\\

Con lo que podemos hacer una combinación lineal de filas de la siguiente manera:
\[\left(\begin{array}{rrr}
	0&1&0\\
	3&2&-2\\
	2&3&1
\end{array}\right) \xrightarrow{\quad F_2 \leftarrow 2*F_1 + 3*F_2\quad} 
\left(\begin{array}{rrr}
	0&1&0\\
	9&8&-6\\
	2&3&1
\end{array}\right)\]
Con \small{$\xrightarrow{\quad F_2 \leftarrow 2*F_1 + 3*F_2\quad}$} indicamos que hemos sustituido la fila 2 ($F_2$) por una combinación lineal de ella misma y la fila 1 ($F_1$).\\

Una vez aclarado lo que es una combinación lineal de filas, para columnas sería similar, podemos enunciar qué consecuencias tiene esto para los determinantes:
\begin{itemize}
	\item \textbf{Si en una matriz una fila (o columna) es combinación lineal de las otras su determinante vale 0.} Por ejemplo:
	\[\left|\begin{array}{rrr}
		3&4&-5\\
		1&-4&7\\
		2&0&1\\
	\end{array}\right| = 0\quad \text{\parbox{.6\linewidth}{Porque la fila 2 es igual al doble de la fila 3 menos la fila 1.\\Es decir, la fila 2 es combinación lineal de otras dos filas.}}\]
	Esto es porque es como si tuviésemos una fila (o columna) de ceros.
	\item \textbf{Si a una fila (o columna) le sumamos una combinación lineal de las otras el determinante no cambia.} Vamos a verlo:
	\[\left|\begin{array}{rrr}
		0&1&0\\
		3&2&-2\\
		2&3&1
	\end{array}\right| = -7 \quad\text{Ahora sumo a la fila 1 las otras dos}\quad\left|\begin{array}{rrr}
	5&6&-1\\
	3&2&-2\\
	2&3&1
\end{array}\right| = -7
	\]
\end{itemize}

\subsection{Determinante de un producto de matrices.}
\textbf{El determinante de un producto es igual al producto de los determinantes.}
\[|A*B| = |A|*|B|\]

\subsection{Determinante de la matriz inversa.}
Sabemos que la matriz inversa cumple que:
\[A*A^{-1} = I\]
Y como hemos visto que $|I| = 1$, aplicamos la propiedad del determinante del producto y resulta:
\[|A|*|A^{-1}| = 1\]

Con lo que:
\[\boldsymbol{|A^{-1}| = \frac{1}{|A|} = |A|^{-1}}\]
De donde sacamos la conclusión que \textbf{para que exista $\boldsymbol{A^{-1}}$ tiene que ocurrir que $\boldsymbol{|A| \neq 0}$}.

\subsection{Determinante de la traspuesta.}
El \textbf{determinante de la matriz traspuesta es el mismo} que el de la matriz original:
\[\boldsymbol{|A^t| = |A|}\]

\subsection{Determinante del producto por un escalar.}
Esta propiedad va a tener dos partes:\\

Si \textbf{multiplicamos una fila }(o una columna) \textbf{por un valor el determinante queda multiplicado por ese valor}.
\[\left|\begin{array}{cccc}
a_{1\,1}&a_{1\,2}&\cdots&a_{1\,n}\\
\vdots&\vdots&&\vdots\\
k*a_{i\,1}&k*a_{i\,2}&\cdots&k*a_{i\,n}\\
\vdots&\vdots&&\vdots\\
a_{n\,1}&a_{n\,2}&\cdots&a_{n\,n}\\
\end{array}\right| = k*\left|\begin{array}{cccc}
a_{1\,1}&a_{1\,2}&\cdots&a_{1\,n}\\
\vdots&\vdots&&\vdots\\
a_{i\,1}&a_{i\,2}&\cdots&a_{i\,n}\\
\vdots&\vdots&&\vdots\\
a_{n\,1}&a_{n\,2}&\cdots&a_{n\,n}\\
\end{array}\right|\]

Y ahora la segunda parte que es conclusión de la primera: \textbf{el determinante de una matriz por un escalar es igual al determinante de la matriz por el escalar elevado a la dimensión de la matriz}:
\[\left|k*\left(\begin{array}{cccc}
a_{1\,1}&a_{1\,2}&\cdots&a_{1\,n}\\
\vdots&\vdots&&\vdots\\
a_{i\,1}&a_{i\,2}&\cdots&a_{i\,n}\\
\vdots&\vdots&&\vdots\\
a_{n\,1}&a_{n\,2}&\cdots&a_{n\,n}\\
\end{array}\right)\right| = \left|\begin{array}{cccc}
k*a_{1\,1}&k*a_{1\,2}&\cdots&k*a_{1\,n}\\
\vdots&\vdots&&\vdots\\
k*a_{i\,1}&k*a_{i\,2}&\cdots&k*a_{i\,n}\\
\vdots&\vdots&&\vdots\\
k*a_{n\,1}&k*a_{n\,2}&\cdots&k*a_{n\,n}\\
\end{array}\right|= k^n*\left|\begin{array}{cccc}
a_{1\,1}&a_{1\,2}&\cdots&a_{1\,n}\\
\vdots&\vdots&&\vdots\\
a_{i\,1}&a_{i\,2}&\cdots&a_{i\,n}\\
\vdots&\vdots&&\vdots\\
a_{n\,1}&a_{n\,2}&\cdots&a_{n\,n}\\
\end{array}\right|\]

\subsection{Transformación en suma de determinantes.}
Esta propiedad es un poco difícil de describir con palabra, así que lo que haremos será enunciar el caso general y luego haremos un caso de ejemplo.\\

El caso general:
\[\left|\begin{array}{ccccc}
a_{1\,1}&\cdots&a_{1\,j} + b_{1\,j}&\cdots&a_{1\,n}\\
a_{2\,1}&\cdots&a_{2\,j} + b_{2\,j}&\cdots&a_{2\,n}\\
\vdots&&\vdots&&\vdots\\
a_{i\,1}&\cdots&a_{i\,j} + b_{i\,j}&\cdots&a_{i\,n}\\
\vdots&&\vdots&&\vdots\\
a_{n\,1}&\cdots&a_{n\,2}+b_{n\,j}&\cdots&a_{n\,n}\\
\end{array}\right| = \left|\begin{array}{ccccc}
a_{1\,1}&\cdots&a_{1\,j}&\cdots&a_{1\,n}\\
a_{2\,1}&\cdots&a_{2\,j}&\cdots&a_{2\,n}\\
\vdots&&\vdots&&\vdots\\
a_{i\,1}&\cdots&a_{i\,j}&\cdots&a_{i\,n}\\
\vdots&&\vdots&&\vdots\\
a_{n\,1}&\cdots&a_{n\,2}&\cdots&a_{n\,n}\\
\end{array}\right| + \left|\begin{array}{ccccc}
a_{1\,1}&\cdots&b_{1\,j}&\cdots&a_{1\,n}\\
a_{2\,1}&\cdots&b_{2\,j}&\cdots&a_{2\,n}\\
\vdots&&\vdots&&\vdots\\
a_{i\,1}&\cdots&b_{i\,j}&\cdots&a_{i\,n}\\
\vdots&&\vdots&&\vdots\\
a_{n\,1}&\cdots&b_{n\,j}&\cdots&a_{n\,n}\\
\end{array}\right| \]

Con un \textbf{ejemplo}:
\[\left|\begin{array}{rrr}
1&3&5\\
3&0&-1\\
0&-4&2
\end{array}\right| = \left|\begin{array}{rrr}
1&3&5\\
0&0&-1\\
0&-4&2
\end{array}\right| + \left|\begin{array}{rrr}
0&3&5\\
3&0&-1\\
0&-4&2
\end{array}\right|\]
Y por las propiedades que hemos visto, esos determinantes se pueden reducir al producto del valor de la primera columna que está entre ceros por su adjunto:
\[=1*(-1)^{1+1}\left|\begin{array}{rr}
0&-1\\
-4&2
\end{array}\right| + 3*(-1)^{2+1}\left|\begin{array}{rr}
3&5\\
-4&2
\end{array}\right| = -4 -78 = -82\]

\section{Métodos para calcular determinantes.}
\subsection{Desarrollo por filas o columnas.}
Por lo visto en las propiedades, el determinante.
\[\left|\begin{array}{rrrr}
a_{1\,1}&\cdots&a_{1\,n}\\
\vdots&&\vdots\\
a_{n\,1}&\cdots&a_{n\,n}
\end{array}\right|\]
Lo podemos calcular como:
\[\left|\begin{array}{rrrr}
a_{1\,1}&\cdots&a_{1\,n}\\
\vdots&&\vdots\\
a_{n\,1}&\cdots&a_{n\,n}
\end{array}\right| = a_{i\,1}*\adjunto{a_{i\,1}} + a_{i\,2}*\adjunto{a_{i\,2}} + \cdots
+ a_{i\,n}*\adjunto{a_{i\,n}} = \sum_{j=1}^n a_{i\,j}*\adjunto{a_{i\,j}}\]
Y esto sería el desarrollo del determinante por la fila $i$.\\

Si lo que queremos es desarrollar el determinante por la columna $i$ quedaría:
\[\left|\begin{array}{rrrr}
a_{1\,1}&\cdots&a_{1\,n}\\
\vdots&&\vdots\\
a_{n\,1}&\cdots&a_{n\,n}
\end{array}\right| = a_{1\,i}*\adjunto{a_{1\,i}} + a_{2\,i}*\adjunto{a_{2\,i}} + \cdots
+ a_{n\,i}*\adjunto{a_{n\,i}} = \sum_{j=1}^n a_{j\,i}*\adjunto{a_{j\,i}}\]


Vamos con un \textbf{ejemplo al más grande}:\\
Calcula el determinante:
\[\left|\begin{array}{rrrr}
1&3&0&-1\\
-2&0&4&3\\
-1&-1&2&-2\\
2&1&3&0
\end{array}\right|\]
\begin{solution}
  Lo más lógico es desarrollarlo a través de una fila o una columna que tenga ceros, ya que
  eso nos va a eliminar términos y nos evitará el calcular algunos determinantes.
  Así que vamos a escoger la primera fila que tiene un 0 en la posición 1,3:
  \begin{small}
  \begin{flalign*}\left|\begin{array}{rrrr}
1&3&0&-1\\
-2&0&4&3\\
-1&-1&2&-2\\
2&1&3&0
  \end{array} \right| &= 1*(-1)^{1+1}*\left|\begin{array}{rrr}
  0&4&3\\
  -1&2&-2\\
  1&3&0
  \end{array}\right| + 3*(-1)^{1+2}\left|\begin{array}{rrr}
  -2&4&3\\
  -1&2&-2\\
  2&3&0
  \end{array}\right|\\& -1 * (-1)^{1+4}\left|\begin{array}{rrr}
  -2&0&4\\
  -1&-1&2\\
  2&1&3
  \end{array}\right|\end{flalign*}
  \begin{center}
    Como el término correspondiente al elemento 1,3 va a dar cero no lo hemos tenido en cuenta.
  \end{center}
  \end{small}
  Y ya solo tenemos que calcular esos tres determinantes, de manera que:
\begin{flalign*}
  1*(-1)^{1+1}*\left|\begin{array}{rrr}
  0&4&3\\
  -1&2&-2\\
  1&3&0
  \end{array}\right| + 3*(-1)^{1+2}\left|\begin{array}{rrr}
  -2&4&3\\
  -1&2&-2\\
  2&3&0
  \end{array}\right| -1 * (-1)^{1+4}\left|\begin{array}{rrr}
  -2&0&4\\
  -1&-1&2\\
  2&1&3
  \end{array}\right|=\\ 1*(-1)^{1+1}*(-23) + 3*(-1)^{1+2}*(-49) -1*(-1)^{1+4}*14 = 138
  \end{flalign*}
\end{solution}

\subsection{Cálculo por reducción de orden.}
Basándonos en el método anterior podemos concluir que si en un determinante todos los elementos de una fila, o columna, son ceros excepto uno de ellos el determinante es igual a ese elemento por su adjunto (esto ya lo habíamos visto como propiedad).\\

Como las propiedades también nos dicen que si a una fila (o columna) le añadimos una combinación lineal de otras el valor del determinante no cambia, podemos ir haciendo ceros utilizando estas combinaciones lineales.\\

Vamos a verlo con un \textbf{ejemplo}:
\[\left| \begin{array}{rrrr}
	3&1&-1&2\\
	-2&-2&2&1\\
	1&3&-1&-1\\
	2&4&-3&2
\end{array} \right|\]
\begin{solution}
	Para que nos resulte sencillo hacer ceros vamos a utilizar un elemento que valga 1 y haremos ceros en toda su fila o columna. Y para que sea aún más sencillo lo vamos a poner el primero.\\
	Elegimos el elemento 1,\,3, que vale 1, y vamos a cambiar toda su fila por la primera. Tenemos que recordar que al cambiar dos filas el determinante cambia de signo, con lo que:
	\[\left| \begin{array}{rrrr}
		3&1&-1&2\\
		-2&-2&2&1\\
		1&3&-1&-1\\
		2&4&-3&2
	\end{array} \right| \stackrel{F_1 \rightleftarrows F_3}{=} - \left| \begin{array}{rrrr}
	1&3&-1&-1\\
	-2&-2&2&1\\
	3&1&-1&2\\
	2&4&-3&2
\end{array}\right|\]
Y ahora vamos restando la primera fila a las demás multiplicándola por el valor adecuado para lograr los ceros. Vamos a verlo con la segunda fila:
\[- \left| \begin{array}{rrrr}
	1&3&-1&-1\\
	-2&-2&2&1\\
	3&1&-1&2\\
	2&4&-3&2
\end{array}\right| \stackrel{F_2  \leftarrow F_2 + 2*F_1}{=}
- \left| \begin{array}{rrrr}
	1&3&-1&-1\\
	0&4&0&-1\\
	3&1&-1&2\\
	2&4&-3&2
\end{array}\right|\]
Y hacemos lo mismo con la tercera y la cuarta fila:
\[- \left| \begin{array}{rrrr}
	1&3&-1&-1\\
	0&4&0&-1\\
	3&1&-1&2\\
	2&4&-3&2
\end{array}\right| \stackrel{\tiny \begin{array}{c}
	F_3 \leftarrow F_3 - 3*F_1\\
	F_4 \leftarrow F_4 - 2*F_1
\end{array}}{=} - \left| \begin{array}{rrrr}
1&3&-1&-1\\
0&4&0&-1\\
0&-8&2&5\\
0&-2&-1&4
\end{array}\right|\]

Y ahora, como en la primera columna todos los elementos son 0 excepto el primero, el determinante será igual a ese elemento por su adjunto:
\[- \left| \begin{array}{rrrr}
	1&3&-1&-1\\
	0&4&0&-1\\
	0&-8&2&5\\
	0&-2&-1&4
\end{array}\right| =- 1 * (-1)^{1+1}*
\left| \begin{array}{rrr}
	4&0&-1\\
	-8&2&5\\
	-2&-1&4
\end{array}\right| = -1*(40) = -40\]
\end{solution}

\section{Aplicación de los determinantes al cálculo de la inversa.}
Calcular la matriz inversa mediante el uso de los determinantes es relativamente sencillo. Para ello vamos a hacer primero una definición:
\subsection{Matriz adjunta.}
La matriz adjunta de $A$ (se llama $\adjunto{A}$) \textbf{es la matriz formada por los adjuntos de $A$}. Es decir:
\[\adjunto{A} =\left( \begin{array}{rrr}
	\adjunto{a_{1\,1}}&\cdots&\adjunto{a_{1\,n}}\\
	\vdots&&\vdots\\
	\adjunto{a_{n\,1}}&\cdots&\adjunto{a_{n\,n}}
\end{array} \right)\]
\subsection{Cálculo de la matriz inversa.}
La matriz inversa de $A$ es igual a \textbf{la traspuesta de la adjunta partida por el determinante} de $A$:
\[A^{-1} = \frac{\left(\adjunto{A}\right)^t}{|A|}\]

Vamos a ver un par de \textbf{ejemplos}, primero con una de orden 2 y luego con otra de orden 3.
\begin{questions}
	\question Calcular la inversa de $\left(\begin{array}{rr}
		1&3\\
		1&2
	\end{array}\right)$.
\begin{solution}
\begin{small}
Lo primero que tenemos que hacer es calcular el determinante de la matriz, ya que si es 0 entonces no tiene inversa y no tiene sentido seguir haciendo ningún cálculo.
\[\left|\begin{array}{rr}
	1&3\\
	1&2
\end{array}\right| = -1\]
Con lo que la matriz tiene inversa. Calculemos ahora la adjunta:
\begin{itemize}
	\item $\adjunto{a_{1\,1}} = 2$
	\item $\adjunto{a_{1\,2}} = -1$
	\item $\adjunto{a_{2\,1}} = -3$
	\item $\adjunto{a_{1\,1}} = 1$
\end{itemize}
Con lo que la matriz adjunta es:
\[\adjunto{A} = \left(\begin{array}{rr}
	2&-1\\
	-3&1
\end{array}\right)\]
Y ya es sencillo calcular la inversa:
\[A^{-1} = \frac{\left(\begin{array}{rr}
		2&-1\\
		-3&1
	\end{array}\right)^t}{-1} = \left(\begin{array}{rr}
-2&3\\
1&-1
\end{array}\right)\]
Y es fácil comprobar que es la inversa:
\[\left(\begin{array}{rr}
	1&3\\
	1&2
\end{array}\right) * \left(\begin{array}{rr}
-2&3\\
1&-1
\end{array}\right) = \left(\begin{array}{rr}
1&0\\
0&1
\end{array}\right) \]
\end{small}
\end{solution}

\question Calcular la inversa de $\left(\begin{array}{rrr}
	1&-1&0\\
	0&1&0\\
	2&0&1
\end{array}\right)$.
\begin{solution}
\begin{small}
Empezamos por calcular el determinante para ver si tiene inversa:
\[\left|\begin{array}{rrr}
	1&-1&0\\
	0&1&0\\
	2&0&1
\end{array}\right| = 1\]
Como sí que tiene inversa calculamos los adjuntos:
\begin{multicols}{3}
	\begin{itemize}
		\item $\adjunto{a_{1\,1}} = \left|\begin{array}{rr}
			1&0\\
			0&1
		\end{array}\right| = 1$
		\item $\adjunto{a_{2\,1}} = -\left|\begin{array}{rr}
		-1&0\\
		0&1
	\end{array}\right| = 1$
		\item $\adjunto{a_{3\,1}} = \left|\begin{array}{rr}
			-1&0\\
			1&0
		\end{array}\right| = 0$
		\item $\adjunto{a_{1\,2}} = -\left|\begin{array}{rr}
			0&0\\
			2&1
		\end{array}\right| = 0$
		\item $\adjunto{a_{2\,2}} = \left|\begin{array}{rr}
			1&0\\
			2&1
		\end{array}\right| = 1$
		\item $\adjunto{a_{3\,2}} = -\left|\begin{array}{rr}
			1&0\\
			0&0
		\end{array}\right| = 0$
		\item $\adjunto{a_{1\,3}} = \left|\begin{array}{rr}
			0&1\\
			2&0
		\end{array}\right| = -2$
		\item $\adjunto{a_{2\,3}} = -\left|\begin{array}{rr}
			1&-1\\
			2&0
		\end{array}\right| = -2$
		\item $\adjunto{a_{3\,3}} = \left|\begin{array}{rr}
			1&-1\\
			0&1
		\end{array}\right| = 1$
	\end{itemize}
\end{multicols}
Con lo que:
\[\adjunto{A} = \left(\begin{array}{rrr}
	1&0&-2\\
	1&1&-2\\
	0&0&1
\end{array}\right)\]
Y la inversa:
\[A^{-1} = \frac{\left(\begin{array}{rrr}
		1&0&-2\\
		1&1&-2\\
		0&0&1
	\end{array}\right)^t}{1} = \left(\begin{array}{rrr}
1&1&0\\
0&1&0\\
-2&-2&1
\end{array}\right)\]
Y se comprueba que su producto por la matriz del enunciado es la identidad:
\[\left(\begin{array}{rrr}
	1&1&0\\
	0&1&0\\
	-2&-2&1
\end{array}\right) * \left(\begin{array}{rrr}
1&-1&0\\
0&1&0\\
2&0&1
\end{array}\right) = \left(\begin{array}{rrr}
1&0&0\\
0&1&0\\
0&0&1
\end{array}\right)\]
\end{small}
\end{solution}
\end{questions}

\end{document}
