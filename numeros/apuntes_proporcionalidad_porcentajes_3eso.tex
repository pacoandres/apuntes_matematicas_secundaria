\documentclass[a4paper,11pt,answers]{exam}
\usepackage{graphicx}
\usepackage{hyperref}
\usepackage[utf8]{inputenc}
\usepackage[spanish]{babel}
\usepackage[T1]{fontenc}
%textcomp es para el símbolo del euro
\usepackage{lmodern, textcomp}
\usepackage[left=1in, right=1in, top=1in, bottom=1in]{geometry}
%\usepackage{mathexam}
\usepackage{amsmath}
\usepackage{amssymb}
\usepackage{multicol}
%para la última página
\usepackage{lastpage}
\usepackage[normalem]{ulem}
%Creative commons
%\usepackage{ccicons}
\usepackage[type={CC}, modifier={by-nc-sa}, version={4.0}, %imagemodifier={-eu-80x25},
lang={spanish}]{doclicense}


\usepackage{color,colortbl}
\definecolor{Gray}{gray}{0.9}
\newcolumntype{g}{>{\columncolor{Gray}}c}
%\pagestyle{headandfoot}
\pagestyle{headandfoot}
\newcommand\ExamNameLine{
\par
\vspace{\baselineskip}
Nombre:\hrulefill\relax
\par}

\renewcommand{\solutiontitle}{\noindent\textbf{Solución:}\par\noindent}

\everymath{\displaystyle}
\newcommand\ddfrac[2]{\frac{\displaystyle #1}{\displaystyle #2}}

\def \autor{Paco Andrés}
\def \titulo{Apuntes de proporcionalidad y porcentajes 3º ESO}
\def \titulofichas {\textbf {\titulo}}
\def \cursofichas {}
\def \fechaexamen {}
%\firstpageheader{\cursofichas}{\titulofichas}{\fechaexamen}
\header{\cursofichas}{\titulofichas}{\fechaexamen}
%\header{\cursofichas}{\titulofichas}{\fechaexamen}
%\firtspagefooter{}{\thepage}{}
%Por alguna razón no sale lo del cc en el pie
\firstpagefootrule
\footrule
\footer{\autor}{\thepage}{\doclicenseIcon}
\pointpoints{punto}{puntos}

\shadedsolutions
%\definecolor{SolutionColor}{rgb}{0.99,0.99,.99}
\renewcommand{\baselinestretch}{1.3}

%Use * instead of \cdot
\mathcode`\*="8000
{\catcode`\*\active\gdef*{\cdot}} 
\newcommand{\Card}{\,\mathrm{Card}}

\renewcommand{\questionlabel}{\textbf{Ejemplo \thequestion:}}
\newcommand{\eur}{\,\text{€}}

\begin{document}

%\author{Paco Andrés}
\title{\titulo}
\date{}
\author{\autor}
\maketitle

\begin{center}
\doclicenseLongText\\
\vspace{.25cm}
\doclicenseImage
\end{center}
\tableofcontents
\newpage
% Para quitar el sangrado
\setlength{\parindent}{0cm}



\section{Proporcionalidad simple.}
\subsection{Magnitud.}
A lo largo de gran parte de este tema vamos a estar hablando de magnitudes, con lo que conviene
definir primero lo que es para poder entender bien de qué estamos hablando tanto en la teoría como
en los ejemplos.
\begin{quote}
\emph{\textbf{Una magnitud es algo que puede ser medido de una forma numérica y expresado con
    una unidad.}}
\end{quote}

Las unidad en la que midamos puede ser cualquiera: las unidades del sistema métrico, euros, personas, o lo que sea.

\subsection{Magnitudes proporcionales.}
Se dice que \textbf{dos magnitudes son proporcionales cuando aumentan o disminuyen de la misma manera}, es decir, si una se multiplica por dos la otra también se multiplica por dos, o se divide entre
dos dependiendo del tipo de proporcionalidad.
\subsubsection{Proporcionalidad directa.}
Se dice que \textbf{dos magnitudes son directamente proporcionales si las dos aumentan de la misma
  manera}, es decir, si una se multiplica por dos la otra también.\\

Vamos a verlo con \textbf{un ejemplo}:\\
Indica la relación que existe entre el número de animales de una granja y la cantidad de comida
que necesitan.
\begin{solution}
  Aquí tenemos dos magnitudes:
  \begin{itemize}
  \item El número de animales, que se mide en animales.
  \item La cantidad de comida, que se mide en kilogramos.
  \end{itemize}
  Es evidente que si el número de animales se multiplica por dos los kg de comida necesaria
  también se multiplica por dos, con lo cual aumentan de la misma manera y tienen una relación de
  proporcionalidad directa.
\end{solution}
\subsubsection{Proporcionalidad inversa.}
Se dice que \textbf{dos magnitudes son directamente proporcionales si al aumentar una la otra
  disminuye de la misma manera}, es decir, si una se multiplica por dos la otra se divide entre
dos.\\

Vamos a verlo con \textbf{un ejemplo}:\\
Indica la relación que existe entre la velocidad a la que va un vehículo y el tiempo que tarda en
recorrer una distancia determinada.
\begin{solution}
  Aquí tenemos dos magnitudes:
  \begin{itemize}
  \item La velocidad, que podemos medirla en km/h.
  \item El tiempo, que se puede medir en h.
  \end{itemize}
  Es evidente que si la velocidad se duplica el tiempo que tardamos se va a reducir a la mitad.
  Es decir, cuando una aumenta la otra disminuye de la misma manera, con lo que tienen una relación
  de proporcionalidad inversa.
\end{solution}

\subsection{Resolución de problemas sencillos de proporcionalidad simple.}
Los enunciados de los problemas de proporcionalidad simple son bastantes similares, normalmente
siguen el siguiente esquema:
\begin{itemize}
\item Nos presentan una situación en la que aparecen dos magnitudes con valores conocidos entre las
  que existe una relación de proporcionalidad. Las llamaremos $a_i$ y $b_i$ ($i$ de inicial).
\item Nos preguntan cuanto debe valer una de estas magnitudes conociendo el valor de la otra. A
  estas magnitudes finales las llamaremos $a_f$ y $b_f$ ($f$ de final).
\item En algunos problemas nos pueden hacer varias preguntas que tendremos que tratar como problemas
  independientes, no podemos contestarlas a todas a la vez.
\end{itemize}

Y el mecanismo para resolverlos es similar en todos:
\begin{enumerate}
\item En primer lugar tenemos que identificar las dos magnitudes que tenemos que manejar, que
  por simplicidad las llamaremos $\boldsymbol{a}$ y $\boldsymbol{b}$.
\item Una vez identificadas las magnitudes tenemos que averiguar qué relación de proporcionalidad
  existe entre ellas haciéndonos la siguiente pregunta:\\
  \emph{¿Al doble de $\boldsymbol{a}$ le corresponde el doble de $\boldsymbol{b}$ o la mitad?}
  Y, por lo que hemos visto, si es \textbf{el doble} se trata de \textbf{proporcionalidad directa}
  mientras que si es \textbf{la mitad} lo que tenemos es \textbf{proporcionalidad inversa}.
\item Una vez identificado el tipo de proporcionalidad que tenemos aplicaremos el método que
  corresponda.
\end{enumerate}
\subsubsection{Problemas de proporcionalidad directa.}
Tal y como hemos indicado en el apartado anterior tenemos las magnitudes $a$ y $b$ y hemos
deducido que tienen una relación de proporcionalidad directa.\\
En este caso los valores iniciales y finales guardan la siguiente relación:
\begin{Large}
  \[\boldsymbol{\frac{b_i}{a_i} = \frac{b_f}{a_f}}\]
\end{Large}
Que podemos escribir también.
\[\boldsymbol{\frac{a_f}{a_i} = \frac{b_f}{b_i}}\]
Y esta relación es la que utilizaremos para resolver los problemas sencillos.\\

Vamos a ver un par de ejemplos:
\begin{questions}
\question Hemos comprado $3$\,kg de manzanas por $5.40$\,€. ¿Cuánto nos costarán $5$\,kg?.
  \begin{solution}
    En este caso las magnitudes son el peso de las peras y el dinero que nos cuesta.\\
    Para que sea más sencillo de entender asignamos los nombres que hemos puesto en el método:
    \begin{itemize}
    \item $a \equiv$ peso de las peras.
    \item $b \equiv$ dinero que nos cuestan.
    \end{itemize}
    Y nos hacemos la pregunta:
    \begin{quote}
      \emph{¿El doble de peso de peras nos costará el doble o la mitad?}
    \end{quote}
    La respuesta es evidente, el doble, con lo que estamos en el caso de proporcionalidad directa
    que es el que estamos estudiando ahora.\\

    
    Del enunciado sacamos que los valores iniciales y finales son:
    \begin{itemize}
    \item $a_i = 3\,\text{kg}\quad a_f = 5\,\text{kg}$
    \item $b_i = 5.40\,\text{€}\quad b_f = x$
    \end{itemize}
    Sustituimos estos valores en la relación y tenemos:
    \[\frac{5\,\text{kg}}{3\,\text{kg}} = \frac{x}{5.40\,\text{€}}\]
    Y tenemos que despejar la $x$, con lo que el $5.40\,\text{€}$ que está dividiendo pasa al
    otro lado multiplicando, con lo que queda:
    \[x = 5.40\,\text{€}*\frac{5}{3} = 9.00\,\text{€}\]
  \end{solution}
\question Una persona es capaz de teclear 7 correos cada $35$\,min. ¿Cuántos escribirá en $10$\,
  min? ¿Cuánto tardará en escribir 50 correos?
  \begin{solution}
    Aquí tenemos un ejemplo en el que hay dos preguntas, con lo que las trataremos como dos
    problemas separados en los que los valores iniciales serán los mismos pero los finales no.\\

    Las magnitudes que tenemos son los correos enviados y el tiempo, con lo que la pregunta es:
    \begin{quote}
      \emph{¿El doble de correos llevará el doble de tiempo o la mitad?}
    \end{quote}
    Lógicamente respuesta es el doble, con lo que volvemos a tener proporcionalidad directa.\\

    Llamaremos $a$ a los correos y $b$ al tiempo, y dividiéndolo en dos preguntas tenemos:
    \begin{parts}
    \part Para la primera pregunta:
      \begin{itemize}
      \item $a_i = 7 \quad a_f = x$
      \item $b_i = 35\,\text{min} \quad b_f = 10\,\text{min}$
      \end{itemize}
      Escribimos la relación:
      \[\frac{x}{7} = \frac{10}{35}\]
      Despejamos:
      \[x = 7*\frac{10}{35} = 2\]
      Con lo que en $10$\,min escribirá dos correos.
    \part Para la segunda pregunta:
      \begin{itemize}
      \item $a_i = 7 \quad a_f = 50$
      \item $b_i = 35\,\text{min} \quad b_f = x$
      \end{itemize}
      Escribimos la relación:
      \[\frac{50}{7} = \frac{x}{35\,\text{min}}\]
      Despejamos:
      \[x = 35\,\text{min}*\frac{50}{7} = 250\,\text{min}\]
      Con lo que en escribir $50$ correos tardará $250$\,min, que son $4$ horas y $10$ minutos.
    \end{parts}
  \end{solution}
\end{questions}

\subsubsection{Problemas de proporcionalidad inversa.}
En este caso la relación que guardan los valores iniciales y finales es:
\begin{Large}
  \[\boldsymbol{a_i*b_i = a_f*b_f}\]
\end{Large}
Y la vamos a usar en \textbf{los siguientes ejemplos}:
\begin{questions}
\question Para construir un muro en tres días hacen falta cuatro personas. Calcula cuantas
  personas harán falta para construirlo en dos días.
  \begin{solution}
    Llamaremos $\boldsymbol{a}$ a las personas y $\boldsymbol{b}$ con lo que los valores iniciales
    y finales son:
    \begin{itemize}
    \item $a_i = 3\,\text{días}\quad a_f = 2\,\text{días}$
    \item $b_i = 4\,\text{personas}\quad b_f=x$
    \end{itemize}
    Entonces la relación queda:
    \[3\,\text{días}*4\,\text{personas} = 2\,\text{días}*x\]
    Y despejando:
    \[x = \frac{3\,\text{días}*4\,\text{personas}}{2\,\text{días}}\]
    \[x = 6\,\text{personas}\]

    Con lo que hacen falta $6$\,personas para construir un muro en $2$\,días.
  \end{solution}
\question Con un depósito de agua se abastece a $20$\,casas durante $15$\,días. Calcula cuantos días
  durará si los habitantes de $8$\,casas se van de vacaciones.
  \begin{solution}
    En este caso llamaremos $\boldsymbol{a}$ a las casas y $b$ a los días que dura el depósito.\\
    Los valores iniciales y finales son:
    \begin{itemize}
    \item $a_i = 20\,\text{casas}\quad a_f=(20 - 8)\,\text{casas} = 12\,\text{casas}$
    \item $b_i = 15\,\text{días}\quad b_f = x$.
    \end{itemize}
    Con esto escribimos la relación:
    \[20\,\text{casas}*15\,\text{días} = 12\,\text{casas}*x\]
    Y despejando:
    \[x = \frac{20\,\text{casas}*15\,\text{días}}{12\,\text{casas}}\]
    \[x = 25\,\text{días}\]
    Si se van $8$\,casas de vacaciones el depósito dura $25$\,días.
  \end{solution}
\end{questions}
\section{Proporcionalidad compuesta.}
En este tipo de proporcionalidad lo que tenemos es que hay más de dos magnitudes, con lo cual
la cosa no es tan sencilla como en la simple, aunque tampoco se complica mucho más.\\

Vamos a ver cómo se resuelven los problemas de proporcionalidad compuesta con un ejemplo:
\begin{quote}
\emph{En $8$\,días, $6$\,excavadoras hacen una zanja de $2\,100$\,m de largo. Calcula cuántas
  máquinas serán necesarias para cavar una zanja de $525$\,m en $3$\,días.}
\end{quote}
Como hemos dicho ahora tenemos más de dos magnitudes, en este caso:
\begin{itemize}
\item Días.
\item Número de excavadoras.
\item Metros de zanja.
\end{itemize}
Y para resolverlo vamos a seguir los siguientes pasos:
\begin{enumerate}
\item \textbf{Identificamos las magnitudes del problema y por cual de ellas nos preguntan}.\\
  En este caso hemos visto que las magnitudes son los días, las excavadoras y los metros de zanja,
  y nos preguntan por el número de excavadoras.
\item \textbf{Establecemos la relación que existe (directa o inversa) entre la magnitud por la que nos
  preguntan y cada una de las demás}.
  \begin{itemize}
  \item \emph{Excavadoras $\longleftrightarrow$ días}: ¿el doble de excavadoras tardará el doble
    o la mitad de días?. Tardará la mitad, con lo cual es inversa.
  \item \emph{Excavadoras $\longleftrightarrow$ metros}: ¿el doble de excavadoras harán el doble
    o la mitad de zanja?. Evidentemente harán el doble, entonces es directa.
  \end{itemize}
\item \textbf{Para cada magnitud construimos una fracción} de la siguiente manera:
  \begin{itemize}
  \item Si la \textbf{relación es inversa}: $\frac{\text{\textbf{valor inicial}}}
    {\text{\textbf{valor final}}}$\vspace*{2mm}
  \item Si la \textbf{relación es directa}: $\frac{\text{\textbf{valor final}}}
    {\text{\textbf{valor inicial}}}$
  \end{itemize}
  Con lo cual en el ejemplo tendremos:
  \begin{itemize}
  \item Para los días, al ser inversa: $\frac{8\,\text{días}}{3\,\text{días}}$.\vspace*{2mm}
  \item Para los metros, al ser directa: $\frac{525\,\text{m}}{2\,100\,\text{m}}$.
  \end{itemize}
\item Por último el valor de \textbf{la magnitud buscada es el resultado de multiplicar el valor inicial
  de esta magnitud por las fracciones del paso anterior}.\\
  De esta manera el número de excavadoras que harán falta es:
  \[x = 6\,\text{excavadoras}*\frac{8}{3}*\frac{525}{2\,100} = 4\,\text{excavadoras}\]
\end{enumerate}
Vamos a ver un par de ejemplos más:
\begin{questions}
\question El alquiler de dos coches durante $9$ días cuesta $675$\,€. Calcula cuanto cuesta alquilar
  tres coches durante $7$ días.
  \begin{solution}
    Identificamos cuales son las magnitudes y por cual nos preguntan:
    \begin{itemize}
    \item Coches.
    \item Días.
    \item Precio. Esta es por la que nos preguntan.
    \end{itemize}
    Averiguamos la relación que existe entre la magnitud por la que nos preguntan y las demás, y de
    paso construimos las fracciones correspondientes:
    \begin{itemize}
    \item Precio $\longleftrightarrow$ coches, directa. La fracción es $\frac{3}{2}$.
    \item Precio $\longleftrightarrow$ días, directa. La fracción es $\frac{7}{9}$
    \end{itemize}
    Y con esto ya tenemos que:
    \[\text{Precio pedido} = 675\,\text{€} * \frac{3}{2}*\frac{7}{9}\]
    \[\text{Precio pedido} = 787,50\,\text{€}\]
  \end{solution}
\question Dieciocho personas han tardado $6$ días en instalar $300$\,m de tubería trabajando $6$\,h/día. ¿Cuántas horas diarias tendrían que trabajar $24$ personas durante $14$ días para instalar
  $700$\,m de tubería?.
  \begin{solution}
    A pesar de que haya más magnitudes el procedimiento es el mismo.\\

    Identificamos las magnitudes:
    \begin{itemize}
    \item Personas.
    \item Días.
    \item Horas al día. Ésta es la preguntada.
    \item Metros de tubería.
    \end{itemize}
    Averiguamos las relaciones y construimos las fracciones:
    \begin{itemize}
    \item h/día $\longleftrightarrow$ personas, inversa. La fracción es $\frac{18}{24}$.
    \item h/día $\longleftrightarrow$ días, inversa. La fracción: $\frac{6}{14}$.
    \item h/día $\longleftrightarrow$ metros de tubería, directa. La fracción: $\frac{700}{300}$
    \end{itemize}
    Entonces:
    \[\text{Horas al día} = 6\,\text{h/día}\,* \frac{18}{24} * \frac{6}{14} * \frac{700}{300} =
      4,5\,\text{h/día}\]
  \end{solution}

\end{questions}
No tiene mucho sentido poner más ejemplos porque todos estos problemas se resuelven dando los mismos
pasos. La manera de aprender a hacerlos lo mejor posible es practicando.
\section{Repartos proporcionales.}
AL igual que todo lo que estamos viendo en este tema, los repartos proporcionales son algo que
se utiliza mucho en el mundo real, veamos un ejemplo:
\begin{quote}
  Tres personas ponen en marcha una empresa aportando $1\,500$, $2\,000$ y $3\,000\eur$ respectivamente.\\
  \emph{Si la empresa obtiene unos beneficios de $3\,900\eur$ ¿cuánto le corresponde a cada una?}
\end{quote}
Lo que corresponde a cada una tiene que guardar una relación de proporcionalidad con el dinero que
ha aportado, y sería una proporcionalidad directa (al doble de dinero aportado le corresponde
el doble de beneficio). También se podría hacer un reparto conforma a las horas trabajadas o con
respecto a otros parámetros que entre esas tres personas considerasen justos.\\

Con lo cual tenemos que: \textbf{en un reparto proporcional tenemos dos magnitudes: una con un
  valor para repartir y otra con distintos valores que es la magnitud que reparte.}
\begin{itemize}
\item Si al \textbf{doble de la magnitud que reparte} le corresponde \textbf{el doble de la magnitud
    repartida} es un \textbf{reparto directamente proporcional}.
\item Si al \textbf{doble de la magnitud que reparte} le corresponde \textbf{la mitad de la magnitud
    repartida} es un \textbf{reparto inversamente proporcional}.
\end{itemize}
Pues vamos a ver cuales son los mecanismos para realizar estos repartos.

\subsection{Repartos directamente proporcionales.}
El ejemplo que acabamos de poner es un buen ejemplo para definir y ver cómo se hace un reparto
directamente proporcional.\\
Y como siempre vamos a verlo por pasos.
\begin{enumerate}
\item Llamemos $a$ a la magnitud repartida y $b_1$, $b_2$, \dots, a los valores de la magnitud que
  reparte.\\
  En el ejemplo tenemos:
  \begin{itemize}
  \item $a=3\,900\eur$, que es el beneficio que queremos repartir.
  \item $b_1 = 1\,500\eur$, $b_2 = 2\,000\eur$, $b_3 = 3\,000\eur$ son los valores de la magnitud
    que reparte.
  \end{itemize}
\item \textbf{Dividimos la magnitud repartida entre la suma de los valores que reparten}.
  Al resultado se le llama ``\emph{razón del reparto}''\\
  \[\boldsymbol{r = \frac{a}{b_1 + b_2 + \dots + b_n}}\]
  En el ejemplo tenemos:
  \[r = \frac{3\,900\eur}{1\,500\eur + 2\,000\eur + 3\,000\eur}\]
  Con lo que la razón del reparto es:
  \[r = 0.6\]
\item \textbf{Multiplicamos cada valor de la magnitud que reparte por la razón del reparto}, el resultado
  es lo que corresponde a cada una de las que participa en el reparto.\\
  Siguiendo con el ejemplo:
  \begin{itemize}
  \item A la que puso $1\,500\eur$ la corresponden $1\,500\eur*0.6 = 900\eur$.
  \item A la que puso $2\,000\eur$ la corresponden $2\,000\eur*0.6 = 1\,200\eur$.
  \item A la que puso $3\,000\eur$ la corresponden $3\,000\eur*0.6 = 1\,800\eur$.
  \end{itemize}
\item \textbf{Comprobamos que está bien sumando los valores que corresponden a cada una} y viendo que
  \textbf{coinciden con el valor de la magnitud repartida} (salvo errores propagados por el redondeo).\\
  Sumamos los resultados del reparto del ejemplo:
  \[900\eur + 1\,200\eur + 1\,800\eur = 3\,900\eur\]
  y vemos que coincide con el valor de los beneficios que había que repartir.
\end{enumerate}

\subsection{Repartos inversamente proporcionales.}
En este caso sucede lo contrario que en el anterior, a menor valor de la magnitud que reparte le
corresponde mayor valor de la magnitud a repartir.\\
Vamos a ver los pasos que hay que dar con un ejemplo.\\
\begin{quote}
  \emph{Tenemos tres gatos a los que queremos  una ración extra de $121$\,g de comida repartida
    de manera inversamente proporcional a la cantidad que han comido hoy. Si el primero comió
  $60$\,g, el segundo $80$\,g y el tercero $60$\,g, ¿Cuánto corresponde a cada uno?}
\end{quote}
\begin{enumerate}
\item Llamemos $a$ a la magnitud repartida y $b_1$, $b_2$, \dots, a los valores de la magnitud que
  reparte.\\
  En el ejemplo tenemos:
  \begin{itemize}
  \item $a=121$\,g, que es el la comida que queremos repartir.
  \item $b_1 = 60$\,g, $b_2 = 80$\,g, $b_3 = 60$\,g son los valores de la magnitud que reparte.
  \end{itemize}
\item \textbf{Dividimos la magnitud a repartir entre la suma de las inversas de la
    magnitud que reparte}. A este resultado también se le llama ``\emph{razón el reparto}''.
  \[\boldsymbol{r = \ddfrac{a}{\ddfrac{1}{b_1} + \ddfrac{1}{b_2} + \dots + \ddfrac{1}{b_n}}}\]
  Con los datos del ejemplo:
  \[r = \ddfrac{121}{\ddfrac{1}{60} + \ddfrac{1}{80} + \ddfrac{1}{60}}\]
  Operamos las fracciones como ya sabemos:
  \[r = \ddfrac{121}{\ddfrac{4}{240} + \ddfrac{3}{240} + \ddfrac{4}{240}}\]
  \[r = \ddfrac{\ 121\ }{\ddfrac{11}{240}} = 121:\frac{11}{240}\]
  \[r = 2\,640\]
\item \textbf{Dividimos la razón del reparto entre cada uno de los valores que reparten} para
  obtener lo que corresponde a cada uno.\\
  En nuestro ejemplo:
  \begin{itemize}
  \item Al primero le corresponde $\frac{2\,640}{60} = 44$\,g.
  \item Al segundo $\frac{2\,640}{80} = 33$\,g.
  \item Y al tercero lo mismo que al primero, $44$\,g.
  \end{itemize}
\item \textbf{Comprobamos que está bien sumando los valores que corresponden a cada una} y viendo que
  \textbf{coinciden con el valor de la magnitud repartida} (salvo errores propagados por el redondeo).\\
  Sumamos los resultados del ejemplo:
  \[44\,\text{g}+33\,\text{g} + 44\,\text{g} = 121\,\text{g}\]
\end{enumerate}
\section{Porcentajes.}
El hecho de que un porcentaje es una parte de algo es algo que se aprende desde muy pequeños en el
mundo en el que vivimos, ya que hay porcentajes por todas partes. Esto hace que sea necesario saber manejarlos lo mejor posible para ser conscientes de qué manera pueden afectarnos.\\

Para ello vamos a empezar por definir qué es exactamente un porcentaje:
\begin{quote}
  \textbf{\emph{Un porcentaje es una fracción cuyo denominador es 100.}}
\end{quote}
Es decir, al hacer un porcentaje de algo estamos dividiéndolo en cien partes y cogiendo las que
indica el valor del porcentaje.\\

El hecho de que sea una fracción nos simplifica bastantes cosas, ya que podemos aplicar lo que
sabemos de fracciones, de manera que para calcular un porcentaje vamos a utilizar la relación que
usamos en fracciones que nos dice:
\[\text{Fracción}*\text{Valor total} = \text{Valor de la parte}\]
Con lo que calcular un porcentaje se reduce a multiplicar por una fracción. Por ejemplo, calcular el $37\%$ de $2\,500$ es:
\[\frac{37}{100}*2\,500 = 0.37*2\,500 = 925\]
\subsection{Tanto por uno.}
En el cálculo del ejemplo anterior se puede observar que lo primero que hemos hecho es la división
$\frac{37}{100} = 0.37$.\\
A este resultado se le llama tanto por uno y es lo que más se utiliza para calcular porcentajes de
una manera efectiva, ya que simplifica los cálculos y la escritura de relaciones y ecuaciones.\\

\textbf{\emph{De aquí en adelante los porcentajes que utilicemos en los cálculos serán siempre en
    tanto por uno.}}
Los porcentajes ``clásicos'' solo los utilizaremos en los textos escritos, como enunciados y respuestas.

\subsection{Ejemplos de cálculo de porcentajes simples.}
Con lo que hemos visto en la definición de porcentajes junto con lo del tanto por uno, el resolver
problemas de porcentajes se convierte en algo bastantes sencillo.\\
Vamos a verlo en unos ejemplos.

\begin{questions}
\question Un pantano tiene una capacidad de $5$ millones de metros cúbicos. Si ahora mismo está al
  $63\%$ de su capacidad ¿cuántos $\text{m}^3$ contiene?
  \begin{solution}
    Pues tenemos que el total son $5\,000\,000\,\text{m}^3$, el porcentaje es $0.63$ y nos piden
    el valor de la parte. Es aplicación directa de la relación que hemos visto:
    \[\text{parte} = 0.63*5\,000\,000\,\text{m}^3 = 3\,150\,000\,\text{m}^3\]
    Y esa es la cantidad de agua que contiene.
  \end{solution}
\question El $35\%$ de los habitantes de una población tienen menos de $40$ años. Si hay $1\,430$
  personas con más de $40$ años, ¿cuántos habitantes tiene en total?
  \begin{solution}
    En este caso conocemos el valor de una parte, pero no de la que indica el porcentaje y
    necesitamos que ambas signifiquen lo mismo para poder utilizar la misma relación que en el
    anterior.\\
    En la citada población solo hay dos posibilidades, tener menos de $40$ años o más, de manera
    que la suma de esos dos porcentajes tiene que ser el $100\%$.\\
    Por lo tanto tenemos que:
    \begin{itemize}
    \item Hay un $35\%$ que tiene menos de $40$ años.
    \item Hay un $65\%$ que tiene más de $40$ años.
    \end{itemize}
    Y como el valor que nos dan corresponde a los que tienen más de 40 años tenemos que utilizar el
    porcentaje que se corresponda, que este caso es $65\%$.\\

    Escribimos la relación con los datos que tenemos:
    \[0.65*\text{total} = 1\,430\]
    El $0,65$ pasa al otro lado dividiendo.
    \[\text{total} = \frac{1\,430}{0.65} = 2\,200\]
    \[\text{total} = 2\,200\]
  \end{solution}
\question En un edificio viven $56$ personas de las cuales $14$ proceden de Teruel. ¿Qué porcentaje representan?
  \begin{solution}
    Escribimos la relación con los datos que tenemos:
    \[\text{porcentaje}*56 = 14\]
    El $56$ pasa al otro lado dividiendo.
    \[\text{porcentaje} = \frac{14}{56} = 0.25\]
    Entonces el $25\%$ son de Teruel.
  \end{solution}
\end{questions}

\section{Variaciones porcentuales.}
Este es, probablemente, la manera de la que más se utilizan los porcentajes en el mundo actual: para
expresar variaciones.\\

Es difícil que no hayamos oído hablar de unas rebajas del $x\%$, o del IVA, o del IPC, o del PIB,
o de los intereses bancarios, \dots\\
Todas las variaciones que se producen en los valores que acabamos de mencionar, y en muchos otros
más, se miden porcentualmente.\\

La manera de calcular cómo queda un valor después de una variación porcentual es la siguiente:
\begin{enumerate}
\item Calculamos el porcentaje del valor que varia.
\item Se lo sumamos, o restamos, al valor original de manera que obtenemos el valor tras la
  variación.
\end{enumerate}
Con un ejemplo:
\begin{quote}
\emph{El precio de los tomates ha subido un $7\%$. Si antes valían a $3.50$\,€/kg, ¿cuánto cuesta
  el kg ahora?}
\begin{enumerate}
\item Calculamos el $7\%$ de $3.50$\,€:
  \[0.07*3.50\,\text{€/kg} = 0.245\,\text{€/kg}\]
\item Como es una subida, se lo sumamos al valor original:
  \[3.50\,\text{€/kg} + 0.245\,\text{€/kg} = 3.745\,\text{€/kg cuestan ahora.}\]
\end{enumerate}
\end{quote}

Así que hemos visto que no es excesivamente complicado de hacer. Pero si nos esforzamos un poco
podemos hacer que sea aún más sencillo.\\

Para ello vamos a utilizar variables en vez de números, que es la mejor manera de sacar conclusiones
válidas para cualquier caso:\\
Tenemos una cantidad inicial, que llamaremos $c_i$, que sufre una variación porcentual, que
llamaremos $i$, indice de variación. Este indice de variación será positivo o negativo según sea un aumento (como el ejemplo que acabamos de ver) o una disminución (como son unas rebajas).\\
Si llamamos $c_f$ a la cantidad final, podemos escribir las operaciones que hemos realizado en el
ejemplo de la siguiente manera:
\[c_f = c_i + i*c_i\]
Y si recordamos la técnica de sacar factor común nos queda:
\begin{Large}
  \[\boldsymbol{c_f = c_i*(1+i)}\]
\end{Large}
Que es la \textbf{relación} que vamos a utilizar en las \textbf{variaciones porcentuales} a partir de ahora.\\

Si utilizamos los datos del ejemplo nos queda:
\[c_f = 3.50\,\text{€/kg}*(1+0.07)\]
\[c_f = 3.50\,\text{€/kg}*1.07\]
\[c_f= 3.745\,\text{€/kg}\]
La suma se ha reducido a sumar $1$ al tanto por uno del indice de variación, con lo que le hemos quitado
parte de la complejidad. Y además hemos obtenido una relación que nos permite jugar con las variables para hacer cálculos más complejos.\\
Vamos a ver unos ejemplos:
\begin{questions}
\question Un pantalón que tiene una rebaja del $30\%$ cuesta $24.5$\,€.
  ¿Cuánto costaba antes de la rebaja?.
  \begin{solution}
    Lo importante es este tipo de problemas es identificar bien si la cantidad que nos dan es la
    cantidad inicial o la final.\\
    En este caso \textbf{el precio que nos dan es después de la rebaja}, así que tiene que ser la
    \textbf{cantidad final}, con lo que $c_f = 24.5\,\text{€}$.\\
    Además hay que tener en cuenta que \textbf{es una rebaja}, con lo que la variación es negativa:
    $i = -0.3$.\\
    Con esto, la relación de las variaciones porcentuales queda:
    \[24.5\,\text{€} = c_i*(1-0.3)\]
    \[24.5\,\text{€} = c_i*0.7\]
    El $0.7$ pasa al otro lado dividiendo y nos queda:
    \[\frac{24.5\,\text{€}}{0.7} = c_i\]
    \[c_i = 35\,\text{€}\]
    Con lo que los pantalones costaban $35$\,€ antes de la rebaja.
  \end{solution}
\question El beneficio de una empresa han pasado de $105\,300$\,€ el año pasado a $112\,107$\,€
  este año. Calcula cual ha sido el porcentaje de variación.
  \begin{solution}
    En este caso el tiempo nos dice cual es el inicial y cual es el final.\\
    \begin{itemize}
    \item La cantidad inicial será la del año pasado: $c_i = 105\,300$\,€.\\
    \item La cantidad final es el beneficio de este año: $c_f = 112\,107$\,€.
    \end{itemize}
    Con lo que la relación de las variaciones queda:
    \[112\,107\eur = 105\,300\eur*(1 + i)\]
    Si recordamos un poco de cómo se resuelven ecuaciones de primer grado con paréntesis sabemos
    que lo primero que tenemos que hacer es desarrollar el paréntesis:
    \[112\,107\eur = 105\,300\eur + 105\,300\eur*i\]

    Luego llevamos los términos con la incógnita (en este caso $i$) a un lado y lo que no tiene
    incógnita al otro, y lo que cambia de lado hace la operación contraria:
    \[112\,107\eur - 105\,300\eur = 105\,300\eur*i\]
    \[6\,807\eur= 105\,300\eur*i)\]
    Y finalmente despejamos la incógnita:
    \[\frac{6\,807\eur}{105\,300\eur} = i\]
    \[i \simeq 0.065\]
    Es decir, el beneficio ha subido (porque $i$ ha salido positiva) un $6.5\%$.
  \end{solution}
\end{questions}
\subsection{Variaciones porcentuales encadenadas. Indice de variación total.}
En el mundo actual son muchos los casos en los que las variaciones porcentuales de algo son
consecutivas.\\
Un ejemplo claro de ello es la variación de los precios, cada cierto tiempo las cosas suben o
bajan (rara vez) un porcentaje, con lo que a lo largo de un año han tenido unas cuantas variaciones
y a veces nos interesa estudiarlas todas juntas como una sola variación ya que hace que todo sera
más sencillo.\\

¿Qué hay que hacer para estudiarlo como una sola variación,
\sout{sumar las subidas y restar las bajadas}?\\
Si hacemos esto que está tachado está completamente mal. Vamos a verlo con un ejemplo sencillo:
\begin{quote}
  \emph{Un artículo que costaba $100\eur$ en enero, subió un $10\%$ en febrero y bajó
    un $10\%$ en marzo. ¿Cuánto cuesta en abril?}\\
  Es fácil pensar que si subió un $10\%$ y bajó otro $10\%$ uno anula al otro y al final se quedó
  como estaba, pero \textbf{no se queda como estaba}. Vamos a hacerlo paso a paso:
  \begin{enumerate}
  \item Primero subió el $10\%$, con lo que si costaba $100\eur$ subió $10\eur$ y entonces
    después de febrero cuesta $110\eur$.
  \item Empieza marzo costando $110\eur$ y baja un $10\%$, pero un $10\%$ de $110\eur$ son $11\eur$,
    con lo que al restárselo tenemos que $100\eur - 11\eur = 99\eur$.
  \end{enumerate}
  Es decir, \textbf{no se queda como estaba}, en realidad \textbf{todo junto hace una bajada del}
  $\boldsymbol{1\%}$.
\end{quote}
\subsubsection{Cálculo de variaciones encadenadas.}
El método para calcular variaciones porcentuales encadenadas es el siguiente:
\begin{itemize}
\item Llamaremos $c_i$ a la cantidad inicial, antes de cualquier variación.
\item Llamaremos $i_1$ a la primera variación, $i_2$ a la segunda, $i_3$ a la tercera, \dots
\end{itemize}
Entonces la relación entre la cantidad final y la inicial después de todas las variaciones es:
\begin{Large}
  \[\boldsymbol{c_f = c_i *(1+ i_1)*(1+i_2)*(1+i_3) \cdots (1+i_n)}\]
\end{Large}
Si aplicamos esto al ejemplo tenemos:
\begin{itemize}
\item La cantidad inicial es $c_i=100\eur$.
\item La primera variación es una subida del $10\%$: $i_1 = 0.1$.
\item La segunda variación es una bajada del $10\%$: $i_2 = -0.1$.
\end{itemize}
De manera que la relación queda:
\[c_f = 100\eur*(1 + 0.1)*(1-0.1)\]
\[c_f = 100\eur*1.1*0.9\]
\[c_f= 99\eur\]
Que es lo que habíamos obtenido al hacer paso a paso el cálculo.
\subsubsection{Indice de variación total. Cálculo.}
Al principio del punto de las variaciones porcentuales encadenadas hemos dicho que todo esto
tenía el objetivo de tratar todas las variaciones juntas como si se tratasen de una sola.\\
Para eso tenemos que definir y calcular el \emph{indice de variación total} ($i_t$).\\

Este indice es el que nos va a permitir calcular la cantidad final a partir de la inicial con una
sola variación:
\[c_f = c_i *(1 + i_t)\]
Y tenemos que para calcularlo con todas las variaciones hay que usar:
\[c_f = c_i *(1+ i_1)*(1+i_2)*(1+i_3) \cdots (1+i_n)\]
Como las $c_f$ de las dos relaciones son iguales y lo mismo sucede con las $c_i$ tendrá que
ocurrir que:
\[1 + i_t = (1+ i_1)*(1+i_2)*(1+i_3) \cdots (1+i_n)\]
Con lo que
\begin{Large}
  \[\boldsymbol{i_t = (1+ i_1)*(1+i_2)\cdots (1+i_n) -1}\]
\end{Large}
También se puede utilizar, porque es equivalente:
\begin{Large}
  \[\boldsymbol{i_t = \frac{c_f}{c_i} - 1}\]
\end{Large}

Si aplicamos las dos fórmulas al artículo del ejemplo anterior que costaba $100\eur$ tenemos que
\begin{itemize}
\item Con la primera fórmula:
  \[i_t = (1+0.1)*(1-0.1)-1\]
  \[i_t = 1.1*0.9-1\]
  \[i_t = 0.99 - 1\]
  \[i_t = -0.01\]
  Es decir, el artículo baja (porque nos ha salido negativo) un $1\%$.
\item Con la segunda fórmula:
  \[i_t = \frac{99\eur}{100\eur} -1\]
  \[i_t = 0.99 - 1\]
  \[i_t = -0.01\]
  Tenía que salirnos lo mismo porque hemos dicho que es equivalente.
\end{itemize}
El usar una u otra dependerá del contexto en el que nos encontremos y los datos de los que
dispongamos.
\end{document}




