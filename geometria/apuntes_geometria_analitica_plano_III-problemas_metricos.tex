\documentclass[a4paper,11pt,answers]{exam}

\usepackage{hyperref}
\usepackage{graphicx}
%\usepackage{pstricks}
\usepackage[utf8]{inputenc}
\usepackage[spanish]{babel}
\usepackage[T1]{fontenc}
%textcomp es para el símbolo del euro
\usepackage{lmodern, textcomp}

\usepackage[left=1in, right=1in, top=1in, bottom=1in]{geometry}
%\usepackage{mathexam}
\usepackage{amsmath}
\usepackage{amssymb}
\usepackage{multicol}
\usepackage{longtable}
%para la última página
%\usepackage{lastpage}

%Para padding en celdas
\usepackage{cellspace}
\setlength\cellspacetoplimit{1mm}
\setlength\cellspacebottomlimit{1mm}

%Para hacer tachados
\usepackage[makeroom]{cancel}

%Creative commons
%\usepackage{ccicons}
\usepackage[type={CC}, modifier={by-nc-sa}, version={4.0}, %imagemodifier={-eu-80x25},
lang={spanish}]{doclicense}

%Para las gráficas:
\usepackage{tikz}
\usepackage{pgfplots}
\pgfplotsset{compat = newest}
\usetikzlibrary{babel} %Si no da errores con algunas cosas al compilar los gráficos.
\usetikzlibrary{arrows.meta,shapes,positioning}
\usetikzlibrary{matrix}
\usepgfplotslibrary{fillbetween}
\usetikzlibrary{arrows.meta}
\usetikzlibrary{fit}
\usetikzlibrary{quotes,angles}
%\usepackage{nicematrix}

\usepackage{color,colortbl}
\definecolor{Gray}{gray}{0.9}
\newcolumntype{g}{>{\columncolor{Gray}}c}
\usepackage{arydshln} %Este pone la línea punteada en la matriz ampliada. TIENE QUE ESTAR DESPUÉS DEL colortbl porque si no casca.
%\pagestyle{headandfoot}
\pagestyle{headandfoot}
\newcommand\ExamNameLine{
\par
\vspace{\baselineskip}
Nombre:\hrulefill\relax
\par}

\renewcommand{\solutiontitle}{\noindent\textbf{Solución:}\par\noindent}

\everymath{\displaystyle}
\newcommand\ddfrac[2]{\frac{\displaystyle #1}{\displaystyle #2}}

\def \autor{Paco Andrés}
\def \titulo{Apuntes de geometría analítica en el plano III.\\Producto escalar.
Problemas métricos.}
\def \titulofichas {\textbf {\titulo}}
\def \cursofichas {}
\def \fechaexamen {}
%\firstpageheader{\cursofichas}{\titulofichas}{\fechaexamen}
\header{\cursofichas}{\begin{small}
\titulofichas
\end{small}}{\fechaexamen}
%\header{\cursofichas}{\titulofichas}{\fechaexamen}
%\firtspagefooter{}{\thepage}{}
%Por alguna razón no sale lo del cc en el pie
\firstpagefootrule
\footrule
\footer{\autor}{\thepage}{\doclicenseIcon}
\pointpoints{punto}{puntos}

\shadedsolutions
%\definecolor{SolutionColor}{rgb}{0.99,0.99,.99}
\renewcommand{\baselinestretch}{1.3}

%Use * instead of \cdot
\mathcode`\*="8000
{\catcode`\*\active\gdef*{\cdot}} 
\newcommand{\Card}{\,\mathrm{Card}}

%For e number
\newcommand{\e}{\,\mathrm{e}}

%Para trigonometría
\newcommand{\asen}{\,\mathrm{asen}\,}
\newcommand{\acos}{\,\mathrm{acos}\,}
\newcommand{\atg}{\,\mathrm{atg}\,}
\newcommand{\degree}{^\circ}
%Para el diferencial y la integral:
\newcommand\dif[1]{\mathrm{d}#1}
\newcommand\integral[2]{\int #1\,\dif{#2}}
\newcommand\integrald[4]{\int_{#3}^{#4} #1\,\dif{#2}}
\newcommand\adjunto[1]{#1^\text{*}}
\newcommand\rango[1]{\mathrm{rg}(#1)}

%Geometría:
\newcommand\vectort[3]{#1*\vec i + #2*\vec j + #3*\vec k}
\newcommand\distancia[2]{\text{d}(#1,#2)}
%Para escribir explicaciones encima del igual:
%\newcommand\igexpl[1]{{\mathrel{\overset{\makebox{\mbox{\normalfont\tiny\sffamily $#1$}}}{=}}}}
%Parece que mejor con stackrel

%Para las unidades:
\newcommand{\unidad}[1]{\,\text{#1}}


\renewcommand{\questionlabel}{\textbf{Ejemplo \thequestion:}}

%Colores
\definecolor{gridgray}{gray}{0.7}
\pgfplotsset{grid style={color=gridgray}}
\begin{document}


%\author{Paco Andrés}
\title{\titulo}
\date{}
\author{\autor}
\maketitle

\begin{center}
\doclicenseLongText\\
\vspace{.25cm}
\doclicenseImage
\end{center}
\tableofcontents
\newpage
\setlength{\parindent}{0cm}

\section{Producto escalar.}
El producto escalar de dos vectores, $\vec{v}$ y $\vec{w}$, se define de la siguiente manera:
\[\boldsymbol{\vec{v}*\vec{w} = |\vec{v}|*|\vec{w}|*\cos \alpha}\]
Donde $\alpha$ es el ángulo que forman los dos vectores.
\begin{center}
  \begin{tikzpicture}
    \coordinate (O) at (0,0);
    \coordinate (V) at (2,1);
    \coordinate (W) at (1,3);
    \draw[-latex] (O)--(V) node[midway, sloped,below] {$\vec{v}$};
    \draw[-latex] (O)--(W) node[midway, sloped,above] {$\vec{w}$};
    \pic["$\alpha$", draw=darkgray, angle eccentricity=.7, angle radius=.7cm]
    {angle=V--O--W};
  \end{tikzpicture}
\end{center}
\subsection{Propiedades del producto escalar.}\label{propiedades}
El producto escalar de dos vectores tiene las siguientes propiedades:
\begin{itemize}
\item Es conmutativo: $\vec{v}*\vec{w} = \vec{w}*\vec{v}$
\item Propiedad asociativa con respecto al producto por un escalar: $(a*\vec{v})*\vec{w} =
  a*(\vec{v}*\vec{w})$
\item Es fácil comprobar que $\vec{v}*\vec{v} = |\vec{v}|^2$
\item Cumple la propiedad distributiva respecto de la suma: $\vec{u}*(\vec{v}+\vec{w}) =
  \vec{u}*\vec{v} + \vec{u}*\vec{w}$.
\item Si $\vec{v}\perp\vec{w}$ tenemos que $\cos \frac{\pi}{2}= 0$ con lo que $\vec{v}*\vec{w} = 0$.\\
  Es decir, si $\vec{v}*\vec{w} = 0$ y ninguno de los dos es el vector nulo entonces son
  perpendiculares.
\end{itemize}

\subsection{Producto escalar por componentes.}
Tal y como está definido el producto escalar tiene pinta de que es un poco complicado calcularlo
ya que, además de los módulos de los vectores, necesitamos conocer el ángulo que forma y eso no
tiene pinta de ser sencillo.\\

Además tenemos que tener en cuenta que normalmente solemos operar con los vectores utilizando sus
componentes en lugar de sus módulos.\\

Si recordamos lo que vimos sobre las bases tenemos qué:
\[\boldsymbol{\vec{v} = (v_x, v_y) = v_x*\vec{u}_x + v_y*\vec{u}_y}\]
Donde $\vec{u}_x$ y $\vec{u}_y$ son la base ortonormal del plano (ortonormal quiere decir perpendicular y de
módulo 1):
\begin{itemize}
\item $\vec{u}_x = (1,0)$
\item $\vec{u}_y = (0,1)$
\end{itemize}
Y es fácil comprobar que el producto escalar con la base ortonormal cumple:
\begin{itemize}
\item $\vec{u}_x*\vec{u}_x = |\vec{u}_x|*|\vec{u}_x|*\cos 0 = 1$
\item $\vec{u}_y*\vec{u}_y = 1$
\item $\vec{u}_x*\vec{u}_y = |\vec{u}_x|*|\vec{u}_y|*\cos \frac{\pi}{2} = 0$
\end{itemize}

De esta manera tenemos que los vectores $\vec{v}$ y $\vec{w}$ se pueden escribir como:
\begin{itemize}
\item $\vec{v} = v_x*\vec{u}_x + v_y*\vec{u}_y$
\item $\vec{w} = w_x*\vec{u}_x + w_y*\vec{u}_y$
\end{itemize}
De manera que
\[\vec{v}*\vec{w} = (v_x*\vec{u}_x + v_y*\vec{u}_y)*(w_x*\vec{u}_x + w_y*\vec{u}_y)\]
que desarrollado se convierte en:
\[\vec{v}*\vec{w} = v_x*w_x*\overbrace{\vec{u}_x*\vec{u}_x}^{\boldsymbol{1}} +
  v_x*w_y*\overbrace{\vec{u}_x*\vec{u}_y}^{\boldsymbol{0}} +
  v_y*w_x*\overbrace{\vec{u}_y*\vec{u}_x}^{\boldsymbol{0}} +
  v_y*w_y*\overbrace{\vec{u}_y*\vec{u}_y}^{\boldsymbol{1}}\]
Y simplificando nos queda que el producto escalar por componentes se calcula así:
\[\boldsymbol{\vec{v}*\vec{w} = v_x*w_x + v_y*w_y}\]\label{por_componentes}

\subsection{Interpretación geométrica del producto escalar.} \label{interpretacion_producto_escalar}
Tenemos dos vectores $\vec{v}$ y $\vec{w}$ de la siguiente manera:
\begin{center}
  \begin{tikzpicture}
    \coordinate (O) at (0,0);
    \coordinate (V) at (5,0);
    \coordinate (W) at (2,2);
    \coordinate (P) at (2,0);
    \draw[-latex] (O)--(V) node[midway, below] {$\vec{v}$};
    \draw[-latex] (O)--(W) node[midway, above, sloped] {$\vec{w}$};
    \draw[ultra thick] (O)--(P) node[midway, above] {$p$};
    \draw[dashed] (P)--(W);
    \pic["$\alpha$", draw=darkgray, angle eccentricity=.7, angle radius=.7cm]
    {angle=V--O--W};
  \end{tikzpicture}
\end{center}
Se ve que podemos construir un triángulo rectángulo proyectando el extremo de $\vec{w}$ sobre
$\vec{v}$ (para proyectar se traza una perpendicular, con lo que ya tenemos el ángulo recto).
Por lo que recordamos de trigonometría tenemos que el tamaño de la proyección $p$ es:
\[\boldsymbol{p = |\vec{w}|*\cos \alpha}\]

Si atendemos a la definición de producto escalar tenemos:
\[\vec{v}*\vec{w} = |\vec{v}|*\overbrace{|\vec{w}|*\cos \alpha}^{p} = |\vec{v}|*p\]
Como $p$ es la proyección de $\vec{w}$ sobre $\vec{v}$ se llega a la conclusión de que \emph{
  \textbf{el producto escalar de dos vectores coincide con la proyección de un vector sobre la dirección del otro
    multiplicada por el módulo del otro}}.\\


\section{Problemas métricos.}
\subsection{Problemas métricos con vectores.}
En este punto vamos a ver cómo resolver una serie de problemas utilizando todo lo que hemos visto
hasta ahora de geometría analítica, que comprende:
\begin{itemize}
\item Vectores y operaciones.
\item Rectas, propiedades y posiciones relativas.
\item Producto escalar y su interpretación geométrica.
\end{itemize}
Para resolver algunos de ellos vamos a tener que utilizar conceptos que corresponden a varios de los
puntos anteriores y, por supuesto, de trigonometría ya que es la base de todos ellos.\\

Empezamos:
\begin{questions}
\question Dados los vectores $\vec{a}=(-3,4)$ y $\vec{b}=(5,-1)$, calcula:
  \begin{parts}
  \part El producto escalar $\vec{a}*\vec{b}$.
  \part $|\vec{a}|$.
  \part $|\vec{b}|$.
  \part El angulo que forman los vectores $\vec{a}$ y $\vec{b}$.
  \end{parts}
  \begin{solution}
    Es un ejercicio básico. Está desglosado en apartados pero la idea es aprender a calcular el
    ángulo que forman dos vectores y los apartados están dispuestos en el orden en el que hay que
    dar los pasos para llegar a calcular el ángulo.
    \begin{parts}
    \part Para calcular el producto escalar utilizamos el método que hemos visto para calcularlo
      por componentes en el punto \ref{por_componentes} (página \pageref{por_componentes}).
      \[\vec{a}*\vec{b} = (-3,4)*(5,-1) = -3*5 + 4*(-1) = -15-4 = -19\]
    \part El módulo de $\vec{a}$ es $|\vec{a}| = \sqrt{(-3)^2 + 4^2} = \sqrt{25} = 5$
    \part El módulo de $\vec{b}$ es $|\vec{b}| = \sqrt{5^2 + (-1)^2} = \sqrt{26}$
    \part Con esto ya podemos calcular el coseno del ángulo que forman, al que llamaremos $\alpha$,
      ya partiendo de la definición de producto escalar
      \[\vec{a}*\vec{b} = |\vec{a}|*|\vec{b}|*\cos \alpha\]
      \[\cos \alpha = \frac{\vec{a}*\vec{b}}{|\vec{a}|*|\vec{b}|}\]
      Sustituyendo:
      \[\cos \alpha = \frac{-19}{5*\sqrt{26}}\simeq -0.7452\]
      Y para calcular el ángulo solo tenemos que utilizar el arco de coseno en la calculadora:
      \[\boldsymbol{\alpha = \acos -0.7452 \simeq 138.176\degree}\]
      Aunque ya tenemos que empezar a acostumbrarnos a usar siempre radianes, de manera que
      \[\alpha \simeq 138.176\degree \simeq 2.412\quad\text{\small{(aprovechamos para recordar
            que los radianes no tienen unidad)}}\]

      También sabemos que para el arco de coseno podemos tener dos soluciones, la que nos da la calculadora que
      llamaremos $\boldsymbol{\alpha}$, y $\boldsymbol{2\pi - \alpha}$. Que sea uno u otro depende del sentido
      en el que midamos el ángulo, tal y como se puede apreciar en la siguiente representación gráfica:
      \begin{center}
        \begin{tikzpicture}
          \coordinate (O) at (0,0);
          \coordinate (V) at (2,1);
          \coordinate (W) at (1,3);
          \draw[-latex] (O)--(V) node[midway, sloped,below] {$\vec{a}$};
          \draw[-latex] (O)--(W) node[midway, sloped,above] {$\vec{b}$};
          \pic["$\alpha$", draw=darkgray, angle eccentricity=.7, angle radius=.7cm]
          {angle=V--O--W};
          \pic["$2\pi - \alpha$", draw=darkgray, angle eccentricity=1.7, angle radius=.5cm]
          {angle=W--O--V};
        \end{tikzpicture}
      \end{center}
    \end{parts}
  \end{solution}
\question Dado el vector $\vec{v} = (-5, 3)$ calcula los vectores que cumplan cada una de las condiciones siguientes:
  \begin{parts}
  \part Unitarios y de la misma dirección que $\vec{v}$.
  \part Ortogonales a $\vec{v}$ y con su mismo módulo.
  \part Ortonormales a $\vec{v}$.
  \end{parts}
  \begin{solution}
    Este ejercicio es un breve repaso de otros que ya hemos hecho y nos va a servir para repasar
    algunos conceptos y mecanismos.
    \begin{parts}
    \part Un vector unitario es el que tiene módulo 1, y para calcular el vector unitario en la
      dirección y sentido de uno dado lo que tenemos que hacer es dividir el vector entre su
      módulo:
      \[\vec{u} = \frac{\vec{v}}{|\vec{v}|}\]
      
      Entonces lo primero es calcular el módulo:
      \[|\vec{v}| = \sqrt{(-5)^2 + 3^2} = \sqrt{34}\]

      Con lo que el vector unitario con la dirección y el sentido de $\vec{v}$ es:
      \[\vec{u} = \left(\frac{-5}{\sqrt{34}}, \frac{3}{\sqrt{34}}\right)\]

      El otro vector que cumple las condiciones de este apartado es el vector
      \[-\vec{u} = \left(\frac{5}{\sqrt{34}}, \frac{-3}{\sqrt{34}}\right)\]
    \part Recordemos que ortogonal es lo mismo que perpendicular.\\
      En su momento vimos que si tenemos un vector $\vec{v} = (v_x, v_y)$ los vectores
      $\vec{w}_1 = (-v_y, v_x)$ y $\vec{w}_2 = (v_y, -v_x)$ son perpendiculares a $\vec{v}$ (es
      fácil comprobarlo con el producto escalar.\\
      Además estos vectores tienen el mismo módulo que $\vec{v}$ ya que tiene las mismas componentes
      (aunque en distinto orden y con distinto signo, pero al ir elevadas al cuadrado no importa).

      De esta manera los vectores pedidos en este apartado son:
      \begin{itemize}
      \item $\vec{w}_1 = (3, 5)$
      \item $\vec{w}_2 = (-3,-5)$
      \end{itemize}
      \begin{center}
        \begin{tikzpicture}[scale=.5]
          \coordinate (O) at (0,0);
          \coordinate (V) at (-5,3);
          \coordinate (W1) at (3,5);
          \coordinate (W2) at (-3,-5);
          \draw[-latex] (O)--(V) node[midway, sloped, above] {$\vec{v}(-5,3)$};
          \draw[-latex] (O)--(W1) node[midway, sloped,above] {$\vec{w}_1(3,5)$};
          \draw[-latex] (O)--(W2) node[midway, sloped,above] {$\vec{w}_2(-3,-5)$};
        \end{tikzpicture}
      \end{center}
    \part Recordemos que ortonormal significa ``ortogonal y de módulo 1''.\\
      Entonces solo tenemos que dividir los vectores del apartado anterior entre su módulo, y
      como éste es el mismo que el de $\vec{v}$ ya lo tenemos del primer apartado, con lo que si
      los llamamos $\vec{n}_1$ y $\vec{n}_2$ sus componentes son:
      \begin{itemize}
      \item $\vec{n}_1 = \left(\frac{3}{\sqrt{34}}, \frac{5}{\sqrt{34}}\right)$
      \item $\vec{n}_2 = \left(\frac{-3}{\sqrt{34}}, \frac{-5}{\sqrt{34}}\right)$
      \end{itemize}
    \end{parts}
  \end{solution}
\question Tenemos dos vectores $\vec{a}$ y $\vec{b}$ de los que sabemos que:
  \begin{itemize}
  \item $|\vec{a}| = 3$.
  \item $|\vec{b}| = 2$.
  \item Forman un ángulo de $30\degree$.
  \end{itemize}
  Calcula cuanto valen $|\vec{a}+\vec{b}|$ y $|\vec{a}-\vec{b}|$.
  \begin{solution}
    En las propiedades del producto escalar (apartado \ref{propiedades}, página \pageref{propiedades})
    vimos que $\vec{v}*\vec{v} = |\vec{v}|^2$, con lo que
    \[|\vec{a}+\vec{b}|^2 = (\vec{a}+\vec{b})*(\vec{a}+\vec{b}) =
      \vec{a}*\vec{a} + \vec{b}*\vec{b}+2\vec{a}*\vec{b} =
      |\vec{a}|^2 + |\vec{b}|^2 + 2|\vec{a}|*|\vec{b}|*\cos \alpha\]
    Y sustituyendo obtenemos que
    \[|\vec{a}+\vec{b}|^2 = 3^2 + 2^2 + 2*3*2*\cos 30\degree = 13+6\sqrt{3} \simeq 23.39\]
    \[|\vec{a}+\vec{b}| \simeq \sqrt{23.39} \simeq 4,84\]

    Haciendo lo mismo para la el otro módulo:
    \[|\vec{a}-\vec{b}|^2 = 3^2 + 2^2 - 2*3*2*\cos 30\degree = 13-6\sqrt{3} \simeq 1.732\]
    \[|\vec{a}-\vec{b}| \simeq \sqrt{1.732} \simeq 1,316\]
  \end{solution}
\question Dados los vectores $\vec{v} = (-2,-1)$ y $\vec{w}=(2,-2)$ calcula cuanto vale la proyección de
  $\vec{v}$ sobre la dirección de $\vec{w}$.
  \begin{solution}
    Tal y como vimos en el punto \ref{interpretacion_producto_escalar} (página \pageref{interpretacion_producto_escalar}) el producto escalar de dos vectores se puede interpretar como la proyección de uno de ellos sobre la
    dirección del otro multiplicada por el módulo del otro.\\
    Es decir, si llamamos $p_w$ a la proyección de $\vec{v}$ sobre la dirección de $\vec{w}$
    \begin{center}
      \begin{tikzpicture}
        \coordinate (O) at (0,0);
        \coordinate (V) at (5,0);
        \coordinate (W) at (2,2);
        \coordinate (P) at (2,0);
        \draw[-latex] (O)--(V) node[midway, below] {$\vec{v}$};
        \draw[-latex] (O)--(W) node[midway, above, sloped] {$\vec{w}$};
        \draw[ultra thick] (O)--(P) node[midway, above] {$p_w$};
        \draw[dashed] (P)--(W);
        \pic["$\alpha$", draw=darkgray, angle eccentricity=.7, angle radius=.7cm]
        {angle=V--O--W};
      \end{tikzpicture}
    \end{center}
    tenemos que según lo dicho ocurre que:
    \[\vec{v}*\vec{w} = p_w*|\vec{w}|\]
    Con lo que:
    \[p_w = \frac{\vec{v}*\vec{w}}{|\vec{w}|}\]

    Así que calculamos el producto escalar:
    \[\vec{v}*\vec{w} = -2*2+(-1)*(-2) = -2\]
    El módulo de $\vec{w}$
    \[|\vec{w}| = \sqrt{2^2 + (-2)^2} = \sqrt{8} = 2\sqrt{2}\]

    Con lo que la proyección es:
    \[p_w = \frac{-2}{2\sqrt{2}} = -\frac{\sqrt{2}}{2}\]

    Si lo que queremos es el tamaño de la proyección entonces tenemos que quitar el signo.
  \end{solution}
\question Calcula el valor de $x$ para que los vectores $\vec{a} = (2, 5)$ y $\vec{b} = (x, 3)$ sean perpendiculares.
  \begin{solution}
    Sabemos que si dos vectores son perpendiculares su producto escalar vale cero (punto \ref{propiedades}, página
    \pageref{propiedades}), con lo que:
    \[\vec{a}*\vec{b} = 2x + 15 = 0\]
    Y resolviendo la ecuación queda que $x = -\frac{15}{2}$.
  \end{solution}
\question Calcula un vector ortonormal al vector $\vec{v} = (-3, 4)$.
  \begin{solution}
    Por lo que hemos visto anteriormente un vector \emph{ortonormal} es el que tiene módulo 1 y es perpendicular
    a la vez.\\

    Sabemos que para conseguir un vector perpendicular a otro solo tenemos que intercambiar las componentes
    y cambiar el signo a una de ellas. Por ejemplo, el vector
    \[\vec{w} = (4,3)\]
    es perpendicular a $\vec{v}$. El problema que tiene es que su módulo no es 1.\\

    Ahora tenemos que utilizar el cálculo del vector unitario, y:
    \[vec{n} = \frac{\vec{w}}{|\vec{w}|} = \frac{(4,3)}{\sqrt{4^2 + 3^2}} = \left(\frac{4}{5}, \frac{3}{5}\right)\]
    ya cumple las dos cosas, con lo que ese es un vector ortonormal a $\vec{v}$.
  \end{solution}
\end{questions}
\subsection{Problemas métricos con puntos y rectas.}
Antes de empezar a resolver problemas que involucran puntos y rectas vamos a introducir un par de fórmulas que
van a ser de mucha utilidad. Y dado el nivel al que nos encontramos vamos a deducir las fórmulas, ya que a lo largo
de estos apuntes hemos visto toda la teoría para poder hacerlo.
\subsubsection{Distancia entre punto y recta.}
Si tenemos un punto $P$ y una recta $r$ podemos tener muchas distancias, tal y como indica el dibujo con $d_1$,
$d_2$, \dots
\begin{center}
  \begin{tikzpicture}
    \coordinate (A) at (-2, -2);
    \coordinate (B) at (2, 2);
    \coordinate (P) at (-1,3);
    \draw (A)--(B) node[right] {$r$};
    \draw[fill=black] (P) circle(2pt) node[above] {$P$};
    \draw[latex-latex, dashed, color=gray] (P)--(-1.5,-1.5) node[midway, above left, sloped] {$d_1$};
    \draw[latex-latex, dashed, color=gray] (P)--(-0.3,-0.3) node[midway, above right, sloped] {$d_2$};
    \draw[latex-latex] (P)--(1,1) node[midway, below right, sloped] {$\boldsymbol{d}$};
    \draw[latex-latex, dashed, color=gray] (P)--(1.7,1.7) node[midway, above right, sloped] {$d_3$};
  \end{tikzpicture}
\end{center}
Pero, tal y como hemos marcado en el dibujo, vamos a definir la \textbf{distancia de un punto a una recta} como \textbf{\emph{la distancia que está medida sobre la perpendicular a la recta que pasa por el punto}}.\\

Una vez definida cual es la distancia que tenemos que utilizar vamos a empezar a razonar con las herramientas que
tenemos, y vamos a calcular la distancia entre:
\begin{itemize}
\item $P(P_x, P_y)$
\item $r:\,Ax + By + C = 0$
\end{itemize}
Como acabamos de decir que la distancia se mide en la dirección perpendicular a la recta necesitamos un vector en
esa dirección, y este vector es:
\[\vec{n} = (A, B)\quad \text{\small(Si no lo recordamos tenemos que repasar los apuntes de rectas)}\]

Con esto tenemos:
\begin{center}
  \begin{tikzpicture}
    \coordinate (A) at (-2, -2);
    \coordinate (B) at (2, 2);
    \coordinate (P) at (-1,3);
    \coordinate (R) at (-1.5, -1.5);
    \draw (A)--(B) node[right] {$r$};
    \draw[fill=black] (P) circle(2pt) node[above] {$P$};
    \draw[fill=black] (R) circle(2pt) node[below, right] {$R(R_x,R_y)$};
    \draw (P)--(1,1) node[midway, below right, sloped] {$\boldsymbol{d}$};
    \draw[-Latex] (R)--(P) node[midway, sloped, above left] {$\overrightarrow{RP}$};
    \draw[-Latex] (1.2,1.2)--(0.2,2.2) node[midway, sloped, above right] {$\vec{n}$};
    \node at (-2, -2.5) {\scriptsize{El punto $R$ es un punto cualquiera de la recta}};
  \end{tikzpicture}
\end{center}
Y del dibujo se puede deducir que la distancia a la recta es la proyección de $\overrightarrow{RP}$ sobre la
dirección perpendicular a la recta.\\

Si $R$ es un punto de la recta lo podemos obtener dando valor a una de las coordenadas,
por ejemplo haciendo $x=0$ (podemos darle cualquier otro valor, pero este es el más sencillo):
\[A*0 + B*y + C = 0\]
\[y = -\frac{C}{B}\]
Con lo que $R\left(0, -\frac{C}{B}\right)$ es el punto de la recta que vamos a utilizar.\\

Por lo que hemos visto, la proyección sobre la dirección perpendicular se va a calcular como:
\[p=\frac{\overrightarrow{RP}*\vec{n}}{|\vec{n}|}\]
Sustituyendo y desarrollado:
\[p=\frac{\left(P_x, P_y + \frac{C}{B}\right)*(A,B)}{\sqrt{A^2 + B^2}}\]
\[p = \frac{P_x*A + P_y * B + \frac{C}{B}*B}{\sqrt{A^2 + B^2}}\]
\[p = \frac{A*P_x + B*P_y + C}{\sqrt{A^2 + B^2}}\]
Y la distancia sería el tamaño de esta proyección, para lo cual tenemos que poner un valor absoluto.\\

De manera que la distancia entre el punto $P(P_x, Py)$ y la recta $r:\,Ax + By + C = 0$ se calcula con la
fórmula:
\[\boldsymbol{d(P, r) = \frac{|A*P_x + B*P_y + C|}{\sqrt{A^2 + B^2}}}\]
Y ésta es la fórmula que vamos a utilizar para resolver los problemas.

\subsubsection{Distancia entre dos rectas.}
Partiendo de la definición que hemos dado para la distancia entre un punto y una recta, el
concepto de \textbf{distancia entre dos rectas solo tiene sentido si las dos rectas son paralelas}.\\
\emph{En ese caso la distancia va a ser la medida sobre la perpendicular común a las dos rectas.}
\begin{center}
  \begin{tikzpicture}
    \coordinate (A) at (-2, -2);
    \coordinate (B) at (2, 2);
    \coordinate (P) at (-2,2);
    \coordinate (Q) at (0,0);
    \draw (A)--(B) node[right] {$r$};
    %\draw[fill=black] (P) circle(2pt) node[above] {$P$};
    \draw[latex-latex] (P)--(Q) node[midway, below right, sloped] {$\boldsymbol{d}$};
    %\draw[fill=black] (Q) circle(2pt) node[below] {$Q$};
    \coordinate (A1) at (-4, 0);
    \coordinate (B1) at (0, 4);
    \draw (A1)--(B1) node[right] {$s$};
  \end{tikzpicture}
\end{center}

Reutilizando el resultado del apartado anterior, podemos calcular la distancia entre las rectas $r$
y $s$ calculando la distancia de $r$ a un punto cualquiera de $s$.\\

Partamos del las ecuaciones de las dos rectas:
\begin{itemize}
\item $r:\, Ax+By+C=0$
\item $s:\, Ax+By+C'=0$ (como son paralelas $A$ y $B$ tienen que ser iguales si ambas ecuaciones
  están lo más simplificadas posible).
\end{itemize}
Entonces vamos a calcular un punto $P$ de $s$ y después calcular su distancia a $r$.\\
Para calcular $P$ hacemos $x=0$ en $s$ (vale cualquier otro valor, pero este es el más sencillo).
\[A*0 + B*y + C' = 0\]
Con lo que
\[y = -\frac{C'}{B}\]
Con lo que el punto es $P\left(0, -\frac{C'}{B}\right)$.\\
Sustituimos en la fórmula de la distancia:
\[d(r,s) = \frac{\left|A*0 + B*\left(-\frac{C'}{B}\right) + C\right|}{\sqrt{A^2 + B^2}}\]
Y simplificando queda:
\[\boldsymbol{d(r,s) = \frac{|C - C'|}{\sqrt{A^2 + B^2}}}\]

\emph{\textbf{Lógicamente, para resolver problemas hace falta saberse de memoria esta fórmula
    y la de la distancia de un punto a una recta.}}\\

Una vez vistas estas dos fórmulas vamos a ver unos cuantos ejemplos de resolución de problemas
en los que se ven implicados puntos y rectas.
\subsubsection{Ejercicios de ejemplo.}
\begin{questions}
\question Calcula la distancia del punto $P(3,7)$ y la recta $r:\,2x-y + 3 = 0$.
  \begin{solution}
    En primer lugar obtenemos el vector perpendicular a la recta, que es:
    \[\vec{n} = (A, B) = (2, -1)\]
    Y ahora aplicamos la fórmula que hemos visto:
    \[d(P, r) = \frac{|2*3 -1*7 + 3|}{\sqrt{2^2 + (-1)^2}}\]
    \[d(P, r) = \frac{2}{\sqrt{5}} = \frac{2\sqrt{5}}{5}\]
    Y con esto ya habríamos resuelto el problema.
  \end{solution}
\question Calcula la distancia entre las rectas $r:\,3x +2y -5 = 0$ y
  $s:\,
  \begin{cases}
    x =& 1+2t\\
    y =& 2-3t
  \end{cases}$
  \begin{solution}
    En primer lugar \emph{tenemos que escribir las dos rectas con su ecuación implícita}, ya que la
    fórmula solo funciona si tenemos las dos rectas escritas de esa forma.\\
    La recta $r$ ya está en forma implícita.\\
    Para la recta $s$ primero pasamos de la paramétrica a la continua:
    \[s:\, \frac{x-1}{2} = \frac{y - 2}{-3}\]
    Eliminamos denominadores:
    \[s:\, -3(x-1) = 2(y - 2)\]
    Desarrollamos y lo llevamos todo al lado izquierdo del la ecuación:
    \[s:\, -3x - 2y +7 = 0\]
    Vemos que los parámetros $A$ y $B$ son de signo contrario a los de $r$, con lo que cambiamos de
    signo toda la ecuación:
    \[s:\, 3x + 2y -7 = 0\]

    Ahora deberíamos comprobar si son paralelas, pero como los parámetros $A$ y $B$ son iguales en
    las dos rectas está claro que son paralelas.\\
    Aplicamos la fórmula de la distancia entre dos rectas:
    \[d(r,s) = \frac{|-5-(-7)|}{\sqrt{2^2 + 3^2}}\]
    \[d(r,s) = \frac{2}{\sqrt{13}} = \frac{2\sqrt{13}}{13}\]

    Esa es la distancia entre las dos rectas.
  \end{solution}
\question Calcula la distancia entre las rectas $r:\,2x - y + 3 = 0$ y $s:\,4x - 2y + 5 = 0$.
  \begin{solution}
    En este caso las dos rectas están en su forma implícita, no tenemos que hacer ninguna
    transformación en ninguna de ellas.\\
    Tenemos que comprobar si son paralelas, y lo vamos a hacer a través de sus vectores
    perpendiculares, ya que si son paralelas sus vectores perpendiculares también tienen que serlo.\\

    Hemos visto que el obtener un vector perpendicular a partir de la ecuación implícita es
    sencillo, siendo $\vec{n} = (A,B)$, con lo que los vectores perpendiculares a cada recta
    son:
    \begin{itemize}
    \item $\vec{n_r} = (2, -1)$.
    \item $\vec{n_s} = (4, -2)$.
    \end{itemize}
    Comprobamos que son proporcionales para saber si son paralelos:
    \[\frac{4}{2} = \frac{-2}{-1}\]
    y resulta que lo son, con lo que $\boldsymbol{r \parallel s}$ y podemos calcular la distancia
    porque tiene sentido.\\

    Ahora se nos presenta un problema, que tal y como hemos deducido la fórmula necesitamos que los
    coeficientes de $x$ e $y$ sean iguales en las dos rectas, y eso es algo que ahora mismo no
    está pasando.
    \begin{itemize}
    \item $r:\,2x - y + 3 = 0$
    \item $s:\,4x - 2y + 5 = 0$
    \end{itemize}
    Pero es fácil ver que si dividimos la ecuación de $s$ entre $2$ vamos a tener los mismos
    coeficientes en $x$ y en $y$ en ambas ecuaciones.
    \begin{itemize}
    \item $r:\,2x - y + 3 = 0$
    \item $s:\,2x - y + \frac{5}{2} = 0$
    \end{itemize}

    Con lo que la distancia entre las dos rectas es:
    \[d(r,s) = \frac{\left|3 - \frac{5}{2}\right|}{\sqrt{2^2 + 1^2}} = \frac{\frac{1}{2}}{
      \sqrt{5}} = \frac{\sqrt{5}}{10}\]
  \end{solution}
\question Calcula la distancia entre las rectas $r:\,y-2 = 3x+1$ y $s:\,\frac{x}{3} = \frac{y-2}{2}$
  \begin{solution}
    Como hemos visto en alguno de los ejercicios anteriores lo primero es escribir ambas rectas en
    forma implícita y comprobar si son paralelas.
    \begin{itemize}
    \item Para $r$:
      \[r:\, y-2 = 3x + 1\]
      Llevamos todo a la izquierda:
      \[r:\,-3x + y -3 = 0\]
    \item Para $s$:
      \[s:\,\frac{x}{3} = \frac{y-2}{2}\]
      Cambiamos los denominadores al otro lado multiplicando:
      \[s:\, 2x = 3(y-2)\]
      Desarrollamos y llevamos todo a la izquierda:
      \[s:\,2x-3y+6 = 0\]
    \end{itemize}
    Comprobamos si los vectores perpendiculares son proporcionales para ver si las rectas son
    paralelas:
    \[\frac{-3}{2}\neq\frac{1}{-3}\]
    Con lo cual \textbf{no son paralelas} y \textbf{no podemos calcular la distancia}.
  \end{solution}
\question Los puntos $A(0, 1)$, $B(3, -2)$ y $C(1, 3)$ forman un triángulo. Calcula la altura sobre
  el lado $\overline{AB}$
  \begin{solution}
    Con lo que sabemos ahora este problema va a ser mucho más sencillo que como lo hicimos en
    los apuntes de rectas.\\
    
    Lo primero es comprobar que esos tres puntos forman un triángulo viendo que los vectores que
    van de uno de los puntos a los otros dos no son proporcionales.\\
    Tenemos que $\overrightarrow{AB} = (3,-3)$ y que $\overrightarrow{AC} = (1, -2)$, con lo que no
    son proporcionales, no están alineados y forman un triángulo.

    Nos pide la altura sobre el lado $\overline{AB}$, que es la distancia de $C$ sobre el lado
    $\overline{AB}$. Y para ello vamos a utilizar la fórmula de la distancia de un punto a una
    recta.\\
    
    Escribimos ecuación punto-pendiente de la recta que pasa por $A$ y $B$:
    \[r_{AB}:\,y - 1 = -x\]
    Y la pasamos a la implícita:
    \[r_{AB}:\,x+y-1 = 0\]
    Con lo que la altura del triángulo es:
    \[h = d(C,r_{AB}) = \frac{|1+3-1|}{\sqrt{2}} = \frac{3\sqrt{2}}{2}\]
    Se puede comprobar que es la misma solución que la obtenida en los ejemplos de los
    apuntes de rectas.    
  \end{solution}
\question Dados los puntos $A(-2,3)$, $B(2,2)$ y $C(10, 0)$ calcula la recta que pasa por los
  puntos $A$ y $B$ y calcula la distancia de $C$ a la recta calculada.
  \begin{solution}
    Como nos piden la distancia de $C$ a la recta que pasa por $A$ y $B$ vamos a escribir esta
    última directamente en su forma implícita.\\

    Hemos visto que en la ecuación implícita $Ax+By+C=0$ el vector $\vec{n} =(A,B)$ es perpendicular
    a la recta, con lo que vamos a construir un vector perpendicular a la recta para llegar a la
    ecuación implícita directamente.\\
    El vector $\overrightarrow{AB}=(4,-1)$ es un vector director de la recta, con lo que si
    hacemos $\vec{n} = (1, 4)$ ya tenemos un vector perpendicular a la recta.\\
    Entonces la ecuación de la recta es $x+4y+C=0$ y tenemos que calcular $C$.\\
    Teniendo en cuenta que la recta tiene que pasar por $A$ se tiene que cumplir:
    \[(-2) + 4*3+C=0\]
    Con lo que $C=-10$\\

    Una vez que tenemos la ecuación de la recta, $r:\,x+4y-10=0$, ya podemos calcular su distancia a
    $C(10, 0)$
    \[d(C, r) = \frac{|10+4*0-10|}{\sqrt{17}} = 0\]
    Con lo que $C$ es un punto de la recta, ya que es la única manera de la que puede estar a una
    distancia 0.
  \end{solution}
\question Calcula el ángulo que forman las rectas $r:\,3x-y + 2 = 0$ y $s:\,y =2x - 1$.
  \begin{solution}
    Con el producto escalar podemos calcular el ángulo que forman dos vectores, y si estos son
    vectores directores de las rectas el ángulo que calculemos va a ser el ángulo que formen
    esas dos rectas.
    \begin{itemize}
    \item Para $r$, como tenemos la ecuación implícita podemos sacar un vector perpendicular, que
      es $\vec{n} = (3, -1)$, y de hay sacamos un vector director para $r$ que es
      \[\vec{v} = (1,3)\]
    \item Para $s$, es la ecuación explícita, con lo que el coeficiente de $x$ es la pendiente de
      recta, y como la pendiente es $m=\frac{w_y}{w_x}=\frac{2}{1}$ tenemos que un vector director
      de $s$ es:
      \[\vec{w} = (1, 2)\]
    \end{itemize}
    Con todo esto el ángulo que forman las dos rectas es:
    \[\alpha = \acos \frac{\vec{v}*\vec{w}}{|\vec{v}|*|\vec{w}|}\]
    \[\alpha = \acos \frac{1*1 + 3*2}{\sqrt{1^2+3^2}*\sqrt{1^2+2^2}}\]
    \[\alpha = \acos \frac{7}{\sqrt{10}\sqrt{5}} = \acos \frac{7}{\sqrt{50}}\]
    \[\alpha \simeq 0.1419\]
    \small{(Esto son $8.13\degree$ aproximadamente, pero ya estamos trabajando con radianes)}\\
    Con lo que el ángulo que forman $r$ y $s$ es de $0.1419$\,radianes aproximadamente.
  \end{solution}
\question Calcula el simétrico de $P(2,3)$ respecto de la recta $r:\,2x - y +2 = 0$ \label{punto_simetria_axial}
  \begin{solution}
    El simétrico de $P$ respecto de $r$ es un punto $P'$ que se encuentra en la perpendicular a
    $r$ que pasa por $P$ y cuya distancia a $r$ es la misma que la de $P$.\\
    Cuando el objeto respecto del cual se hace una simetría es una recta, como es el caso, estamos hablando
    de una \emph{simetría axial}.\\
    Gráficamente:
    \begin{center}
      \begin{tikzpicture}
        \coordinate (P) at (-1,1);
        \coordinate (A) at (-2,-2);
        \coordinate (B) at (2,2);
        \coordinate (P1) at (1, -1);
        \coordinate (M) at (0,0);
        \draw (A)--(B) node[right] {$r$};
        \draw[fill=black] (P) circle(1pt) node[left] {$P$};
        \draw[fill=black] (P1) circle(1pt) node[right] {$P'$};
        \draw[dashed, color=darkgray] (P)--(P1);
        \draw[fill=darkgray,color=darkgray] (M) circle(1pt) node[left,xshift=-2] {\small{$M$}};
      \end{tikzpicture}
    \end{center}
    En el dibujo se puede observar que la recta que une $P$ y $P'$ es perpendicular a $r$ y que la
    distancia de ambos puntos a la recta es la misma.\\

    En gris hemos dibujado un punto $M$, que es el punto de corte de la recta $r$ con la que une
    $P$ y su simétrico, ya que la idea que vamos a seguir es la siguiente: como $P$ y $P'$ se
    encuentran a la misma distancia de $r$ el punto de corte de la recta que los une es el punto
    medio del segmento que forman $P$ y $P'$.\\

    Entonces vamos a calcular la recta que pasa por $P$ y $P'$, que es perpendicular a $r$ por la
    definición de simétrico.\\
    De la ecuación de $r$ obtenemos el vector perpendicular $\vec{n} = (2, -1)$, que va a ser vector
    director de la recta que une $P$ y $P'$ a la que vamos a llamar $s$.\\
    Como $s$ tiene que pasar por $P$ escribimos la ecuación punto pendiente de $s$:
    \[s:\,y-3=\frac{-1}{2}(x-2)\]
    El punto $M$ es el punto de corte de $r$ y $s$, para poder calcularlo mejor pasamos $s$ a
    forma implícita:
    \[s:\, y - 3 = -\frac{x}{2} +1\]
    \[s:\, x +2y -8 = 0\]
    Y resolvemos el sistema para obtener $M$:
    \[
      \begin{cases}
        2x - y &=-2\\
        x + 2y &= 8
      \end{cases}\]
    Que nos da $M\left(\frac{4}{5},\frac{18}{5}\right)$.\\

    Por la fórmula del punto medio sabemos que:
    \[M = \frac{P+P'}{2}\]
    Con lo que
    \[P' = 2M -P\]
    \[P' = 2*\left(\frac{4}{5},\frac{18}{5}\right) - (2,3) = \left(\frac{-6}{5}, \frac{3}{5}
      \right)\]
    Y ya tenemos al simétrico de $P$ respecto de $r$.
  \end{solution}
\question Calcula las bisectrices de las rectas $r:\, 2x -3y  = 0$ y $s:\,-x+y -2 = 0$.
  \begin{solution}
    \textbf{IMPORTANTE}: para empezar a resolver este problema hay que comprobar que las rectas no
    son paralelas, porque si son paralelas no tendría sentido. Es fácil ver que los vectores
    perpendiculares que se obtienen de las ecuaciones no son proporcionales, con lo que las rectas
    no son paralelas.\\
    
    Una vez hecha esta comprobación vamos a  recordar lo que es la bisectriz de un ángulo:
    \emph{la bisectriz de un ángulo
      es la recta que divide el ángulo en dos ángulos iguales.}.\\
    Entonces la bisectriz de dos rectas hablamos de la recta que divide el ángulo que forman en
    dos partes iguales, gráficamente:
    \begin{center}
      \begin{tikzpicture}
        \coordinate (R) at (2,1.15);
        \coordinate (R1) at (-2,-1.15);
        \coordinate (S) at (1.15,2);
        \coordinate (S1) at (-1.15,-2);

        \coordinate (B11) at (2,2);
        \coordinate (B12) at (-2,-2);

        \coordinate (B21) at (-2,2);
        \coordinate (B22) at (2,-2);

        \draw (R)--(R1) node[left] {$r$};
        \draw (S)--(S1) node[left] {$s$};

        \draw[dashed] (B11)--(B12) node[left] {$b_1$};
        \draw[dashed] (B21)--(B22) node[below right] {$b_2$};
      \end{tikzpicture}
    \end{center}
    Como dos rectas secantes dividen el plano en cuatro ángulos iguales dos a dos, vamos a tener dos
    bisectrices: $b_1$ y $b_2$, que van a ser perpendiculares.\\

    Para calcular estas dos bisectrices vamos a tener en cuenta que, al dividir el ángulo en dos
    partes iguales, cada uno de los puntos de estas rectas se encuentra a la misma distancia de $r$
    que de $s$.\\
    Entonces utilizaremos un punto genérico $P(x,y)$, de manera que su distancia a $r$ es:
    \[d(P,r) = \frac{|2x - 3y|}{\sqrt{13}}\]
    y a la recta $s$:
    \[d(P,s) = \frac{|-x + y - 2|}{\sqrt{2}}\]
    Y por lo que acabamos de decir tiene que ocurrir:
    \[d(P,r) = d(P,s)\]
    \[\frac{|2x - 3y|}{\sqrt{13}} = \frac{|-x + y - 2|}{\sqrt{2}}\]

    Como tenemos una ecuación con valores absolutos se va a dividir en dos ecuaciones que
    resolveremos por separado y cada una de ellas nos llevará a una de las dos bisectrices que
    hemos visto que tienen que salir.
    \begin{itemize}
    \item La primera ecuación va a ser quitando los valores absolutos sin hacer nada más:
      \[\frac{2x - 3y}{\sqrt{13}} = \frac{-x + y - 2}{\sqrt{2}}\]
      \[2\sqrt{2}x - 3\sqrt{2}y = -\sqrt{13}x + \sqrt{13}y - 2\sqrt{13}\]
      \[(2\sqrt{2} + \sqrt{13})x -(3\sqrt{2}+\sqrt{13})y +2\sqrt{13} = 0\]
      Que será la ecuación de la bisectriz $b_1$, por ejemplo (el orden da igual).
    \item La segunda ecuación es quitando los valores absolutos y cambiando de signo uno de los
      lados de la igualdad:
      \[\frac{2x - 3y}{\sqrt{13}} = -\frac{-x + y - 2}{\sqrt{2}}\]
      \[2\sqrt{2}x - 3\sqrt{2}y = \sqrt{13}x - \sqrt{13}y + 2\sqrt{13}\]
      \[(2\sqrt{2} - \sqrt{13})x -(3\sqrt{2}-\sqrt{13})y -2\sqrt{13} = 0\]
      Que será la ecuación de la segunda bisectriz, $b_2$.
    \end{itemize}
  \end{solution}
\question Calcula la recta simétrica de $r:\,x-y = 7$ respecto de la recta $s:\,x-y = 4$.
  \begin{solution}
    Empecemos por entender qué es eso de la recta simétrica de una respecto de otra:\\
    \emph{La simétrica de $r$ respecto de $s$ es una la recta $r'$ construida con los simétricos de cada punto de
      $r$.}
    En el caso que nos ocupa es fácil ver que las rectas son paralelas, con lo que la situación sería la siguiente:
    \begin{center}
      \begin{tikzpicture}
        \coordinate (Ar) at (-1,-2);
        \coordinate (Br) at (2, 1);
        \coordinate (As) at (-1,-1);
        \coordinate (Bs) at (2, 2);
        \coordinate (Ar1) at (-1,0);
        \coordinate (Br1) at (2, 3);
        \draw (Ar)--(Br) node[right] {$r$};
        \draw (As)--(Bs) node[right] {$s$};
        \draw (Ar1)--(Br1) node[right] {$r'$};
        \coordinate (Pr) at (1,0);
        \coordinate (Ps) at (.5,.5);
        \coordinate (Pr1) at (0,1);
        \draw[dashed,latex-latex] (Pr)--(Ps) node[midway, above right] {$d$};
        \draw[fill=black] (Pr) circle (2pt) node[below] {$P$};
        \draw[dashed,latex-latex] (Pr1)--(Ps) node[midway, above right] {$d$};
        \draw[fill=black] (Pr1) circle (2pt) node[above] {$P'$};
      \end{tikzpicture}
    \end{center}
    Según se ve en el gráfico, cada punto $P$ de $r$ tiene su simétrico $P'$ en la recta $r'$.

    Al ser rectas paralelas podemos utilizar la fórmula de la distancia entre rectas: si $r:\,Ax + By + C_r = 0$ y
    $s:\,Ax + By + C_s = 0$ ($A$ y $B$ son iguales por ser paralelas) su distancia es:
    \[d(r, s) = \frac{|C_r - C_s}{\sqrt{A^2 + B^2}}\]
    Entonces la distancia de $r$ a $s$ es:
    \[d(r, s) = \frac{|-(7) - (-4)|}{\sqrt{1^2 + (-1)^2}} = \frac{3}{\sqrt{2}}\]
    Y la distancia de $s$ a $r'$ es:
    \[d(s, r') = \frac{|-4 - C_{r'}|}{\sqrt{2}}\]
    Y por la definición de la simetría axial esas distancias tienen que ser iguales:
    \[\frac{|4 - C_{r'}|}{\sqrt{2}} = \frac{3}{\sqrt{2}}\]
    Como hay un valor absoluto hay que separarlo en dos ecuaciones, una quitando el valor absoluto sin más y otra
    quitándole y cambiando lo de dentro de signo.
    \begin{itemize}
    \item Quitando el valor absoluto sin más:
      \[\frac{-4-C_{r'}}{\sqrt{2}} = \frac{3}{\sqrt{2}}\]
      Quitamos denominadores:
      \[-4-C_{r'} = 3\]
      \[C_{r'} = -7\]
      Con lo que quedaría que una posible recta simétrica es $x - y -7 = 0$, que es la misma que $r$, con lo que esta
      solución no nos vale.
    \item Quitando el valor absoluto y cambiando de signo lo de dentro:
      \[\frac{-(-4-C_{r'})}{\sqrt{2}} = \frac{3}{\sqrt{2}}\]
      Quitamos denominadores:
      \[-(-4 - C_{r'} ) = 3\]
      \[4+C_{r'} = 3\]
      \[C_{r'} = -1\]
      Que nos da como solución la recta $x -y -1 = 0$, que sí es una solución válida.
    \end{itemize}
    Por tanto la recta simétrica que nos piden es:
    \[\boldsymbol{r':\,x-y = 1}\]
  \end{solution}
\question Calcula la simétrica de $r:\,x+2y - 2= 0$ respecto de $s:\,x - y = 2$.
  \begin{solution}
    El problema es similar al anterior solo que esta vez las rectas no son paralelas, la situación es:
    \begin{center}
      \begin{tikzpicture}
        \coordinate (As) at (-2,-2);
        \coordinate (Bs) at (2,2);
        \coordinate (Ar) at (-2,-1);
        \coordinate (Br) at (2,1);
        \coordinate (Ar1) at (-1,-2);
        \coordinate (Br1) at (1,2);

        \draw (Ar)--(Br) node[right] {$r$};
        \draw (As)--(Bs) node[right] {$s$};
        \draw (Ar1)--(Br1) node[right] {$r'$};
        \coordinate (Pr) at (1,0.5);
        \coordinate (Ps) at (.75,.75);
        \coordinate (Pr1) at (0.5,1);
        \draw[dashed,latex-latex] (Pr)--(Ps) node[midway, above right] {$d$};
        \draw[fill=black] (Pr) circle (2pt) node[below] {$P$};
        \draw[dashed,latex-latex] (Pr1)--(Ps) node[midway, above right] {$d$};
        \draw[fill=black] (Pr1) circle (2pt) node[above] {$P'$};

        \coordinate (C) at (0,0);
        \draw[fill=black] (C) circle(1pt) node[below right] {\scriptsize{$C$}};
      \end{tikzpicture}
    \end{center}
    Se ve que $s$ es el eje de simetría y cada punto $P$ de $r$ tiene su simétrico con respecto a $s$, $P'$ de $r'$.\\
    
    
    El método que tenemos que utilizar aquí es un poco más largo que el anterior, que va a consistir en calcular $r'$
    a partir de dos de sus puntos.\\
    Uno de ellos va a ser el común a las tres rectas, que en el dibujo lo hemos llamado $C$, y para calcularlo
    resolvemos el sistema que forman $r$ y $s$.
    \[
      \begin{cases}
        x+ 2y & = 2\\
        x-y &= 2
      \end{cases}
    \]
    Es un sistema sencillo que tiene de solución el punto $C(2, 0)$.\\

    Nos hace falta otro punto más, y para ello vamos a calcular el simétrico respecto de $s$ de un punto $P$ de $r$.
    Para ello vamos a hacer $x=0$ en la ecuación de $r$, y obtenemos el punto $P(0,1)$. Calculamos el simétrico con
    lo visto en el \emph{ejemplo \ref{punto_simetria_axial}} (página \pageref{punto_simetria_axial}):
    \begin{enumerate}
    \item Calculamos un vector perpendicular a $s$: $\vec{n} = (1, -1)$.
    \item Calculamos la recta perpendicular a $s$ que pasa por $P$ (vamos a empezar desde la punto-pendiente):
      \[y - 1 = -1*(x - 0)\]
      \[x + y = 1\]
    \item Calculamos el corte ($M$) de $s$ y su perpendicular:
      \[
        \begin{cases}
          x - y &= 2\\
          x+y & = 1
        \end{cases}
      \]
      De donde $M\left(\frac{3}{2}, -\frac{1}{2}\right)$
    \item Calculamos $P'$ simétrico de $P$ respecto de $M$ (utilizando que $M$ es el punto medio de $P$ y $P'$):
      \[M= \frac{P + P'}{2}\]
      \[\left(\frac{3}{2}, -\frac{1}{2}\right) = \frac{(0, 1) + P'}{2}\]
      \[(3, -1) = (0,1) + P'\]
      \[P' = (3, -2)\]
    \end{enumerate}
    Con esto ya tenemos los dos puntos por los que pasa $r'$, $C(2, 0)$ y $P'(3, -2)$, con lo que la ecuación de $r'$
    es (vamos a repasar el método del cálculo la recta que pasa por dos puntos):
    \begin{enumerate}
    \item Calculamos un vector director: $\vec{v} = P' - C = (1, -2)$
    \item Escribimos la punto-pendiente (o cualquier otra que sepamos) con el punto $C$:
      \[y = -2(x - 2)\]
    \item Y la transformamos en la que queramos (por ejemplo la implícita):
      \[y = -2x + 4\]
      \[2x + y = 4\]
    \end{enumerate}
    Con lo que la recta que nos piden es $\boldsymbol{r':\,2x + y = 4}$
  \end{solution}
\question Dada la recta $r:\, 2x - 3y = 0$ y el punto $C(1, 1)$ escribe la ecuación de la recta simétrica de $r$ con
  respecto a $C$.
  \begin{solution}
    En este caso, en el que el objeto respecto al cual se hace la simetría es un punto, se habla de \emph{
      simetría central o radial}.\\
    En este caso la recta simétrica $r'$ es la formada por los puntos $P'$ simétricos a los puntos $P$ de $r$\\
    Gráficamente:
    \begin{center}
      \begin{tikzpicture}
        \coordinate (C) at (0,0);
        \coordinate (A) at (-2, -3);
        \coordinate (B) at (2, 1);
        \coordinate (A1) at (-2, -1);
        \coordinate (B1) at (2, 3);
        \coordinate (P) at (1, 0);
        \coordinate (P1) at (-1, 0);

        \draw[fill=black] (C) circle (2pt) node[above] {$C$};
        \draw[fill=black] (P) circle (2pt) node[right] {$P$};
        \draw[fill=black] (P1) circle (2pt) node[left] {$P'$};
        \draw (A)--(B) node[right] {$r$};
        \draw (A1)--(B1) node[right] {$r'$};

        \draw[dashed, latex-latex] (P)--(C) node[midway, above] {$d$};
        \draw[dashed, latex-latex] (P1)--(C) node[midway, below] {$d$};
      \end{tikzpicture}
    \end{center}
    Que es como habíamos dicho, cada punto $P$ de $r$ tiene su simétrico respecto de $C$, $P'$, en $r'$.\\

    Con lo que podemos utilizar el mecanismo para calcular un punto simétrico respecto a otro sobre los puntos de
    $r$. Para ello escribimos $r$ en forma paramétrica, ya que así tenemos la expresión de todos los puntos de $r$.\\
    \begin{enumerate}
    \item Un vector perpendicular a $r$ es $\vec{n}=(2, -3)$, con lo que un vector director es $\vec{v} = (3,2)$.
    \item Haciendo $x=0$ en $r$ obtenemos
      \[0 - 3y = 0\]
      Con lo que un punto de $r$ es el $(0,0)$.
    \item Con lo anterior construimos las ecuaciones paramétricas de $r$:
      \[r:\,
        \begin{cases}
          x &=3t\\
          y &= 2t
        \end{cases}
      \]
    \end{enumerate}
    Entonces cualquier punto de $r$ es $P(3t, 2t)$.\\

    Si $P'$ es el simétrico de $P$ respecto de $C$, entonces $C$ es el punto medio de $P$ y $P'$:
    \[C = \frac{P + P'}{2}\]
    \[(1, 1) = \frac{(3t, 2t) + P'}{2}\]
    \[(2, 2) = (3t, 2t) + P'\]
    \[P' = (2-3t, 2-2t)\]
    Con lo que las ecuaciones paramétricas de $r'$ son:
    \[r':\,
      \begin{cases}
        x &= 2-3t\\
        y &= 2- 2t
      \end{cases}
    \]
    Y si lo pasamos a la forma implícita:
    \[r':\,\frac{x-2}{-3} = \frac{y-2}{-2}\quad\text{\small{Ecuación continua}}\]
    \[r':\,2x - 4 = 3y - 6\]
    \[r':\,2x-3y = -2\]
  \end{solution}
\question Dada la recta $r:\, x - 3y +1= 0$ y el punto $C(2, 1)$ escribe la ecuación de la recta simétrica de $r$ con respecto
a $C$.
\begin{solution}
  El enunciado de este problema es idéntico al anterior, con lo que habrá que hacer lo mismo.
  Escribimos la ecuación de $r$ en paramétricas:
  \begin{enumerate}
    \item Un vector perpendicular a $r$ es $\vec{n}=(1, -3)$, con lo que un vector director es $\vec{v} = (3,1)$.
    \item Haciendo $y=0$ en $r$ obtenemos
      \[x - 3*0 + 1= 0\]
      Con lo que un punto de $r$ es el $(-1,0)$.
    \item Con lo anterior construimos las ecuaciones paramétricas de $r$:
      \[r:\,
        \begin{cases}
          x &=-1 + 3t\\
          y &= t
        \end{cases}
      \]
    \end{enumerate}
    Entonces cualquier punto de $r$ es $P(-1+ 3t, t)$.\\

    Tal y como vimos en el anterior, los puntos de la recta simétrica $r'$ están a la misma distancia de $C$ que los de $r$, con lo que:
    \[C = \frac{P+P'}{2}\]
    \[(2,1) = \frac{(-1+3t, t) + (x, y)}{2}\]
    \[(4, 2) = (-1+3t, t) + (x, y)\]
    \[(x, y) = (9-3t,2 - t)\]
    Con lo que las paramétricas de $r'$ son:
    \[r':\,
      \begin{cases}
        x &= 5-3t\\
        y &= 1 - t
      \end{cases}\]
    Y pasándola a la implícita queda:
    \[r':\,\frac{x-5}{-3} = \frac{y-2}{-1}\]
    \[r':\, -x+5 = -3y +6\]
    \[r':\, -x + 3y - 1 = 0\]
    O, lo que es lo mismo:
    \[r':\,x - 3y + 1 = 0\]

    Es decir, hemos obtenido que $r$ y $r'$ son la misma recta. ¿Por qué?\\
    Porque en este caso el punto $C$ pertenece a $r$, y \textbf{la recta simétrica con respecto a un punto que pertenece a esa
      recta es ella misma}.\\

    De aquí se saca una \textbf{conclusión}: \emph{si en primer lugar comprobamos si el centro de simetría pertenece a la recta, nos ahorramos
      el resto del cálculo ya que en este caso la simétrica es ella misma}.
\end{solution}
\question Dados los puntos $A(1,4)$, $B(-3,0)$ y $C(3, -2)$, calcula las coordenadas del punto que equidista de los
  tres.
  \begin{solution}
    Si $P$ es equidistante de $A$ y $B$ tiene que estar en la mediatriz del segmento $\overline{AB}$, ya que esta
    contiene todos los puntos que equidistan de $A$ y $B$.\\

    Por la misma razón tiene que estar en la mediatriz de $\overline{BC}$ (se podría hacer el mismo
    razonamiento con la mediatriz de $\overline{AC}$).\\

    Y si está en ambas mediatrices no tiene más remedio que ser el corte de las dos.\\
    \begin{center}
      \begin{tikzpicture}
        \coordinate (A) at (-1, 1);
        \coordinate (B) at (0, 0);
        \coordinate (C) at (1, 0);
        \coordinate (M11) at (-2, -1);
        \coordinate (M12) at (2, 3);
        \coordinate (M21) at (.5, -1);
        \coordinate (M22) at (.5, 3);
        \coordinate (P) at (.5, 1.5);

        \draw[dotted] (M11)--(M12);
        \draw[dashed] (M21)--(M22);

        \draw[fill=black] (A) circle (2pt) node[left] {$A$};
        \draw[fill=black] (B) circle (2pt) node[below] {$B$};
        \draw[fill=black] (C) circle (2pt) node[right] {$C$};
        \draw[fill=black] (P) circle (2pt) node[left] {$P$};
        \node at (5, 1) {\parbox[t]{6cm}{\small{
              \begin{itemize}
              \item $\cdots\cdots$ Mediatriz de $\overline{AB}$
              \item $---$ Mediatriz de $\overline{BC}$
              \item $P$ es equidistante a $A$, $B$ y $C$
              \end{itemize}
              
            }
          }
        };
      \end{tikzpicture}

    \end{center}

    Este razonamiento nos conduce a que si los tres puntos están alineados no va a existir el punto equidistante,
    ya que las mediatrices son paralelas:
    \begin{center}
      \begin{tikzpicture}
        \coordinate (A) at (-1, 1);
        \coordinate (B) at (0, 0);
        \coordinate (C) at (1, -1);
        \coordinate (M11) at (-2, -1);
        \coordinate (M12) at (2, 3);
        \coordinate (M21) at (-1, -2);
        \coordinate (M22) at (2, 1);


        \draw[dotted] (M11)--(M12);
        \draw[dashed] (M21)--(M22);

        \draw[fill=black] (A) circle (2pt) node[left] {$A$};
        \draw[fill=black] (B) circle (2pt) node[below] {$B$};
        \draw[fill=black] (C) circle (2pt) node[right] {$C$};

        \node at (5, 1) {\parbox[t]{6cm}{\small{
              \begin{itemize}
              \item $\cdots\cdots$ Mediatriz de $\overline{AB}$
              \item $---$ Mediatriz de $\overline{BC}$
              \end{itemize}
              
            }
          }
        };
      \end{tikzpicture}

    \end{center}
    Con lo cual vamos a comprobar si están alineados:
    \begin{itemize}
    \item $\overrightarrow{AB} = (-4, -4)$
    \item $\overrightarrow{BC} = (6, -2)$
    \end{itemize}
    Y vemos que no lo están ya que no son proporcionales, con lo que podemos continuar con el ejercicio.\\

    Calculamos la mediatriz de $A$ y $B$, que llamaremos $r$:
    \[\sqrt{(x-1)^2 + (y -4)^2} = \sqrt{(x+3)^2 + y^2}\]
    \[(x-1)^2 + (y -4)^2 = (x+3)^2 + y^2\]
    \[x^2 - 2x + 1 + y^2 - 8y + 16 = x^2 + 6x + 9 + y^2\]
    \[-8x -8 y + 8 = 0\]
    Que simplificamos a
    \[r:\, x + y - 1 = 0\]

    La mediatriz de $\overline{BC}$, que llamaremos $s$:
    \[\sqrt{(x-3)^2 + (y +2)^2} = \sqrt{(x+3)^2 + y^2}\]
    \[(x-3)^2 + (y +2)^2 = (x+3)^2 + y^2\]
    \[x^2 - 6x + 9 + y^2 + 4y + 4 = x^2 + 6x + 9 + y^2\]
    \[-12x + 4y +4 = 0\]
    Que se simplifica a
    \[s:\, 3x - y - 1 = 0\]

    El el punto pedido es el corte de las dos rectas:
    \[
      \begin{cases}
        x + y - 1 &= 0\\
        3x - y - 1 = 0
      \end{cases}
    \]
    Si sumamos ambas ecuaciones queda:
    \[4x - 2 = 0\]
    \[x = \frac{1}{2}\]
    Y sustituyendo en $r$:
    \[\frac{1}{2} + y - 1 = 0\]
    \[y = \frac{1}{2}\]

    Con lo que el punto $P\left(\frac{1}{2}, \frac{1}{2}\right)$ equidista de $A$, $B$ y $C$.
  \end{solution}
  
\end{questions}
\end{document}


% LocalWords:  escríbelo º geométricamente cuadritos cuadradito v x R
% LocalWords:  w b n j AB k C AC BC P Q PQ PR M AM BA DC
% LocalWords:  overrightarrow
