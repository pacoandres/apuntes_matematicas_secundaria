\documentclass[a4paper,10pt,answers]{exam}
\usepackage{graphicx}
\usepackage[utf8]{inputenc}
\usepackage[spanish]{babel}
\usepackage[T1]{fontenc}
%textcomp es para el símbolo del euro
\usepackage{lmodern, textcomp}

\usepackage[left=1in, right=1in, top=1in, bottom=1in]{geometry}
%\usepackage{mathexam}
\usepackage{amsmath}
\usepackage{amssymb}
\usepackage{multicol}
%para la última página
\usepackage{lastpage}

%Para hacer tachados
\usepackage[makeroom]{cancel}

%Creative commons
%\usepackage{ccicons}
\usepackage[type={CC}, modifier={by-nc-sa}, version={4.0}, %imagemodifier={-eu-80x25},
lang={spanish}]{doclicense}


\usepackage{color,colortbl}
\definecolor{Gray}{gray}{0.9}
\newcolumntype{g}{>{\columncolor{Gray}}c}
%\pagestyle{headandfoot}
\pagestyle{headandfoot}
\newcommand\ExamNameLine{
\par
\vspace{\baselineskip}
Nombre:\hrulefill\relax
\par}

\renewcommand{\solutiontitle}{\noindent\textbf{Solución:}\par\noindent}

\everymath{\displaystyle}
\newcommand\ddfrac[2]{\frac{\displaystyle #1}{\displaystyle #2}}

\def \autor{Paco Andrés}
\def \titulo{Apuntes sobre logaritmos}
\def \titulofichas {\textbf {\titulo}}
\def \cursofichas {}
\def \fechaexamen {}
%\firstpageheader{\cursofichas}{\titulofichas}{\fechaexamen}
\header{\cursofichas}{\titulofichas}{\fechaexamen}
%\header{\cursofichas}{\titulofichas}{\fechaexamen}
%\firtspagefooter{}{\thepage}{}
%Por alguna razón no sale lo del cc en el pie
\firstpagefootrule
\footrule
\footer{\autor}{\thepage}{\doclicenseIcon}
\pointpoints{punto}{puntos}

\shadedsolutions
%\definecolor{SolutionColor}{rgb}{0.99,0.99,.99}
\renewcommand{\baselinestretch}{1.3}

%Use * instead of \cdot
\mathcode`\*="8000
{\catcode`\*\active\gdef*{\cdot}} 
\newcommand{\Card}{\,\mathrm{Card}}
\begin{document}

%For e number
\newcommand{\e}{\mathrm{e}}

%\author{Paco Andrés}
\title{\titulo}
\date{}
\author{\autor}
\maketitle

\begin{center}
\doclicenseLongText\\
\vspace{.25cm}
\doclicenseImage
\end{center}

\section{Necesidad de los logaritmos}
Hasta ahora hemos visto que hay seis operaciones aritméticas: suma resta, multiplicación división, potencia y raíz.\\
También hemos visto que cada operación tiene su contraria:
\begin{itemize}
	\item Si en una suma desconocemos uno de los sumandos lo hayamos con su contraria, que es la resta:
	\[a+b=c \Leftrightarrow a=c-b\ (\text{o también } b = a-c)\]
	\item Si en una multiplicación desconocemos uno de los factores lo hayamos con su contraria, la división:
	\[a * b = c \Leftrightarrow a= \frac{c}{b}\  \left( \text{o también } b = \frac{c}{a} \right)\]
	\item Si en una potencia desconocemos la base la calculamos con la raíz, que es su contraria:
	\[a^n = b \Leftrightarrow a = \sqrt[n]{b}\]
\end{itemize}
Pero, ¿cómo podemos obtener un exponente desconocido? ¿Es necesario?\\
Hay muchas fórmulas en las que aparecen exponentes, y a veces serán desconocidos. Pongamos un ejemplo:\\
En el cálculo del interés compuesto (el que utilizan los bancos) se utiliza la fórmula:
\[Cantidad\ final = Cantidad\ final * (1 + inter\acute{e}s)^n\]
En esta fórmula $n$ es el número de años (o de meses, dependerá de lo que contratemos con el banco) y es fácil imaginarse que habrá situaciones en las que necesitemos calcular esa $n$ y, ¿con cual de las operaciones anteriores lo haríamos?\\

El problema es que ninguna de las que hemos dicho nos sirve, y por eso existe otra operación que también es contraria a la potencia (con lo que la potencia tiene dos contrarias) que se llama \textbf{logaritmo}.

\section{Definición de logaritmo}
Con lo que hemos dicho antes el logaritmo se define de la siguiente manera:
\Large
\[\boldsymbol {\log_a b = c \Leftrightarrow a^c = b}\]
\normalsize
Es decir, el resultado del logaritmo es \textbf {el valor $c$ al que hay que elevar la base $a$ para que nos dé como resultado el argumento $b$}.\\

Vamos a ver unos ejemplos:
\[\log_3 81 = 4 \Leftrightarrow 3^4 = 81\]
\[\log_5 125 = 3 \Leftrightarrow 5^3 = 125\]
\[\log_4 2 = \frac{1}{2} = 0.5 \Leftrightarrow 4^\frac{1}{2} = 2\]
\[\log_2 \frac{1}{4} = -2 \Leftrightarrow 2^{-2} = \frac{1}{4}\]
\[\log_a 1 = 0 \Leftrightarrow a^0 = 1\]

Aunque no lo parezca, ésta definición no es sencilla de entender y requiere memorizarla y tener muy claro que significa cada cosa (base, argumento y resultado) para poder utilizar los logaritmos correctamente.

Debido a a que no todos los exponentes conducen a números negativos y que hay exponentes (raíces) que no pueden tener bases negativas, la base y el argumento de un algoritmo no pueden ser negativos. 

\subsection{Logaritmos especiales} \label{logaritmos especiales}
Aparte de memorizar la definición anterior y tener claro lo que significa cada valor que aparece en ella, es necesario saber que existen dos logaritmos especiales, que lo son por razones históricas, que no vienen al caso, y porque son los logaritmos que podemos calcular con cualquier calculadora científica.\\
El hecho de ser especiales hace que los escribamos de manera distinta al resto. Y hay que saber cómo se escriben porque si no, no podremos localizarlos en la calculadora ni podremos utilizarla para cualquier cosa que tenga que ver con logaritmos.
\subsubsection{Logaritmo decimal}
Los logaritmos decimales son \textbf{aquellos que tienen base 10}, y la manera de diferenciarlos de los demás es que se escriben sin base ($\boldsymbol {\log_{10} = \log}$) Ejemplos:
\[\log_{10} 2 = \log 2\]
\[\log_{10} 5 =  \log 5\]
\[\log 67 = \log_{10} 67\]
\[\log 45 = \log_{10} 45\]
\subsubsection{Logaritmo neperiano}
Estos logaritmos son los que \textbf{tienen base e}.\\

$\e$ es una constante matemática, al igual que $\pi$, y es un número irracional (tiene infinitas cifras) cuyo valor es $\e = 2.71828...$ El porqué de utilizar este número excede el propósito de estos apuntes, pero si alguien tiene interés en saber por qué es tan importante puede consultar las siguientes páginas:
\begin{itemize}
	\item \url{https://soymatematicas.com/numero-e/}
	\item \url{https://es.wikipedia.org/wiki/N\%C3\%BAmero_e} (ésta es bastante más compleja)
\end{itemize}

Lo importante en este punto es que los logaritmos que tienen como base a $\e$ tienen una manera especial de escribirse, y esta es $\boldsymbol{\ln}$. Ejemplos:
\[\log_\e 13 = \ln 13\]
\[\log_\e 18 = \ln 18 \]
\[\ln 4 = \log_\e 4\]
\[\ln 10 = \log_\e 10\]

\subsection{Prioridad de los logaritmos}
A la hora de hacer operaciones combinadas con logaritmos hay que tener en cuenta que \textbf{tienen la misma prioridad que un paréntesis}. Es decir:
\[\log_2 7*3 = \left( \log_2 7 \right) * 3 \]

Debido a esto hay que tener en cuenta que tendremos que poner paréntesis para indicar un orden distinto al fijado. De todas maneras vamos a ver los distintos casos que nos pueden aparecer:
\begin{itemize}
	\item Para indicar que queremos hacer primero un producto y luego el logaritmo tendremos que escribir $\log_a (b*c)$, porque si no ponemos paréntesis tendríamos que hacer primero el logaritmo y luego la multiplicación.
	\item Para indicar que queremos hacer primero una división y luego el logaritmo podemos escribir $\log_a \left(\frac{b}{c} \right)$ y también sin paréntesis, $\log_a \frac{b}{c}$, porque en este caso la prioridad nos la da la el tamaño y la posición de línea de fracción.\\
	Si quisiésemos indicar que hay que hacer el logaritmo antes de la división lo escribiríamos así: $\frac{\log_a b}{c}$.
	\item Para indicar que queremos hacer un logaritmo y elevarlo a una potencia se escribe $\log_a^n b$, y esto es equivalente a escribir $\left( \log_a b \right)^n$.\\
	Si lo que queremos indicar es que hay que hacer primero la potencia y luego el logaritmo escribiremos $\log_a b^n$.
\end{itemize}
\subsection{Ejercicios sobre la definición de logaritmos}
En este apartado vamos a ver como utilizar la definición para calcular algunos logaritmos sencillos. Para ello hay que recordar la definición: $\log_a b = c \Leftrightarrow a^c = b$.\\
De la definición se obtienen dos conclusiones importantes que nos sirven para hacer ejercicios:
\begin{itemize}
	\item $\boldsymbol{\log_a a^n = n}$
	\item $\boldsymbol{\log_a 1 = 0}$, independientemente del valor de $a$
\end{itemize}

Con esto y la factorización de números podemos empezar a hacer algunos ejercicios de logaritmos.\\
\textbf{Ejemplos}:
\begin{itemize}
	\item $\log_2 16 = \log_2 2^4 = 4$
	\item $\log 1\,000 = \log 10^3 = 3$
	\item $\log_3 \frac{1}{27} = \log_3 3^{-3} = -3$
	\item $\log_{\sqrt{5}} 5 = \log_{\sqrt{5}} (\sqrt{5})^2 = 2$
	\item $\log_\frac{1}{4} 16 = \log_\frac{1}{4} \left(\frac{1}{4} \right)^{-2} = -2$
	\item $\log_{27} 3 = \log_{27} \sqrt[3]{27} = \log_{27} 27^\frac{1}{3} = \frac{1}{3}$
\end{itemize}
\subsection{Posibles resultados de los logaritmos}
En los ejemplos y ejercicios de los apartados anteriores hemos visto que los resultados de los logaritmos son de todo tipo, positivos, negativos, cero coma, ...\\
Aunque no lo parezca esto tiene una lógica y es conveniente conocerla para saber si los resultados que estamos obteniendo son coherentes.\\
Si tenemos un logaritmo $\boldsymbol{\log_a b}$ los posibles resultados son (recuerda que la base $a$ y el argumento $b$ siempre son positivos):
\begin{itemize}
	\item Si la base ($a$) es mayor que 1:
	\begin{itemize}
		\item Si el argumento ($b$) es mayor que la base el resultado es positivo y mayor que 1.
		\item Si el argumento es positivo pero menor que 1 el resultado será entre 0 y 1 (cero coma)
		\item Si el argumento es menor que 1 el resultado será negativo.
	\end{itemize}
	\item Si la base es menor que 1:
	\begin{itemize}
		\item Si el argumento es menor que la base el resultado es positivo y mayor que 1.
		\item Si el argumento es mayor que la base pero menor que 1 el resultado estará entre 0 y 1.
		\item Si el argumento es mayor que 1 el resultado será negativo.
	\end{itemize}
	\item Si la base es 1 no tiene sentido, porque no podemos obtener ningún número distinto de 1 haciendo potencias de 1.
\end{itemize}
\section{Propiedades de los logaritmos}
Teniendo en cuenta que los logaritmos son una operación contraria a las potencias y que estas tienen unas propiedades muy utilizadas, es lógico pensar que los logaritmos también las tienen. Y además se utilizan mucho más que las de las potencias.\\
Vamos a ir viendo estas propiedades.
\subsection{Logaritmo de un producto}
El logaritmo de un producto es igual que la suma de logaritmos:
\large
\[\boldsymbol{\log_a (b*c) = \log_a b + \log_a c}\]
\normalsize
\textbf{Ejemplos}: 
\begin{itemize}
	\item $\log_2 (3*5) = \log_2 3 + \log_2 5$
	\item $\log_\frac{2}{3} (4*9) = \log_\frac{2}{3} 4 + \log_\frac{2}{3} 9$
	\item $\log 6 = \log (2*3) = \log 2 + \log 3$
	\item $\log 20 = \log 2 + \log 10 = \log 2 + 1$
\end{itemize}

\subsection{Logaritmo de un cociente}
Haciendo una analogía con las potencias se ve claramente que el logaritmo de un cociente será la resta de los logaritmos:
\large
\[\boldsymbol{\log_a \frac{b}{c} = \log_a b - \log_a c}\]
\normalsize
\textbf{Ejemplos}:
\begin{itemize}
	\item $\log_3 \frac{2}{5} = \log_3 2 - \log_3 5$
	\item $\log 0.25 = \log \frac{25}{100} = \log 25 - \log 100 = \log 25 - 2$
	\item $\ln \frac{1}{5} = \ln 1 - \ln 5 = -\ln 5$
\end{itemize}

\subsection{Logaritmo de una potencia}
Cuando el argumento está elevado a un exponente, éste sale multiplicando fuera del logaritmo:
\large
\[\boldsymbol{\log_a b^n = n*\log_a b}\]
\normalsize
Si en vez de una potencia tenemos una raíz:
\[\boldsymbol{\log_a \sqrt[n]{b} = \log_a b^{\frac{1}{n}} = \frac{1}{n}*\log_a b = \frac{\log_a b}{n}}\]
Con lo que podríamos decir que en el logaritmo de una raíz el índice sale dividiendo al logaritmo.\\

\textbf{Ejemplos}:
\begin{itemize}
	\item $\log_5 25 = \log_5 5^2 = 2* \log_5 5 = 2*1 = 2$
	\item $\log 1\,000 = \log 10^3 = 3* \log 10 = 3$
	\item $\log_2 \sqrt{3} = \frac{\log_2 3}{2}$
	\item $\log_3 \sqrt[5]{17^2} = \frac{2}{5} * \log_3 17$
\end{itemize}
\subsection{Cambio de base}
Como hemos dicho en el apartado sobre logaritmos especiales (\S\ref{logaritmos especiales}) en las calculadoras solo suele haber dos logaritmos:
\begin{itemize}
	\item El de base 10 o decimal ($\log$).
	\item El de base $\e$ o neperiano ($\ln$).
\end{itemize}
Y esto presenta un problema: ¿cómo calculamos un logaritmo en cualquier otra base?.\\
Para eso está esta propiedad que nos permite calcular un logaritmo en cualquier base a partir de otra:
\large
\[\boldsymbol{\log_b c = \frac{\log_a c}{\log_a b}}\]
\normalsize
Vamos a ver unos ejemplos de cómo se utiliza:
\begin{itemize}
	\item $\log_3 4 = \frac{\log 4}{\log 3}$\\
	
	Con la calculadora obtenemos $\log 4 \simeq 0.602$ y $\log 3 \simeq 0.477$, que sustituyendo en lo anterior nos da:
	\[\log_3 4 \simeq \frac{0.602}{0.477} \simeq 1.262\]
	Y con la calculadora comprobamos que $3^{1.262} = 4.0006$ (no sale exactamente igual a 4 porque hemos redondeado)
	
	\item $\log_{35} 7 = \frac{\ln 7}{\ln 35}$\\
	
	Como antes, cogemos la calculadora y obtenemos $\ln 7 \simeq 1.9459$ y $\ln 35 \simeq 3.5553$, que sustituyendo nos da:
	\[\log_{35} 7 = \frac{1.9459}{3.5553} \simeq 0.5473\]
\end{itemize}
\end{document}
