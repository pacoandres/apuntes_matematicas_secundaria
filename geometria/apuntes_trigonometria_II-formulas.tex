\documentclass[a4paper,11pt,answers]{exam}

\usepackage{graphicx}
\usepackage{hyperref}
\usepackage{wrapfig}
\usepackage[utf8]{inputenc}
\usepackage[spanish]{babel}
\usepackage[T1]{fontenc}
%textcomp es para el símbolo del euro
\usepackage{lmodern, textcomp}
%\usepackage{textgreek}
\usepackage[left=1in, right=1in, top=1in, bottom=1in]{geometry}
%\usepackage{mathexam}
\usepackage{amsmath}
\usepackage{amssymb}
\usepackage{multicol}
\usepackage{longtable}
%para la última página
%\usepackage{lastpage}

%Para padding en celdas
\usepackage{cellspace}
\setlength\cellspacetoplimit{1mm}
\setlength\cellspacebottomlimit{1mm}

%Para hacer tachados
\usepackage[makeroom]{cancel}

%Creative commons
%\usepackage{ccicons}
\usepackage[type={CC}, modifier={by-nc-sa}, version={4.0}, %imagemodifier={-eu-80x25},
lang={spanish}]{doclicense}

%Para las gráficas:
\usepackage{tikz}
\usepackage{pgfplots}
\pgfplotsset{compat = newest}
\pgfplotsset{compat=1.18}
\usetikzlibrary{babel} %Si no da errores con algunas cosas al compilar los gráficos.
\usetikzlibrary{arrows,shapes,positioning}
\usetikzlibrary{matrix}
\usepgfplotslibrary{fillbetween}
\usetikzlibrary{arrows.meta}
\usetikzlibrary{fit}
\usetikzlibrary{quotes,angles}
%\usepackage{nicematrix}

\usepackage{color,colortbl}
\definecolor{Gray}{gray}{0.9}
\newcolumntype{g}{>{\columncolor{Gray}}c}
\usepackage{arydshln} %Este pone la línea punteada en la matriz ampliada. TIENE QUE ESTAR DESPUÉS DEL colortbl porque si no casca.
%\pagestyle{headandfoot}
\pagestyle{headandfoot}
\newcommand\ExamNameLine{
\par
\vspace{\baselineskip}
Nombre:\hrulefill\relax
\par}

\renewcommand{\solutiontitle}{\noindent\textbf{Solución:}\par\noindent}

\everymath{\displaystyle}
\newcommand\ddfrac[2]{\frac{\displaystyle #1}{\displaystyle #2}}

\def \autor{Paco Andrés}
\def \titulo{Apuntes de trigonometría II.\\Fórmulas trigonométricas.}
\def \titulofichas {\textbf {\titulo}}
\def \cursofichas {}
\def \fechaexamen {}
%\firstpageheader{\cursofichas}{\titulofichas}{\fechaexamen}
\header{\cursofichas}{\begin{small}
\titulofichas
\end{small}}{\fechaexamen}
%\header{\cursofichas}{\titulofichas}{\fechaexamen}
%\firtspagefooter{}{\thepage}{}
%Por alguna razón no sale lo del cc en el pie
\firstpagefootrule
\footrule
\footer{\autor}{\thepage}{\doclicenseIcon}
\pointpoints{punto}{puntos}

\shadedsolutions
%\definecolor{SolutionColor}{rgb}{0.99,0.99,.99}
\renewcommand{\baselinestretch}{1.3}

%Use * instead of \cdot
\mathcode`\*="8000
{\catcode`\*\active\gdef*{\cdot}} 
\newcommand{\Card}{\,\mathrm{Card}}
\newcommand{\degree}{^\circ}
%For e number
\newcommand{\e}{\,\mathrm{e}}
\newcommand{\asen}{\,\mathrm{asen}\,}
\newcommand{\acos}{\,\mathrm{acos}\,}
\newcommand{\atg}{\,\mathrm{atg}\,}

%Para el diferencial y la integral:
\newcommand\dif[1]{\mathrm{d}#1}
\newcommand\integral[2]{\int #1\,\dif{#2}}
\newcommand\integrald[4]{\int_{#3}^{#4} #1\,\dif{#2}}
\newcommand\adjunto[1]{#1^\text{*}}
\newcommand\rango[1]{\mathrm{rg}(#1)}
\newcommand\vectort[3]{#1*\vec i + #2*\vec j + #3*\vec k}
\newcommand\unidad[1]{\,\text{#1}}
%Para escribir explicaciones encima del igual:
%\newcommand\igexpl[1]{{\mathrel{\overset{\makebox{\mbox{\normalfont\tiny\sffamily $#1$}}}{=}}}}
%Parece que mejor con stackrel

%Colores
\definecolor{midgray}{rgb}{0.4,0.4,0.4}


% Aumenta el interlineado en aligns y demás
\setlength{\jot}{1.5em}

% Permite poner textos a las etiquetas: \labeltext{texto}{etiqueta}
\makeatletter
\newcommand{\labeltext}[2]{%
  \@bsphack
  \MakeLinkTarget*{#1}%
  \def\@currentlabel{#1}{\label{#2}}%
  \@esphack%
}\makeatother

% paragraphs como subsubsubsections
\makeatletter
\renewcommand\paragraph{\@startsection{paragraph}{4}{\z@}%
% display heading, like subsubsection
                                     {-3.25ex\@plus -1ex \@minus -.2ex}%
                                     {1.5ex \@plus .2ex}%
                                     {\normalfont\normalsize\bfseries}}
 \setcounter{secnumdepth}{4}

                                   
 \makeatother
 % \setcounter{tocdepth}{3} %Solo parts, sections y subsections en el índice.

 \renewcommand{\questionlabel}{\textbf{Ejemplo \thequestion:}}


 %Para que nombre bien los anexos.
 \makeatletter
\newcommand\appendix@section[1]{%
  \refstepcounter{section}%
  \orig@section*{Anexo \@Roman\c@section: #1}%
  \addcontentsline{toc}{section}{Anexo \@Roman\c@section: #1}%
}
\let\orig@section\section
\g@addto@macro\appendix{\let\section\appendix@section}
\makeatother

 
\begin{document}


%\author{Paco Andrés}
\title{\titulo}
\date{}
\author{\autor}
\maketitle

\begin{center}
\doclicenseLongText\\
\vspace{.25cm}
\doclicenseImage
\end{center}
\tableofcontents
\newpage
\setlength{\parindent}{0cm}

\section{Introducción.}
Esta es la continuación avanzada de los apuntes de trigonometría básica.\\

Como bien indicamos en la frase anterior estos apuntes pertenecen a lo que se podría llamar
matemáticas avanzadas, que requieren de métodos y destrezas que hasta ahora apenas se utilizaban
por parte del alumnado.\\

Es por esto que cosas como las demostraciones y las deducciones, en las que apenas se entraba en
los apuntes de ESO, son importantes ya que muestran la manera de hacer las cosas en las matemáticas
avanzadas.\\
No es necesario que el alumnado memorice estas demostraciones o deducciones pero sí que haga el
esfuerzo de entenderlas, preguntando todas las dudas que surjan, ya que habrá ejercicios en los que
haya que utilizar razonamientos y/o metodologías parcialmente similares a las utilizadas en ellas.\\

Y para empezar vamos a ver los radianes, que es algo relativamente sencillo y muy útil.

\section{Grados y radianes.}
Hasta ahora hemos utilizado los grados para medir ángulos, pero a partir de ahora tenemos que
utilizar una unidad que tiene una serie de ventajas que vamos a ver ahora (además de que es la
unidad con la que funciona el análisis que veremos en cursos posteriores).

Con los grados tenemos la circunferencia dividida en $360\degree$ de la siguiente manera:
\begin{center}
  \begin{tikzpicture}
    \begin{axis}[width=8cm, height=8cm, xmin=-5, xmax=5, ymin=-5, ymax=5, ticks=none, axis x line=center, axis y line=center]
      \draw (0,0) circle(3.5);
      \node[above right] at (3.5,0) {$0\degree$};
      \node[above left] at (0,3.5) {$90\degree$};
      \node[above left] at (-3.5,0) {$180\degree$};
      \node[below right] at (0,-3.5) {$270\degree$};
      \node[below right] at (3.5,0) {$360\degree$};
    \end{axis}
  \end{tikzpicture}
\end{center}
Y si el ángulo es mayor de $360\degree$ es porque estamos dando más de una vuelta.

Además sabemos que si $r$ es el radio de la circunferencia la longitud de ésta viene dada por:
\[\boldsymbol{l = 2\pi r}\]
Y el área del círculo que encierra es:
\[\boldsymbol{S = \pi r^2}\]

Cada uno de los cuadrantes es un cuarto de la circunferencia, con lo que su longitud y su área
serían las correspondientes a $360\degree$ divididas entre cuatro.

Si tenemos un sector con un ángulo $\alpha$:
\begin{center}
  \begin{tikzpicture}
    \draw (0,0) coordinate (O) -- (3.5,0) coordinate (A)
    arc[start angle=0, end angle=60,radius=3.5cm] coordinate (B) -- (0,0);
    \draw (0:0.5) arc (0:60:0.5) node[right, midway] {$\alpha$};
  \end{tikzpicture}
\end{center}
la longitud y el área de ese arco serán la parte proporcional de la longitud y el área totales:
\[l = \frac{\alpha}{360\degree}*2\pi r\]
\[S = \frac{\alpha}{360\degree}*\pi r^2\]

Para evitar esa fracción tan fea podemos hacer lo siguiente, dividir la circunferencia en ángulos
que van de $0$ a $2\pi$ de la siguiente manera:
\begin{center}
  \begin{tikzpicture}
    \begin{axis}[width=8cm, height=8cm, xmin=-5, xmax=5, ymin=-5, ymax=5, ticks=none, axis x line=center, axis y line=center]
      \draw (0,0) circle(3.5);
      \node[above right] at (3.5,0) {$0$};
      \node[above left] at (0,3.5) {$\frac{\pi}{2}$};
      \node[above left] at (-3.5,0) {$\pi$};
      \node[below right] at (0,-3.5) {$\frac{3\pi}{2}$};
      \node[below right] at (3.5,0) {$2\pi$};
    \end{axis}
  \end{tikzpicture}
\end{center}

Como ahora la circunferencia entera es un ángulo de $2\pi$, las fórmulas para la longitud
y la superficie del arco que veíamos antes quedan así:
\[l = \frac{\alpha}{2\pi}*2\pi r = \boldsymbol{\alpha*r}\]
\[S = \frac{\alpha}{2\pi}*\pi r^2 = \boldsymbol{\frac{\alpha*r^2}{2}}\]
que son mucho más sencillas que las que obteníamos con los grados.

\textbf{Las unidades que dividen la circunferencia en ángulos de $0$ a $2\pi$ se llaman radianes}
y tienen una relación directamente proporcional con los grados, con lo que es fácil pasar de
una unidad a otra.\\
Si $\alpha_r$ es la medida del ángulo en radianes y $\alpha_g$ es el mismo ángulo en grados
la relación de proporcionalidad es:
\[\boldsymbol{\frac{\alpha_g}{180\degree} = \frac{\alpha_r}{\pi}}\]

Vamos a ver un par de ejemplos de conversión:
\begin{questions}
\question Escribe la medida en radianes de un ángulo de $135\degree$.
  \begin{solution}
    Simplemente tenemos que aplicar la proporcionalidad indicada y resolver:
    \[\frac{135\degree}{180\degree} = \frac{\alpha}{\pi}\]
    \[\frac{3}{4}*\pi = \alpha\]
    Con lo que $135\degree$ son $\frac{3}{4}\pi$ radianes.
  \end{solution}
\question Convierte a grados un ángulo de $\frac{\pi}{3}$ radianes.
  \begin{solution}
    Pues hacemos lo mismo que en el anterior:
    \[\frac{\alpha}{180\degree} = \dfrac{\frac{\pi}{3}}{\pi}\]
    \[\alpha = \frac{1}{3}*180\degree\]
    \[\alpha = 60\degree\]
    Entonces $\frac{\pi}{3}$ radianes son $60\degree$.
  \end{solution}
\end{questions}

En principio a los radianes no se les pone ningún circulito ($\degree$) ni otra unidad. En caso de
que necesiten algún tipo de desambiguación se les pone el diminutivo \textbf{rad}, por ejemplo
``$\frac{\pi}{4}$\,rad'', pero lo normal es no ponerles nada.

\section{Deducción de razones trigonométricas.}
Cuando explicamos trigonometría básica vimos que había una tabla con las razones trigonométricas
de ángulos famosos que no dijimos de donde salían.\\
Pues este es el momento en el que conviene que veamos de donde salen, ya que implican una serie
de formas de razonar matemáticas y geométricas que es conveniente conocer para poder entender
el tema principal de estos apuntes.\\

\textbf{NOTA}: \emph{este apartado va a ser el último en el que utilicemos grados, después
  todo será en radianes.}

\subsection{Deducción de las razones del ángulo de $30\degree$
  $\left( \frac{\pi}{6}\text{ en radianes} \right)$.}
Empecemos por dibujar un triángulo rectángulo con un ángulo de $30\degree$.
\begin{center}
  \begin{tikzpicture}
    \coordinate (O) at (0,0);
    \coordinate (A) at (5.196, 0);
    \coordinate (B) at (5.196,3);
    \draw (A)--(O) node[midway, below] {$x$};
    \draw (B)--(O) node[midway, sloped, above] {$h$};
    \draw (A)--(B) node[midway, right] {$y$};
    \pic["$30\degree$", draw=darkgray, angle eccentricity=1.4, angle radius=1cm]
    {angle=A--O--B};
  \end{tikzpicture}
\end{center}
Y por definición
\begin{itemize}
\item $\cos 30\degree = \frac{x}{h}$
\item $\sen 30\degree = \frac{y}{h}$
\end{itemize}
Pero desconocemos cuanto valen $x$, $y$, y $h$.\\

Si hacemos el simétrico del triángulo respecto del cateto horizontal nos queda:
\begin{center}
  \begin{tikzpicture}
    \coordinate (O) at (0,0);
    \coordinate (A) at (5.196, 0);
    \coordinate (B) at (5.196,3);
    \coordinate (B1) at (5.196,-3);
    \draw (A)--(O);
    \draw (B)--(O) node[midway, sloped, above] {$h$};
    \draw (A)--(B) node[midway, right] {$y$};
    \draw[dashed] (A)--(B1) node[midway, right] {$y$};
    \draw[dashed] (B1)--(O) node[midway, sloped, below] {$h$};
    \pic["\scriptsize{$30\degree$}"xshift=-1mm, draw=darkgray, angle eccentricity=1.4,
    angle radius=1cm] {angle=A--O--B};
    
    \pic["$60\degree$"yshift=-3mm, draw=darkgray, angle eccentricity=1.2, angle radius=1.6cm]
    {angle=B1--O--B};

    \pic["\scriptsize{$\alpha$}"color=midgray, draw=midgray, angle eccentricity=1.4,
    angle radius=.6cm] {angle=O--B--A};
    \pic["\scriptsize{$\alpha$}"color=midgray, draw=midgray, angle eccentricity=1.4,
    angle radius=.6cm] {angle=A--B1--O};
  \end{tikzpicture}
\end{center}
que es un triángulo isósceles cuyo ángulo formado por los lados iguales es de $60\degree$.\\
Al ser isósceles los ángulos sobre el lado desigual tienen que ser iguales, con lo que:
\[60\degree + 2\alpha = 180\degree\]
\[\alpha = 60\degree\]
Y si tiene todos sus ángulos iguales es porque es equilátero, con lo que los tres lados son
iguales y por tanto:
\[2y = h\]
\[\boldsymbol{y = \frac{h}{2}}\]

Llevando esto al triángulo original nos queda:
\begin{center}
  \begin{tikzpicture}
    \coordinate (O) at (0,0);
    \coordinate (A) at (5.196, 0);
    \coordinate (B) at (5.196,3);
    \draw (A)--(O) node[midway, below] {$x$};
    \draw (B)--(O) node[midway, sloped, above] {$h$};
    \draw (A)--(B) node[midway, right] {$y = \frac{h}{2}$};
    \pic["$30\degree$", draw=darkgray, angle eccentricity=1.4, angle radius=1cm]
    {angle=A--O--B};
  \end{tikzpicture}
\end{center}
Con lo que:
\[\sen 30\degree = \frac{y}{h} = \dfrac{\frac{h}{2}}{h}\]
\[\boldsymbol{\sen 30\degree = \frac{1}{2}}\]

Y para obtener el coseno utilizamos la relación fundamental de la trigonometría:
\[\sen^2 30\degree + \cos^2 30\degree = 1\]
\[\frac{1}{4} + \cos^2 30\degree = 1\]
\[\cos^2 30\degree = \frac{3}{4}\]
\[\cos 30\degree = \pm\sqrt{\frac{3}{4}}\]
Como estamos en el primer cuadrante nos quedamos con la raíz positiva, con lo que:
\[\boldsymbol{\cos 30\degree = \frac{\sqrt{3}}{2}}\]
Y la tangente
\[\tg 30\degree = \frac{\sen 30\degree}{\cos 30\degree}\]
\[\boldsymbol{\tg 30\degree = \frac{\sqrt{3}}{3}}\]

\subsection{Deducción de las razones del ángulo de $60\degree$
  $\left( \frac{\pi}{3} \right)$.}
Para deducir las razones del ángulo de $60\degree$ solo tenemos que tener en cuenta que es el
complementario de $30\degree$ y cuando estudiábamos la parte básica de trigonometría vimos que
las relaciones entre las razones de ángulos complementarios son:
\begin{itemize}
\item $\sen (90\degree - \alpha) = \cos \alpha$
\item $\cos (90\degree - \alpha) = \sen \alpha$
\end{itemize}
Entonces tenemos que:
\begin{itemize}
\item $\boldsymbol{\sen 60\degree = \frac{\sqrt{3}}{2}}$
\item $\boldsymbol{\cos 60\degree = \frac{1}{2}}$
\item $\boldsymbol{\tg 60\degree = \sqrt{3}}$
\end{itemize}

\subsection{Deducción de las razones del ángulo de $45\degree$
  $\left( \frac{\pi}{4} \right)$.}
Tenemos un triángulo rectángulo que tiene un ángulo de $45\degree$:
\begin{center}
  \begin{tikzpicture}
    \coordinate (O) at (0,0);
    \coordinate (A) at (4, 0);
    \coordinate (B) at (4,4);
    \draw (A)--(O) node[midway, below] {$x$};
    \draw (B)--(O) node[midway, sloped, above] {$h$};
    \draw (A)--(B) node[midway, right] {$y$};
    \pic["$45\degree$", draw=darkgray, angle eccentricity=1.4, angle radius=1cm]
    {angle=A--O--B};
    \pic["$\alpha$", draw=darkgray, angle eccentricity=1.4, angle radius=1cm]
    {angle=O--B--A};
  \end{tikzpicture}
\end{center}
El ángulo que nos falta es $\alpha = 180\degree - 90\degree - 45\degree =
\boldsymbol{45\degree}$, con lo que es un triángulo isósceles y por tanto $y = x$:
\begin{center}
  \begin{tikzpicture}
    \coordinate (O) at (0,0);
    \coordinate (A) at (4, 0);
    \coordinate (B) at (4,4);
    \draw (A)--(O) node[midway, below] {$x$};
    \draw (B)--(O) node[midway, sloped, above] {$h$};
    \draw (A)--(B) node[midway, right] {$x$};
    \pic["$45\degree$", draw=darkgray, angle eccentricity=1.4, angle radius=1cm]
    {angle=A--O--B};
    \pic["$45\degree$", draw=darkgray, angle eccentricity=1.4, angle radius=1cm]
    {angle=O--B--A};
  \end{tikzpicture}
\end{center}
Y por la definición de seno y coseno tenemos que tienen que ser iguales, $\sen 45\degree =
\cos 45\degree$.\\
Y además tienen que cumplir la relación fundamental, con lo que tenemos un sistema de ecuaciones:
\[
  \begin{cases}
    \sen 45\degree = \cos 45\degree\\
    \sen^2 45\degree + \cos^2 45\degree = 1
  \end{cases}
\]

Resolvemos por sustitución:
\[\sen^2 45\degree + \sen^2 45\degree = 1\]
\[2\sen^2 45\degree =1\]
\[\sen 45\degree = \pm \sqrt{\frac{1}{2}}\]

Como es el primer cuadrante nos quedamos con el positivo, y además lo arreglamos un poco:
\[\boldsymbol{\sen 45\degree = \frac{\sqrt{2}}{2}}\]

Entonces las razones de $45\degree$ son:
\begin{itemize}
\item $\boldsymbol{\sen 45\degree = \frac{\sqrt{2}}{2}}$
\item $\boldsymbol{\cos 45\degree = \frac{\sqrt{2}}{2}}$
\item $\boldsymbol{\tg 45\degree = 1}$
\end{itemize}

\subsection{Deducción de las razones del ángulo de $0\degree$.}
Para este caso vamos a hacer un razonamiento dinámico, vamos a ver qué pasa con el cateto
adyacente y el cateto opuesto cuando el ángulo estando cada vez más cerca de $0\degree$:
\begin{center}
  \begin{tikzpicture}[baseline=(current bounding box.center)]
    \coordinate (O) at (0,0);
    \coordinate (A) at (2.9,0);
    \coordinate (B) at (2.9,0.78);
    \draw (O)--(B) node[midway, sloped, above] {$h$};
    \draw (A)--(B) node[midway, right] {$y$};
    \draw (O)--(A) node[midway, below] {$x$};
    \pic[draw=darkgray, angle radius=.7cm]
    {angle=A--O--B};
  \end{tikzpicture}
  \quad
  $\longrightarrow$
  \quad
  \begin{tikzpicture}[baseline=(current bounding box.center)]
    \coordinate (O) at (0,0);
    \coordinate (A) at (2.95,0);
    \coordinate (B) at (2.95,0.52);
    \draw (O)--(B) node[midway, sloped, above] {$h$};
    \draw (A)--(B) node[midway, right] {$y$};
    \draw (O)--(A) node[midway, below] {$x$};
    \pic[draw=darkgray, angle radius=.7cm]
    {angle=A--O--B};
  \end{tikzpicture}
  \quad
  $\longrightarrow$
  \quad
  \begin{tikzpicture}[baseline=(current bounding box.center)]
    \coordinate (O) at (0,0);
    \coordinate (A) at (2.99,0);
    \coordinate (B) at (2.99,0.26);
    \draw (O)--(B) node[midway, sloped, above] {$h$};
    \draw (A)--(B) node[midway, right] {$y$};
    \draw (O)--(A) node[midway, below] {$x$};
    \pic[draw=darkgray, angle radius=.7cm]
    {angle=A--O--B};
  \end{tikzpicture}
\end{center}

Es fácil observar que según el ángulo se va acercando a $0\degree$ el cateto opuesto va
desapareciendo, mientras que el adyacente va midiendo lo mismo que la hipotenusa.\\
De esta manera cuando el ángulo llegue a $0\degree$ el cateto opuesto habrá desaparecido ($y = 0$)
y el cateto adyacente sera exactamente igual que la hipotenusa ($x = h$), con lo que:
\begin{itemize}
\item $\boldsymbol{\cos 0\degree = \frac{h}{h} = 1}$
\item $\boldsymbol{\sen 0\degree = \frac{0}{h} = 0}$
\item $\boldsymbol{\tg 0\degree = 0}$
\end{itemize}

\subsection{Deducción de las razones del ángulo de $90\degree$
  $\left( \frac{\pi}{2} \right)$.}
Para deducir las razones del ángulo de $90\degree$ vamos a utilizar el ángulo complementario,
como hicimos con las de $60\degree$, ya que $90\degree$ es el complementario de $0\degree$.\\
Y así tenemos:
\begin{itemize}
\item $\boldsymbol{\sen 90\degree = \cos 0\degree = 1}$
\item $\boldsymbol{\cos 90\degree = \sen 0\degree = 0}$
\item $\boldsymbol{\tg 90\degree = \frac{1}{0}} = \nexists$ \textbf{no se puede hacer}.
\end{itemize}

\subsection{Deducción de las razones de otros ángulos.}
Como regla general, deducir las razones para otros ángulos no es algo sencillo salvo para casos
concretos.\\

Estos casos concretos son las transformaciones de los ángulos que acabamos de ver, $\frac{\pi}{6}$,
$\frac{\pi}{4}$, $\frac{\pi}{3}$, a ángulos de otros cuadrantes utilizando las transformaciones
de cuadrante que vimos en trigonometría básica.\\
Por ejemplo las razones de $\frac{5\pi}{6}$:
\begin{itemize}
\item $\sen \frac{5\pi}{6} = \sen \left(\pi - \frac{\pi}{6}\right) = \sen \frac{\pi}{3} =
  \frac{1}{2}$
\item $\cos \frac{5\pi}{6} = \cos \left(\pi - \frac{\pi}{6}\right) = -\cos \frac{\pi}{3} =
  -\frac{\sqrt{3}}{2}$
\item $\tg \frac{5\pi}{6} = \tg \left(\pi - \frac{\pi}{6}\right) = -\tg \frac{\pi}{3} =
  -\frac{\sqrt{3}}{2}$
\end{itemize}

Sin embargo son muy pocos los ángulos que podemos formar así, en concreto todos los del primer
cuadrante que no hemos visto.\\

Pero estos últimos ángulos se pueden construir de otras maneras, por ejemplo el ángulo
$\frac{\pi}{12}$ lo podemos construir como
\begin{itemize}
\item $\frac{\pi}{12} = \ddfrac{\frac{\pi}{6}}{2}$
\item $\frac{\pi}{12} = \frac{\pi}{4} - \frac{\pi}{6}$
\end{itemize}

Pero para calcular las razones de una operación con ángulos no podemos operar con la misma operación
que con los ángulos, no hay proporcionalidad. Se ve claramente porque
\begin{itemize}
\item $\sen \frac{\pi}{3} \neq 2*\sen \frac{\pi}{6}$
\item $\cos \pi \neq \cos \frac{\pi}{2} + \cos \frac{\pi}{2}$
\end{itemize}
Así que vamos a ver cómo tenemos que hacer para calcular las razones de las operaciones con
ángulos.
\section{Fórmulas trigonométricas.} \label{fórmulas_trigonométricas}
\subsection{Razones de la suma y de la resta de ángulos.}
Empezamos por calcular cuando vale el \textbf{seno de la suma} de dos ángulos.\\
Tenemos la siguiente situación (hemos indicado todos los elementos que vamos a necesitar para
hacer la deducción):
\begin{center}
  \begin{tikzpicture}
    \coordinate (A) at (0, 0);
    \coordinate (B) at (6, 3);
    \coordinate (C) at (6, 0);
    \coordinate (D) at (4.5, 6);
    \coordinate (F) at (4.5, 0);
    \coordinate (E) at (6, 6);

    \draw (A)--(B)--(C)--cycle;
    \draw (A)--(B)--(D)--cycle;
    \draw[dotted, color=midgray] (F)--(D);
    \draw[dashed] (D)--(E)--(B)--cycle;
    \node[left] at (A) {$A$};
    \node[right] at (B) {$B$};
    \node[right] at (C) {$C$};
    \node[above] at (D) {$D$};
    \node[above] at (E) {$E$};

    \node[below, color=midgray] at (F) {$F$};
 
    \pic["$\alpha$", draw=darkgray, angle eccentricity=1.4, angle radius=7mm]
    {angle=C--A--B};
    \pic["$\beta$", draw=darkgray, angle eccentricity=1.4, angle radius=8mm]
    {angle=B--A--D};
    \pic[draw=darkgray, angle eccentricity=.5, angle radius=5mm, "."]
    {right angle=D--B--A};
    \pic[draw=darkgray, angle eccentricity=.5, angle radius=5mm, "."]
    {right angle=B--C--A};
    \pic[draw=darkgray, angle eccentricity=.5, angle radius=3mm, "."]
    {right angle=D--E--B};

    \pic["$\widehat{B_1}$", draw=darkgray, angle eccentricity=1.6, angle radius=7mm]
    {angle=A--B--C};
    \pic["$\widehat{B_2}$", draw=darkgray, angle eccentricity=1.5, angle radius=7mm]
    {angle=E--B--D};
    \pic["$\widehat{D_1}$", draw=darkgray, angle eccentricity=1.4, angle radius=8mm]
    {angle=A--D--B};
    \pic["$\widehat{D_2}$", draw=darkgray, angle eccentricity=1.5, angle radius=7mm]
    {angle=B--D--E};
  \end{tikzpicture}
\end{center}
Queremos calcular el seno del ángulo $\alpha + \beta$, y por la definición
\[\sen (\alpha + \beta) = \frac{\overline{FD}}{\overline{AD}}\]
Pero para hacer la deducción vamos a utilizar el hecho de que $\overline{CE}$ mide lo mismo que
$\overline{FD}$, con lo que:
\[\sen (\alpha + \beta) = \frac{\overline{CE}}{\overline{AD}}\]

$\overline{CE}$ lo podemos descomponer como
$\boldsymbol{\overline{CE} = \overline{CB}+\overline{BE}}$.

Por trigonometría sabemos que $\boldsymbol{\overline{CB} = \overline{AB}*\sen \alpha}$, pero
¿cuánto vale $\overline{BE}$. Para ello vamos a utilizar razonamientos con ángulos.\\

Tenemos que $\widehat{B_1} = \frac{\pi}{2} - \alpha$ y además tenemos que $\widehat{B_1} +
\widehat{B_2} + \frac{\pi}{2} = \pi$, con lo que $\widehat{B_2} = \alpha$ y entonces
\[\boldsymbol{\overline{BE} = \overline{BD}*\cos \alpha}\]

De manera que al sustituir $\overline{CE}$ en el seno queda:

\[\sen (\alpha + \beta) = \frac{\overline{AB}*\sen \alpha + \overline{BD}*
    \cos \alpha}{\overline{AD}}\]

Utilizando las propiedades de las fracciones vamos a separarlo de esta manera:
\[\sen (\alpha + \beta) = \sen \alpha * \frac{\overline{AB}}{\overline{AD}} +
  \cos \alpha*\frac{\overline{BD}}{\overline{AD}}\]

Por la definición de las razones trigonométricas tenemos que
\begin{itemize}
\item $\frac{\overline{AB}}{\overline{AD}} = \cos \beta$
\item $\frac{\overline{BD}}{\overline{AD}} = \sen \beta$
\end{itemize}
Y al sustituir nos queda que:
\[\boldsymbol{\sen (\alpha + \beta) = \sen \alpha * \cos \beta +
    \cos \alpha*\sen \beta}\]

Y para el \textbf{seno de la resta}:
\[\sen (\alpha - \beta) = \sen \alpha * \cos (-\beta) +
  \cos \alpha*\sen (-\beta)\]
Recordando que $\cos (-\beta) = \cos \beta$ y $\sen (-\beta) = -\sen \beta$:
\[\boldsymbol{\sen (\alpha - \beta) = \sen \alpha * \cos \beta -
    \cos \alpha*\sen \beta}\]\vspace{3mm}

El \textbf{coseno de la suma} es más sencillo de calcular ya que vamos a aprovechar la relación con las
razones del ángulo complementario, que nos dicen que:
\[\cos \alpha = \sen \left( \frac{\pi}{2} - \alpha \right)\]
Entonces:
\[\cos (\alpha + \beta) = \sen \left( \frac{\pi}{2} - (\alpha + \beta) \right)\]
\[\cos (\alpha + \beta) = \sen \left( \frac{\pi}{2} - \alpha - \beta \right)\]
Que lo podemos agrupar de la siguiente manera:
\[\cos (\alpha + \beta) = \sen \left( \left(\frac{\pi}{2} - \alpha\right) - \beta \right)\]
Donde podemos aplicar la fórmula de la resta que hemos visto un poco más arriba y nos queda:
\[\cos (\alpha + \beta) = \sen \left(\frac{\pi}{2} - \alpha\right) *\cos \beta
  -\cos \left(\frac{\pi}{2} - \alpha\right) *\sen \beta\]
Que por las relaciones con el ángulo complementario se transforma en:
\[\boldsymbol{\cos (\alpha + \beta) =\cos \alpha * \cos \beta - \sen \alpha *\sen \beta}\]
Y entonces el \textbf{coseno de la resta}:
\[\boldsymbol{\cos (\alpha - \beta) =\cos \alpha * \cos \beta + \sen \alpha *\sen \beta}\]\vspace{3mm}

La deducción de la \textbf{tangente de la suma} requiere de un dominio importante de las reglas
algebraicas y de las fracciones.\\
Partimos de la definición:
\[\tg (\alpha + \beta) = \frac{\sen (\alpha +  \beta)}{\cos (\alpha + \beta)}\]
\[\tg (\alpha + \beta) = \frac{\sen \alpha * \cos \beta +
    \cos \alpha*\sen \beta}{\cos \alpha * \cos \beta - \sen \alpha *\sen \beta}\]
Que podemos separar en:
\[\tg (\alpha + \beta) = \frac{\sen \alpha * \cos \beta
  }{\cos \alpha * \cos \beta - \sen \alpha *\sen \beta} +
  \frac{\cos \alpha*\sen \beta}{\cos \alpha * \cos \beta - \sen \alpha *\sen \beta}\]

Ahora vamos a simplificar las dos fracciones entre $\cos \alpha *\cos \beta$:
\[\tg (\alpha + \beta) = \ddfrac{
    \ddfrac{\sen \alpha * \cancel{\cos \beta}}{\cos \alpha *\cancel{\cos \beta}}}
  {\ddfrac{\cancel{\cos \alpha * \cos \beta}}{\cancel{\cos \alpha *\cos \beta}} -
    \ddfrac{\sen \alpha *\sen \beta}{\cos \alpha *\cos \beta}} +
  \ddfrac{\ddfrac{\cancel{\cos \alpha}*\sen \beta}{\cancel{\cos \alpha} *\cos \beta}}
  {\ddfrac{\cancel{\cos \alpha * \cos \beta}}{\cancel{\cos \alpha *\cos \beta}} -
    \ddfrac{\sen \alpha *\sen \beta}{\cos \alpha *\cos \beta}}\]
Que aplicando la definición de la tangente se puede escribir:
\[\tg (\alpha + \beta) = \frac{\tg \alpha}{1 - \tg \alpha * \tg \beta}
  + \frac{\tg \beta}{1 - \tg \alpha * \tg \beta}\]
Y juntándolo todo:
\[\boldsymbol{\tg (\alpha + \beta) = \frac{\tg \alpha + \tg \beta}{1 - \tg \alpha * \tg \beta}}\]
Para calcular la \textbf{tangente de la resta} utilizamos que $\tg (-\beta) = -\tg \beta$:
\[\boldsymbol{\tg (\alpha - \beta) = \frac{\tg \alpha - \tg \beta}{1 + \tg \alpha * \tg \beta}}\]\vspace{3mm}

A veces estas fórmulas vienen agrupadas de la siguiente manera:
\[\sen (\alpha \pm \beta) = \sen \alpha * \cos \beta \pm \cos \alpha *\sen \beta\]
\[\cos (\alpha \pm \beta) = \cos \alpha * \cos \beta \mp \sen \alpha *\sen \beta\]
\[\tg (\alpha \pm \beta) = \frac{\tg \alpha \pm \tg \beta}{1 \mp \tg \alpha * \tg \beta}\]
\begin{center}
  \small{\emph{(Es importante darse cuenta de que en el coseno y en la tangente algunos signos del
      miembro derecho están cambiados respecto a los del izquierdo).}}
\end{center}\vspace{1cm}

Vamos a ver un ejemplo de cómo utilizar estas fórmulas:\\
\textbf{Calcula las razones del ángulo de $\frac{5\pi}{12}$ ($75\degree$) a partir de la
  tabla de ángulos famosos.}
\begin{solution}
  Tenemos que construir el ángulo que nos piden con ángulos famosos, y es fácil llegar a que
  $\frac{5\pi}{12} = \frac{\pi}{4} + \frac{\pi}{6}$.\\
  Y ya solo tenemos que aplicar las fórmulas y operar un poco.
  \[\sen \frac{5\pi}{12} = \sen \left(\frac{\pi}{4} + \frac{\pi}{6}\right) = \sen \frac{\pi}{4}*\cos \frac{\pi}{6} +
    \cos \frac{\pi}{4}*\sen \frac{\pi}{6}\]
  \[\sen \frac{5\pi}{12} = \frac{\sqrt{2}}{2} *\frac{\sqrt{3}}{2} + \frac{\sqrt{2}}{2}*\frac{1}{2}\]
  \[\boldsymbol{\sen \frac{5\pi}{12} = \frac{\sqrt{6}+\sqrt{2}}{4}}\]\vspace{3mm}

  De la misma manera obtenemos el coseno:
  \[\cos \frac{5\pi}{12} = \cos \frac{\pi}{4}*\cos \frac{\pi}{6} -
    \sen \frac{\pi}{4}*\sen \frac{\pi}{6}\]
  \[\cos \frac{5\pi}{12} = \frac{\sqrt{2}}{2} *\frac{\sqrt{3}}{2} -
    \frac{\sqrt{2}}{2}*\frac{1}{2}\]
  \[\boldsymbol{\cos \frac{5\pi}{12} = \frac{\sqrt{6}-\sqrt{2}}{4}}\]\vspace{3mm}

  Y para la tangente vamos a utilizar la definición en vez de la fórmula y así recordamos
  racionalización y operaciones con radicales:
  \[\tg \frac{5\pi}{12} = \frac{\sen \frac{5\pi}{12}}{\cos \frac{5\pi}{12}} =
    \frac{\sqrt{6}+\sqrt{2}}{\sqrt{6}-\sqrt{2}}\]
  Recordamos que para racionalizar hay que multiplicar por el conjugado del denominador:
  \[\tg \frac{5\pi}{12} = \frac{(\sqrt{6}+\sqrt{2})^2}{(\sqrt{6}-\sqrt{2})(\sqrt{6}+\sqrt{2})}\]
  \[\tg \frac{5\pi}{12} = \frac{6 + 2 + 2\sqrt{12}}{6-2}\]
  Sacamos factores de $\sqrt{12}$:
  \[\tg \frac{5\pi}{12} = \frac{8 + 4\sqrt{3}}{4}\]
  Y al simplificar queda:
  \[\boldsymbol{\tg \frac{5\pi}{12} = 2 + \sqrt{3}}\]
  
\end{solution}

\subsection{Razones del ángulo doble.}
Con las fórmulas que acabamos de ver vamos a hacer una deducción rápida de las razones del ángulo
doble, ya que $2\alpha = \alpha + \alpha$.\\

Entonces el \textbf{seno del ángulo doble}:
\[\sen 2\alpha = \sen \alpha * \cos \alpha + \cos \alpha * \sen \alpha\]
\[\boldsymbol{\sen 2\alpha = 2\sen \alpha \cos \alpha}\]\vspace{3mm}

El \textbf{coseno del ángulo doble}:
\[\cos 2\alpha = \cos \alpha *\cos \alpha - \sen \alpha*\sen \alpha\]
\[\boldsymbol{\cos 2\alpha = \cos^2 \alpha - \sen^2 \alpha}\]\vspace{3mm}

Y la \textbf{tangente del ángulo doble}:
\[\tg 2\alpha = \frac{\tg \alpha + \tg \alpha}{1 - \tg \alpha * \tg \alpha}\]
\[\boldsymbol{\tg 2\alpha = \frac{2\tg \alpha }{1 - \tg^2 \alpha}}\]\vspace{1cm}

Al igual que con las formulas del punto anterior vamos a ver un ejemplo:\\
\textbf{Calcula las razones del ángulo $\frac{\pi}{3}$ a partir de las de $\frac{\pi}{6}$ y
  comprueba que el resultado coincide con lo que tenemos en la tabla de ángulos famosos.}
\begin{solution}
  De la tabla de ángulos famosos tenemos:
  \begin{itemize}
  \item $\sen \frac{\pi}{6} = \frac{1}{2}$
  \item $\cos \frac{\pi}{6} = \frac{\sqrt{3}}{2}$
  \item $\tg \frac{\pi}{6} = \frac{\sqrt{3}}{3}$
  \end{itemize}
  Y esto es lo que tenemos que aplicar, porque $\frac{\pi}{3} = 2*\frac{\pi}{6}$.\\

  Empezamos por el seno:
  \[\sen \frac{\pi}{3} = 2*\sen \frac{\pi}{6}*\cos \frac{\pi}{6}\]
  \[\sen \frac{\pi}{3} = 2*\frac{1}{2}*\frac{\sqrt{3}}{2}\]
  \[\boldsymbol{\sen \frac{\pi}{3} = \frac{\sqrt{3}}{2}}\]

  El coseno:
  \[\cos \frac{\pi}{3} = \cos^2 \frac{\pi}{6} - \sen^2 \frac{\pi}{6}\]
  \[\cos \frac{\pi}{3} = \left(\frac{\sqrt{3}}{2}\right)^2 -
    \left( \frac{1}{2}\right)^2\]
  \[\cos \frac{\pi}{3} = \frac{3}{4} - \frac{1}{4}\]
  \[\boldsymbol{\cos \frac{\pi}{3} = \frac{1}{2}}\]
  
  Y la tangente:
  \[\tg \frac{\pi}{3} = \frac{2\tg \frac{\pi}{6} }{1 - \tg^2 \frac{\pi}{6}}\]
  \[\tg \frac{\pi}{3} = \frac{2*\frac{\sqrt{3}}{3} }{1 - \left(\frac{\sqrt{3}}{3}\right)^2}\]
  \[\tg \frac{\pi}{3} = \frac{\frac{\sqrt{3}*2}{3} }{1 - \frac{1}{3}}\]
  \[\tg \frac{\pi}{3} = \frac{\sqrt{3}*\frac{2}{3} }{\frac{2}{3}}\]
  \[\boldsymbol{\tg \frac{\pi}{3} = \sqrt{3}}\]
\end{solution}

\subsection{Razones del ángulo mitad.}
Ahora vamos a deducir las razones de el ángulo mitad utilizando las siguiente sustitución:
\[\beta = 2\alpha\]
\[\frac{\beta}{2} = \alpha\]

Con lo que el coseno de $2\alpha$ queda:
\[\cos \beta = \cos^2 \frac{\beta}{2} - \sen^2 \frac{\beta}{2}\]
Y a partir de aquí vamos a deducir el \textbf{seno del ángulo doble}
utilizando la relación fundamental de la trigonometría:
\[\cos \beta = 1 - \sen^2 \frac{\beta}{2} - \sen^2 \frac{\beta}{2}\]
\[\cos \beta = 1 - 2\sen^2 \frac{\beta}{2}\]
Intercambiamos de miembro seno y coseno:
\[2\sen^2 \frac{\beta}{2} = 1 - \cos \beta\]
Y despejado queda:
\[\boldsymbol{\sen \frac{\beta}{2} = \pm\sqrt{\frac{1 -\cos \beta}{2}}}\]

Si en
\[\cos \beta = \cos^2 \frac{\beta}{2} - \sen^2 \frac{\beta}{2}\]
aplicamos la relación de la trigonometría en el seno vamos a obtener
\[\cos \beta = \cos^2 \frac{\beta}{2} - \left(1 - \cos^2 \frac{\beta}{2} \right)\]
que nos sirve para calcular el \textbf{coseno del ángulo mitad}:
\[\cos \beta = \cos^2 \frac{\beta}{2} - 1 + \cos^2 \frac{\beta}{2}\]
\[\cos \beta = 2\cos^2 \frac{\beta}{2} - 1\]
Y operando llegamos a
\[\boldsymbol{\cos \frac{\beta}{2} = \pm \sqrt{\frac{1+\cos \beta}{2}}}\]

Y aplicando la definición de la tangente con los dos anteriores se obtiene la
\textbf{tangente del ángulo mitad}:
\[\boldsymbol{\tg \frac{\beta}{2} = \pm \sqrt{\frac{1 - \cos \beta}{1+\cos \beta}}}\]

Podemos ver que en todas hay un $\pm$ delante de la raíz, esto nos quiere decir que tendremos que elegir el signo dependiendo del cuadrante en el que este el ángulo resultante.\\
Vamos a verlo con un ejemplo:\\
\textbf{Calcula las razones del ángulo que es la mitad de $\frac{5\pi}{3}$.}
\begin{solution}
  Primero necesitamos saber cuanto valen las razones del ángulo que nos dan, $\frac{5\pi}{3}$.\\
  Tenemos claro que ese ángulo no está en la tabla de ángulos famosos, ni es del primer cuadrante.\\
  Entonces vamos a ver de qué cuadrante es, y vemos que es mayor de $\frac{3\pi}{2}$ y menor
  que $2\pi$, con lo que está en el tercer cuadrante.\\

  Si le restamos $\frac{3\pi}{2}$ queda:
  \[\frac{5\pi}{3} - \frac{3\pi}{2} = \frac{\pi}{6}\]
  Que sí está en la tabla de ángulos famosos, con lo cual podemos obtener las razones de
  $\frac{5\pi}{3}$ aplicando las relaciones entre cuadrantes, en donde obtuvimos que:
  \begin{itemize}
  \item $\sen \left(\frac{3\pi}{2} + \alpha \right) = -\cos \alpha$
  \item $\cos \left(\frac{3\pi}{2} + \alpha \right) = \sen \alpha$
  \item $\tg \left(\frac{3\pi}{2} + \alpha \right) = -\cotg \alpha$
  \end{itemize}

  Con lo que:
  \begin{itemize}
  \item $\sen \frac{5\pi}{3} = -\cos \frac{\pi}{6} = -\frac{\sqrt{3}}{2}$
  \item $\cos \frac{5\pi}{3} = \sen \frac{\pi}{6} = \frac{1}{2}$ (realmente ésta es la única que
    necesitamos)
  \item $\tg \frac{5\pi}{3} = -\cotg \frac{\pi}{6} = -\sqrt{3}$
  \end{itemize}

  Y ahora ya podemos resolver las razones del ángulo $\frac{\frac{5\pi}{3}}{2} = \frac{5\pi}{6}$
  que es lo que nos pide el ejercicio.\\

  En los ejercicios que involucran al ángulo mitad siempre tenemos que saber en qué cuadrante
  está el ángulo para elegir los signos, y para ello comparamos con los límites de cada cuadrante.\\
  En este caso tenemos que $\frac{\pi}{2} < \frac{5\pi}{6} < \pi$, con lo que está en el
  segundo cuadrante:
  \begin{itemize}
  \item El seno es positivo.
  \item El coseno es negativo.
  \item La tangente es negativa.
  \end{itemize}
  Con lo que ya sabemos qué signo tenemos que coger en cada uno.\\

  Empezamos con el seno:
  \[\sen \frac{5\pi}{6} = \sqrt{\frac{1 - \cos \frac{5\pi}{3}}{2}}\]
  \[\sen \frac{5\pi}{6} = \sqrt{\frac{1 - \frac{1}{2}}{2}}\]
  \[\sen \frac{5\pi}{6} = \sqrt{\frac{\ \frac{1}{2}\ }{2}}\]
  \[\sen \frac{5\pi}{6} = \sqrt{\frac{1}{4}}\]
  \[\boldsymbol{\sen \frac{5\pi}{6} = \frac{1}{2}}\]

  El coseno (este es negativo)
  \[\cos \frac{5\pi}{6} = -\sqrt{\frac{1 + \cos \frac{5\pi}{3}}{2}}\]
  \[\cos \frac{5\pi}{6} = -\sqrt{\frac{1 + \frac{1}{2}}{2}}\]
  \[\cos \frac{5\pi}{6} = -\sqrt{\frac{\ \frac{3}{2}\ }{2}}\]
  \[\cos \frac{5\pi}{6} = -\sqrt{\frac{3}{4}}\]
  \[\boldsymbol{\cos \frac{5\pi}{6} = -\frac{\sqrt{3}}{2}}\]

  Y la tangente (también es negativa):
  \[\tg \frac{5\pi}{6} = -\sqrt{\frac{1 - \cos \frac{5\pi}{3}}
      {1 + \cos \frac{5\pi}{3}}}\]
  \[\tg \frac{5\pi}{6} = -\sqrt{\frac{1 - \frac{1}{2}}
      {1 + \frac{1}{2}}}\]
  \[\tg \frac{5\pi}{6} = -\sqrt{\frac{\ \frac{1}{2}\ }
      {\frac{3}{2}}}\]
  \[\tg \frac{5\pi}{6} = -\sqrt{\frac{1}{3}}\]
  Y tras racionalizar
  \[\boldsymbol{\tg \frac{5\pi}{6} = -\frac{\sqrt{3}}{3}}\]
\end{solution}

\subsection{Transformaciones de productos en sumas.}
En principio estas fórmulas pueden parecer un capricho o un intento de volver todo más farragoso
si cabe, pero no es así.\\
En niveles posteriores (o puede que en el que estemos ahora porque haya tocado repasar) veremos
algo llamado \emph{cálculo integral}, que es razonablemente sencillo de hacer cuando hay sumas
pero que se a veces es imposible cuando hay productos.\\
Es por esto que a veces es necesario poder convertir un producto en una suma, así que vamos a ver
cómo se puede hacer esto en trigonometría.\\

Empezamos por tener dos ángulos, $x$ e $y$, y hemos visto las razones de su suma y su resta.\\
Vamos a fijarnos en el coseno:
\begin{itemize}
\item $\cos (x+y) = \cos x \cos y - \sen x \sen y$
\item $\cos (x-y) = \cos x \cos y + \sen x \sen y$
\end{itemize}
Si sumamos ambas igualdades nos queda:
\[\cos (x+y) + \cos (x-y) = 2*\cos x \cos y\]
Que arreglándolo un poco queda:
\[\boldsymbol{\cos x \cos y = \frac{\cos (x+y) + \cos (x-y)}{2}}\]
Ésta es la primera transformación de producto en suma que vamos a ver.\\

Si en vez de sumar restamos las igualdades de los cosenos y lo arreglamos nos va a quedar:
\[\boldsymbol{\sen x \sen y = \frac{\cos (x-y) - \cos (x+y)}{2}}\]

Ahora vamos a hacer lo mismo pero con las fórmulas del seno de la suma y la resta:
\begin{itemize}
\item $\sen (x+y) = \sen x \cos y + \cos x \sen y$
\item $\sen (x-y) = \sen x \cos y - \cos x \sen y$
\end{itemize}
Sumamos las igualdades, las arreglamos y queda:
\[\boldsymbol{\sen x \cos y = \frac{\sen (x+y) + \sen (x-y)}{2}}\]

Y al restarlas llegamos a
\[\boldsymbol{\cos x \sen y = \frac{\sen (x+y) - \sen (x-y)}{2}}\]

Más adelante haremos algunos ejemplos en los que tendremos que utilizar estas fórmulas.

\subsection{Transformaciones de sumas en productos.}
También puede ser que nos interese hacer justamente lo contrario que en el apartado anterior,
convertir una suma en un producto.\\
Obtener estas transformaciones a partir de las vistas en el apartado anterior es sencillo, basta
con que hagamos el siguiente cambio de variables:
\begin{itemize}
\item $a = x+y$
\item $b = x-y$
\end{itemize}
Y el cambio inverso es:
\begin{itemize}
\item $x =\frac{a + b}{2}$
\item $y =\frac{a - b}{2}$
\end{itemize}
Entonces cogemos la primera transformación del apartado anterior y aplicamos el cambio:
\[\cos x \cos y = \frac{\cos (x+y) + \cos (x-y)}{2}\]
\[\cos \frac{a+b}{2} \cos \frac{a-b}{2} = \frac{\cos a + \cos b}{2}\]
\[\boldsymbol{\cos a + \cos b = 2\cos \frac{a+b}{2} \cos \frac{a-b}{2}}\]

Haciendo el mismo cambio de variables en el resto de transformaciones del apartado anterior
llegamos a:
\[\boldsymbol{\cos a - \cos b = -2\sen \frac{a+b}{2} \sen \frac{a-b}{2}}\]
\[\boldsymbol{\sen a + \sen b = 2\sen \frac{a+b}{2} \cos \frac{a-b}{2}}\]
\[\boldsymbol{\sen a - \sen b = 2\sen \frac{a-b}{2} \cos \frac{a+b}{2}}\]

\section{Recopilación.}
Por ahora hemos visto bastantes cosas y estaría bien tenerlo todo recopilado en una chuleta
o algo parecido.\\
Lo ideal sería que cada cual confeccione la suya ya que es una buena manera de estudiar. Pero
para situaciones en las que es imposible hacerse una se puede acceder a ella a través del
siguiente enlace (si el visor de PDF lo permite):
\href{https://cloud.educa.madrid.org/s/5dDbEPyXYa8WWgt}{\underline{Tablas trigonométricas}}.\\

O copiando el siguiente enlace en el navegador:
https://cloud.educa.madrid.org/s/5dDbEPyXYa8WWgt

\section{Ejercicios de ejemplo.}
Lógicamente, todo esto que hemos visto nos va a servir para resolver ejercicios.\\
El problema es que los ejercicios que se refieren a todo esto no suelen ser sencillos, hay que
tener en cuenta muchas cosas y acordarse de todas las definiciones y las fórmulas.\\

Los ejercicios de ejemplo que vamos a ver se dividen en dos tipos, \emph{identidades
  trigonométricas} y \emph{ecuaciones trigonométricas}.\\
En ambas vamos a tener una igualdad, pero en el caso de las igualdades tenemos que demostrar que
son lo mismo independientemente del valor del ángulo (o los ángulos) mientras que en el caso de
las ecuaciones hay que hallar el valor del ángulo (o los ángulos).\\

Vamos a empezar con las ecuaciones que nos suenan más y así vamos calentando.

\subsection{Calculo de razones trigonométricas.}
Vamos a ver unos cuantos ejercicios sobre cómo usar las fórmulas que hemos visto para calcular
las razones de diversos ángulos.
\begin{questions}
\question Sabiendo que $\cos 84\degree = 0.1$ calcula el resto de razones de $84\degree$ y del
  ángulo de $42\degree$.
  \begin{solution}
    Para calcular el resto de razones de $84\degree$ vamos a utilizar la relación fundamental de
    la trigonometría:
    \[\sen^2 84\degree + \cos^2 84\degree = 1\]
    \[\sen^2 84\degree + 0.01 = 1\]
    \[\sen^2 84\degree = 0.99\]
    \[\boldsymbol{\sen 84\degree \simeq 0.995}\]
    \[\boldsymbol{\tg 84\degree \simeq \frac{0.1}{0.995} \simeq 0.1005 }\]\vspace{3mm}

    Y ahora calculamos las razones de $42\degree$ utilizando las fórmulas del ángulo mitad. Como
    está en el primer cuadrante todas son positivas.
    \[\sen 42\degree = \sqrt{\frac{1 -  \cos 84\degree}{2}}\]
    \[\sen 42\degree = \sqrt{\frac{1 -  0.1}{2}}\]
    \[\sen 42\degree = \sqrt{0.45}\]
    \[\boldsymbol{\sen 42\degree \simeq 0.671}\]

    \[\cos 42\degree = \sqrt{\frac{1 +  \cos 84\degree}{2}}\]
    \[\cos 42\degree = \sqrt{\frac{1 +  0.1}{2}}\]
    \[\cos 42\degree = \sqrt{0.55}\]
    \[\boldsymbol{\cos 42\degree \simeq 0.742}\]

    Y la tangente es muy sencilla:
    \[\boldsymbol{\tg 42\degree \simeq 0.905}\]
    
  \end{solution}
\question Sabiendo que $\sen 5\degree = 0.087$ y que $\cos 20\degree = 0.9397$ calcula las razones
  de $15\degree$ y $50\degree$.
  \begin{solution}
    Está claro que vamos a tener que utilizar varias de las fórmulas que hemos visto, ya que
    \begin{itemize}
    \item $15\degree = 20\degree - 5\degree$
    \item $50\degree = 2*(20\degree + 5\degree)$
    \end{itemize}

    Como en todas esas fórmulas aparecen tanto el seno como el coseno de los dos ángulos, lo primero
    que tenemos que hacer es calcular lo que nos falta de cada uno utilizando la relación
    fundamental de la trigonometría.

    Empezamos con el coseno de $5\degree$:
    \[\cos^2 5\degree = 1 - 0.087^2\]
    \[\cos 5\degree = \sqrt{0.992431} \simeq 0.9962\]

    Y el seno de $20\degree$:
    \[\sen^2 20\degree = 1 - 0.9397^2\]
    \[\sen 20\degree \simeq 0.342\]

    Con esto ya podemos empezar a calcular las razones de los ángulos que nos piden.
    \begin{itemize}
    \item Para el ángulo de $15\degree$:\\
      \[\sen 15\degree = \sen 20\degree \cos 5\degree - \cos 20\degree sen 5\degree\]
      \[\sen 15\degree \simeq 0.342*0.9962 - 0.9397*0.087\]
      \[\boldsymbol{\sen 15\degree \simeq 0.25894}\]\vspace{1mm}
      \[\cos 15\degree = \cos 20\degree \cos 5\degree + \sen 20\degree \sen 5\degree\]
      \[\cos 15\degree \simeq 0.9397*0.9962 + 0.342*0.087\]
      \[\boldsymbol{\cos 15\degree \simeq 0.9659}\]\vspace{1mm}
      Y por último:
      \[\boldsymbol{\tg 15\degree \simeq \frac{0.25894}{0.9659} \simeq 0.2681}\]
      
    \item Para el ángulo de $50\degree$ primero necesitamos conocer el seno y el coseno de
      $25\degree$, así que eso es lo primero que tenemos que calcular:
      \[\sen 25\degree = \sen 20\degree \cos 5\degree + \cos 20\degree sen 5\degree\]
      \[\sen 25 \simeq 0.4225\]\vspace{1mm}
      \[\cos 25\degree = \cos 20\degree \cos 5\degree - \sen 20\degree \sen 5\degree\]
      \[\cos 25\degree \simeq 0.9064\]

      Y con esto ya podemos calcular las razones de $50\degree$:
      \[\sen 50\degree = 2\sen 25\degree \cos 25\degree\]
      \[\sen 50\degree \simeq 2*0.4225*0.9064\]
      \[\boldsymbol{\sen 50\degree \simeq 0.7659}\]\vspace{1mm}
      \[\cos 50\degree = \cos^2 25\degree - \sen^2 25\degree\]
      \[\cos 50\degree \simeq 0.9064^2 - 0.4225^2\]
      \[\boldsymbol{\cos 50\degree \simeq 0.6431}\]\vspace{1mm}
      \[\boldsymbol{\tg 50\degree \simeq 1.191}\]
    \end{itemize}
  \end{solution}
\end{questions}
\subsection{Ecuaciones trigonométricas.}
Antes de empezar a resolver ecuaciones trigonométricas tenemos que ver unos detalles importantes
acerca de las soluciones de las ecuaciones trigonométricas.\\

A la hora de resolver una ecuación trigonométrica, en el paso final vamos a tener que hallar el ángulo que resuelve la ecuación utilizando una razón inversa, por ejemplo:
\[\alpha = \asen 0\]

Para ello acudiremos a la calculadora, o las tablas, y obtendremos que
\[\boldsymbol{\alpha = 0}\]

Pero, ¿es 0 el único ángulo que tiene un seno que vale cero?.\\
No, sabemos que $\boldsymbol{\sen \pi = 0}$, y también $\boldsymbol{\sen 2\pi = 0}$ y si
seguimos dando vueltas vamos a obtener muchos otros ángulos cuyo seno vale cero.\\

La conclusión es que una ecuación trigonométrica no va a tener solución única, sino infinitas
soluciones. La regla es que o no tiene solución o tiene infinitas y, en el caso de que tenga
infinitas, estás están determinadas por si la última operación es un arco de seno, de coseno o de
tangente.\\

Cuando vimos la trigonometría en la circunferencia en la parte de trigonometría básica, vimos las
relaciones entre las razones de ángulos de distintos cuadrantes, y de ahí es fácil obtener que:
\begin{itemize}
\item Hay dos ángulos que tienen el \textbf{mismo seno}:
  $\boldsymbol{\alpha}$ y $\boldsymbol{\pi - \alpha}$.
\item Hay dos ángulos que tienen el \textbf{mismo coseno}:
  $\boldsymbol{\alpha}$ y $\boldsymbol{2\pi - \alpha}$ (o también $-\alpha$).
\item Hay dos ángulos que tienen la \textbf{misma tangente}: $\boldsymbol{\alpha}$ y
  $\boldsymbol{\pi + \alpha}$.
\end{itemize}
Y esto es sin dar ninguna vuelta a la circunferencia, que si se la damos tendremos que a cada
uno de los anteriores le podemos sumar $2\pi$, $4\pi$, $6\pi$, \dots, según el número de vueltas
que demos. Esto se escribe $\boldsymbol{2\pi*k}$, donde $\boldsymbol{k}$ es el \textbf{número
  de vueltas}.\\

De modo que nos queda que \textbf{las soluciones de una ecuación trigonométrica vienen dadas de
  la siguiente manera según el arco que hay que obtener}:
\begin{itemize}
\item Si es un \textbf{arco de seno} las soluciones son:
  \begin{itemize}
  \item $\boldsymbol{\alpha + 2\pi*k}$\\
    y
  \item $\boldsymbol{\pi - \alpha + 2\pi*k}$
  \end{itemize}
\item Si es un \textbf{arco de coseno} las soluciones son:
  \begin{itemize}
  \item $\boldsymbol{\alpha + 2\pi*k}$\\
    y
  \item $\boldsymbol{2\pi - \alpha + 2\pi*k}$\quad(que también se puede escribir  $\boldsymbol{-\alpha + 2\pi*k}$).
  \end{itemize}
\item Si es un \textbf{arco de tangente} las soluciones son $\boldsymbol{\alpha + \pi*k}$.
\end{itemize}

Lo anterior es la regla general, pero hay un par de situaciones que se pueden particularizar a
partir de lo anterior para obtener unas reglas más sencillas que incluyan todas las soluciones
posibles:
\begin{itemize}
\item $\boldsymbol{\asen 0 = \pi k}$.
\item $\boldsymbol{\acos 0 = \frac{\pi}{2} + \pi k}$.
\end{itemize}

Y con todo esto y las fórmulas anteriores ya podemos empezar a resolver ecuaciones trigonométricas.
\begin{questions}
\question Escribe las soluciones de la ecuación $\sen x = \frac{1}{2}$.
  \begin{solution}
    Si echamos mano a la tabla vemos que el ángulo que tiene de seno $\frac{1}{2}$ es
    $\frac{\pi}{6}$, de manera que las soluciones de la ecuación son:
    \begin{itemize}
    \item $\boldsymbol{x = \frac{\pi}{6} + 2\pi k}$.\\
      y
    \item $\boldsymbol{x = \frac{5\pi}{6} + 2\pi k}$.
    \end{itemize}
  \end{solution}
\question Resuelve la ecuación $\cos \left(2\alpha + \frac{\pi}{4} \right) = \frac{1}{2}$.
  \begin{solution}
    Hacemos lo mismo que en el anterior ya que es un coseno que tenemos en la tabla, y
    obtenemos las soluciones
    \begin{itemize}
    \item $2\alpha + \frac{\pi}{4} = \frac{\pi}{3} + 2\pi k$
    \item $2\alpha + \frac{\pi}{4} = \frac{2\pi}{3} + 2\pi k$
    \end{itemize}
    
    Pero lo que buscamos es $\alpha$, con lo que tenemos que despejar en cada una de ellas.
    \begin{itemize}
    \item Para la primera solución
      \[2\alpha = \frac{\pi}{3} - \frac{\pi}{4} + 2\pi k\]
      \[2\alpha = \frac{\pi}{12}+ 2\pi k\]
      \[\alpha = \ddfrac{\ddfrac{\pi}{12} + 2\pi k}{2}\]
      \[\boldsymbol{\alpha = \frac{\pi}{24} + \pi k}\]
    \item Y para la segunda:
      \[2\alpha = \frac{5\pi}{12}+ 2\pi k\]
      \[\boldsymbol{\alpha = \frac{5\pi}{24} + \pi k}\]
    \end{itemize}
  \end{solution}
\question Resolver la ecuación $\sen a + \frac{4}{3} \cos^2 a = \frac{3}{2}$.
  \begin{solution}
    Aquí tenemos una ecuación en la que aparecen un seno y un coseno y para poder resolverla
    haciendo un arco solo podemos tener una razón, con lo que tenemos que convertir el seno
    en coseno o al revés (o los dos en tangente, pero no es el caso).\\
    Para ello vamos a aprovechar la relación fundamental de la trigonometría:
    \[\sen^2 a + \cos^2 a = 1\]
    \[\cos^2 a = 1 - \sen^2 a\]
    Y la ecuación se convierte en:
    \[\sen a + \frac{4}{3}(1 - \sen^2 a) = \frac{3}{2}\]
    
    Para escribir menos vamos a hacer el cambio de variable $\sen a = x$, y la ecuación queda:
    \[x + \frac{4}{3}(1 - x^2) = \frac{3}{2}\]
    que es mucho más amigable para resolver.

    Quitamos paréntesis:
    \[x + \frac{4}{3} - \frac{4}{3}x^2 = \frac{3}{2}\]
    Quitamos denominadores:
    \[6x + 8 - 8x^2 = 9\]
    Reordenamos y ajustamos signos:
    \[8x^2 - 6x + 1 = 0\]
    Y las soluciones de esta ecuación de segundo grado son:
    \begin{itemize}
    \item $x = \frac{1}{2}$.
    \item $x = \frac{1}{4}$.
    \end{itemize}

    Y ahora tenemos que deshacer el cambio para obtener todas las soluciones del ángulo $a$:
    \begin{itemize}
    \item Para $x = \frac{1}{2}$
      \[a = \asen \frac{1}{2}\]
      Y por lo que hemos visto al principio de este apartado, tras recordar la tabla de
      ``ángulos famosos'':
      \begin{itemize}
      \item $\boldsymbol{a = \frac{\pi}{6} + 2\pi k}$
      \item $a = \pi - \frac{\pi}{6} + 2 \pi k$\\
        $\boldsymbol{a= \frac{5\pi}{6} + 2\pi k}$
      \end{itemize}
    \item Y para $x = \frac{1}{4}$
      \[a = \asen \frac{1}{4}\]
      Este no está en la tabla, con lo que hay que utilizar la calculadora. Y ésta nos dice que:
      \[\asen \frac{1}{4} \simeq 0.2527\quad\text{(en radianes, por eso no tienen unidad)}\]
      Con lo que las soluciones son:
      \begin{itemize}
      \item $\boldsymbol{a \simeq 0.2527 + 2\pi k}$
      \item $\boldsymbol{a \simeq \pi - 0.2527 + 2\pi k}$
      \end{itemize}
    \end{itemize}\vspace{3mm}

    Y por recopilar y tenerlas todas juntas:
    \begin{itemize}
    \item $\boldsymbol{a = \frac{\pi}{6} + 2\pi k}$
    \item $\boldsymbol{a= \frac{5\pi}{6} + 2\pi k}$
    \item $\boldsymbol{a \simeq 0.2527 + 2\pi k}$
    \item $\boldsymbol{a \simeq \pi - 0.2527 + 2\pi k}$
    \end{itemize}
  \end{solution}
\question Resuelve $\tg x + \sen 2x = 0$
  \begin{solution}
    Aquí tenemos varios problemas, tenemos dos razones distintas y dos ángulos distintos, con lo
    que lo mejor es abordar los problemas de uno en uno e ir viendo cual será el siguiente
    paso cada vez.\\

    Empezamos por convertir todo al mismo ángulo recordando que $\sen 2x = 2\sen x \cos x$:
    \[\tg x + 2\sen x \cos x = 0\]
    Aplicamos la definición de tangente:
    \[\frac{\sen x}{\cos x} + 2\sen x \cos x = 0\]
    \[\sen x + 2\sen x \cos^2 x = 0\]
    Sacamos factor común:
    \[\sen x * (1 + 2\cos^2 x) = 0\]

    Nos ha quedado un producto cuyo resultado es cero, y es solo es posible si uno de los
    factores es cero.\\
    Es decir:
    \begin{itemize}
    \item Ó $\sen x = 0$
    \item Ó $1 + 2\cos^2 x = 0$
    \end{itemize}
    Así que tenemos que resolver esas dos ecuaciones y la solución será la unión de las soluciones
    de las dos.\\
    \begin{itemize}
    \item La solución de $\sen x =0$ la hemos visto ya en la introducción de este apartado:
      \[x = \pi k\]
    \item Al despejar en la segunda llegamos a $\cos x = \sqrt{-\frac{1}{2}}$ que no podemos hacer.
    \end{itemize}
    Por tanto, las soluciones de $\tg x + \sen 2x = 0$ son $\boldsymbol{x= \pi k}$.
  \end{solution}
\question Resuelve la ecuación $\sen 2x - \cos 3x = 0$.
  \begin{solution}
    Tenemos que resolver una ecuación que a simple vista no es nada sencilla, entre otras cosas
    porque no hemos visto que hacer con las razones del ángulo triple. Pero tenemos que uno
    de los miembros de la ecuación es cero, y hay una
    operación con la que si es sencillo resolver ecuaciones que tienen 0 en uno de sus miembros,
    y es el producto ya que si $a*b =0$ tiene que ocurrir que $a$ o $b$ sean 0.\\

    Pero, ¿podemos escribir esa resta como un producto? Sí, con las transformaciones de sumas
    en productos. Así que buscamos cual nos viene bien y vemos que no hay ninguna que mezcle
    seno y coseno.\\
    Pero podemos utilizar las relaciones entre cuadrantes y convertirlo así:
    \[\sen 2x - \cos 3x = \sen 2x + \sen \left( \frac{3\pi}{2} + 3x \right)\]
    Y ahora utilizamos $\sen a + \sen b = 2\sen \frac{a+b}{2} \cos \frac{a-b}{2}$
    para convertirlo en producto:
    \[\sen 2x + \sen \left( \frac{3\pi}{2} + 3x \right) = 2\sen \dfrac{2x + 3x +\frac{3\pi}{2}}{2}
      \cos \dfrac{2x - 3x -\frac{3\pi}{2}}{2}\]

    De esta manera la ecuación queda:
    \[2\sen \dfrac{2x + 3x +\frac{3\pi}{2}}{2}
      \cos \dfrac{2x - 3x -\frac{3\pi}{2}}{2} = 0\]

    Con lo que tiene que ocurrir que el seno o el coseno sean cero para que el producto valga cero.\\
    Entonces tenemos que resolver dos ecuaciones:
    \begin{itemize}
    \item $\sen \dfrac{5x +\frac{3\pi}{2}}{2} = 0$
    \item $\cos \dfrac{-x -\frac{3\pi}{2}}{2} = 0$\\Y teniendo en cuenta que $\cos \alpha =
      \cos (-\alpha)$ podemos escribirla: $\cos \dfrac{x +\frac{3\pi}{2}}{2} = 0$
    \end{itemize}

    Que vamos a resolver por separado, y la solución a la ecuación del problema será la unión de
    las soluciones de las dos. Empezamos por la del seno.
    \begin{itemize}
    \item Aplicamos el arco de seno obteniendo:
      \[\dfrac{2x + 3x +\frac{3\pi}{2}}{2} = \asen 0\]
      y por lo que hemos visto de las soluciones al principio de este apartado (es algo que hay
      que saberse):
      \[\dfrac{5x +\frac{3\pi}{2}}{2} = \pi k\]
      \[5x +\frac{3\pi}{2} = 2\pi k\]
      \[\boldsymbol{x = \dfrac{2\pi k - \frac{3\pi}{2}}{5}}\]

    \item Y para la segunda, con el arco de coseno se tiene que:
      \[\dfrac{x +\frac{3\pi}{2}}{2} = \acos 0\]
    \[\dfrac{x +\frac{3\pi}{2}}{2} = \frac{\pi}{2} + \pi k\]
    \[x +\frac{3\pi}{2} = \pi + 2\pi k\]
    \[\boldsymbol{x = -\frac{\pi}{2} + 2\pi k}\]
  \end{itemize}

  Entonces las soluciones de $\sen 2x - \cos 3x = 0$ son
  \begin{itemize}
  \item $x = \dfrac{2\pi k - \frac{3\pi}{2}}{5}$
  \item $x = -\frac{\pi}{2} + 2\pi k$
  \end{itemize}
  \end{solution}
\question Resolver la ecuación $\sen 2x * \cos x = 6 \sen^3 x$.
  \begin{solution}
    Al igual que hemos hecho con las ecuaciones anteriores, vamos a convertir el seno de $2x$ en
    razones de $x$ aplicando la fórmula. La ecuación queda:
    \[2 \sen x \cos x *\cos x = 6 \sen^3 x\]
    \[2 \sen x \cos^2 x = 6 \sen^3 x\]
    Podemos simplificar entre $2$ y llevar todo al lado izquierdo:
    \[\sen x \cos^2 x - 3 \sen^3 x = 0\]
    \small{\emph{\textbf{IMPORTANTE}: podríamos tener la tentación de simplificar entre $\sen x$, pero
      nunca podemos simplificar una ecuación entre algo que contiene la incógnita ya que si un
      de las soluciones es cero el simplificar entre ella puede llevar a resultados inconsistentes}}\vspace{2mm}\\
  Si sacamos factor común nos queda
  \[\sen x *(\cos^2 x - 3 \sen^2 x) = 0\]
  Con lo que tenemos que un producto de dos factores tiene que ser cero y lo dividimos en dos
  ecuaciones que resolvemos por separado:
  \begin{itemize}
  \item $\sen x = 0$\\
    La solución de esta ecuación es $x = \pi k$.
  \item $\cos^2 x - 3\sen^2 x = 0$.\\
    Aplicamos la relación fundamental:
    \[1 - \sen^2 x - 3\sen^2 x = 0\]
    \[\sen^2 x = \frac{1}{4}\]
    \[\sen x = \pm \frac{1}{2}\]
    Y ahora tenemos que hacer el arco para el positivo y para el negativo:
    \begin{itemize}
    \item Para $\sen x = \frac{1}{2}$ las soluciones son $x = \frac{\pi}{6} + 2\pi k$ y $x =
      \frac{5\pi}{6} + 2\pi k$.
    \item Para $\sen x = -\frac{1}{2}$ las soluciones son $x = -\frac{\pi}{6} + 2\pi k$ y $x =
      \frac{7\pi}{6} + 2\pi k$.
    \end{itemize}
  \end{itemize}
  \end{solution}
\question Resuelve la ecuación $\cos 8x + \cos 6x = 2 \cos \frac{7\pi}{6}*\cos x$
  \begin{solution}
    Como en algún otro caso que hemos visto, nos vuelven a aparecer razones de operaciones sobre
    ángulos para las que no hemos visto fórmula, en este caso son $6x$ y $8x$.\\
    
    Tal y como hemos visto en casos anteriores, la transformación de sumas en productos a veces
    ayuda, con lo que vamos a probar a ver.\\
    En este caso tenemos una suma de cosenos, y la transformación es:
    \[\cos a + \cos b = 2\cos \frac{a+b}{2} *\cos \frac{a - b}{2}\]

    Y al aplicarla a la ecuación la convierte en:
    \[2 \cos \frac{8x + 6x}{2} *\cos \frac{8x - 6x}{2} = 2\cos \frac{7\pi}{6} \cos x\]
    \[2 \cos 7x \cos x = 2\cos \frac{7\pi}{6} \cos x\]

    Podemos simplificar el $2$, pero el $\cos x$ no porque ya hemos visto que nos puede llevar
    a resultados incorrectos, lo que hacemos es llevar todo a un lado y dejar cero al otro.
    \[\cos 7x \cos x - \cos \frac{7\pi}{6}\cos x = 0\]
    \[\cos x \left(\cos 7x - \cos \frac{7\pi}{6}\right) = 0\]
    Y volvemos a tener un producto cuyo resultado es cero.\\
    
    Lo separamos en dos ecuaciones que resolvemos por separado:
    \begin{itemize}
    \item $\cos x = 0$\\
      Sabemos que las soluciones de esta ecuación son $x = \frac{\pi}{2} + \pi k$.
    \item $\cos 7x - \cos \frac{7\pi}{6} = 0$.\\
      La reescribimos con el valor del coseno del ángulo conocido:
      \[\cos 7x - \left(-\frac{\sqrt{3}}{2} \right) = 0\]
      \[\cos 7x = -\frac{\sqrt{3}}{2}\]
      \[7x = \acos -\frac{\sqrt{3}}{2}\]
      Para hacer este arco de coseno podemos utilizar la calculadora o recordar (mirar) la tabla,
      y obtendremos que una solución es:
      \[7x = \frac{5\pi}{6} + 2 \pi k\]
      \[\boldsymbol{x = \dfrac{\frac{5\pi}{6} + 2 \pi k}{7}}\]
      Y la otra solución:
      \[7x = 2\pi - \frac{5\pi}{6} + 2 \pi k\]
      \[7x = \frac{7\pi}{6} + 2 \pi k\]
      \[\boldsymbol{x = \frac{\pi}{6} +\frac{2 \pi k}{7}}\]

      Y haciendo una compilación como solemos hacer, tenemos que las soluciones son:
      \begin{itemize}
      \item $\boldsymbol{x = \frac{\pi}{2} + 2 \pi k}$.
      \item $\boldsymbol{x = \dfrac{\frac{5\pi}{6} + 2 \pi k}{7}}$.
      \item $\boldsymbol{x = \frac{\pi}{6} +\frac{2 \pi k}{7}}$.
      \end{itemize}
    \end{itemize}
  \end{solution}
\question Indica las soluciones del la ecuación $\sen 2x - \tg x = 0$ estén en el intervalo
  $\left[0, \frac{\pi}{2}\right]$
  \begin{solution}
    Tenemos que resolver la ecuación, como en los anteriores, y quedarnos solo con las soluciones
    que estén en el primer cuadrante.\\

    Como hay un seno de $2x$ y una tangente de $x$ vamos a utilizar la fórmula del ángulo doble:
    \[2\sen x \cos x - \tg x = 0\]
    Aplicamos la definición de la tangente:
    \[2\sen x \cos x - \frac{\sen x}{\cos x} = 0\]
    \[2\sen x \cos^2 x - \sen x = 0\]
    Sacamos factor común:
    \[\sen x (2 \cos^2 x -1) = 0\]

    Y hacemos cómo el resto de veces que tenemos un producto que da de resultado cero, dividirlo
    en ecuaciones a resolver por separado:
    \begin{itemize}
    \item $\sen x = 0$\\
      Sabemos que las soluciones son $x = \pi k$, y para que esté en el intervalo indicado
      solo puede ser $\boldsymbol{x = 0}$
    \item $2\cos^2 x - 1= 0$.\\
      Despejamos:
      \[\cos^2 x = \frac{1}{2}\]
      \[\cos x = \pm \frac{\sqrt{2}}{2}\]
      Cuyas soluciones son:
      \begin{itemize}
      \item $x = \frac{\pi}{4} + 2\pi k$
      \item $x = \frac{3\pi}{4} + 2\pi k$
      \item $x = \frac{5\pi}{4} + 2\pi k$
      \item $x = \frac{7\pi}{4} + 2\pi k$
      \end{itemize}
      Y de todas esas la única que está en el intervalo que nos piden es
      $\boldsymbol{x = \frac{\pi}{4}}$
    \end{itemize}
    De modo que las soluciones de la ecuación que están en el intervalo $\left[0, \frac{\pi}{2}
    \right]$ son:
    \begin{itemize}
    \item $\boldsymbol{x = 0}$
    \item $\boldsymbol{x = \frac{\pi}{4}}$
    \end{itemize}
  \end{solution}
\question Resuelve la ecuación $\sen 3x - {4} \sen x = -1$.
  \begin{solution}
    Volvemos a tener una razón para la que no tenemos fórmula, y convertir la resta en producto
    tampoco nos sirve de mucho porque al otro lado no hay un cero.\\

    Vamos a tratar de desarrollar el seno de $3x$ utilizando lo que conocemos. Empezamos por el
    hecho de que $3x = x + 2x$:
    \[\sen 3x = \sen (x + 2x) = \sen x*\cos 2x + \cos x*\sen 2x\]
    Ahora podemos sustituir las razones de $2x$ por su fórmula:
    \[\sen 3x = \sen x(\cos^2 x - \sen^2 x) + \cos x * 2\cos x \sen x\]
    \[\sen 3x = \sen x \cos^2 x - \sen^3 x + 2\sen x \cos^2 x\]
    \[\sen 3x = 3\sen x \cos^2 x - \sen^3 x\]
    Aplicamos la relación fundamental:
    \[\sen 3x = 3\sen x (1- \sen^2 x) - \sen^3 x\]
    Y finalmente queda:
    \[\sen 3x = 3\sen x - 4\sen^3 x\]

    Aplicamos a la ecuación del enunciado y tenemos:
    \[3\sen x - 4\sen^3 x -4 \sen x = -1\]
    \[-\sen x - 4\sen^3 x = -1\]
    Que reordenándola se convierte en
    \[4\sen^3 x + \sen x - 1 = 0\]

    Para resolverla vamos a hacer el cambio $t = \sen x$, de manera que queda:
    \[4t^3 + t - 1 = 0\]
    Una ecuación cúbica que vamos a tener que resolver tanteando por Ruffini. \emph{¿Con qué valores
      tanteamos? Con los de las tablas de ángulos famosos}:
    \begin{itemize}
    \item $0$
    \item $\pm 1$
    \item $\pm \frac{1}{2}$
    \end{itemize}
    Con los de las raíces no vamos a tantear porque es excesivamente complicado para este nivel.\\

    Es fácil comprobar que el único que funciona es $\frac{1}{2}$:
    \begin{center}
      \begin{tabular}{r|rrrr}
        &$4$&$0$&$1$&$-1$\\
        $\frac{1}{2}$&&$2$&$1$&$1$\\
        \hline
        &$4$&$2$&$2$&$0$
      \end{tabular}
    \end{center}
    El polinomio que queda, $4t^2 + 2t + 1$ no tiene raíces reales.\\

    Como hemos obtenido que $t=\frac{1}{2}$ las soluciones que
    estamos buscando son las que hacen $\sen x = \frac{1}{2}$, y estas son:
    \begin{itemize}
    \item $\boldsymbol{x = \frac{\pi}{6} + 2\pi k}$
    \item $\boldsymbol{x = \frac{5\pi}{6} + 2\pi k}$
    \end{itemize}
  \end{solution}
\end{questions}
\subsubsection{Sobre la resolución de ecuaciones trigonométricas.}
A la vista de los ejemplos que hemos hecho no se puede decir que haya un método (o varios) para
resolver ecuaciones trigonométricas.\\

Lo que sí que queremos dejar claro es que hay que tener en la cabeza un par de cosas para
poder resolverlas:
\begin{itemize}
\item Acordarse bien de todo lo que hemos visto sobre resolución de ecuaciones en general: factores
  comunes, productos que dan cero, uso de Ruffini, \dots
\item Saberse razonablemente bien las fórmulas trigonométricas que hemos visto, además de
  todas las relaciones entre cuadrantes, ángulos complementarios y demás.
\end{itemize}

Aparte de todo esto está el objetivo, que es obtener una o varias ecuaciones en las que solo
aparezca una razón para poder utilizar el arco para sacar los ángulos que son solución de la
ecuación.\\

Evidentemente vamos a tener muchas ocasiones en las que el camino que hemos elegido en un principio
complica más la ecuación o no nos lleva a nada. No pasa nada por empezar de nuevo con otro camino,
de hecho es de lo más normal.\\

Y cuando hayamos practicado lo suficiente seremos capaces de anticipar el resultado al que nos
puede llevar un camino para ver si es el indicado o no.

\subsection{Identidades trigonométricas y simplificaciones.}
Quizá la palabra ``identidades'' nos traiga recuerdos, las famosas \emph{identidades notables} que
tanta lucha nos han dado.\\

Pues las identidades trigonométricas son algo parecido, son dos expresiones exactamente iguales.
La diferencia es que en lugar de aprendérnoslas (aunque con el tiempo alguna se te quedará) lo que
tenemos que hacer en estos ejercicios es demostrar que realmente son iguales.\\
Es decir, \textbf{tenemos que demostrar que lo que nos dan en el enunciado es realmente una
  identidad} utilizando las identidades que conocemos, que son todas las fórmulas que aparecen
en estos apuntes y en los de \emph{trigonometría básica}.\\

Y, ¿Cómo hacemos eso?. Pues podemos hacerlo de varias maneras:
\begin{itemize}
\item Partiendo de uno de los miembros de la igualdad llegar al otro mediante las operaciones
  que conocemos.
\item Operar con los dos miembros a la vez y comprobar que se llega al mismo resultado.
\end{itemize}
En unos casos una será la manera más rápida y en otros será la otra, pero la manera de que podamos
hacer bien estos ejercicios es practicando.\\

Tenemos que tener en cuenta que todas las fórmulas que hemos visto en el apartado \ref{fórmulas_trigonométricas} (que empieza en la página \pageref{fórmulas_trigonométricas}) no son otra cosa que
identidades, con lo que el estudiar las distintas deducciones que se hacen en ese apartado es muy útil para poder resolver bien este tipo de ejercicios.\\

En el caso de las simplificaciones vamos a tener que hacer algo parecido, pero tendremos que partir
de la expresión que nos dan porque no hay otra. Y hay que hacer lo mismo, ir haciendo operaciones
hasta que nos quede una relación trigonométrica o algo muy sencillo. Al igual que con las
identidades la única manera de saber qué hacer es practicando.\\

Así que vamos a ver unos cuantos ejemplos de cada tipo de ejercicio para que
después podamos practicar.

\begin{questions}
\question Demuestra que $\sec^2 x = 1 + \tg^2 x$
  \begin{solution}
    Esta es la identidad trigonométrica más famosa con diferencia, y la más sencilla para empezar.\\

    ¿Por dónde empezamos?, por el lado derecho que hay una operación que podemos hacer, una suma.
    Para poder hacer esa suma vamos a cambiar la tangente por su definición.
    \[\sec^2 x = 1 + \frac{\sen^2 x}{\cos^2 x}\]
    Así que hacemos la suma de fracciones:
    \[\sec^2 x = \frac{\cos^2 x}{\cos^2 x} + \frac{\sen^2 x}{\cos^2 x}\]
    \[\sec^2 x = \frac{\cos^2 x + \sen^2 x}{\cos^2 x}\]
    \[\sec^2 x = \frac{1}{\cos^2 x}\]
    Y si cambiamos la secante por su definición nos queda:
    \[\frac{1}{\cos^2 x} = \frac{1}{\cos^2 x}\]
    Con lo que tenemos \textbf{lo mismo a ambos lados del igual}, es decir, \textbf{hemos
      demostrado que es una identidad}.
  \end{solution}
\question Simplifica la expresión $\frac{2\cos \left(\frac{\pi}{4} - \alpha\right)
    \cos \left(\frac{\pi}{4} + \alpha\right)}{\cos 2\alpha}$
  \begin{solution}
    Aquí tenemos razones trigonométricas de distintos ángulos, pero en todas aparece $\alpha$.
    Así que lo que vamos a hacer es utilizar las fórmulas de la suma y la resta de ángulos, así como
    la del ángulo doble, para que todas las razones sean de $\alpha$.\\
    \[\frac{2\left(\cos \frac{\pi}{4}\cos \alpha +\sen \frac{\pi}{4}\sen \alpha\right)*
        \left(\cos \frac{\pi}{4}\cos \alpha -\sen \frac{\pi}{4}\sen \alpha\right)}
      {\cos^2 \alpha - \sen^2 \alpha}\]
    De la tabla (que ya nos sabemos) tenemos que $\sen \frac{\pi}{4} = \frac{\sqrt{2}}{2}$ y
    $\cos \frac{\pi}{4} = \frac{\sqrt{2}}{2}$, así que sustituimos:

    \[\frac{2\left(\frac{\sqrt{2}}{2}\cos \alpha +\frac{\sqrt{2}}{2}\sen \alpha\right)*
        \left(\frac{\sqrt{2}}{2}\cos \alpha -\frac{\sqrt{2}}{2}\sen \alpha\right)}
      {\cos^2 \alpha - \sen^2 \alpha}\]
    Y en el numerador tenemos una suma por diferencia, una identidad notable que nos sabemos, así
    que la aplicamos y queda:
    \[\frac{2\left(\frac{2}{4}\cos^2 \alpha - \frac{2}{4}\sen^2 \alpha\right)}
      {\cos^2 \alpha - \sen^2 \alpha}\]
    \[\frac{\cos^2 \alpha - \sen^2 \alpha}{\cos^2 \alpha - \sen^2 \alpha} = 1\]
    Con lo que la expresión del enunciado se simplifica a $\boldsymbol{1}$.
  \end{solution}
\question Comprueba la igualdad $\frac{\tg a + \tg b}{\tg a - \tg b} =
  \frac{\sen (a+b)}{\sen (a - b)}$.
  \begin{solution}
    Este ejemplo lo vamos a hacer dos veces, una partiendo del miembro izquierdo para llegar al
    derecho y otra al revés. La idea es que veamos algunas de las técnicas que podemos utilizar
    para hacer este tipo de ejercicios.
    \begin{itemize}
    \item \textbf{Empezando con el lado izquierdo}:\\
      Aplicamos la definición de tangente, y queda:
      \[\ddfrac{\ddfrac{\sen a}{\cos a} + \ddfrac{\sen b}{\cos b}}
        {\ddfrac{\sen a}{\cos a} - \ddfrac{\sen b}{\cos b}} = \frac{\sen (a+b)}{\sen (a - b)}\]
      
      \[\ddfrac{\ \ddfrac{\sen a*\cos b}{\cos a \cos b} + \ddfrac{\cos a \sen b}{\cos a \cos b}\ }
        {\ddfrac{\sen a*\cos b}{\cos a \cos b} - \ddfrac{\cos a \sen b}{\cos a \cos b}} =
        \frac{\sen (a+b)}{\sen (a - b)}\]
      
      \[\ddfrac{\ \ddfrac{\sen a*\cos b + \cos a \sen b}{\cos a \cos b}\ }
        {\ddfrac{\sen a*\cos b - \cos a \sen b}{\cos a \cos b}} =
        \frac{\sen (a+b)}{\sen (a - b)}\]
      
      \[\frac{\sen a \cos b + \cos a \sen b}{\sen a \cos b - \cos a \sen b} =
        \frac{\sen (a+b)}{\sen (a - b)}\]
      
      Y lo que tenemos en el numerador y denominador del miembro izquierdo son las fórmulas
      del seno de la suma y resta de ángulos respectivamente, con lo que la identidad ya está
      demostrada.
    \item \textbf{Empezando con el miembro derecho}:\\
      En este caso aplicamos las fórmulas del seno de la suma y de la resta de ángulos:
      \[\frac{\tg a + \tg b}{\tg a - \tg b} =
        \frac{\sen a\cos b + \cos a \sen b}{\sen a \cos b - \cos a \sen b}\]
      En el lado izquierdo tenemos tangentes, que en realidad son una fracción.
      Como nos hace falta este denominador se lo tenemos que poner en el numerador y el denominador
      de la derecha, para que siga siendo una fracción equivalente.\\
      Por tanto dividimos el numerador y el denominador de la derecha entre $\cos a \cos b$, y
      queda:
      \[\frac{\tg a + \tg b}{\tg a - \tg b} =
        \frac{\dfrac{\sen a\cos b + \cos a \sen b}{\cos a \cos b}}
        {\dfrac{\sen a \cos b - \cos a \sen b}{\cos a \cos b}}\]
      Separamos los numeradores:
      \[\frac{\tg a + \tg b}{\tg a - \tg b} =
        \frac{\dfrac{\sen a\cos b}{\cos a \cos b} + \dfrac{\cos a \sen b}{\cos a \cos b}}
        {\dfrac{\sen a \cos b}{\cos a \cos b} - \dfrac{\cos a \sen b}{\cos a \cos b}}\]
      Y simplificamos:
      \[\frac{\tg a + \tg b}{\tg a - \tg b} =
        \frac{\dfrac{\sen a}{\cos a} + \dfrac{\sen b}{\cos b}}
        {\dfrac{\sen a}{\cos a} - \dfrac{\sen b}{\cos b}}\]
      Y es fácil ver que tenemos lo mismo en ambos lados, con lo que la identidad está demostrada.
    \end{itemize}
  \end{solution}
\question Simplifica la expresión $\cos^4 y - \sen^4 y$.
  \begin{solution}
    En este caso no tenemos más remedio que hacer uso también de las identidades notables,
    y con ellas podemos escribir:
    \[\cos^4 y - \sen^4 y = \left(\cos^2 y + \sen^2 y\right)*\left(\cos^2 y - \sen^2 y\right)\]
    Teniendo en cuenta la relación fundamental de la trigonometría, queda:
    \[\left(\cos^2 y + \sen^2 y\right)*\left(\cos^2 y - \sen^2 y\right) = \cos^2 y - \sen^2 y\]
    Que no es más que la fórmula del coseno del ángulo doble, con lo que:
    \[\boldsymbol{\cos^4 y - \sen^4 y = \cos 2y}\]
  \end{solution}
\question Simplifica la expresión $(\cos x + \sen x )^2$.
  \begin{solution}
    Desarrollamos la identidad notable:
    \[(\cos x + \sen x )^2 = \cos^2 x + \sen^2 x + 2\sen x \cos x\]
    Que teniendo en cuenta la relación fundamental de la trigonometría y el seno del ángulo
    doble, resulta ser:
    \[\boldsymbol{(\cos x + \sen x )^2 = 1 + \sen 2x}\]
  \end{solution}
\question Simplifica $\frac{1 - \cos 2\alpha}{\cos^2 \alpha}$
  \begin{solution}
    Vamos a empezar por poner todo con $\alpha$:
    \[\frac{1 - \cos 2\alpha}{\cos^2 \alpha} =
      \frac{1 - \left(\cos^2 \alpha - \sen^2 \alpha\right)}{\cos^2 \alpha} =
      \frac{1 - \cos^2 \alpha + \sen^2 \alpha}{\cos^2 \alpha}\]
  
  Teniendo en cuenta la relación fundamental:
  \[\frac{1 - \cos^2 \alpha + \sen^2 \alpha}{\cos^2 \alpha} =
    \frac{\sen^2 \alpha + \sen^2 \alpha}{\cos^2 \alpha} = 2\tg^2 \alpha\]
  Con lo que la simplificación es:
  \[\boldsymbol{\frac{1 - \cos 2\alpha}{\cos^2 \alpha}  = 2\tg^2 \alpha}\]
  \end{solution}
\question Demuestra que $\sen^2 a - \cos^2 b = \sen^2 b - \cos^2 a$
  \begin{solution}
    Teniendo en cuenta la relación fundamental de la trigonometría podemos escribir:
    \[\left(1 - \cos^2 a \right) - \left( 1 - \sen^2 b \right) = \sen^2 b - \cos^2 a\]
    Quitamos los paréntesis:
    \[1 - \cos^2 a - 1 + \sen^2 b = \sen^2 b - \cos^2 a\]
    Y está claro que en ambos miembros tenemos lo mismo.
  \end{solution}
\question Comprueba que $\sen \alpha = \frac{2 \tg \dfrac{\alpha}{2}}
  {1 + \tg^2 \dfrac{\alpha}{2}}$
  \begin{solution}
    Aquí poco podemos hacer en el lado izquierdo, ya que solo tenemos el seno de un ángulo simple.
    Así que vamos a por el lado izquierdo, utilizando las fórmulas del ángulo mitad queda:
    \[\sen \alpha = \frac{\pm 2 \sqrt{\dfrac{1 - \cos \alpha}{1+\cos \alpha}}}
      {1 + \dfrac{1 - \cos \alpha}{1+\cos \alpha}} =
      \frac{\pm 2 \sqrt{\dfrac{1 - \cos \alpha}{1+\cos \alpha}}}
      {\dfrac{1+\cos \alpha}{1 + \cos \alpha} + \dfrac{1 - \cos \alpha}{1+\cos \alpha}} =
      \frac{\pm 2 \sqrt{\dfrac{1 - \cos \alpha}{1+\cos \alpha}}}
      {\dfrac{2}{1+\cos \alpha}}\]
    Simplificamos la fracción grande y queda:
    \[\sen \alpha = \pm (1+\cos \alpha)*\sqrt{\frac{1 - \cos \alpha}{1+\cos \alpha}}
      =\pm \sqrt{\frac{(1+\cos \alpha)^2 (1-\cos \alpha)}{1 + \cos \alpha}} =
      \pm \sqrt{(1 + \cos \alpha) (1 - \cos \alpha)}\]
    Utilizamos la identidad notable:
    \[\sen \alpha = \pm \sqrt{1 - \cos^2 \alpha} = \pm \sqrt{\sen^2 \alpha}\]
    Y en este caso tendríamos que es una identidad que solo se cumple a veces, cuando el seno
    del ángulo mitad sea el mismo que el del seno del ángulo. Es decir, se cumplirá siempre que
    $0 \leq \alpha \leq \pi$, ya que en otro caso el seno del ángulo mitad va a ser positivo
    mientras que el seno del ángulo va a ser negativo.
  \end{solution}
\question Demuestra la siguiente identidad: $\sen x + \cos x = \frac{1 + \tg x}{\sec x}$.
  \begin{solution}
    Vamos a empezar por utilizar la definición de la secante, $\sec x = \frac{1}{\cos x}$:
    \[\sen x + \cos x = \ddfrac{1+ \tg x}{\ddfrac{1}{\cos x}}\]
    \[\sen x + \cos x = \cos x * (1 + \tg x)\]
    \[\sen x + \cos x = \cos x + \cos x* \tg x\]
    Y teniendo en cuenta que $\tg x = \frac{\sen x}{\cos x}$ ya hemos demostrado que ambos lados
    son iguales, con lo cual es una identidad.
  \end{solution}
\question Simplifica la siguiente expresión:
  \[\frac{\sen 2a}{\sen a} + \frac{1 + \cos 2a}{\cos a}\]
  \begin{solution}
    En primer lugar vamos a sustituir las razones de los ángulos dobles por sus fórmulas:
    \[\frac{2\sen a \cos a}{\sen a} + \frac{1 + \cos^2 a - \sen^2 a}{\cos a}\]
    Teniendo en cuenta que $1 - \sen^2 a = \cos^2 a$ nos queda:
    \[2*\cos a + \frac{\cos^2 + \cos^2 a}{\cos a}\]
    \[2*\cos a + 2\cos a = \boldsymbol{4\cos a}\]
  \end{solution}
\question Comprueba si es cierta la identidad $\frac{\tg x}{\tg 2x - \tg x} = \cos 2x$.
  \begin{solution}
    Empezamos por aplicar las fórmulas del ángulo doble:
    \[\ddfrac{\tg x}{\ddfrac{2\tg x}{1-\tg^2 x} -\tg x} = \cos^2 x - \sen^2 x\]
    \[\ddfrac{\tg x}{\ddfrac{2\tg x - \tg x +\tg^3 x}{1 - \tg^2 x}} = \cos^2 x - \sen^2 x\]
    \[\left( 1 - \tg^2 x \right)*\frac{\tg x}{\tg x + \tg^3 x} = \cos^2 x - \sen^2 x\]
    \[\frac{1 - \tg^2 x}{1 + \tg^2 x} = \cos^2 x- \sen^2 x\]
    En el primer ejercicio de este punto hemos visto que $1 + \tg^2 x = \sec^2 x$, con lo que:
    \[\ddfrac{1 - \tg^2 x}{\ddfrac{1}{\cos^2 x}} = \cos^2 x - \sen^2 x\]
    \[\cos^2 x - \cos^2 x * \tg^2 x = \cos^2 x - \sen^2 x\]
    Y como $\tg^2 x = \frac{\sen^2 x}{\cos^2 x}$ acabamos obteniendo lo mismo en los dos lados, con
    lo que la identidad queda demostrada.
  \end{solution}
\end{questions}
Creemos que con los ejemplos anteriores se puede entender el mecanismo de simplificación y de
demostración de identidades trigonométricas, pero la única manera de que salgan bien es practicando.
En internet es fácil encontrar ejercicios para hacer, muchos de los cuales están con su solución.\\
Y siempre puedes preguntar a tu profesor/a por más ejercicios para practicar.
\end{document}
