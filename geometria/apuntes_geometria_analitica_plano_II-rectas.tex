\documentclass[a4paper,11pt,answers]{exam}

\usepackage{hyperref}
\usepackage{graphicx}
%\usepackage{pstricks}
\usepackage[utf8]{inputenc}
\usepackage[spanish]{babel}
\usepackage[T1]{fontenc}
%textcomp es para el símbolo del euro
\usepackage{lmodern, textcomp}
%\usepackage{tgbonum} %TeX Gyre Bonum
\usepackage[left=1in, right=1in, top=1in, bottom=1in]{geometry}
%\usepackage{mathexam}
\usepackage{amsmath}
\usepackage{amssymb}
\usepackage{multicol}
\usepackage{longtable}
%para la última página
%\usepackage{lastpage}

%Para padding en celdas
\usepackage{cellspace}
\setlength\cellspacetoplimit{1mm}
\setlength\cellspacebottomlimit{1mm}

%Para hacer tachados
\usepackage[makeroom]{cancel}

%Creative commons
%\usepackage{ccicons}
\usepackage[type={CC}, modifier={by-nc-sa}, version={4.0}, %imagemodifier={-eu-80x25},
lang={spanish}]{doclicense}

%Para las gráficas:
\usepackage{tikz}
\usepackage{pgfplots}
\pgfplotsset{compat = newest}
\usetikzlibrary{babel} %Si no da errores con algunas cosas al compilar los gráficos.
\usetikzlibrary{arrows.meta,shapes,positioning}
\usetikzlibrary{matrix}
\usepgfplotslibrary{fillbetween}
\usetikzlibrary{arrows.meta}
\usetikzlibrary{fit}
\usetikzlibrary{quotes,angles}
%\usepackage{nicematrix}

\usepackage{color,colortbl}
\definecolor{Gray}{gray}{0.9}
\newcolumntype{g}{>{\columncolor{Gray}}c}
\usepackage{arydshln} %Este pone la línea punteada en la matriz ampliada. TIENE QUE ESTAR DESPUÉS DEL colortbl porque si no casca.
%\pagestyle{headandfoot}
\pagestyle{headandfoot}
\newcommand\ExamNameLine{
\par
\vspace{\baselineskip}
Nombre:\hrulefill\relax
\par}

\renewcommand{\solutiontitle}{\noindent\textbf{Solución:}\par\noindent}

\everymath{\displaystyle}
\newcommand\ddfrac[2]{\frac{\displaystyle #1}{\displaystyle #2}}

\def \autor{Paco Andrés}
\def \titulo{Apuntes de geometría analítica en el plano II.\\Rectas.}
\def \titulofichas {\textbf {\titulo}}
\def \cursofichas {}
\def \fechaexamen {}
%\firstpageheader{\cursofichas}{\titulofichas}{\fechaexamen}
\header{\cursofichas}{\begin{small}
\titulofichas
\end{small}}{\fechaexamen}
%\header{\cursofichas}{\titulofichas}{\fechaexamen}
%\firtspagefooter{}{\thepage}{}
%Por alguna razón no sale lo del cc en el pie
\firstpagefootrule
\footrule
\footer{\autor}{\thepage}{\doclicenseIcon}
\pointpoints{punto}{puntos}

\shadedsolutions
%\definecolor{SolutionColor}{rgb}{0.99,0.99,.99}
\renewcommand{\baselinestretch}{1.3}

%Use * instead of \cdot
\mathcode`\*="8000
{\catcode`\*\active\gdef*{\cdot}} 
\newcommand{\Card}{\,\mathrm{Card}}

%For e number
\newcommand{\e}{\,\mathrm{e}}

%Para trigonometría
\newcommand{\asen}{\,\mathrm{asen}\,}
\newcommand{\acos}{\,\mathrm{acos}\,}
\newcommand{\atg}{\,\mathrm{atg}\,}
\newcommand{\degree}{^\circ}
%Para el diferencial y la integral:
\newcommand\dif[1]{\mathrm{d}#1}
\newcommand\integral[2]{\int #1\,\dif{#2}}
\newcommand\integrald[4]{\int_{#3}^{#4} #1\,\dif{#2}}
\newcommand\adjunto[1]{#1^\text{*}}
\newcommand\rango[1]{\mathrm{rg}(#1)}

%Geometría:
\newcommand\vectort[3]{#1*\vec i + #2*\vec j + #3*\vec k}
\newcommand\distancia[2]{\text{d}(#1,#2)}
%Para escribir explicaciones encima del igual:
%\newcommand\igexpl[1]{{\mathrel{\overset{\makebox{\mbox{\normalfont\tiny\sffamily $#1$}}}{=}}}}
%Parece que mejor con stackrel

%Para las unidades:
\newcommand{\unidad}[1]{\,\text{#1}}


\renewcommand{\questionlabel}{\textbf{Ejemplo \thequestion:}}

%Colores
\definecolor{gridgray}{gray}{0.7}
\pgfplotsset{grid style={color=gridgray}}

% Aumenta el interlineado en aligns y demás
\setlength{\jot}{1.5em}

\begin{document}
%Como cargo dos fuentes tengo que decirle cual usar porque si no usa la última cargada.
%\fontfamily{lmr}\selectfont 

%\author{Paco Andrés}
\title{\titulo}
\date{}
\author{\autor}
\maketitle

\begin{center}
\doclicenseLongText\\
\vspace{.25cm}
\doclicenseImage
\end{center}
\tableofcontents
\newpage

% Para quitar las sangrías que me vuelven loco.
\setlength{\parindent}{0cm}

\section{La recta. Definición.}
Junto con el punto y el plano, las rectas son los objetos fundamentales de la geometría y por esto
son complicados (o imposibles) de definir sin acudir a otros objetos que dependen de ellos.\\

Al igual que hicimos con la definición de punto vamos a dar una definición un poco cogida por
los pelos, pero que intuitivamente comprenda las propiedades de la recta que vamos a utilizar.\\

Vamos a definir la recta como \textbf{una línea en la que no hay cambio de dirección}. Es decir,
cualquier segmento recto de la línea que cojamos va a formar el mismo ángulo con la horizontal.\\

\begin{center}
  \begin{tikzpicture}[scale=0.7]
    \draw (-6, -4) -- (6,4);
    \draw[ultra thick, -Latex] (-4,-2.67) coordinate (P1) -- (-1.51,-1) coordinate (P2);
    \draw[ultra thick, -Latex] (0, 0) coordinate (P3) -- (4,2.67) coordinate (P4);
    \draw[fill] (P1) circle(2pt) node[above] {$P_1$};
    \draw[fill] (P2) circle(2pt) node[above] {$P_2$};
    \draw[fill] (P3) circle(2pt) node[above] {$P_3$};
    \draw[fill] (P4) circle(2pt) node[above] {$P_4$};
    \draw[dashed, color=darkgray] (P1) -- (-2, -2.67) coordinate (A1);
    \draw[dashed, color=darkgray] (P3) -- (2, 0) coordinate (A3);
    \pic["$\alpha$", draw=darkgray, angle eccentricity=1.2, angle radius=1cm]
    {angle=A1--P1--P2};
    \pic["$\alpha$", draw=darkgray, angle eccentricity=1.2, angle radius=1cm]
    {angle=A3--P3--P4};
    \node[text width=12cm, align=center, below] at (0,-4)
    {\small{\emph{Los segmentos se pueden representar como vectores, y se ve que al estar en
      la misma recta forman el mismo ángulo con la horizontal}}};
  \end{tikzpicture}
\end{center}

Y con esta definición vamos empezar a sacar conclusiones y así poder hacer operaciones.
\section{La ecuación vectorial de la recta.} \label{sec:ecuacion-vectorial}
En la definición hemos dicho que la recta no cambia de dirección, por tanto una de sus propiedades
es la dirección que tiene.\\
Hemos visto que la dirección se indica con la pendiente, que es la tangente del ángulo que forma
con la horizontal, pero el trabajar con ángulos es un poco engorroso.\\
Otra manera de describir la dirección es mediante un vector, porque en sus componentes está
la pendiente de manera implícita (recuerda que la pendiente en un vector venía dada por el cociente
$\frac{v_y}{v_x}$).
Entonces para describir una recta vamos a necesitar \textbf{un vector que indique la dirección} de
esta recta. A este vector se le llama \textbf{vector director de la recta} y le vamos a llamar
$\vec{v}$ por simplicidad.\\

En la definición que hemos hecho también hemos hablado de segmentos rectos que son fragmentos de
recta limitados por dos puntos, con lo que necesitamos dos puntos para dar indicar cada segmento. A estos puntos los vamos a llamar $X(x, y)$ y $P(P_x, P_y)$, también por simplicidad.\\

Como el segmento que forman $X$ y $P$ tiene la dirección del vector $\overrightarrow{PX}$, este
tiene que ser paralelo a $\vec{v}$, tiene que ser proporcional:
\[\boldsymbol{\overrightarrow{PX} = t*\vec{v}\quad \text{con } t \in \mathbb{R}}\]
\begin{center}
  \begin{tikzpicture}
    \draw (0,0) -- (6,4);
    \draw[ultra thick, -Latex] (1,0.67) coordinate (P) -- (3,2) coordinate (X)
    node[sloped, midway, above] {$t*\vec{v}$};
    \draw[ultra thick, -Latex] (1,-.33) -- (2, .33) node[sloped, midway, above] {$\vec{v}$};
    \draw[fill] (P) circle(2pt) node[above] {$P$};
    \draw[fill] (X) circle(2pt) node[above] {$X$};
  \end{tikzpicture}
\end{center}

Si tenemos en cuenta que $\overrightarrow{PX} = X -P$ nos queda:
\[X - P = t*\vec{v}\]

Entonces:
\[\boldsymbol{X = P + t*\vec{v}} \label{eq:ecuacion-vectorial}
  \tag{\small{\textbf{Ecuación vectorial de la recta}}}\]

Que se puede escribir:
\[\boldsymbol{(x,y) = (P_x, P_y) + t*\vec{v}}\]

Y lo que esta ecuación nos dice es que \textbf{si tenemos un punto $\boldsymbol{P}$ de una recta
  y un vector $\boldsymbol{\vec{v}}$ en la dirección de ésta, podemos conseguir cualquier otro
  punto de la recta sumando al punto $\boldsymbol{P}$ un vector proporcional a
  $\boldsymbol{\vec{v}}$}.\\
A punto $P$ no se le da ningún nombre en especial y ya hemos visto que al vector con la dirección
de la recta se le llama vector director.\\

Esta ecuación es muy importante ya que, como vamos a ver a continuación, es de donde salen el
resto de las ecuaciones de la recta.

\section{Las otras ecuaciones de la recta.}
Una recta tiene varias maneras de describirse matemáticamente y a cada una de esas descripciones se
la llama ecuación porque tiene una igualdad entre dos expresiones.\\

\emph{Es importante saberse todas las ecuaciones de la recta y como pasar de una a otra}, ya que
dependiendo del contexto nos convendrá usar una u otra.\\
Por ejemplo, para calcular el punto de corte de dos rectas tenemos que resolver un sistema y
dependiendo del tipo de ecuación que tengamos será más sencillo o más complicado el resolverlo.
Pero si sabemos pasar de una ecuación a otra podremos utiliza la más sencilla.\\
Y \textbf{lo fundamental} para poder hacer estas conversiones \textbf{es
  tener identificado dónde está el punto y donde está el vector director en cada ecuación}.\\


Una vez dicho esto vamos a por el resto de ecuaciones de la recta \textbf{partiendo de la
  ecuación vectorial}:
\[(x,y) = (P_x, P_y) + t*\vec{v}\]

\subsection{Ecuación paramétrica.} \label{sec:ecuacion-parametrica}
Cuando hemos practicado las operaciones con vectores hemos visto que si nos aparece una ecuación
la separamos por componentes obteniendo un sistema. Pues aquí vamos a hacer lo mismo, y nos queda:
\[
  \boldsymbol{\begin{cases}
    x & = P_x + t*v_x\\
    y &= P_y + t*v_y
  \end{cases}} \label{eq:ecuacion-parametrica} \tag{\small{\textbf{Ecuación paramétrica de la
      recta}}}
\]
Al igual que la ecuación vectorial, esta ecuación nos \textbf{permite obtener cualquier punto de la
  recta} si conocemos un punto de esta y el vector director, solo tenemos que dar
\textbf{distintos valores al parámetro $\boldsymbol{t}$}.\\
También nos permite saber si un punto está en la recta o no dependiendo de si el sistema
tiene solución o no, veremos ejemplos más adelante.\\

Y, tal y como hemos dicho antes es fundamental tener identificado el punto y el vector director\\
En esta ecuación \textbf{las componentes del vector director se encuentran multiplicando al
  parámetro y el punto sumando a esta multiplicación}.

Esta ecuación es muy útil a la hora de hacer un uso automatizado de las rectas, para utilidades
gráficas, herramientas automáticas (tornos, sierras, \dots).

\subsection{Ecuación continua.} \label{sec:ecuacion-continua}
En la ecuación paramétrica podemos despejar $t$ en ambas ecuaciones, con lo que queda:
\[
  \begin{cases}
    t &= \frac{x - P_x}{v_x}\\
    t &= \frac{y - P_y}{v_y}
  \end{cases}
\]
Utilizamos igualación y queda:
\[\boldsymbol{\frac{x - P_x}{v_x} = \frac{y - P_y}{v_y}}\label{eq:ecuacion-continua}
  \tag{\small{\textbf{Ecuación continua de la recta}}}\]
Con esta ecuación podemos conocer una de las coordenadas de un punto de la recta a partir de la
otra coordenada.\\
También se puede comprobar si un punto pertenece a la recta porque si pertenece tenemos que
obtener el mismo resultado a ambos lados del igual.\\

En esta ecuación \textbf{el punto se encuentra restando en el numerador y el vector director en
  el denominador. Cada componente está en un lado de la igualdad}.\\

Al igual que la ecuación anterior, con esta ecuación podemos obtener una coordenada a partir de la
otra, y también nos sirve para saber si un punto está en la recta porque tiene que salir lo mismo a
ambos lados.\\
Es una ecuación fácil de construir y de la que es sencillo pasar a otras formas.\\

\large{\emph{\textbf{Nota sobre la ecuación continua}}}: esta ecuación es la única situación en matemáticas en la
que se puede dejar un signo en el denominador, por ejemplo:
\[\frac{x - 3}{-2} = \frac{y + 1}{3}\]
porque así sabemos que el vector director es $\vec{v} = (-2, 3)$.\\

También es la única situación en la que puede haber un cero en el denominador, por ejemplo:
\[\frac{x}{0} = \frac{y - 2}{3}\]
cuando el vector director es $\vec{v} = (0, 3)$.

\subsection{Ecuación punto-pendiente.} \label{sec:ecuacion-puntopendiente}
Partiendo de la ecuación continua:
\[\frac{y - P_y}{v_y} = \frac{x - P_x}{v_x}\]
Pasamos $v_y$ al otro lado multiplicando:
\[y - P_y = v_y*\frac{x - P_x}{v_x}\]
Que podemos escribir así;
\[y - P_y = \frac{v_y}{v_x}*(x - P_x)\]
Y haciendo $m = \frac{v_y}{v_x}$, que es la \textbf{pendiente}, nos queda:
\[\boldsymbol{y -P_y = m*(x-P_x)}\label{eq:ecuacion-puntopendiente}
  \tag{\small{\textbf{Ecuación punto-pendiente}}}\]

En esta ecuación \textbf{el vector director se puede identificar como
  $\boldsymbol{\vec{v} = (1, m)}$ y el punto es el que está restando a la $\boldsymbol{x}$ y a
  la $\boldsymbol{y}$}.\\

Esta ecuación tiene la misma utilidad que las anteriores.\\
Es también muy fácil de construir y es de la que se parte para la forma siguiente.

\subsection{Ecuación explícita.} \label{sec:ecuacion-explicita}
Ahora partimos de la punto-pendiente:
\[y -P_y = m*(x-P_x)\]
Desarrollamos el paréntesis y despejamos la $y$:
\[y = mx -m*P_x+P_y\]
Hacemos $n = P_y - m*P_x$ y la ecuación queda:
\[\boldsymbol{y = mx + n}\label{eq:ecuacion-explicita}\tag{\small{\textbf{Ecuación explícita}}}\]
En esta ecuación $m$ es la \textbf{pendiente} y a $n$ se la llama $ordenada en el origen$.\\

En esta ecuación \textbf{el vector director de la recta es $\boldsymbol{\vec{v} = (1, m)}$
  y el punto por el que pasa es $\boldsymbol{P(0, n)}$}.\\

Esta ecuación es la más reducida de todas las ecuaciones de la recta, es la mejor cuando necesitamos
construir una tabla de valores, ya que la obtención de $y$ para distintos valores de $x$ es casi inmediata.

\subsection{Ecuación implícita.} \label{sec:ecuacion-implicita}
Partiendo de la ecuación continua:
\[\frac{y - P_y}{v_y} = \frac{x - P_x}{v_x}\]
Quitamos los denominadores
\[v_x*(y - P_y) = v_y*(x-P_x)\]
Desarrollamos y llevamos todo a un lado
\[v_xy -v_xP_y - v_yx+v_yP_x = 0\]

Hacemos
\begin{itemize}
\item $a = -v_y$
\item $b = v_x$
\item $c = v_yP_x - v_xP_y$
\end{itemize}
Y nos queda
\[\boldsymbol{ax+by+c = 0}\label{eq:ecuacion-implicita}
  \tag{\small{\textbf{Ecuación implícita}}}\]

En esta ecuación \textbf{el vector director es $\boldsymbol{\vec{v} = (-b,a)}$} (o
  $\vec{v} = (b, -a)$ porque lo que nos interesa es la dirección y el sentido
  nos da igual) \textbf{pero aquí la obtención del punto no es inmediata (daríamos un valor a una
coordenada para obtener la otra)}.\\

La mayor utilidad de esta ecuación es cuando necesitamos resolver sistemas para encontrar puntos de
corte.

Esta ecuación se puede obtener a partir de la punto-pendiente o de la explícita, pero haciendo
denominador común y quitando todos los denominadores.

\section{Nomenclatura de las rectas.}
A las rectas también se las da nombre, al igual que todo en matemáticas, ya que esto nos permite
identificar de manera unívoca a qué nos estamos refiriendo en cada situación y evitamos equívocos.\\

\textbf{Las rectas se nombran con una recta minúscula que generalmente empieza desde la $r$}, y a
continuación se escribe una de las ecuaciones, de esta manera:
\[\boldsymbol{r: \frac{y - 2}{5} = \frac{x+4}{3}}\]

Da igual qué ecuación de las anteriores sea, la que conozcamos.\\

También se puede escribir de la siguiente manera:

\[\boldsymbol{r\equiv \frac{y - 2}{5} = \frac{x+4}{3}}\]
\begin{center}
  \small{(el símbolo $\equiv$ significa equivalente, que es lo mismo.)}
\end{center}

Por ver otro ejemplo, si la ecuación de $s$ que conocemos es la paramétrica quedará:
\[s \equiv
  \begin{cases}
    x &= 3 + t\\
    y &= -1 -5t
  \end{cases}
\]
Donde solo cambia el tipo de ecuación, pero la manera de nombrarla es la misma.

\section{Rectas especiales.}
Existen dos tipos de rectas que son especiales, las \textbf{rectas horizontales} y las
\textbf{rectas verticales}.\\

Para estas dos rectas solo vamos a utilizar un tipo de ecuación que además es bastante sencilla,
pero tenemos que ver de dónde sale y sus consecuencias.

\subsection{Rectas horizontales.} \label{sec:recta-horizontal}
Una recta horizontal es aquella cuya dirección es horizontal así que \textbf{su vector director}
tiene que ser horizontal, con lo que \textbf{no puede tener componente vertical}.
\[\vec{v} = (v_x, 0)\]
Con lo que si pasa por el punto $P(P_x, P_y)$ su ecuación vectorial será:
\[(x, y) = (P_x, P_y) + t*(v_x, 0)\]
La paramétrica:
\[
  \begin{cases}
    x &= P_x + t*v_x\\
    y &= P_y
  \end{cases}
\]
Y a partir de aquí empezamos a tener problemas, ya que para obtener la ecuación continua deberíamos
despejar $t$ en ambas ecuaciones pero en la segunda no tenemos.\\
A menos que la escribamos de la siguiente manera:
\[
  \begin{cases}
    x &= P_x + t*v_x\\
    y &= P_y + t*0
  \end{cases}
\]
Y aquí ya podemos despejar $t$, aunque nos va a quedar algo un poco raro ya que aparece un cero
en el denominador:
\[
  \begin{cases}
    t &= \frac{x - P_x}{v_x}\\
    t &= \frac{y - P_y}{0}
  \end{cases}
\]
Y al pasar a la continua:
\[\frac{y - P_y}{0} = \frac{x - P_x}{v_x}\]
En general esto no tiene sentido matemáticamente, pero en la ecuación de la recta se puede hacer ya
que nos aporta información acerca del vector director.\\
Y si pasamos el cero al otro lado multiplicando queda (esto tampoco se puede hacer, salvo en el
caso de la recta):
\[y - P_y = 0*\frac{x -P_x}{v_x}\]
\[y -P_y = 0\]
\[\boldsymbol{y = P_y}\]
\emph{¿Qué quiere decir esto?}. Pues quiere decir que en la recta horizontal la $x$ puede tomar
el valor que sera, la única restricción es que la $y$ solo puede tener un valor.\\
Una explicación gráfica:
\begin{center}
  \begin{tikzpicture}
    \begin{axis}[xmin=-5, xmax=5, ymin=-3, ymax=3, ytick=\empty,
      axis x line=center, axis y line=center, grid=none]
      \draw[ultra thick] (-5, 1.5) -- (5,1.5);
      \draw[fill] (-2, 1.5) circle(2pt) node[above] {$(-2, P_y)$};
      \draw[fill] (1, 1.5) circle(2pt) node[above] {$(1, P_y)$};
      \draw[fill] (3.8, 1.5) circle(2pt) node[above] {$(3.8; P_y)$};
    \end{axis}
  \end{tikzpicture}
\end{center}

Con lo cual \textbf{la construcción de la ecuación de una recta horizontal} es muy sencilla:\\

\emph{Escribe la ecuación de la recta horizontal que pasa por el punto $P(3, -2)$.}
\begin{solution}
  Basta con hacer
  \[y = -2\]
  Y ya está.
\end{solution}

\subsection{Recta vertical.}
Haciendo un razonamiento similar al de la recta horizontal llegamos a la ecuación:
\[\boldsymbol{x = P_x}\]
Y gráficamente:
\begin{center}
  \begin{tikzpicture}
    \begin{axis}[xmin=-5, xmax=5, ymin=-5, ymax=5, xtick=\empty,
      axis x line=center, axis y line=center, grid=none]
      \draw[ultra thick] (-1.5, -5) -- (-1.5,5);
      \draw[fill] (-1.5, -3) circle(2pt) node[left] {$(P_x, -3)$};
      \draw[fill] (-1.5, 1) circle(2pt) node[left] {$(P_x, 1)$};
      \draw[fill] (-1.5, 2.5) circle(2pt) node[left] {$(P_x; 2,5)$};
    \end{axis}
  \end{tikzpicture}
\end{center}
Es decir, la coordenada vertical de un punto de la recta puede ser cualquier valor, pero la
horizontal es la misma para todos.\\

Y escribir la ecuación de una recta vertical es igual de sencillo que construir la de la recta
horizontal:\\


\emph{Escribe la ecuación de la recta vertical que pasa por el punto $P\left(-\frac{1}{2},
    \frac{2}{13} \right)$.}
\begin{solution}
  Solo tenemos que hacer:
  \[x = -\frac{1}{2}\]
\end{solution}

\section{Aplicaciones de la ecuación de la recta.}
En este apartado vamos a ver una serie de ejemplos en los que haremos uso de las distintas formas
de la ecuación de la recta.\\

Aunque no hace falta, no está de más insistir en que para entender estos ejemplos hay que haberse
estudiado todo lo anterior referente a la obtención de las distintas formas así como de la
localización del vector director y un punto por el que pasa en cada una de las ecuaciones.\\

Y sin más empezamos por los ejemplos:
\begin{questions}
\question Escribe todas las ecuaciones de la recta que pasa por los puntos $P(-1,2)$ y $Q(3,0)$.
  \begin{solution}
    Por lo que hemos visto, para escribir cualquiera de las ecuaciones de la recta necesitamos
    un punto y un vector mientras que aquí nos dan dos puntos. Pero es fácil obtener el vector
    entre esos dos puntos:
    \[\overrightarrow{PQ} = (4, -2)\]

    De este vector solo nos interesa la dirección, con lo que podemos dividirlo entre $2$ obteniendo
    un vector con la misma dirección pero con números más sencillos (\emph{esto no es obligatorio,
      pero va a hacer las cosas más sencillas}).\\
    Entonces el vector director va a ser:
    \[\vec{v} = \frac{\overrightarrow{PQ}}{2} = (2, -1)\]

    Con respecto al punto, nos da igual hacer las ecuaciones con $P$ que con $Q$. Vamos a hacerlas
    con ambos puntos a la vez y veremos que algunas acaban siendo iguales (en este caso lo lógico
    sería hacerlas con $Q$ porque es más sencillo ya que una de sus coordenadas es cero).
    \begin{center}
      \renewcommand{\arraystretch}{2}
      \begin{longtable}{cc}
        \multicolumn{2}{l}{Empezamos por la \textbf{ecuación vectorial}}\\
        \textbf{\tiny{Con $P(-1,2)$}} & \textbf{\tiny{Con $Q(3,0)$}} \\
        $\boldsymbol{(x,y) = (-1,2) + t*(2,-1)}$&$\boldsymbol{(x,y) = (3, 0) + t*(2,-1)}$ \\
        \multicolumn{2}{l}{Pasamos a la \textbf{ecuación paramétrica}:}\\
        \textbf{\tiny{Con $P(-1,2)$}} & \textbf{\tiny{Con $Q(3,0)$}} \\
        $
        \boldsymbol{\begin{cases}
          x &= -1 + 2t\\
          y&= 2-t
        \end{cases}}$ &
                       $
                       \boldsymbol{\begin{cases}
                         x &= 3 + 2t\\
                         y&= -t
                       \end{cases}}$\\
        \multicolumn{2}{l}{Despejamos $t$ e igualamos para obtener la
        \textbf{ecuación continua}.}\\
        \textbf{\tiny{Con $P(-1,2)$}} & \textbf{\tiny{Con $Q(3,0)$}} \\
        $
        \begin{cases}
          t&=\frac{x+1}{2}\\
          t &= \frac{y-2}{-1}
        \end{cases}$ &
                       $
                       \begin{cases}
                         t&=\frac{x-3}{2}\\
                         t &= \frac{y}{-1}
                       \end{cases}$\\
        $\boldsymbol{\frac{y-2}{-1} = \frac{x+1}{2}}$
                                      &$\boldsymbol{\frac{y}{-1} = \frac{x-3}{2}}$\\
        \multicolumn{2}{l}{Llevamos $v_y$ al otro lado para obtener la \textbf{ecuación
        punto-pendiente}.}\\
        \textbf{\tiny{Con $P(-1,2)$}} & \textbf{\tiny{Con $Q(3,0)$}} \\
        $\boldsymbol{y - 2 = \frac{-1}{2}*(x+1)}$&$\boldsymbol{y = \frac{-1}{2}(x-3)}$\\
        \multicolumn{2}{l}{\parbox{\columnwidth}{Desarrollamos, despejamos $y$ y reducimos para obtener la
        \textbf{ecuación explícita} (en ésta ya coinciden).}}\\
        \textbf{\tiny{Con $P(-1,2)$}} & \textbf{\tiny{Con $Q(3,0)$}} \\
        $y = -\frac{1}{2}x -\frac{1}{2} + 2$&$y = -\frac{1}{2}x + \frac{3}{2}$\\
        $\boldsymbol{y = -\frac{1}{2}x + \frac{3}{2}}$&$\boldsymbol{y = -\frac{1}{2}x + \frac{3}{2}}$\\
        \multicolumn{2}{l}{\parbox{\columnwidth}{Y para obtener la \textbf{ecuación implícita}
        volvemos a partir de la ecuación continua.}}\\
        \textbf{\tiny{Con $P(-1,2)$}} & \textbf{\tiny{Con $Q(3,0)$}} \\
        $\frac{x+1}{2} = \frac{y-2}{-1}$&$\frac{x-3}{2} = \frac{y}{-1}$\\
        $-1*(x+1) = 2*(y -2)$&$-1*(x-3) = 2*y$\\
        $-x - 1 = 2y -4$&$-x+3 = 2y$\\
        $\boldsymbol{-x - 2y + 3 = 0}$&$\boldsymbol{-x-2y+3 = 0}$\\
        $x + 2y - 3 = 0$&$x+2y-3 = 0$\\
      \end{longtable}
    \end{center}

    Si nos fijamos bien las únicas en las que coinciden son en la implícita y en la explícita
    (y si no coinciden es porque podemos simplificarlas), pero aunque en las otras no coincidan
    siguen siendo ecuaciones de la misma recta y van a dar los mismos valores.
  \end{solution}
\question Con la recta del ejercicio anterior, calcula la coordenada que falta utilizando la
  ecuación que indica en cada uno.
  \begin{parts}
  \part $x = -2$ con la continua.
  \part $y = 1$ con la paramétrica.
  \part $x = 0$ con la implícita.
  \end{parts}
  \begin{solution}
    La idea de este ejercicio es comprobar que se obtiene el mismo valor para la coordenada que
    falta utilicemos la ecuación que utilicemos y la hayamos calculado con el punto que sea,
    mientras sea de la misma recta vamos a obtener el mismo valor.
    \begin{parts}
    \part En este apartado nos falta la $y$, y nos pide que lo calculemos con la continua.\\
      Para comprobar que da lo mismo vamos a utilizar las dos ecuaciones continuas que hemos
      obtenido en el ejercicio anterior.\\
      
      Habíamos obtenido las ecuaciones $\frac{y-2}{-1} = \frac{x+1}{2}$ y
      $\frac{y}{-1} = \frac{x-3}{2} $.
      \begin{itemize}
      \item Con $\frac{x+1}{2} = \frac{y-2}{-1}$ sustituimos el valor que nos dan de la $x$ y
        despejamos la $y$:
        \[\frac{y-2}{-1} = \frac{-2+1}{2}\]
        \[-y + 2 = -\frac{1}{2} \]
        \[y = 2 + \frac{1}{2}\]
        \[\boldsymbol{y = \frac{5}{2}}\]
      \item Ahora hacemos lo mismo con $\frac{y}{-1} = \frac{x-3}{2} $:
        \[\frac{y}{-1} = \frac{-2-3}{2}\]
        \[-y = -\frac{5}{2} \]
        \[\boldsymbol{y = \frac{5}{2}}\]
      \end{itemize}
      Es decir, nos da igual qué ecuación utilicemos que el resultado va a ser el mismo.\\
      A partir de ahora solo lo haremos con una para no ser demasiado pesados.

      Y el resultado es que la recta pasa por el punto $\left(-2, \frac{5}{2}\right)$.
      
    \part Aquí nos falta la $x$ y nos pide que la obtengamos con la paramétrica. De las dos
      paramétricas que habíamos obtenido vamos a coger la más sencilla:
      \[
        \begin{cases}
          x &= 3 + 2t\\
          y&= -t
        \end{cases}
      \]
      La ecuación paramétrica no es la más adecuada para este tipo de ejercicios,
      ya que para calcular $x$ primero tenemos que obtener $t$.\\
      En este caso es sencillo, ya que al sustituir el valor del enunciado $y=1$ nos queda:
      \[
        \begin{cases}
          x &= 3 + 2t\\
          1&= -t
        \end{cases}
      \]
      Con lo que $t=-1$ y entonces
      \[x = 3 + 2*(-1)\]
      \[x = 1\]

      Entonces la recta pasa por el punto $(1,1)$.
    \part Cogemos la ecuación implícita simplificada:
      \[x+2y-3 = 0\]
      Y sustituimos el valor del enunciado $x = 0$:
      \[0 + 2y - 3 = 0\]
      \[2y= 3\]
      \[y = \frac{3}{2}\]

      Con lo que la recta pasa por el $\left(0,\frac{3}{2}\right)$.
    \end{parts}
  \end{solution}
\question Discute por cuales de estos puntos pasa la recta $r: y = 2x - 1$:
  \begin{parts}
  \part $A(1,1)$.
  \part $B\left(\frac{5}{2}, 4\right)$.
  \part $C(3, 1)$.
  \part $D\left(-\frac{5}{4}, -\frac{7}{2}\right)$.
  \end{parts}
  \begin{solution}
    Tal y como hemos indicado antes, y también como conclusión del ejercicio anterior, si un
    punto pertenece a una recta debe cumplir la ecuación de la recta.\\
    Entonces solo tendremos que sustituir cada punto y comprobar si cumple la ecuación o no.
    \begin{parts}
    \part Sustituimos el punto:
      \[1 = 2*1 - 1\]
      \[ 1 =1\]
      Con lo que cumple la ecuación, entonces $A \in r$ (el punto $A$ está en la recta $r$).
    \part Lo mismo con el punto $B$:
      \[4 = 2*\frac{5}{2} - 1\]
      \[4 = 5 - 1\]
      Y también cumple la ecuación, entonces $B \in r$.
    \part Sustituimos $C$ en la ecuación de la recta:
      \[1 = 2*3 - 1\]
      \[1 = 6 -1\]
      Y eso es falso, con lo que este punto no cumple la ecuación, así que $C \notin r$
    \part Y para $D$:
      \[-\frac{7}{2} = 2*\frac{-5}{4} - 1\]
      \[-\frac{7}{2} = -\frac{5}{2} - 1\]
      Que sí es cierto. Por tanto $D \in r$.
    \end{parts}

    Resumiendo, nos ha salido que $A$, $B$ y $D$ están en la recta, mientras que $C$ no está.
  \end{solution}
\question Dada la recta $r: y = -x + \frac{1}{2}$, escribe su ecuación paramétrica.
  \begin{solution}
    La dificultad de resolución de este ejercicio radica en que hay que saberse todas las ecuaciones
    de la recta y lo que significan los parámetros que tiene cada una de ellas.\\

    En este caso la ecuación $y = -x + \frac{1}{2}$ es la explícita (apartado
    \ref{sec:ecuacion-explicita}, página \pageref{sec:ecuacion-explicita}), que tiene la forma:
    \[y = mx + n\]
    Donde $m$ es la pendiente y $n$ es la ordenada en el origen.\\

    La ecuación que nos piden es la paramétrica (apartado \ref{sec:ecuacion-parametrica},
    página \pageref{sec:ecuacion-parametrica}) y hemos visto que los parámetros que tiene son
    el vector director y el punto por el que pasa.\\

    En el apartado de la ecuación explícita vimos que el vector director se puede construir como
    $\vec{v} = (1, m)$ y el punto por el que pasa es $P(0, n)$, con lo que el vector director
    de $r$ es $\vec{v} = (1, -1)$ y el punto es $P\left(0, \frac{1}{2}\right)$, con lo que
    sustituimos en la paramétrica y queda:
    \[r:
      \begin{cases}
        x &= 0 + 1*t\\
        y &= \frac{1}{2} + (-1)* t
      \end{cases}
    \]
    Y como sabemos que los ceros sumando y los unos multiplicando no se escriben, nos quedará:
    \[\boldsymbol{r:
        \begin{cases}
          x &= t\\
          y &= \frac{1}{2} - t
        \end{cases}}
    \]
    
    Siempre que tengamos que cambiar de ecuación de la recta tendremos que obtener el vector
    director y un punto a partir de la ecuación que nos den, y con eso ya podemos construir la que
    sea.
  \end{solution}
\question Dada la recta $s: 3x - 2y + 2 = 0$, escribe el resto de sus ecuaciones.
  \begin{solution}
    La ecuación que nos dan es la implícita (apartado \ref{sec:ecuacion-implicita}, página
    \pageref{sec:ecuacion-implicita}) y en ésta sabemos que un vector director es
    $\vec{v} = (2, 3)$. Para obtener un punto tenemos que dar un valor a una de las coordenadas,
    aunque lo normal es dar varios valores sencillos para coger el punto que salga más sencillo.\\
    Vamos a ver que nos sale con un par de intentos:
    \begin{itemize}
    \item Si hacemos $y = 0$ queda
      \[3x +2 = 0\]
      \[x = -\frac{2}{3}\]
      Con lo que el punto sería $\left(-\frac{2}{3}, 0\right)$.
      
    \item Si hacemos $x = 0$ queda:
      \[-2y + 2 = 0\]
      \[y = 1\]
      Con lo que el punto sería $(0 , 1)$.
    \end{itemize}
    Evidentemente vamos a utilizar el segundo que es mucho más sencillo, con lo que procedemos a
    construir todas las ecuaciones de la recta que tiene la dirección de $\vec{v} =(2, 3)$ y que
    pasa por el punto $(0,1)$.
    \begin{itemize}
    \item Ecuación vectorial:
      \[\boldsymbol{s: (x, y) = (0,1) + t*(2,3)}\]
    \item Ecuación paramétrica:
      \[\boldsymbol{s:
          \begin{cases}
            x &= 2t\\
            y &= 1 + 3t
          \end{cases}
        }
      \]
    \item Ecuación continua:
      \[\boldsymbol{s: \frac{y - 1}{3} = \frac{x}{2}}\]
    \item Ecuación punto-pendiente:
      \[\boldsymbol{s: y - 1 = \frac{3}{2}*x}\]
    \item Ecuación explícita:
      \[\boldsymbol{s: y = \frac{3}{2}x + 1}\]
    \end{itemize}
    La ecuación implícita no hay que construirla porque es la que nos dan en el enunciado.
  \end{solution}
\question Escribe un vector director y un punto de cada una de las siguientes rectas:
  \begin{parts}
  \part $r: y = -4x + 2$
  \part $s: 3y = 2x -5$
  \part $t: y = -1$
  \part $a:
    \begin{cases}
      x &= 5 - 2t\\
      y &=\frac{t}{3}
    \end{cases}$
  \part $b: (x, y) = (2-5t, 2t)$
  \end{parts}
  \begin{solution}
    Vamos a ver la solución de este ejercicio suponiendo que ya nos sabemos las ecuaciones de la
    recta bastante bien, incluidas las especiales.
    \begin{parts}
    \part La recta $r$ está en forma explícita, con lo que un vector director es
      $\vec{v} = (1, -4)$ y un punto por el que pasa es $(0, 2)$.
    \part La recta $s$ no tiene ninguna de las formas que hemos visto, pero si pasamos todo al
      lado izquierdo se convierte en la implícita:
      \[3y -2x + 5 = 0\]
      Con lo que un vector director es el $\vec{v} = (3, 2)$ y para obtener un punto hacemos
      $y = 1$, con lo que queda:
      \[3 - 2x + 5 = 0\]
      \[-2x = -8\]
      \[x = 4\]
      Y un punto por el que pasa es el $(4, 1)$.
    \part La recta $t$ es una recta especial, en concreto es una recta horizontal.\\
      Por lo que vimos en el apartado \ref{sec:recta-horizontal} (página
      \pageref{sec:recta-horizontal}) cualquier vector horizontal es director,
      por ejemplo $\vec{v} = (1,0)$, y como punto podemos coger cualquiera cuya coordenada
      vertical sea $-1$, por ejemplo el $(0, -1)$.
    \part Como la recta $a$ nos la dan en paramétricas es sencillo obtener lo que nos piden,
      un vector director es $\vec{v} = \left(-2, \frac{1}{3}\right)$ y pasa por el $(5, 0)$.
    \part La ecuación de la recta $b$ se parece a la vectorial pero no tiene exactamente la forma.
      Lo que hacemos es separarla de la siguiente manera:
      \[(x,y) = (2, 0) + t*(-5, 2)\]
      Y ya sí que coincide con la vectorial, y sacar de ella el punto y el vector es trivial.
    \end{parts}
  \end{solution}
\question Escribe la ecuación de la recta $s$ que es paralela a $r: y +1 = 3(x -2)$ y pasa por el
  punto $A(1, 0)$.
  \begin{solution}
    Como vimos ya en la parte de vectores, si dos direcciones son paralelas tienen la misma
    pendiente, y sabiendo esto es fácil resolver el ejercicio.\\

    El verdadero problema de este ejercicio es que hay que recordar todas las ecuaciones de las
    rectas y qué significa cada parámetro en cada una de ellas.\\

    La ecuación en la que nos dan $r$ es la \emph{ecuación punto-pendiente} (apartado
    \ref{sec:ecuacion-puntopendiente}, \pageref{sec:ecuacion-puntopendiente}),
    y en el caso de $r$ la pendiente es $m= 3$ y para el resto del ejercicio el punto no nos
    interesa.\\
    
    Con lo cual la ecuación de $s$ tiene que tener la misma pendiente pero pasando por el punto
    $A(1, 0)$, con lo que la ecuación de $s$ con las condiciones que nos piden es:
    \[\boldsymbol{s: y - 0 = 3(x - 1)}\]
    Que se puede simplificar a
    \[s: y = 3(x-1)\]

    Como no nos han pedido ninguna ecuación en concreto con esto bastaría. Si nos piden otra
    ecuación habría que pasar de la que tenemos a la que nos piden.
  \end{solution}
\question Dada la recta $r: 2x - y + 4 = 0$ escribe la ecuación continua de la perpendicular a
  $r$ que pasa por el $(1, 0)$.
  \begin{solution}
    Para escribir la ecuación que nos piden necesitamos un vector director y un punto.\\
    El punto nos lo dan en el enunciado, solo nos falta el vector director.\\

    De la recta $r$ sacamos su vector director $\vec{v}_r = (1, 2)$, así que vamos a ver un
    recordatorio de la parte de vectores sobre cómo se obtiene una dirección perpendicular
    a otra:
    \begin{quote}
      \emph{%\fontfamily{qcr}\selectfont
        Si tenemos un vector $\vec{u} = (u_x, u_y)$ los vectores
        \begin{itemize}
        \item $\vec{w}_1 = (-u_y, u_x)$
        \item $\vec{w}_2 = (u_y, -u_x)$
        \end{itemize}
        Son perpendiculares a $\vec{u}$
      }
    \end{quote}
    Con lo que cogemos como vector director de $s$ a $\vec{v}_s = (-2,1)$.\\

    Entonces ya tenemos todo para poder escribir la ecuación continua de $s \perp r$:
    \[\boldsymbol{s: \frac{y}{1} = \frac{x - 1}{-2}} \]
  \end{solution}
\end{questions}

\section{Posiciones relativas de dos rectas.}
Si tenemos dos rectas, $r$ y $s$, sabemos que pueden ser:
\begin{center}
  \begin{tikzpicture}[baseline=(current bounding box.center)]
    \draw (0,0) -- (4,3) node[above, sloped] {$r$};
    \draw (0,2) -- (3.5,1.5) node[above, sloped] {$s$};
    \node[font=\bfseries] at (2,3.5) {Secantes};
  \end{tikzpicture}
  \quad
  \begin{tikzpicture}[baseline=(current bounding box.center)]

    \draw (0,0) -- (4,3) node[above, sloped] {$r$};
    \draw (0,1) -- (4,4) node[above, sloped] {$s$};
    \node[font=\bfseries] at (2,4) {Paralelas};
  \end{tikzpicture}
  \quad
  \begin{tikzpicture}[baseline=(current bounding box.center)]

    \draw (0,0) -- (4,3) node[above, sloped] {$r$};
    \draw (0,0) -- (4,3) node[below, sloped] {$s$};
    \node[font=\bfseries] at (2,3.5) {Coincidentes};
  \end{tikzpicture}
\end{center}
Es decir:
\begin{itemize}
\item Las rectas \textbf{secantes tienen un punto en común}.
\item Las rectas \textbf{paralelas no tienen puntos en común}.
\item Las rectas \textbf{coincidentes tienen todos sus puntos en común}.
\end{itemize}

\emph{¿Cómo relacionamos esto con las ecuaciones que estamos viendo?}\\
Vamos a pensar un poco en lo que significan las ecuaciones y la relación saldrá fácil.\\

Uno de los usos que hemos dado a las ecuaciones es el de saber si un punto está en la recta o no.
Es decir, podemos considerar que la ecuación de una recta es un criterio que deben cumplir los
puntos que están en ella.\\

Cuando dos rectas son secantes tienen un punto en común, hay un punto que cumple las ecuaciones
de las dos rectas. \textbf{La manera de encontrar un punto que cumpla dos ecuaciones es resolver
  el sistema que forman estas ecuaciones}.\\

Si recordamos la teoría que vimos en los sistemas de ecuaciones lineales (porque las ecuaciones de
las rectas son lineales ya que solo tienen monomios de primer grado) teníamos que podían pasar tres cosas:
\begin{itemize}
\item El sistema tenga una solución. Sistema compatible determinado.
\item El sistema tenga infinitas soluciones. Sistema compatible indeterminado.
\item El sistema no tenga solución. Sistema incompatible.
\end{itemize}
que coinciden con las tres situaciones en las que pueden estar dos rectas:
\begin{itemize}
\item \textbf{Rectas secantes}: un punto en común que es la única solución del sistema,
  \textbf{sistema compatible determinado}.
\item \textbf{Rectas paralelas}: ningún punto en común, el sistema no tiene solución, \textbf{
    sistema incompatible}.
\item \textbf{Rectas coincidentes}: todos sus puntos en común, el sistema tiene infinitas
  soluciones, \textbf{sistema compatible indeterminado}.
\end{itemize}

Vamos a ver unos cuantos ejemplos de esto:
\begin{questions}
\question Discute la posición relativa de las siguientes rectas:
  \begin{parts}
  \part $r: -x + 2y + 3 = 0$, $s: 2x - y +3 =0$.
  \part $r: x - y +2 = 0$, $s: 2y - 2x - 3 = 0$.
  \part $r: 2x-3y +1 = 0$, $s: -4x + 6y - 2 = 0$.
  \end{parts}
  \begin{solution}
    Pues tal y como acabamos de ver, lo que tenemos que hacer es resolver el sistema que forman
    las dos rectas e interpretar el resultado.
    \begin{parts}
    \part Escribimos el sistema para resolverlo:
      \[
        \begin{cases}
          -x + 2y &= -3\\
          2x - y &= -3
        \end{cases}
      \]
      Multiplicamos la primera por dos y hacemos reducción:
      \[\left\lbrace
        \begin{array}{rrrcr}
          &-2x &+ 4y &=& -6\\
          +&&&&\\
          &2x &- y &=&-3\\
          \hline
          &&3y &=& -9
        \end{array}
        \right.
      \]
      Ya tenemos que $y= -3$ y es fácil obtener que $x = -3$, con lo que las rectas
      \textbf{son secantes} y se cortan en el punto $(-3, -3)$.
    \part Escribimos el sistema:
      \[
        \begin{cases}
          x-y &= -2\\
          -2x + 2y &= 3
        \end{cases}
      \]
      En este caso vamos a resolver por sustitución, despejamos en la primera $x = y -2$ y
      sustituimos en la segunda:
      \[-2*(y - 2) + 2y = 3\]
      \[-2y +4 + 2y = 3\]
      \[0 = -1\]
      Y sabemos que cuando nos sale algo así es porque es un \textbf{sistema incompatible}, que no
      tiene solución y por tanto \textbf{las rectas son paralelas}.
    \part Procedemos igual que en los otros dos:
      \[
        \begin{cases}
          2x-3y&= -1\\
          -4x + 6y&= 2
        \end{cases}
      \]
      Multiplicamos la primera por dos y reducimos:
      \[\left\lbrace
        \begin{array}{rrrcr}
          &4x&-6y&=&-2\\
          +&&&&\\
          &-4x&+6y&=&2\\
          \hline
          &&0&=&0
        \end{array}
        \right.
      \]
      Que quiere decir que el sistema tiene \textbf{infinitas soluciones}, con lo que son
      \textbf{rectas coincidentes}.
    \end{parts}
  \end{solution}
\end{questions}

\section{Ejemplos de ejercicios típicos.}
Gran parte de la utilidad de toda la teoría que acabamos de ver necesita de un poco de práctica
con el razonamiento espacial relacionado con el álgebra, algo que a primera vista no es evidente.\\
Además hay que tener en cuenta que hay que acordarse de bastantes cosas que se han visto en otros
momentos, como puede ser la trigonometría o la resolución de ecuaciones.\\

Por esto, y porque a veces no es evidente cómo utilizar las ecuaciones de la recta, vamos a ver
unos cuantos ejemplos de ejercicios típicos para practicar todo lo que hemos visto hasta ahora.
\begin{questions}
\question Escribe la ecuación de la bisectriz del segundo cuadrante.
  \begin{solution}
    En primer lugar vamos a recordar qué es la bisectriz de un ángulo: \emph{``La bisectriz
      de un ángulo es la recta que divide el ángulo en dos partes iguales''}.\\

    Entonces tenemos que escribir la ecuación de una recta que forme el mismo ángulo con el eje
    horizontal negativo que con el vertical positivo y que pase por el origen de coordenadas.\\
    Vamos a verlo gráficamente con todos los ángulos implicados:
    \begin{center}
      \begin{tikzpicture}
        \begin{axis}[xmin=-3, xmax=3, ymin=-3, ymax=3, ytick=\empty, xtick=\empty,
          axis x line=center, axis y line=center, grid=none]
          \draw[ultra thick] (-2.5,2.5) --(2.5, -2.5);
          \coordinate (A) at (-1,1);
          \coordinate (O) at (0,0);
          \coordinate (B) at (-1, 0);
          \coordinate (C) at (0, 1);
          \coordinate (D) at (1, 0);
          \pic["$45\degree$", draw=darkgray, angle eccentricity=1.25, angle radius=1cm]
          {angle=A--O--B};
          \pic["$45\degree$", draw=darkgray, angle eccentricity=1.25, angle radius=.8cm]
          {angle=C--O--A};
          \pic["$135\degree$", draw=black, angle eccentricity=1.25, angle radius=1.5cm]
          {angle=D--O--A};
        \end{axis}
      \end{tikzpicture}\\
      \small{\emph{(No está de más que recordemos que en todos los ejercicios
        es necesario dibujar para entender qué está pasando para saber qué tenemos que hacer)}}
    \end{center}

    Es evidente que el ángulo que tiene que formar la recta es $135\degree$,
    con lo que la pendiente de la recta tiene que ser $m = \tg 135\degree = -1$.
    Y como conocemos además un punto por el que tiene
    que pasar, el $(0,0)$ lo más lógico es utilizar la ecuación punto pendiente ya que no nos piden
    ninguna en concreto.\\
    
    La ecuación de la bisectriz es entonces:
    \[\boldsymbol{y - 0 = -1*(x - 0)}\]
    O más simple:
    \[\boldsymbol{y = -x}\]
  \end{solution}
\question Escribe la ecuación de la mediatriz del segmento que tiene por extremos los puntos
  $A(2,-1)$ y $B(4, 5)$.
  \begin{solution}
    Empecemos por recordar qué es la mediatriz: \emph{``la mediatriz de un segmento es una
      recta perpendicular a este que pasa por su punto medio''}.
    Y ahora vamos a dibujar la situación para ver qué tenemos que hacer:
    \begin{center}
      \begin{tikzpicture}
        \coordinate (A) at (2, -1);
        \coordinate (B) at (4,5);
        \coordinate (M) at (3, 2);
        \coordinate (C) at (0,3);
        \coordinate (D) at (6,1);

        \draw (A)--(B);
        \draw[fill] (A) circle(2pt) node[left] {$A$};
        \draw[fill] (B) circle(2pt) node[right] {$B$};
        \draw[fill] (M) circle(2pt) node[below left] {$M$};
        \draw[dashed] (C)--(D) node[right] {$r$};
        \pic[draw=black, angle eccentricity=.5, angle radius=.4cm, pic text=.]
        {right angle=B--M--C};
      \end{tikzpicture}
    \end{center}

    Hemos dibujado la mediatriz punteada y la hemos llamado $r$.\\
    Para escribir la ecuación de esta recta \textbf{necesitamos el punto $M$ y un vector
      en la dirección de la recta}
    (o la pendiente si queremos escribir la ecuación punto-pendiente).\\

    El punto $M$ sabemos calcularlo, hemos visto el método cuando explicábamos vectores. Y también
    podemos obtener un vector director porque sabemos calcular un vector perpendicular al vector
    $\overrightarrow{AB}$. Así que vamos a por ello.

    El punto medio se calcula con la fórmula:
    \[M = \frac{A+B}{2}\]
    Con lo cual:
    \[M = (3,2)\]

    Obtenemos el vector $\overrightarrow{AB} = (2, 6)$, con lo que un vector $\vec{v} \perp
    \overrightarrow{AB}$ es:
    \[\vec{v} = (-6,2)\]

    Y con esto podemos escribir, por ejemplo, la ecuación continua (ya que no nos piden ninguna en
    concreto):
    \[\boldsymbol{r: \frac{y - 2}{2} = \frac{x - 3}{-6}}\]

    Y esa es la ecuación de la mediatriz del segmento $\overline{AB}$.\\

    Se podría simplificar, ya que del vector director solo nos interesa la dirección, con lo que
    podríamos elegir:
    \[\vec{u} = \frac{\vec{v}}{2} = (-3, 1)\]
    y la ecuación continua quedaría así:
    \[r: \frac{y -2}{1} = \frac{x-3}{-3}\]
  \end{solution}
\question Los puntos $A(0, 1)$, $B(3, -2)$ y $C(1, 3)$ forman un triángulo. Calcula la altura sobre
  el lado $\overline{AB}$\\
  \small{(Tal y como vamos a ver en la resolución, éste es un ejercicio bastante completo, en el que
  hay que aplicar distintos conocimientos de vectores y rectas, geometría y álgebra, aparte
  de hacer razonamiento espacial).}
  \begin{solution}
    Empecemos por recordar qué es exactamente la altura de un triángulo: \emph{``una altura de
      un triángulo es la distancia de un lado al vértice opuesto siguiendo la perpendicular
      al lado''}.

    Así que la situación que tenemos es la siguiente:
    \begin{center}
      \begin{tikzpicture}
        \coordinate (A) at (0,1);
        \coordinate (B) at (3,-2);
        \coordinate (C) at (1,3);
        \coordinate (H) at (-.5, 1.5);

        \coordinate (A1) at (-1,2);
        \coordinate (B1) at (4,-3);
        \coordinate (C1) at (2,4);
        \coordinate (H1) at (-1, 1);

        \draw (A)--(B);
        \draw (A)--(C);
        \draw (B)--(C);
        \draw[thick] (H)--(C) node[midway, sloped, above] {$\boldsymbol{h}$};

        \draw[dashed] (A1)--(B1) node[below] {$r$};
        \draw[dashed] (H1)--(C1) node[above] {$s$};
        \pic[draw=black, angle eccentricity=.5, angle radius=.4cm, pic text=.]
        {right angle=A--H--C};

        \draw[fill] (A) circle(2pt) node[below left] {$A$};
        \draw[fill] (B) circle(2pt) node[below left] {$B$};
        \draw[fill] (C) circle(2pt) node[above] {$C$};
        \draw[fill] (H) circle(2pt) node[left] {$H$};
      \end{tikzpicture}
    \end{center}

    Nos piden calcular cuanto mide el segmento $h$, que es la distancia entre el punto que hemos
    llamado $H$ y el vértice $C$.\\
    Entonces necesitamos saber qué coordenadas tiene el punto $H$.\\

    Teniendo en cuenta la definición de la altura que acabamos de recordar, el punto $H$ es
    la intersección entre la recta que contiene el lado $\overline{AB}$ y la recta que contiene
    la altura sobre ese lado.\\
    Y para calcular la intersección de dos rectas tenemos que resolver el sistema que forman, con
    lo que necesitamos escribir sus ecuaciones.\\

    Vamos a empezar por la ecuación de $r$, que contiene el lado $\overline{AB}$. 
    Conocemos dos puntos, con lo que podemos calcular un vector director y ya estaría.\\
    El vector director:
    \[\overrightarrow{AB}= (3,-3)\]
    Que lo podemos simplificar a:
    \[\vec{u} = (1, -1)\]
    Y cogiendo el punto $A$ la ecuación de la recta $r$ queda:
    \[\boldsymbol{r: \frac{y - 1}{-1} = \frac{x}{1}}\]

    Para la recta $s$ tenemos que es perpendicular a $r$ y pasa por $C$.\\
    Un vector perpendicular a $r$ es:
    \[\vec{v} = (1,1)\]
    Y al pasar por $C$ su ecuación continua queda:
    \[\boldsymbol{s: \frac{y -3}{1} = \frac{x-1}{1}}\]

    Para calcular $H$ tenemos que resolver el sistema que forman las dos rectas:
    \[
      \begin{cases}
        \frac{y - 1}{-1} &= \frac{x}{1}\\
        \frac{y -3}{1} &= \frac{x-1}{1}
      \end{cases}
    \]
    Como es un sistema bastante sencillo no lo vamos a resolver aquí (además de que alargaría mucho
    el ejercicio). Tras resolverlo nos sale que el punto de corte es $\boldsymbol{
      H\left(-\frac{1}{2},
        \frac{3}{2}\right)}$.\\

    Ya solo nos queda obtener $h$, que es $h = |\overrightarrow{HC}|$, entonces:
    \[h = \left|\left(\frac{3}{2}, \frac{3}{2}\right)\right|\]
    \[h = \sqrt{\left(\frac{3}{2}\right)^2 + \left(\frac{3}{2}\right)^2}\]
    \[h = \sqrt{\frac{18}{4}}\]
    \[h = \frac{\sqrt{18}}{2}\]
    \[\boldsymbol{h = \frac{3\sqrt{2}}{2}}\]
  \end{solution}
\end{questions}
\end{document}

