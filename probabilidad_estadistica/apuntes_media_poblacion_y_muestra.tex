\documentclass[a4paper,11pt,answers]{exam}
\usepackage{graphicx}
\usepackage[utf8]{inputenc}
\usepackage[spanish]{babel}
\usepackage[T1]{fontenc}
%textcomp es para el símbolo del euro
\usepackage{lmodern, textcomp}
\usepackage[left=1in, right=1in, top=1in, bottom=1in]{geometry}
%\usepackage{mathexam}
\usepackage{amsmath}
\usepackage{amssymb}
\usepackage{multicol}
%para la última página
\usepackage{lastpage}

\usepackage{color,colortbl}
\definecolor{Gray}{gray}{0.9}
\newcolumntype{g}{>{\columncolor{Gray}}c}
%\pagestyle{headandfoot}
\pagestyle{headandfoot}
\newcommand\ExamNameLine{
\par
\vspace{\baselineskip}
Nombre:\hrulefill\relax
\par}

\renewcommand{\solutiontitle}{\noindent\textbf{Solución:}\par\noindent}

\everymath{\displaystyle}
\newcommand\ddfrac[2]{\frac{\displaystyle #1}{\displaystyle #2}}

\def \titulofichas {\textbf {Parámetros población y muestras}}
\def \cursofichas {}
\def \fechaexamen {}
%\firstpageheader{\cursofichas}{\titulofichas}{\fechaexamen}
%\runningheader{\cursofichas}{\titulofichas}{\fechaexamen}
\header{\cursofichas}{\titulofichas}{\fechaexamen}
%\firtspagefooter{}{\thepage}{}
\footer{}{\thepage}{}
\pointpoints{punto}{puntos}

\shadedsolutions
%\definecolor{SolutionColor}{rgb}{0.99,0.99,.99}
\renewcommand{\baselinestretch}{1.3}
%Use * instead of \cdot
\mathcode`\*="8000
{\catcode`\*\active\gdef*{\cdot}} 
\begin{document}
\section*{Definiciones previas}
\subsection*{Población}
La población es el conjunto total de individuos a los que se aplica el estudio.\\

Generalmente no se puede hacer el estudio sobre la población total, pero si pudiésemos hacerlo nos saldría una media y una desviación típica que vamos a llamar

\begin{itemize}
\Large
\item $\mu_p$: Media poblacional.
\item $\sigma_p$: desviación típica poblacional.
\end{itemize}

\subsection*{Muestra}
Es una muestra que cogemos de la población y que tiene que cumplir ciertas condiciones (tamaño, forma de seleccionar individuos, etc.) que garantizan que sea representativa. Estas condiciones dependerán del estudio que se esté realizando.\\

Para cada muestra obtendremos unos parámetros que serán:
\begin{itemize}
\Large
\item $\overline{x}_i$: media de la muestra $i$.
\item $\sigma_i$: desviación típica de la muestra $i$.
\end{itemize}

\subsection*{Conjunto de muestras}
Cuando se hace un estudio estadístico no solo se toma una muestra, sino que se toman un conjunto de muestras, cada una de las cuales tendrá su media ($\overline{x}_i$) y su desviación típica ($\sigma_i$).\\
Si nos fijásemos en las medias de todas las muestras veríamos que se distribuyen según una distribución normal con una media y una desviación típica que llamaremos:
\begin{itemize}
\Large
\item $\mu_m$: media de las medias de las muestras.
\item $\sigma_m$: desviación típica de las medias de las muestras.
\end{itemize}

\section*{Relación entre los parámetros. Inferencia}
\subsection*{Población y conjunto de muestras}
Lo que se busca es establecer la relación que hay entre los parámetros de la población y los de un conjunto de muestras.\\
Por lo que hemos visto en otros sitios, se tiene que:
\begin{itemize}
\Large
\item $\mu_m = \mu_p$. \normalsize Es decir la media de las medias de las muestras es igual que la media de la población.\Large
\item $\sigma_m = \frac{\sigma_p}{\sqrt{n}}$
\end{itemize}
Con lo anterior, que es lo que aparece en el teorema del límite central (así es como se llama esto), podemos obtener la media y la desviación de la distribución de las medias muestrales.\\
Si nos dan éstas últimas y nos piden la media y la desviación poblacional, simplemente tenemos que despejar.
\begin{itemize}
\item $\mu_p = \mu_m$.
\item $\sigma_p = \sigma_m * \sqrt{n}$. Simplemente hay que despejar.
\end{itemize}
\subsection*{Población y muestra}
(Esta parte no se suele dar en bachillerato. Aunque no entiendo porqué, ya que me parece sencilla y facilita la comprensión de algunas de las otras cosas)\\
A veces no es posible tener un conjunto de muestras y hay que conformarse con una sola, y de ahí inferir los parámetros para la población total.\\

Si $\overline{x}$, $\sigma$ y $n$ son la media, la desviación y el tamaño de la muestra respectivamente; la media y la desviación de la población se pueden estimar como:
\begin{itemize}
\Large
\item $\mu_p = \overline{x}$ \normalsize A falta de algo mejor.\Large
\item $\sigma_p = \sigma * \sqrt{\frac{n}{n-1}}$


\end{itemize}
\textbf{Lo anterior sería lo correcto, pero en bachillerato se da como bueno hacer}:
\begin{itemize}
\Large
\item $\mu_p = \overline{x}$ \normalsize A falta de algo mejo.\Large
\item $\sigma_p = \sigma$
\end{itemize}
Evidentemente esto coincide con lo anterior para valores de $n$ muy grandes, y por eso se da como una buena la aproximación.
\end{document}
