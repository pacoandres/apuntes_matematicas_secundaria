\documentclass[a4paper,11pt,answers]{exam}

\usepackage{hyperref}
\usepackage{graphicx}
%\usepackage{pstricks}
\usepackage[utf8]{inputenc}
\usepackage[spanish]{babel}
\usepackage[T1]{fontenc}
%textcomp es para el símbolo del euro
\usepackage{lmodern, textcomp}

\usepackage[left=1in, right=1in, top=1in, bottom=1in]{geometry}
%\usepackage{mathexam}
\usepackage{amsmath}
\usepackage{amssymb}
\usepackage{multicol}
\usepackage{longtable}
%para la última página
%\usepackage{lastpage}

%Para padding en celdas
\usepackage{cellspace}
\setlength\cellspacetoplimit{1mm}
\setlength\cellspacebottomlimit{1mm}

%Para hacer tachados
\usepackage[makeroom]{cancel}

%Creative commons
%\usepackage{ccicons}
\usepackage[type={CC}, modifier={by-nc-sa}, version={4.0}, %imagemodifier={-eu-80x25},
lang={spanish}]{doclicense}

%Para las gráficas:
\usepackage{tikz}
\usepackage{pgfplots}
\pgfplotsset{compat = newest}
\pgfplotsset{compat=1.12}
\usetikzlibrary{babel} %Si no da errores con algunas cosas al compilar los gráficos.
\usetikzlibrary{arrows.meta,shapes,positioning}
\usetikzlibrary{matrix}
\usepgfplotslibrary{fillbetween}
\usetikzlibrary{arrows.meta}
\usetikzlibrary{fit}
\usetikzlibrary{quotes,angles}
%\usepackage{nicematrix}

\usepackage{color,colortbl}
\definecolor{Gray}{gray}{0.9}
\newcolumntype{g}{>{\columncolor{Gray}}c}
\usepackage{arydshln} %Este pone la línea punteada en la matriz ampliada. TIENE QUE ESTAR DESPUÉS DEL colortbl porque si no casca.
%\pagestyle{headandfoot}
\pagestyle{headandfoot}
\newcommand\ExamNameLine{
\par
\vspace{\baselineskip}
Nombre:\hrulefill\relax
\par}

\renewcommand{\solutiontitle}{\noindent\textbf{Solución:}\par\noindent}

\everymath{\displaystyle}
\newcommand\ddfrac[2]{\frac{\displaystyle #1}{\displaystyle #2}}

\def \autor{Paco Andrés}
\def \titulo{Apuntes de geometría analítica en el plano I.\\Coordenadas y vectores.}
\def \titulofichas {\textbf {\titulo}}
\def \cursofichas {}
\def \fechaexamen {}
%\firstpageheader{\cursofichas}{\titulofichas}{\fechaexamen}
\header{\cursofichas}{\begin{small}
\titulofichas
\end{small}}{\fechaexamen}
%\header{\cursofichas}{\titulofichas}{\fechaexamen}
%\firtspagefooter{}{\thepage}{}
%Por alguna razón no sale lo del cc en el pie
\firstpagefootrule
\footrule
\footer{\autor}{\thepage}{\doclicenseIcon}
\pointpoints{punto}{puntos}

\shadedsolutions
%\definecolor{SolutionColor}{rgb}{0.99,0.99,.99}
\renewcommand{\baselinestretch}{1.3}

%Use * instead of \cdot
\mathcode`\*="8000
{\catcode`\*\active\gdef*{\cdot}} 
\newcommand{\Card}{\,\mathrm{Card}}

%For e number
\newcommand{\e}{\,\mathrm{e}}

%Para trigonometría
\newcommand{\asen}{\,\mathrm{asen}\,}
\newcommand{\acos}{\,\mathrm{acos}\,}
\newcommand{\atg}{\,\mathrm{atg}\,}
\newcommand{\degree}{^\circ}
%Para el diferencial y la integral:
\newcommand\dif[1]{\mathrm{d}#1}
\newcommand\integral[2]{\int #1\,\dif{#2}}
\newcommand\integrald[4]{\int_{#3}^{#4} #1\,\dif{#2}}
\newcommand\adjunto[1]{#1^\text{*}}
\newcommand\rango[1]{\mathrm{rg}(#1)}
\newcommand\vectort[3]{#1*\vec i + #2*\vec j + #3*\vec k}
%Para escribir explicaciones encima del igual:
%\newcommand\igexpl[1]{{\mathrel{\overset{\makebox{\mbox{\normalfont\tiny\sffamily $#1$}}}{=}}}}
%Parece que mejor con stackrel

%Para las unidades:
\newcommand{\unidad}[1]{\,\text{#1}}


\renewcommand{\questionlabel}{\textbf{Ejemplo \thequestion:}}

%Colores
\definecolor{gridgray}{gray}{0.7}
\pgfplotsset{grid style={color=gridgray}}
\begin{document}


%\author{Paco Andrés}
\title{\titulo}
\date{}
\author{\autor}
\maketitle

\begin{center}
\doclicenseLongText\\
\vspace{.25cm}
\doclicenseImage
\end{center}
\tableofcontents
\newpage

\section{Coordenadas en el plano.}

En matemáticas cada \textbf{posición} en el plano se define con \textbf{una pareja de valores}
que indican la \textbf{distancia horizontal y la altura con respecto} a un punto determinado
llamado \textbf{origen}.\\

Esta pareja de valores se escribe entre paréntesis y separada por una coma o un punto y coma
(esto último es obligatorio cuando trabajemos con decimales).\\

El primer valor siempre es la distancia horizontal y el segundo la altura, con el siguiente
criterio de signos:
\begin{itemize}
\item En horzontal: positivo indica a la derecha del origen, negativo a la izquierda.
\item En vertical: positivo indica por encima del origen, negativo por debajo.
\end{itemize}

De esta manera $(-1,2)$ indica una posición que está $1$ unidad a la izquierda del origen y
$2$ unidades por encima. Gráficamente:
\begin{center}
\begin{tikzpicture}
	\begin{axis}[xmin=-3, xmax=3, ymin=-3, ymax=3, xtick={-3, -2, - 1, 0, 1, 2, 3}, ytick={-3, -2, - 1, 0, 1, 2, 3}, axis x line=center, axis y line=center, grid=both]
		\addplot[mark=*, only marks] coordinates {(-1, 2)};
	\end{axis}
\end{tikzpicture}
\end{center}

Para hacer más sencilla la comprensión gráfica se añaden unos ejes en los que se indican las
distancias horizontales y verticales al origen (que es el punto en el que se cortan los ejes)
y, a veces, se añade una cuadrícula para mayor
simplicidad. A lo largo de este primer cuadernillo de apuntes se intentará representar todo de
esta manera salvo en situaciones en las que consideremos que el dibujo de los ejes o la cuadrícula
pueda dificultar la comprensión de lo que pretendemos explicar.

\section{El punto.}
El punto es el elemento geométrico más elemental, todo se define a partir de él.\\
El punto no tiene dimensiones, sin longitud, ni superficie ni volumen. Se corresponde con una
posición en el espacio, de manera que se indica de la misma manera que ésta pero añadiéndole un
nombre que siempre es una letra mayuscula.\\

De esta manera los puntos $A(3, 0)$, $B(-1, -1)$ y $C\left(2, -\frac{3}{2}\right)$ son:
\begin{center}
  \begin{tikzpicture}
    \begin{axis}[xmin=-4, xmax=4, ymin=-4, ymax=4, xtick={-3, -2, - 1, 0, 1, 2, 3}, ytick={-3, -2, - 1, 0, 1, 2, 3}, axis x line=center, axis y line=center, grid=both]
      % \addplot[mark=*, only marks] coordinates {(3,0) (-1,-1) (2,.5)};
      \addplot[mark=*, only marks] coordinates {(3,0)} node[above] {$A$};
      \addplot[mark=*, only marks] coordinates {(-1,-1)} node[left] {$B$};
      \addplot[mark=*, only marks] coordinates {(2,-1.5)} node[below] {$C$};
    \end{axis}
  \end{tikzpicture}
\end{center}

Dependiendo del contexto los puntos se nombran a partir de la $A$ o de la $P$, o de otra letra
si el contexto lo hace necesario. Lo importante es que es una letra mayúscula.
\section{El vector.}
En matemáticas existen múltiples definiciones de vectores, más sencillas o más complejas,
cada una de ellas prestando más o menos atención a determinadas características (que veremos más
adelante).

La definición que vamos a utilizar aquí es una definición intuitiva, basada en el movimiento:
\begin{center}
  \emph{\textbf{Un vector es el desplazamiento que hay que realizar para ir de un punto a otro.}}  
\end{center}
Y la ventaja de esta definición, además de ser fácilmente comprensible, es que lleva implicito
el mecanismo para calcular un vector, porque \textbf{¿cómo se calcula el desplazamiento de un lugar a otro?}.

Pensemos en cómo calculamos el desplazamiento para ir del 2º al 5º piso, \textbf{lo que hacemos es
  restar} $5-2 = 3$, el desplazamiento es de 3 pisos.

De esta manera el vector que nos lleva desde el punto $A$ al punto $B$ (que se escribe $\overrightarrow{AB}$, con una flecha encima) será:
\[\overrightarrow{AB} = B -A\quad\quad\text{(el desplazamiento siempre es destino menos origen)}\]
\begin{center}
  \small{(En algunos textos en lugar de la flecha lo que se hace es escribir el vector en
    negrita $\overrightarrow{AB} = \boldsymbol{AB}$, pero no es lo normal)}
\end{center}

Pero, ¿cómo podemos restar un punto a otro si tiene dos valores entre paréntesis?.\\
Lo que tenemos que hacer es no mezclar la parte horizontal con la vértical, tenemos que operarlas
siempre por separado.\\
De esta manera, si $A(A_x, A_y)$ y $B(B_x,B_y)$ el vector $\overrightarrow{AB}$ será:
\[\overrightarrow{AB} = B - A = (B_x,B_y) - (A_x,A_y) = (B_x - A_x, B_y - A_y)\]
Es decir, la parte horizontal del vector $\overrightarrow{AB}$, que se llama \textit{componente
  horizontal}, es la resta de las coordenadas horizontales de los puntos. Y lo mismo ocurre para la
componente vertical del vector.\\

A partir de esta definición es fácil darse cuenta de que $\overrightarrow{AB}$ no es el mismo vector
que $\overrightarrow{BA}$, el primero es el que va a $A$ a $B$ y el segundo al revés, de manera que
son vectores opuestos:
\begin{itemize}
\item $\overrightarrow{AB} = B-A$.
\item $\overrightarrow{BA} = A - B$.
\end{itemize}

Vamos a ver unos ejemplos y así vemos cómo se interpreta geométricamente el vector.

\begin{questions}
\question Calcula y dibuja el vector que va del punto $A(3,-2)$ al punto $B(-1, 1)$.
  \begin{solution}
    El cálculo del vector es sencillo, teniendo cuidado con los signos:
    \[\overrightarrow{AB} = (-1, 1) - (3, -2) = (-1-3, 1-(-2)) = (-4, 3)\]
    
    Y la representación de la situación:
    \begin{center}
      \begin{tikzpicture}
        \begin{axis}[xmin=-4, xmax=4, ymin=-4, ymax=4, xtick={-3, -2, - 1, 0, 1, 2, 3}, ytick={-3, -2, - 1, 0, 1, 2, 3}, axis x line=center, axis y line=center, grid=both]
          % \addplot[mark=*, only marks] coordinates {(3,0) (-1,-1) (2,.5)};
          \addplot[mark=*, only marks] coordinates {(3,-2)} node[above] {$A$};
          \addplot[mark=*, only marks] coordinates {(-1,1)} node[left] {$B$};
          \draw[arrows={-Latex[length=3mm]}] (3,-2) -- (-1,1);
        \end{axis}
      \end{tikzpicture}
    \end{center}
  \end{solution}
Ahora vamos a interpretar lo que significa $\overrightarrow{AB} = (-4,3)$ en el dibujo.\\
Si nos fijamos en la parte horizontal el vector recorre cuatro unidades (cuadritos)
hacia la izquierda,
por eso la componente horizontal es $-4$; mientras que la componente vertical recorre tres
unidades hacia arriba y esa componente es $3$ positivo.

A modo de regla tenemos que:
\begin{itemize}
\item Si la componente horizontal es positiva significa que el desplazamiento es hacia la
  derecha, mientras que si es negativo es a la izquierda.
\item Si la componente vertical es positiva significa que el desplazamiento es hacia arriba,
  mientras que si es negativa el desplazamiento es hacia abajo.
\end{itemize}

\question Calcula a qué punto se desplaza el punto $A(3,-2)$ mediante el vector
  $\overrightarrow{AB} = (-4, -1)$.
  \begin{solution}
    Si hemos visto que el vector se calcula como el destino menos el origen
    \[\overrightarrow{AB} = B-A\]
    para calcular el destino solo tenemos que despejar:
    \[B = \overrightarrow{AB} + A\]
    \[B = (-4, -1) + (3, -2)\]
    Y sumaremos como antes, horizontal con horizontal y vertical con vertical:
    \[B = (-4+ 3, -1 + (-2)) = (-1, -3)\]

    De manera gráfica para poder interpretarlo:
    \begin{center}
      \begin{tikzpicture}
        \begin{axis}[xmin=-4, xmax=4, ymin=-4, ymax=4, xtick={-3, -2, - 1, 0, 1, 2, 3}, ytick={-3, -2, - 1, 0, 1, 2, 3}, axis x line=center, axis y line=center, grid=both]
          % \addplot[mark=*, only marks] coordinates {(3,0) (-1,-1) (2,.5)};
          \addplot[mark=*, only marks] coordinates {(3,-2)} node[above] {$A$};
          \addplot[mark=*, only marks] coordinates {(-1,-3)} node[left] {$B$};
          \draw[arrows={-Latex[length=3mm]}] (3,-2) -- (-1,-3);
        \end{axis}
      \end{tikzpicture}
    \end{center}
  \end{solution}
  La interpretación gráfica es que hemos dibujado el punto $A$ en las coordenadas correspondientes,
  después hemos dibujado $\overrightarrow{AB}$ desde $A$, $4$ unidades a la izquierda y una
  hacia abajo como indican sus componentes, y en el punto en el que termina
  hemos situado $B$.
  Y vemos que el resultado gráfico de $B$ coincide con el calculado.
\end{questions}

\subsection{Propiedades de los vectores.}
Al estar compuestos de varios valores y tener el significado que tienen hace que los vectores
tengan unas propiedades fundamentales que no tiene ningún otro objeto matemático que hayamos
visto hasta ahora.\\

Estas propiedades son las siguientes:
\begin{itemize}
\item \textbf{Módulo}: que es el tamaño del vector. Es fácil ver que, geométricamente, el vector y sus componentes forman un triángulo rectángulo en el que el vector es la hiponenusa.
  \begin{center}
    \begin{tikzpicture}
      \coordinate (O) at (0,0);
      \coordinate (A) at (3,2);
      \coordinate (B) at (3,0);
      \draw[-Latex] (O)--(A) node[midway, above, sloped] {$\vec{v}$};
      \draw[dashed] (O)--(B) node[midway, below] {$v_x$};
      \draw[dashed] (B)--(A) node[midway, right] {$v_y$}; 
    \end{tikzpicture}
  \end{center}
  Entonces podemos utilizar el teorema de Pitágoras para calcular el módulo del vector, que
  se escribe poniéndolo entre dos barras verticales:
  \[|\vec{v}| = \sqrt{v_x^2 + v_y^2}\] 
\item \textbf{Dirección}: la dirección de un vector es la recta que lo contiene. Para entenderlo
  mejor podemos imaginar que la dirección de un vector es la inclinación que tiene este con
  respecto a la horizontal.\\
  Muchas veces se utiliza como ``sinónimo'' \textbf{la pendiente}, que es \textbf{la tangente del
    ángulo que forma el vector con la horizontal} y, teniendo en cuenta que el vector con sus
  componentes forma un triángulo rectángulo, la pendiente se calcula como:
  \begin{center}
    Pendiente$=\tg \alpha = \frac{v_y}{v_x}$
    \small{(Aprovechamos esta propiedad para recordar que es necesario saber
      trigonometría básica para poder aprender geometría analítica)}
  \end{center}
  De aquí se saca la conclusión de que \textbf{dos vectores que tienen la misma dirección} forman el
  mismo ángulo con la horizontal y por tanto \textbf{son paralelos}.
\item \textbf{Sentido}: el sentido del vector es hacia que lado de la dirección apunta. El sentido
  viene indicado por el signo de las componentes.
\end{itemize}

\textbf{Para que dos vectores sean iguales tienen que tener el mismo módulo, la misma dirección
  y el mismo sentido.}\\
Por eso en la definición de vectores hemos dicho que $\overrightarrow{AB}$ y $\overrightarrow{BA}$
no son el mismo vector ya que tienen igual módulo y dirección pero distinto sentido, por lo que
hemos dicho que son opuestos.

\begin{questions}
\question Calcula el módulo y el ángulo que forma con la horizontal el vector $\vec{v} = (-1,3)$.
  \begin{solution}
    Por lo que hemos visto el módulo se calcula utilizando el teorema de Pitágoras:
    \[|\vec{v}| = \sqrt{(-1)^2 + 3^2} = \sqrt{10}\]

    Y el ángulo que forma con la horizontal es el arco de tangente de su pendiente:
    \[\tg \alpha = \frac{3}{-1} = -3\]
    \[\alpha = \atg (-3) \simeq -71.57\degree = 288.43\degree\]
    El ángulo lo podemos dejar en negativo o positivo según nos convenga o nos diga el contexto.
  \end{solution}
\end{questions}
\subsection{Vectores unitarios.}
Se dice que que un vector es unitario cuando su módulo es $1$.\\
Más adelante veremos como obtener un vector unitario en la dirección y sentido de otro vector.

\subsection{Vectores ligados y vectores libres.}
Esta es una definición que depende completamente del contexto en el que estemos trabajando y no
afecta a la manera en que realicemos las operaciones.\\

Se dice que un vector es ligado cuando está asociado a un punto de origen. En el ejemplo que hemos
utilizado para la definición de vector hemos usado un vector ligado y por eso en el nombre
hemos puesto el nombre de los puntos ($\overrightarrow{AB}$).\\
El \textbf{punto en el que empieza} el vector se llama \textbf{punto de aplicación}.\\
A los vectores ligados a veces también se les llama vectores fijos.\\

Se dice que un vector es libre cuando no está asociado a un punto en concreto. Este caso sería el
que hemos utilizado para definir las propiedades básicas de los vectores y por eso lo hemos
llamado con una letra minúscula ($\vec{v}$). Cuando un vector es libre siempre se nombra con
una letra minúscula.

\section{Operaciones con vectores.}
Todas las operaciones con vectores tienen dos formas de hacerse, la analítica y la geométrica.\\

La analítica es más sencilla ya que consiste en sumas, restas, \dots Es decir, operaciones
aritméticas básicas. Pero esta manera no nos proporciona información sobre lo que está
sucediendo.\\

La geométrica es más compleja ya que es necesario dibujar todo, pero nos permite saber qué está
sucediendo y así poder interpretar el resultado o saber que operación tenemos que realizar
para conseguir un objetivo.\\

Es evidente que hay que saber y entender las dos maneras ya que el procedimiento a utilizar va a
ser el siguiente:
\begin{enumerate}
\item Dibujar la situación inicial.
\item Pensar en las operaciones geométricas que tenemos que hacer para conseguir el
  objetivo.
\item Realizar de manera analítica las operaciones del punto anterior para llegar al resultado.
\end{enumerate}

Y es por esto que en cada operación se va a explicar el cálculo analítico y la interpretación
geométrica de la operación, tanto en la explicación como en los ejemplos.

\subsection{Opuesto de un vector.}
Dos vectores son opuestos si tienen el mismo módulo y la misma dirección pero distinto sentido.\\
Para calcular el opuesto de un vector solo hay que cambiar de signo a las componentes.\\
Si tenemos $\vec{v} =(v_x, v_y)$ el opuesto será:
\[-\vec{v} = (-v_x, -v_y)\]

La interpretación geométrica es:
\begin{center}
  \begin{tikzpicture}
    \draw[-Latex] (0,0)--(2,1) node[midway, sloped, above] {$\vec{v}$};
    \draw[Latex-] (0,-1)--(2,0) node[midway, sloped, above] {$-\vec{v}$};
  \end{tikzpicture}
\end{center}
Como se puede ver ambos tienen el mismo módulo y la misma dirección, pero distinto sentido.

\begin{questions}
\question Escribe un vector $\vec{w}$ opuesto a $\vec{v} = (-2,3)$.
  \begin{solution}
    Este ejemplo es muy sencillo:
    \[\vec{w} = -\vec{v} = (2,-3)\]
  \end{solution}
\end{questions}
\subsection{Suma de punto y vector.}
Esta operación ya se ha explicado en la definición de vector, ya que está íntimamente ligada
a esta definición, pero no está de más volverla a contar, tanto de manera analítica como
geométrica.\\

Si tenemos un punto $A(A_x, A_y)$ y un vector $\vec{v}= (v_x, v_y)$ el resultado de su suma
($A + \vec{v}$) es un punto cuyas coordenadas son:
\[B(B_x,B_y) = (A_x + v_x, A_y + v_y)\]

Geométricamente es el punto que en el que acaba el vector cuando se aplica en el punto $A$:
\begin{center}
  \begin{tikzpicture}[baseline=(current bounding box.center)]
    \begin{axis}[xmin=-2, xmax=2, ymin=-3, ymax=3, axis lines =none]
      \addplot[mark=*, only marks] coordinates {(-1,-2)} node[left] {$A$};
      \addplot[draw=none] coordinates {(1,2)};
      \node at (-.5,1.5) {\textbf{1.}};
    \end{axis}
  \end{tikzpicture}
  \quad
  $\longrightarrow$
  \quad
  \begin{tikzpicture}[baseline=(current bounding box.center)]
    \begin{axis}[xmin=-2, xmax=2, ymin=-3, ymax=3, axis lines =none]
      \addplot[mark=*, only marks] coordinates {(-1,-2)} node[left] {$A$};
      \draw[arrows={-Latex[length=3mm]}] (-1,-2) -- (1,2) node[midway, above, sloped]
      {$\vec{v}$};
      \node at (-.5,1.5) {\textbf{2.}};
    \end{axis}
  \end{tikzpicture}
  \quad
  $\longrightarrow$
  \quad
  \begin{tikzpicture}[baseline=(current bounding box.center)]
    \begin{axis}[xmin=-2, xmax=2, ymin=-3, ymax=3, axis lines =none]
      \addplot[mark=*, only marks] coordinates {(-1,-2)} node[left] {$A$};
      \addplot[mark=*, only marks] coordinates {(1,2)} node[right] {$B$};
      \draw[arrows={-Latex[length=3mm]}] (-1,-2) -- (1,2) node[midway, above, sloped]
      {$\vec{v}$};
      \node at (-.5,1.5) {\textbf{3.}};
    \end{axis}
  \end{tikzpicture}
\end{center}
Tal y como indica el dibujo la operación geométrica es
\begin{enumerate}
\item Dibujamos el punto de partida.
\item Aplicamos el vector en el punto de partida avanzando en horizontal y vertical los valores
  indicados en las componentes.
\item El resultado es el punto en el que finaliza el vector.
\end{enumerate}
\subsection{Suma de vectores.}
\textbf{La suma de vectores es otro vector cuya componente horizontal es la suma de las componentes horizontales y cuya componente vertical es la suma de componenetes verticales.}\\
Es decir, si $\boldsymbol{\vec{u} = (u_x, u_y)}$ y $\boldsymbol{\vec{v} = (v_x, v_y)}$ entonces:
\[\boldsymbol{\vec{u} + \vec{v} = (u_x + v_x, u_y + v_y)}\]

Geométricamente la suma de vectores es como poner un vector a continuación del otro, el vector
resultante tiene el mismo origen que el primero y el mismo destino que el segundo.
\begin{center}
  \begin{multicols}{2}
    \begin{tikzpicture}
      % \begin{axis}
      \draw[-latex] (0,0) -- (2,3) node[midway,above,sloped] {$\vec u$};
      \draw[-latex] (2,3) -- (4,1) node[midway,above,sloped] {$\vec v$};
      \draw[ultra thick, -latex] (0,0) -- (4,1) node[midway,above,sloped] {$\vec u + \vec{v}$};
      % \end{axis}
    \end{tikzpicture}
    
    \begin{tikzpicture}
      
      \draw[-latex] (0,0) -- (2,-2) node[midway,above,sloped] {$\vec v$};
      \draw[-latex] (2,-2) -- (4,1) node[midway,above,sloped] {$\vec u$};
      \draw[ultra thick, -latex] (0,0) -- (4,1) node[midway,above,sloped] {$\vec u + \vec{v}$};
      
    \end{tikzpicture}
  \end{multicols}
\end{center}

En la represtación geométrica hemos puesto la operación en dos ordenes distintos y es fácil ver
que el resultado es el mismo, lo que quiere decir que la suma de vectores es conmutativa.\\

El igual que ocurre con los enteros, la resta de vectores es equivalente a la suma con el
opuesto del segundo.

Vamos a ver un par de ejemplos de manera analítica y gráfica.
\begin{questions}
\question Dados los vectores $\vec{u} = (3,2)$ y $\vec{v} = (-1, 2)$ calcula y dibuja
  $\vec{w} = \vec{u} + \vec{v}$.
  \begin{solution}
    El cálculo es sencillo:
    \[\vec{w} = (3,2) + (-1,2) = (2,4)\]

    Vamos a hacer ahora el cálculo geométrico, primero empezando por $\vec{u}$ y luego
    empezando por $\vec{v}$ para comprobar que sale lo mismo (cada cuadradito representa
    una unidad):

    \begin{center}
      \begin{tikzpicture}[baseline=(current bounding box.center), scale=.9]
        \begin{axis}[axis equal, xmin=-1, xmax=4, ymin=-1, ymax=5, axis lines = middle,
          axis line style={draw=gridgray}, xticklabel=\empty, yticklabel=\empty,
          tick label style={major tick length=0pt}, grid=major, ytick={-1, 0, 1, 2, 3, 4, 5}]
          \draw[-Latex] (0,0) --(3,2) node[midway, sloped, above] {$\vec{u}=(3,2)$};
          \node at (-1,4) {\textbf{1.}};
        \end{axis}
      \end{tikzpicture}
      \quad
      $\longrightarrow$
      \quad
      \begin{tikzpicture}[baseline=(current bounding box.center), scale=.9]
        \begin{axis}[axis equal, xmin=-1, xmax=4, ymin=-1, ymax=5, axis lines = middle,
          axis line style={draw=gridgray}, xticklabel=\empty, yticklabel=\empty,
          tick label style={major tick length=0pt}, grid=major, ytick={-1, 0, 1, 2, 3, 4, 5}]
          \draw[-Latex] (0,0) --(3,2) node[midway, sloped, below] {$\vec{u}=(3,2)$};
          \draw[-Latex] (3,2) -- (2,4) node[midway, sloped, above] {$\vec{v}=(-1,2)$};
          \draw[ultra thick, -Latex] (0,0) -- (2,4) node[midway, sloped, above] {$\vec{w}$};
          \node at (-1,4) {\textbf{2.}};
        \end{axis}
      \end{tikzpicture}
    \end{center}

    Hemos dibujado primero $\vec{u}$ (tres unidades a la derecha y dos hacia arriba) y a
    continuación $\vec{v}$ (una unidad a la izquierda y dos hacia arriba). El resultado es
    el vector que une el origen de $\vec{u}$ y el destino de $\vec{v}$ y recorre dos unidades a la
    derecha y cuatro hacia arriba, que es lo que nos ha salido en el cálculo.\\

    Vamos a hacerlo ahora al revés y veremos que el resultado es el mismo:
    \begin{center}
      \begin{tikzpicture}[baseline=(current bounding box.center), scale=.9]
        \begin{axis}[axis equal, xmin=-2, xmax=3, ymin=-1, ymax=5, axis lines = middle,
          axis line style={draw=gridgray}, xticklabel=\empty, yticklabel=\empty,
          tick label style={major tick length=0pt}, grid=major, ytick={-1, 0, 1, 2, 3, 4, 5}]
          \draw[-Latex] (0,0) --(-1,2) node[midway, sloped, above] {$\vec{v}=(-1,2)$};
          \node at (-2,4) {\textbf{1.}};
        \end{axis}
      \end{tikzpicture}
      \quad
      $\longrightarrow$
      \quad
      \begin{tikzpicture}[baseline=(current bounding box.center), scale=.9]
        \begin{axis}[axis equal, xmin=-2, xmax=3, ymin=-1, ymax=5, axis lines = middle,
          axis line style={draw=gridgray}, xticklabel=\empty, yticklabel=\empty,
          tick label style={major tick length=0pt}, grid=major, ytick={-1, 0, 1, 2, 3, 4, 5}]
          \draw[-Latex] (0,0) --(-1,2) node[midway, sloped, below] {$\vec{v}=(-1,2)$};
          \draw[-Latex] (-1,2) -- (2,4) node[midway, sloped, above] {$\vec{u}=(3,2)$};
          \draw[ultra thick, -Latex] (0,0) -- (2,4) node[midway, sloped, below] {$\vec{w}$};
          \node at (-2,4) {\textbf{2.}};
        \end{axis}
      \end{tikzpicture}
    \end{center}

    Hemos empezado dibujando $\vec{v}$ y a continuación $\vec{u}$, y el resultado sigue teniendo
    dos unidades a la derecha y cuatro hacia arriba.\\
    Esto nos quiere decir que la suma de vectores es conmutativa.
  \end{solution}
\question Dados los mismos vectores que en el enunciado anterior calcula $\vec{a} = \vec{u} -
  \vec{v}$.
  \begin{solution}
    Recordemos los vectores que nos daban $\vec{u} = (3,2)$ y $\vec{v} = (-1, 2)$.\\

    El cálculo sigue siendo sencillo:
    \[\vec{a} = (3,2) - (-1, 2) = (3-(-1), 2-2) = (4,0)\]

    Para hacer la operación geométricamente vamos a tener en cuenta que la resta es lo mismo
    que sumar con el opuesto. Entonces
    \[\vec{a} = \vec{u} -\vec{v} = \vec{u} + (-\vec{v})\]

    Con lo que para el cálculo geométrico vamos a dibujar $\vec{u}$ y a continuación
    $-\vec{v} = (1, -2)$:

    \begin{center}
      \begin{tikzpicture}[baseline=(current bounding box.center), scale=.9]
        \begin{axis}[axis equal, xmin=0, xmax=5, ymin=-1, ymax=4, axis lines = middle,
          axis line style={draw=gridgray}, xticklabel=\empty, yticklabel=\empty,
          tick label style={major tick length=0pt}, grid=major, ytick={-1, 0, 1, 2, 3, 4}]
          \draw[-Latex] (0,0) --(3,2) node[midway, sloped, above] {$\vec{u}=(3,2)$};
          \node at (0,3) {\textbf{1.}};
        \end{axis}
      \end{tikzpicture}
      \quad
      $\longrightarrow$
      \quad
      \begin{tikzpicture}[baseline=(current bounding box.center), scale=.9]
        \begin{axis}[axis equal, xmin=0, xmax=5, ymin=-1, ymax=4, axis lines = middle,
          axis line style={draw=gridgray}, xticklabel=\empty, yticklabel=\empty,
          tick label style={major tick length=0pt}, grid=major, ytick={-1, 0, 1, 2, 3, 4}]
          \draw[-Latex] (0,0) --(3,2) node[midway, sloped, above] {$\vec{u}=(3,2)$};
          \draw[-Latex] (3,2) -- (4,0) node[midway, sloped, above] {$-\vec{v}=(1,-2)$};
          \draw[ultra thick, -Latex] (0,0) -- (4,0) node[midway, sloped, below] {$\vec{a}$};
          \node at (0,3) {\textbf{2.}};
        \end{axis}
      \end{tikzpicture}
    \end{center}
    Vemos que dibujando $\vec{u}$ y a continuación $-\vec{v}$ el vector que obtenemos tiene las
    mismas componentes que habíamos calculado.\\
    Y esta sería la interpretación geométrica de la resta de vectores.
  \end{solution}
\end{questions}
\subsubsection{Regla del paralelogramo.}
Para entender bien esta regla vamos a utilizar los vectores de los ejemplos de la suma de
vectores.\\

Estos vectores han sido $\vec{u} = (3,2)$ y $\vec{v} = (-1, 2)$, y lo que vamos a hacer es dibujar
el paralelogramo que tiene de lados esos dos vectores:
\begin{center}
  \begin{tikzpicture}
    \begin{axis}[axis equal, xmin=-1, xmax=4, ymin=-1, ymax=5, axis lines = middle,
      axis line style={draw=lightgray}, xticklabel=\empty, yticklabel=\empty,
      tick label style={major tick length=0pt}, grid=major, ytick={-1, 0, 1, 2, 3, 4, 5}]
      \draw[thick, -Latex] (0,0) --(3,2) node[midway, sloped, below] {$\vec{u}$};
      \draw[thick, -Latex] (0,0) --(-1,2) node[midway, sloped, below] {$\vec{v}$};
      \draw[thick,-Latex] (-1,2) --(2,4) node[midway, sloped, above] {$\vec{u}$};
      \draw[thick,-Latex] (3,2) --(2,4) node[midway, sloped, above] {$\vec{u}$};
      \addplot[mark=*, only marks] coordinates {(0, 0)};
    \end{axis}
  \end{tikzpicture}
\end{center}

Y resulta que \textbf{el vector suma $\boldsymbol{\vec{u} + \vec{v}}$ es la diagonal que va del
  punto del que salen los dos vectores al punto al que llegan los dos}:
\begin{center}
  \begin{tikzpicture}
    \begin{axis}[axis equal, xmin=-1, xmax=4, ymin=-1, ymax=5, axis lines = middle,
      axis line style={draw=lightgray}, xticklabel=\empty, yticklabel=\empty,
      tick label style={major tick length=0pt}, grid=major, ytick={-1, 0, 1, 2, 3, 4, 5}]
      \draw[dashed, thick,-Latex] (0,0) --(3,2) node[midway, sloped, below] {$\vec{u}$};
      \draw[dashed,thick,-Latex] (0,0) --(-1,2) node[midway, sloped, below] {$\vec{v}$};
      \draw[dashed,thick,-Latex] (-1,2) --(2,4) node[midway, sloped, above] {$\vec{u}$};
      \draw[dashed,thick,-Latex] (3,2) --(2,4) node[midway, sloped, above] {$\vec{u}$};
      \addplot[mark=*, only marks] coordinates {(0, 0)};
      \draw[ultra thick, -Latex] (0,0) --(2,4) node[midway, sloped, above] {$\vec{u}+\vec{v}$};
    \end{axis}
  \end{tikzpicture}
\end{center}

\textbf{El vector resta $\boldsymbol{\vec{u} - \vec{v}}$ es la diagonal que va desde el destino de
  $\boldsymbol{\vec{v}}$ al destino de $\boldsymbol{\vec{u}}$}:
\begin{center}
  \begin{tikzpicture}
    \begin{axis}[axis equal, xmin=-1, xmax=4, ymin=-1, ymax=5, axis lines = middle,
      axis line style={draw=lightgray}, xticklabel=\empty, yticklabel=\empty,
      tick label style={major tick length=0pt}, grid=major, ytick={-1, 0, 1, 2, 3, 4, 5}]
      \draw[dashed,thick, -Latex] (0,0) --(3,2) node[midway, sloped, below] {$\vec{u}$};
      \draw[dashed,thick, -Latex] (0,0) --(-1,2) node[midway, sloped, below] {$\vec{v}$};
      \draw[dashed,thick,-Latex] (-1,2) --(2,4) node[midway, sloped, above] {$\vec{u}$};
      \draw[dashed,thick,-Latex] (3,2) --(2,4) node[midway, sloped, above] {$\vec{u}$};
      \addplot[mark=*, only marks] coordinates {(0, 0)};
      \draw[ultra thick, -Latex] (-1,2)--(3,2) node[midway, sloped, above] {$\vec{u} - \vec{v}$};
    \end{axis}
  \end{tikzpicture}
\end{center}
\end{document}
