\documentclass[a4paper,11pt,answers]{exam}

\usepackage[utf8]{inputenc}
\usepackage[spanish]{babel}
\usepackage[T1]{fontenc}
% textcomp es para el símbolo del euro
\usepackage{lmodern, textcomp}

\usepackage[left=1.5cm, right=1.5cm, top=1in, bottom=1in]{geometry}
% \usepackage{mathexam}
\usepackage[all]{nowidow} %Quita viudas y huerfanas



\usepackage{amsmath}
\usepackage{amssymb}
\usepackage{multicol}
\usepackage{longtable}

% Para poder referenciar los pasos hay que crear un nuevo tipo de lista:
\usepackage{hyperref}
\usepackage{enumitem,cleveref}

% para la última página
% \usepackage{lastpage}



% Para padding en celdas
\usepackage{cellspace}
\setlength\cellspacetoplimit{1mm}
\setlength\cellspacebottomlimit{1mm}

% Para hacer tachados
\usepackage[makeroom]{cancel}

% Creative commons
% \usepackage{ccicons}
\usepackage[type={CC}, modifier={by-nc-sa}, version={4.0}, %imagemodifier={-eu-80x25},
lang={spanish}]{doclicense}

% Para las gráficas:
\usepackage{tikz}
\usepackage{pgfplots}
\pgfplotsset{compat = newest}
\pgfplotsset{compat=1.18}
\usetikzlibrary{babel} %Si no da errores con algunas cosas al compilar los gráficos.
\usetikzlibrary{arrows,shapes,positioning}
\usetikzlibrary{matrix}
\usepgfplotslibrary{fillbetween}
\usetikzlibrary{arrows.meta}
\usetikzlibrary{fit}
\usetikzlibrary{quotes,angles}
\usetikzlibrary{tikzmark}
\usetikzlibrary{math}


\usepackage{color,colortbl}
\definecolor{Gray}{gray}{0.9}
\newcolumntype{g}{>{\columncolor{Gray}}c}
\usepackage{arydshln} %Este pone la línea punteada en la matriz ampliada. TIENE QUE ESTAR DESPUÉS DEL colortbl porque si no casca.
\usepackage{nicematrix} %Tiene que estar después del arydshln para que no casque.
% \pagestyle{headandfoot}
\pagestyle{headandfoot}
\newcommand\ExamNameLine{
  \par
  \vspace{\baselineskip}
  Nombre:\hrulefill\relax
  \par}

\renewcommand{\solutiontitle}{\noindent\textbf{Solución:}\par\noindent}

\everymath{\displaystyle}
\newcommand\ddfrac[2]{\frac{\displaystyle #1}{\displaystyle #2}}

\def \autor{Paco Andrés}
\def \titulo{Apuntes de álgebra para secundaria.}
\def \titulofichas {\textbf {\titulo}}
\def \cursofichas {}
\def \fechaexamen {}
% \firstpageheader{\cursofichas}{\titulofichas}{\fechaexamen}
\header{\cursofichas}{\begin{small}
    \titulofichas
  \end{small}}{\fechaexamen}
% \header{\cursofichas}{\titulofichas}{\fechaexamen}
% \firtspagefooter{}{\thepage}{}
% Por alguna razón no sale lo del cc en el pie
\firstpagefootrule
\footrule
\footer{\autor}{\thepage}{\doclicenseIcon}
\pointpoints{punto}{puntos}

\shadedsolutions
% \definecolor{SolutionColor}{rgb}{0.99,0.99,.99}
\renewcommand{\baselinestretch}{1.3}

% Use * instead of \cdot
\mathcode`\*="8000
{\catcode`\*\active\gdef*{\cdot}} 

\newcommand{\Card}{\,\mathrm{Card}}

% For e number
\newcommand{\e}{\,\text{\Large e}}
\newcommand{\bolde}{\,\text{\textbf{\large e}}}
\newcommand{\asen}{\,\mathrm{asen}\,}
\newcommand{\acos}{\,\mathrm{acos}\,}
\newcommand{\atg}{\,\mathrm{atg}\,}

% Para el diferencial y la integral:
\newcommand\dif[1]{\mathrm{d}#1}
\newcommand\integral[2]{\int #1\,\dif{#2}}
\newcommand\integrald[4]{\int_{#3}^{#4} #1\,\dif{#2}}
\newcommand\adjunto[1]{#1^\text{*}}
\newcommand\rango[1]{\mathrm{rg}(#1)}
\newcommand\vectort[3]{#1*\vec i + #2*\vec j + #3*\vec k}
\newcommand\distancia[1]{\mathrm{d}(#1)}
\newcommand{\realset}{\mathbb{R}}

\newcommand{\testineq}[1]{\[\text{\Large{¿}}\,#1\,\text{\Large{?}}\]}
% Para escribir explicaciones encima del igual:
% \newcommand\igexpl[1]{{\mathrel{\overset{\makebox{\mbox{\normalfont\tiny\sffamily $#1$}}}{=}}}}
% Parece que mejor con stackrel


% \usepackage{enumerate}
% Aumenta el interlineado en aligns y demás
\setlength{\jot}{1.5em}

% Permite poner textos a las etiquetas: \labeltext{texto}{etiqueta}
\makeatletter
\newcommand{\labeltext}[2]{%
  \@bsphack
  \MakeLinkTarget*{#1}%
  \def\@currentlabel{#1}{\label{#2}}%
  \@esphack%
}\makeatother

% paragraphs como subsubsubsections
\makeatletter
\renewcommand\paragraph{\@startsection{paragraph}{4}{\z@}%
% display heading, like subsubsection
                                     {-3.25ex\@plus -1ex \@minus -.2ex}%
                                     {1.5ex \@plus .2ex}%
                                     {\normalfont\normalsize\bfseries}}
 \setcounter{secnumdepth}{4}
 \makeatother
 % \setcounter{tocdepth}{3} %Solo parts, sections y subsections en el índice.

\renewcommand{\questionlabel}{\textbf{Ejemplo \thequestion:}}
\begin{document}


% \author{Paco Andrés}
\title{\textbf{\titulo}}
\date{}
\author{\autor}
\maketitle

\begin{center}
  \doclicenseLongText\\
  \vspace{.25cm}
  \doclicenseImage
\end{center}
\tableofcontents
\newpage
\renewcommand{\abstractname}{\LARGE Introducción\\}
% \section*{Introducción}
\vspace*{2cm}
\begin{abstract}
  A lo largo de mi experiencia como profesor de matemáticas en centros de
  secundaria me he encontrado con gran cantidad de situaciones del alumnado
  en las que el material que encontraba disponible no era el que la
  situación requería.\\
  Estas situaciones se pueden resumir en
  \begin{itemize}
  \item Desfase curricular importante.
  \item Perdida de material.
  \item Ausencias largas.
  \item \dots
  \end{itemize}

  Y el problema surge cuando la persona afectada por alguna de estas circunstancias quiere aprender, ponerse al día, llegar a titular, y se encuentra en un curso, o en un centro, en el que no puede recibir ningún tipo de apoyo para superar ese bache.\\
  En esos casos no podemos hacernos cargo de explicar completamente ese desfase y, hasta el momento, no he encontrado material que explique detalladamente los procedimientos algebraicos que tiene que dominar el alumnado de secundaria.\\
  Claramente estos conocimientos tienen que ir emparejados con los conocimientos sobre operaciones con enteros, racionales e irracionales, y en las explicaciones de cada nivel se intenta no exceder los conocimientos de esas otras ``áreas''.\\

  Es por esto que me decidí a escribir estos apuntes, para que podamos ofrecer a las personas que se encuentran en las situaciones antes referidas, y están dispuestas a trabajar, lo necesario para unos apuntes que expliquen de manera detallada los procesos que deben conocer.\\
    
  Y a estas personas dedico estos apuntes ya que, aunque han sido pocas, los han inspirado y los merecen.\\\vspace*{2cm}

  \begin{center}
    \small{\textbf{NOTA: Es importante tener en cuenta que estos apuntes están centrados
        principalmente en el componente teórico y la mecánica procedimental, y es necesario
        completarlos con ejercicios y problemas para que sean útiles.}}
  \end{center}
\end{abstract}
\newpage
\part{1º de ESO.}
  El álgebra es una parte de las matemáticas que nos permite realizar operaciones con cantidades desconocidas, para poder hacer:
  \begin{enumerate}
  \item Generalizaciones.
  \item Hallar los valores de las cantidades desconocidas para que se cumplan ciertas condiciones. Esto se llama resolver ecuaciones.
  \end{enumerate}
  Para poder trabajar de la manera indicada tenemos que dejar muy claras las reglas con las que tenemos que trabajar, y para ello hay que empezar definiendo lo que vamos a utilizar.

  \section{Definiciones y generalidades.}
  \subsection{Variable}
  Se llama variable a \textbf{cualquier símbolo, generalmente una letra ($x$, $y$, ...), con la que representamos un valor cualquiera}.\\

  Es importante tener en cuenta que en un determinado contexto una variable es únicamente un valor, no puede ser varios. Es decir: una variable no puede tener varios significados, y si tenemos varios significados a los que asignar variables cada uno de ellos ha de tener una variable distinta.
  \subsection{Expresión algebraica.}
  Se llama expresión algebraica a \textbf{una operación combinada de números y variables}. Por ejemplo:
  \begin{itemize}
  \item $3*x + 1$
  \item $2*x*y^2$
  \item $\frac{x}{y}$
  \end{itemize}

  Vamos a ver un \textbf{ejemplo} de cómo se utiliza una expresión algebraica:
  \emph{En el mercado tenemos las naranjas a 1€ el kilo, y las manzanas a 0,80€ el kilo. Escribir la expresión algebraica que nos da el precio de la compra en función del peso de cada fruta.}
  \begin{solution}
    El peso de cada fruta puede ser cualquiera, luego son variables, con lo que al peso de naranjas le llamaremos $x$ y al de las manzanas $y$.\\

    Si compramos 1\,kg de naranjas el precio será de 1\,€·1=1\,€. Si lo que compramos son 3 kilos, entonces el precio será 1\,€·3=3\,€. Extendiendo este razonamiento, si compramos $x$ kilos el precio será 1\,€·$x$.\\
    Con el mismo razonamiento para las manzanas, el precio de $y$ kilos de manzanas será 0,80€·$y$.\\

    Y lógicamente el precio de la compra será 1\,€·$x$+0,80\,€·$y$. Ésta es la expresión que buscábamos, nos da el precio total simplemente con sustituir $x$ e $y$ por el número de kilos que queramos y operar. Por ejemplo si llevamos 5 kilos de naranjas y 3 de manzanas, sustituimos y nos queda $1\,\text{€}*5+0,80\,\text{€}*3=7,40\,\text{€}$.
  \end{solution}
  \subsection{Término y factor.}
  \textbf{Término} es cualquier cosa que \textbf{está  sumando} o restando (depende del signo que tenga).\\

  \textbf{Factor} es cualquier cosa que \textbf{está multiplicando}.\\

  En principio parece sencillo, pero vamos a verlo con unos ejemplos para tenerlo más claro:
  \textbf{Ejemplos de términos.}
  \begin{itemize}
  \item En $a+b$ hay dos términos, $a$ y $b$.
  \item En $a-b$ hay dos términos, $a$ y $-b$.
  \item En $2*a + 3*b$ hay dos términos, $2*a$ y $3*b$.
  \item En $\frac{2*a}{3} - 5*b$ hay dos términos, $\frac{2*a}{3}$ y $-5*b$
  \end{itemize}
  \textbf{Ejemplos de factores.}
  \begin{itemize}
  \item En $a*b$ hay dos factores, $a$ y $b$.
  \item En $ab$ hay dos factores, $a$ y $b$ (recordemos que si no hay escrita una operación es una multiplicación).
  \item En $2x^2$ hay dos factores, $2$ y $x^2$.
  \item En $(a+1)*(3+x)$ hay dos factores, $(a+1)$ y $(3+x)$.
  \end{itemize}

  Es importante aprenderse y comprender estas definiciones ya que más adelante las vamos a utilizar bastante a la hora de explicar operaciones y mecánicas para resolver ejercicios.

  \subsection{Evaluación de una expresión algebraica.}
  Consiste en \textbf{sustituir las variables por números y realizar la operación completa} para dar un resultado.\\
  \textbf{Ejemplo:}\\
  Evaluar la expresión $\frac{2x^2 y}{3 + y}$ en $x=-1$ e $y=2$.
  \begin{solution}
    Por lo que acabamos de contar solo hay que sustituir las variables por los valores que nos indican y realizar la operación siguiendo el orden.\\
    \textbf{IMPORTANTE: cuando sustituyamos variables por valores negativos tenemos que poner paréntesis, ya que si no es posible que hagamos mal la operación o que escribamos algo erróneo.}\\

    En el caso que nos ocupa hay que cambiar todas las $x$ por $-1$ y las $y$ por $2$:
    \[\frac{2*(-1)^2 * 2}{3 +2} = \frac{2*1 *2}{5} = \frac{4}{5}\]
  \end{solution}
  \subsection{Traducción de enunciados.}
  La potencia del álgebra es las posibilidades que nos ofrece a la hora de expresar cualquier enunciado como una operación combinada con variables. Es decir, nos permite traducir un enunciado a una expresión algebraica.\\
  Esto es algo de uso común aunque no se presente de forma explícita, cualquier fórmula que hayamos visto es en realidad una expresión algebraica.\\
  Por ejemplo el área de un rectángulo, que es $A = b*a$ donde $b$ representa la longitud de la base y $a$ la de la altura.\\
  Esa fórmula es una expresión algebraica en la que $b$ y $a$ son variables y cuando calculamos el área lo que estamos haciendo es evaluarla para unos valores concretos de $a$ y $b$.\\

  No obstante la traducción de enunciados es posiblemente es una de las partes más difíciles a la hora de estudiar álgebra elemental, ya que presupone que sabemos perfectamente cual es el significado de cada una de las operaciones aritméticas y cómo están implicadas en expresiones orales de uso común, que es más complejo de lo que parece.
  Claramente no podemos hacerlo de manera intuitiva (excepto personas contadas que tienen un don), con lo que hay que hacer el razonamiento en pasos pequeños y escribiendo lo más posible. Es la única manera que hay de hacerlo bien, y cualquier persona que sepa matemáticas razona así\\

  Veámoslo con \textbf{unos cuantos ejemplos}:
  \begin{questions}
  \question Escribe la expresión algebraica correspondiente al enunciado ``\emph{el doble de un número}''.\\
    \begin{solution}
      El primer paso es asignar significado a las variables: en el enunciado nos hablan de un número que no conocemos y que puede ser cualquiera. Así que le asignaremos una variable, por ejemplo $x$.\\
      Ahora tenemos que fijarnos en las operaciones que aparecen en el enunciado: el doble. El doble es multiplicar por dos y lo que va multiplicado por dos es el número, es decir, la variable. Entonces:
      \begin{center}
        \emph{El doble de un número $x$ = $2*x$}.
      \end{center}
    \end{solution}

  \question Traduce a una expresión algebraica ``\emph{El dinero que tengo si junto las monedas de euro y de dos euros}''.
    \begin{solution}
      En este caso no conocemos las monedas que hay de cada tipo y, como son dos significados diferentes, tenemos que utilizar dos variables:
      \begin{itemize}
      \item $x$ son las monedas de euro.
      \item $y$ son las monedas de dos euros.
      \end{itemize}
      Ahora veamos las operaciones que hay que hacer.\\
      Como nos está preguntando por el dinero, y no por las monedas, tendremos que multiplicar cada moneda por su valor:
      \begin{itemize}
      \item En monedas de euro hay $1*x$ euros.
      \item En monedas de dos euros hay $2*y$ euros.
      \end{itemize}
      Y para obtener el dinero total, que es lo que nos piden, solo tenemos que sumar el dinero que hay en cada tipo de moneda:
      \begin{center}
        \emph{El dinero total es $1*x+2y$}. \small (más adelante veremos que el 1· no se pone).
      \end{center}
    \end{solution}
  \question Traduce la expresión ``\emph{un número par}''.
    \begin{solution}
      Este, y los tres enunciados que siguen, son enunciados teóricos que aparecen mucho en matemáticas y a la vez son muy poco intuitivos.\\
      La manera de razonar aquí es la siguiente: si un número es par es divisible entre dos, con lo que lo podemos obtener multiplicado su mitad por dos. Luego si conocemos la mitad conocemos el número, así que llamaremos $x$ a la mitad del número y de esta manera:
      \begin{center}
        \emph{La expresión de un número par es $2x$, donde $x$ es la mitad del número}.
      \end{center}
    \end{solution}

  \question Traduce la expresión ``\emph{el siguiente a un número}''.
    \begin{solution}
      Desconocemos el número, con lo que lo llamaremos $n$ (podría ser $x$ pero vamos a cambiar para que se vea que existen otras variables).
      Pero no nos piden el número sino su siguiente, y la distancia entre un número y su siguiente siempre es la unidad. Con lo que \emph{el siguiente a un número es $n+1$}.
    \end{solution}

  \question Traduce ``\emph{un número impar}''.
    \begin{solution}
      Para traducir este enunciado vamos a utilizar los dos anteriores.\\
      Sabemos que los pares y los impartes están intercalados (impar-par-impar-par-impar-...), de manera que un número impar siempre sera el siguiente a un par, con lo que por lo obtenido en los dos ejemplos anteriores:
      \begin{center}
        \emph{Un número impar es $2n+1$}. \small (o $2x+1$, depende de la variable que estemos utilizando) 
      \end{center}

      También se podría pensar que un número impar es el anterior a uno par, quedándonos otra expresión completamente válida para un número impar: $2n-1$.
    \end{solution}

  \question Traduce ``\emph{un número de dos cifras}''.
    \begin{solution}
      En este caso nos hablan de dos ``cosas'' que no conocemos, que son las cifras. Y al ser dos que no tienen porqué ser iguales tenemos que asignar una variable distinta a cada una:
      \begin{itemize}
      \item Llamaremos $x$ a las unidades.
      \item E $y$ a las decenas.
      \end{itemize}
      De esta manera, teniendo en cuenta como se escribe un número a partir de sus unidades, decenas, ... la expresión que nos da el número es $10*y + x$.
    \end{solution}
  \end{questions}

  \section{Monomios.}
  \subsection{Definición}
  \emph{\textbf{Un monomio es un número por potencias naturales de variables}}, por ejemplo:
  \begin{itemize}
  \item $2*x^2$
  \item $-3*x^2 *y$
  \item $-x$
  \item $x*y^2$
  \end{itemize}
  El símbolo de la multiplicación no suele escribirse, de manera que los ejemplos anteriores quedan $2x^2$, $-3x^2 y$, $-x$, $xy^2$.
  \subsection{Partes de un monomio.}
  \begin{center}
    \begin{tikzpicture}[y=0.80pt, x=0.80pt, yscale=-1.000000, xscale=1.000000, inner sep=0pt, outer sep=0pt]
      \path (0,0) node  {\huge $3\ \ x\ \ ^2$};
      \path[draw=black,opacity=0.498,line join=round,line cap=round,line
      width=0.448pt] (-30,0) ellipse (0.4cm and 0.7cm); %elipse coeficiente
      \path[draw=black,opacity=0.498,line join=round,line cap=round,line
      width=0.646pt] (15,0) ellipse (.8cm and 0.7cm); %elipse parte literal
      \path[draw=black,dashed,opacity=0.498,line
      join=round,line cap=round,line width=0.532pt] (27,-5) ellipse
      (0.3cm and 0.3cm); %elipse grado
      \path (-140, 0) node   {\textbf{Coeficiente}};
      \path (30,60) node {\textbf{Parte literal}};
      \path(80,-15) node  {\textbf{Grado}};
      
      \draw[-latex,line width=.5mm]   (-100,0) -- (-45,0); %línea coeficiente
      \draw[-latex,line width=.5mm]  (30,50) -- (20,27); %línea parte literal
      \draw[-latex,line width=.5mm] (57,-15) -- (37,-5); %línea grado

    \end{tikzpicture}
  \end{center}
  \begin{itemize}
  \item \textbf{Coeficiente}: es el número que multiplica incluyendo el signo. En este caso es $\boldsymbol{3}$.\\
    Si no hay ningún número multiplicando el coeficiente es 1.
  \item \textbf{Parte literal}: es lo que queda cuando quitamos el coeficiente: las variables con sus exponentes. En este caso es $\boldsymbol{x^2}$.
  \item \textbf{Grado}: es la suma de los exponentes de las variables. En este caso es $\boldsymbol{2}$.\\
    Recordemos que si no hay exponente es porque vale 1.
  \end{itemize}
  Vamos a ver una tabla con unos ejemplos para que se vea mejor:
  \begin{solution}
    \begin{center}
      \renewcommand{\arraystretch}{1.5}
      \begin{tabular}{|r|r|r|r|}
        \hline
        \multicolumn{1}{|c|}{\textbf{Monomio}} & \multicolumn{1}{c|}{\textbf{Coeficiente}} & \multicolumn{1}{c|}{\textbf{Parte literal}} & \multicolumn{1}{c|}{\textbf{Grado}} \\ \hline
        $5x^2y$&$5$&$x^2y$&$3$\\ \hline
        $-3y$&$-3$&$y$&$1$\\ \hline
        $\frac{3}{2}z^4$&$\frac{3}{2}$&$z^4$&$4$\\\hline
        $\frac{-x}{2}$&$-\frac{1}{2}$&$x$&$1$\\\hline
        $-5$&$-5$&No hay&$0$\\\hline
        $-x^2 y^3$&$-1$&$x^2 y^3$&$5$\\\hline
        $x$&$1$&$x$&$1$\\\hline
      \end{tabular}
    \end{center}
    En los ejemplos de las tablas se pueden observar los siguientes detalles que son importantes:
    \begin{itemize}
    \item Un número sin variable es un monomio de grado cero.
    \item El coeficiente puede ser cualquier tipo de número, incluidas fracciones, raíces, \dots
    \item Si el coeficiente es una fracción de numerador 1 éste se puede sustituir por la parte literal.
    \end{itemize}
  \end{solution}

  \subsection{Operaciones con monomios.}
  \subsubsection{Suma de monomios.}
  A la hora de sumar monomios tenemos que pensar que lo que estamos haciendo es contar cosas iguales. Es decir, la operación
  \[3x + 4x\]
  es juntar un grupo de tres $x$ con un grupo de cuatro $x$, con lo cual el resultado es $7x$.\\
  Esto es similar a decir que tenemos tres cajas en un sitio y cuatro cajas en otro. Si las juntamos todas resulta que tenemos siete cajas.

  Pero, ¿qué sucede cuando las partes literales son distintas? Pues lo mismo que cuando juntamos cosas distintas, que no podemos agruparlas. Es decir
  \[2x + 4y\]
  se queda como está, ya que $x$ e $y$ son dos cosas distintas y no podemos agruparlas.\\

  Conclusión: \textbf{solo podemos sumar monomios que tienen la misma parte literal. En caso de que tengan la misma parte literal lo que hacemos es sumar los coeficientes y dejar la misma parte literal}.\\

  Otra forma de entenderlo es a través de la propiedad distributiva de la multiplicación respecto a la suma:
  \[3x + 4x = (3+4)*x = 7x\]

  Algunos ejemplos más de sumas de monomios:
  \begin{solution}
    \begin{itemize}
    \item $4x^2 + x^2 = 5x^2$ \begin{small}
	(Hay que tener en cuenta que cuando no hay coeficiente es por que vale 1, con lo que $4+1=5$)
      \end{small}
    \item $13y - 7y = 6y$
    \item $5x^2 + 3x = 5x^2 + 3x$ \begin{small}
	(Recordar que la parte literal incluye los exponentes, y si son distintos no se puede sumar)
      \end{small}
    \item $y^4 - 5y^4 = -4y^4$ \begin{small}
	(No hay que olvidarse de utilizar las reglas de signos)
      \end{small}
    \item $-8x^3 + 5x^3 - x^3 = -4x^3$
    \item $3x^2- 2x + 5x^2 + 3x = 8x^2 + x$ \begin{small}
	(Es un recopilatorio de casos anteriores)
      \end{small}
    \end{itemize}
  \end{solution}
  Es \textbf{importante} tener en cuenta que la parte literal incluye los exponentes de las variables, con lo que $x$ y $x^2$ no son la misma parte literal sino que significan dos cosas distintas. De manera que
  \[3x^2 + 5x\]
  \textbf{No se puede hacer y se queda como está} ya que $x$ y $x^2$ son cosas distintas.

  Cuando \textbf{dos monomios} se pueden sumar, \textbf{tienen la misma parte literal}, se dice que son \textbf{monomios semejantes}.
  \subsubsection{Producto de un número por un monomio.}
  Para entender como funciona el producto de un número por un monomio vamos a poner un ejemplo con cajas como en el caso anterior. Si tenemos cajas que tienen cinco paquetes y un palé en el que caben diez cajas, ¿cuántos paquetes hay en un palé? En este caso multiplicaríamos diez por cinco y nos da cincuenta paquetes en un palé.\\

  Pues para multiplicar un monomio por un número tenemos que pensar en lo mismo. En este caso $10*5x$ sería el equivalente al enunciado anterior, donde $x$ equivale al paquete, y el resultado es $50x$.\\
  Es decir, lo que hacemos es \textbf{multiplicar el coeficiente por el número y dejar la parte literal como está}.\\

  Algunos ejemplos más:
  \begin{solution}
    \begin{itemize}
    \item $7*2x^3 = 14x^3$
    \item $4y^2*(-2) = -8y^2$ \begin{small}
	(Se cumple la propiedad conmutativa de la multiplicación)
      \end{small}
    \item $-3*(-5zy) = 15zy$
    \end{itemize}
  \end{solution}

  \subsubsection{Producto de monomios.}
  Para entender cómo se hace el producto de monomios tenemos que tener en cuenta las propiedades siguientes:
  \begin{itemize}
  \item Propiedad conmutativa de la multiplicación. Es decir $a*b = b*a$
  \item Propiedad asociativa de la multiplicación. Es decir, en $a*b*c$ nos da lo mismo hacer primero $a*b$ y multiplicar el resultado por $c$ que hacer primero $b*c$ y multiplicar el resultado por $a$. Y si la juntamos con la anterior nos dice que también da lo mismo hacer primero $a*c$ y luego multiplicar el resultado por $b$.
  \item Cuando hay un producto de potencias de la misma base se suman los exponentes y se deja la misma base. Es decir, $a^n *a^m = a^{n+m}$. Y siempre hay que recordar que cuando no hay exponente es porque vale uno: $a^n *a = a^{n+1}$.
  \end{itemize}

  Con todo esto vamos a hacer el producto $(3x^2 y)* (-5xy)$ utilizando lo anterior:
  \begin{enumerate}
  \item Primero reordenamos todo utilizando las propiedades conmutativa y asociativa: $(3x^2 y)* (-5xy) = (-5*3)*(x^2 *x)*(y*y)$
  \item Multiplicamos cada paréntesis teniendo en cuenta la propiedad del producto de potencias de la misma base en el caso de las variables, y queda: $(3x^2 y)* (-5xy) = -15 x^3 y^2$
  \end{enumerate}
  Es decir, \textbf{para multiplicar dos monomios tenemos que multiplicar por una parte los coeficientes y por otra las partes literales utilizando las propiedades de las potencias}.\\

  Vamos a ver unos cuantos ejemplos más:
  \begin{solution}
    \begin{itemize}
    \item $\boldsymbol{(2x^2) * (-3x)} = (2*(-3))*(x^2 * x) = \boldsymbol{-6x^3}$
    \item $\boldsymbol{5xy*(xy^2)} = (5*1)(x*x)(y*y^2) = \boldsymbol{5x^2y^3}$
    \item $\boldsymbol{\left(-\frac{x^2}{2}\right) *(-4 x^3)} = \left(-\frac{1}{2} * (-4) \right) (x^2*x^3) = \boldsymbol{\frac{4}{2}x^5 = 2x^5}$
    \item $\boldsymbol{\frac{2x^2y}{5}*\frac{3x^2 y^2}{4}} = \left(\frac{2}{5}*\frac{3}{4}\right)(x^2*x^2)(y*y^2) = 
      \boldsymbol{\frac{3}{10}x^4y^3}$
    \item $\boldsymbol{-z^3*z^2} = (-1*1)(z^3 *z^2) = \boldsymbol{-z^5}$
    \end{itemize}
  \end{solution}
  Lógicamente, cuando tengamos práctica nos saltaremos el paso intermedio: \[-2x^3y^4 *(3 x y^2) = -6 x^4y^6\]

  \subsubsection{Cociente de monomios.}
  El mecanismo para realizar una división de monomios es el mismo que en el caso de la multiplicación: dividimos por un lado los coeficientes y por otro las partes literales utilizando las propiedades de las potencias (\emph{recordar que en la división de potencias de la misma base tenemos que restar los exponentes y dejar la misma base}).\\

  Vamos a ver unos ejemplos:
  \begin{itemize}
  \item $\boldsymbol{\frac{9x^4}{3x^2}}= \frac{9}{3}*\frac{x^4}{x^2} = 3 x^{4-2} = \boldsymbol{3x^2}$
  \item $\boldsymbol{\frac{10x^3 y^2}{6xy}} = \frac{10}{6} \frac{x^3}{x} \frac{y^2}{y} = \boldsymbol{\frac{5}{3}x^2 y}$
  \item $\boldsymbol{\frac{x^4}{3x^2}} = \frac{1}{3}\frac{x^4}{x^2} =
    \boldsymbol{\frac{1}{3}x^2}$
  \end{itemize}

  Y al igual que ocurre con la multiplicación de monomios, con la práctica nos saltaremos el paso intermedio:
  \[\frac{18x^5y^3}{8x^2 y^2} =\frac{9x^3 y}{4}\]

  \section{Polinomios I.}
  \subsection{Definición.}
  Uno polinomio \textbf{no es más que una suma de monomios}. Es decir, algo como
  \[4x^3 - 5x^2 + 3x -2\]
  es un polinomio.\\

  \textbf{A cada uno de los sumandos del polinomio se le llama término}. Es decir, cada monomio que aparece es un término e incluye el signo:\\
  en el polinomio de ejemplo ($4x^3 - 5x^2 + 3x -2$) los términos son:
  \begin{multicols}{2}
    \begin{itemize}
    \item $4x^3$
    \item $-5x^2$ (incluye el signo)
    \item $3x$
    \item $-2$ (incluye el signo)
    \end{itemize}
  \end{multicols}

  Se denomina \textbf{grado del polinomio} al mayor grado que aparece. En el ejemplo que hemos puesto tenemos los siguientes términos:
  \begin{multicols}{2}
    \begin{itemize}
    \item $4x^3$, grado 3
    \item $-5x^2$, grado 2
    \item $3x$, grado 1
    \item $-2$, grado 0
    \end{itemize}
  \end{multicols}
  El grado es 3, puesto que es el mayor.
  \subsection{Nombre de un polinomio.}
  A los polinomios se les suele dar nombre ya que a veces son largos de escribir.\\
  Para nombrar un polinomio \textbf{se utiliza una letra mayúscula}, empezando por la P normalmente, \textbf{y entre paréntesis se indican las variables} que participan en el polinomio.\\

  Por \textbf{ejemplo}:
  \[P(x) = x^3 - 2x^2 + x + 5\]

  Y si hay más de una variable:
  \[Q(x,y) = x^2 + xy + 3y\]

  Esta manera de nombrarlos es especialmente útil a la hora de indicar en que valores de las variables se evalúa un polinomio.\\
  Por ejemplo, si queremos evaluar el primer polinomio en $x=3$ basta con escribir $P(3)$, y quedaría:
  \[P(3) = 3^3 - 2*3^2 +3 + 5 = 19\]
  Y si queremos evaluar el segundo en $x=-1$ e $y=-2$ nos queda:
  \[  Q(-1, -2) = (-1)^2 + (-1)*(-2) + 3*(-2) = -3\]
  \subsection{Operaciones con polinomios I.}
  \subsubsection{Reducción de un polinomio.} \label{red_pol}
  Reducir un polinomio consiste en \textbf{sumar los monomios semejantes y ordenar los términos resultantes de mayor a menor grado}.\\

  Vamos a ver \textbf{un ejemplo}: reduce el siguiente polinomio:
  \[-x^2 + 2x^4 + x^3 +3x^2 - 2 - 1 + x^4\]
  Buscamos el grado del polinomio, que en este caso es $4$ y vamos viendo los términos semejantes que hay siguiendo el orden descendentes del grado:
  \begin{itemize}
  \item De grado $4$ tenemos $2x^4 + x^4 = \boldsymbol{3x^4}$.
  \item De grado $3$ solo tenemos $\boldsymbol{x^3}$ con lo que se queda como está.
  \item De grado $2$ están $-x^2 + 3x^2 = \boldsymbol{2x^2}$
  \item De grado $1$ no hay ningún término.
  \item De grado $0$ tenemos $-2-1=\boldsymbol{-3}$.
  \end{itemize}
  Y teniendo en cuenta todo esto lo escribimos de mayor a menor grado:
  \[-x^2 + 2x^4 + x^3 +3x^2 - 2 - 1 + x^4 = 3x^4 + x^3 + 2x^2 - 3\]
  \subsubsection{Suma y resta de polinomios.}


  \paragraph{Nombres de polinomios.}
    Antes de ver la suma y la resta vamos a ver como se da nombre a los polinomios, ya que estos nombres se utilizan bastante a la hora de hacer operaciones con polinomios.\\
    
    Para hacer más sencillas algunas cosas a veces se da nombres a los polinomios. Las reglas son las siguientes: al polinomio se le nombra con una letra mayúscula seguida de un paréntesis en el que se encuentra la lista de variables que aparecen en el polinomio.\\
    Por ejemplo, en un ejercicio nos dicen:
    \[P(x) = x^2 + 3x - 5\]
    Se lee ``Pe de $x$'' y significa que ese polinomio se llama $P$ y en el aparece la variable $x$. Siempre que aparezca $P(x)$ en una operación en ese ejercicio tendremos que sustituirlo por $x^2 + 3x - 5$. A continuación vamos a ver unos ejemplos de esto.
  \paragraph{Suma de polinomios.}
    Tenemos los siguientes polinomios:
    \[P(x) = 2x^2 - 3x -1 \quad\quad Q(x)= 3x^2 + x - 3\]

    Si tenemos que hacer $P(x) + Q(x)$ solo hay que juntar los términos de los dos polinomios y reducir (apartado \ref{red_pol}):
    \[P(x) + Q(x) = \overbrace{2x^2 - 3x -1}^{P(x)} \ \overbrace{+ 3x^2 + x - 3}^{Q(x)} = 5x^2 -2x -4\]

    Vamos a ver otro ejemplo:
    \[P(x) = 4x^2 - 3\quad\quad Q(x) = -2x^2 - 3x +5\]

    Entonces:
    \[P(x) + Q(x) = \overbrace{4x^2 - 3}^{P(x)} \ \overbrace{-2x^2 - 3x +5}^{Q(x)} = 2x^2 - 3x +2\]
    \textbf{IMPORTANTE}: \emph{fíjate en que el primer término del segundo polinomio de la suma ($Q(x)$) lleva signo en los dos ejemplos. Hay que ponerlo siempre.}
    \paragraph{Resta de polinomios.}
    Es igual que la suma salvo que tenemos que \textbf{cambiar el signo de TODOS los monomios del polinomio que esté a la derecha del menos}.\\

    Por ejemplo, si nos dan:
    \[P(x) = 2x^2 - 3x -1 \quad\quad Q(x)= 3x^2 + x - 3\]
    Y nos piden $P(x) - Q(x)$ tenemos que cambiar todos los signos de $Q(x)$ que es el que está a la derecha del menos. Y queda de la siguiente manera:
    \[P(x) -Q(x) = \overbrace{2x^2 - 3x -1}^{P(x)} \ \overbrace{-3x^2-x+3}^{-Q(x)}
      = -x^2 - 4x + 2\]

    Y si nos piden $Q(x) - P(x)$ el que hay que cambiar de signo es $P(x)$:
    \[Q(x) - P(x) = \overbrace{3x^2 + x - 3}^{Q(x)}\ \overbrace{-2x^2+3x+1}^{-P(x)}
      =x^2 + 4x -2\]

    Con los otros polinomios de ejemplo que hemos utilizado antes
    \[P(x) = 4x^2 - 3\quad\quad Q(x) = -2x^2 - 3x +5\]

    Si hacemos $P(X) - Q(x)$ tendremos:
    \[P(x) - Q(x) = \overbrace{4x^2 - 3}^{P(x)} \ \overbrace{+2x^2 + 3x - 5}^{-Q(x)}
      = 6x^2 + 3x -8\]


  \section{Ecuaciones I} \label{ecs_sencillas}
  Una ecuación es una igualdad de dos expresiones, por ejemplo:
  \[3x + 2 = -x^2\]

  Es evidente que esas dos expresiones no son iguales. Por lo que hemos visto no podemos hacer la suma $3x + 2$, con lo que no puede dar $-x^2$ y además ni siquiera concuerdan los signos.\\

  En este caso el igual no representa el resultado de una operación sino que es una pregunta, y nos está preguntando ¿cuánto tiene que valer $x$ para que esas dos cosas sean iguales?.\\
  Para averiguarlo tenemos que \emph{resolver la ecuación}, y para poder hacerlo tenemos que hacer algunas definiciones.
  \subsection{Partes de una ecuación.}
  En una ecuación siempre hay un símbolo de igualdad que la separa en \textbf{dos partes}, a \textbf{cada una} de las cuales se la llama \textbf{miembro}:
  \[\overbrace{3x + 2 }^{\text{\small Primer miembro}}=\overbrace{-x-4}^{\text{\small Segundo miembro}}\]


  \subsection{Ecuaciones de primer grado sencillas.}
  Los conocimientos para resolver ecuaciones de primer grado sencillas son intuitivos, se pueden deducir con unas ecuaciones extremadamente sencillas.\\

  Empecemos con la primera:
  \[x + 12 = 25\]
  Es evidente que en este caso $x$ vale $13$, pero vamos a analizar qué pasos hemos dado para llegar a ello: para saber cuanto vale $x$ hemos restado $12$ a $25$ porque $x$ es lo que falta. Es decir,
  \[x = 25 - 12\]

  Si en vez de una suma fuese una resta:
  \[x - 15 = 13\]
  Tenemos que
  \[x = 13 + 15\]

  Y con una multiplicación:
  \[4x = 36 \text{ (que es lo mismo que } 4*x=36 \text{)}\]
  Es buscar un número que multiplicado por $4$ nos dé $36$, y sabemos que para obtenerlo tenemos que hacer una división:
  \[x = \frac{36}{4}\]

  Y si fuese una ecuación con una división:
  \[\frac{x}{5} = 7\]
  Es lógico que lo resolvamos haciendo una multiplicación:
  \[x = 5*7\]

  Con lo que si nos fijamos en todos los casos la manera de resolver las ecuaciones ha consistido en que todo lo que esta operando con la incógnita ha pasado al otro lado del igual haciendo la operación contraria.\\

  Y esta es la \textbf{primera regla} para resolver ecuaciones: \textbf{\large todo lo que cambia de lado pasa haciendo la operación contraria.}\\

  La \textbf{segunda regla} es el orden que en que hay que hacer cada cosa, y para aprenderlo vamos a verlo con un ejemplo:
  \[3x + 2 = x - 4\]
  Los pasos a dar son los siguientes, y tiene que ser en este orden:
  \begin{enumerate}[ref={paso~\arabic*}]
  \item Llevamos los términos que tienen $x$ a un lado del igual y los que no tienen $x$ al otro (recordemos que los términos son los monomios que están sumando o restando), teniendo en cuenta que todo lo que cambia de lado del igual pasa haciendo la operación contraria:
    \[3x - x = -4 - 2\]
    A este paso se le llama \emph{transposición de términos}.
  \item Reducimos los polinomios que han quedado a cada lado del igual.
    \[2x = -6\]
  \item Dejamos la $x$ sola llevando su coeficiente al otro lado, recordando que tiene que hacer la operación contraria:
    \[x = \frac{-6}{2}\]
    A este paso se le llama \emph{despejar la incógnita}. \label{despejar}
  \item Simplificamos el resultado si es necesario: \label{simpl_ec}
    \[x = -3\]
  \item Comprobamos el resultado. Esto se hace cogiendo la ecuación original ($3x + 2 = x - 4$) cambiando la incógnita por el valor obtenido en el \ref{simpl_ec} (o en el \ref{despejar}) y comprobando que nos da el mismo valor a ambos lados del igual:
    \[3*(-3) + 2 = -3 -4\]
    \[-9 + 2 = -7\]
    \[-7 = -7\]
    Como nos ha dado el mismo resultado ya hemos comprobado que la hemos resuelto bien.\\
    
    \emph{Si no nos diese el mismo resultado es que hemos hecho algo mal en algún paso y tendremos que buscar el fallo para corregirlo.}
  \end{enumerate}

  Vamos a ver \textbf{otro ejemplo}:\\

  Resolver $x - 1 = 3x + 4$.
  \begin{solution}
    \begin{enumerate}
    \item Hacemos la \emph{transposición de términos}:
      \[x - 3x = 4 + 1\]
    \item Reducimos los miembros de la ecuación:
      \[-2x = 5\]
    \item \emph{Despejamos} la $x$:
      \[x = \frac{5}{-2}\]
    \item Simplificamos (en este caso solo es un tema de signos):
      \[x = -\frac{5}{2}\]
    \item Comprobamos:
      \[-\frac{5}{2} - 1 = 3*\left(-\frac{5}{2}\right) + 4\]
      \[-\frac{5}{2} - \frac{2}{2} = -\frac{15}{2} + \frac{8}{2}\]
      \[-\frac{7}{2} = -\frac{7}{2}\]
    \end{enumerate}
    La hemos resuelto bien y la solución es $\boldsymbol{x = -\frac{5}{2}}$.
  \end{solution}
  \newpage
\part{2º de ESO.}
  \section{Polinomios II.}
  \subsection{Operaciones con polinomios II.}
  \subsubsection{Producto de polinomios.}
  Para multiplicar dos polinomios tenemos que hacer uso de la propiedad distributiva del producto respecto de la suma.\\
  Vamos a recordar lo que nos dice esta propiedad, que es que:
  \[a*(b+c) = a*b + a*c\]
  Esto se puede extender a productos más complejos, por ejemplo:
  \[(a+b)*(c+d) = a*(c+d)+b*(c+d) = a*c+a*d+b*c+b*d\]
  Que es lo mismo que decir: \textbf{multiplicamos cada término del primer factor por todos los términos del segundo factor}.\\

  Y así vamos a \textbf{multiplicar polinomios: multiplicamos cada monomio del primero por todos los monomios del segundo y después reducimos}. Es \textbf{importante poner el signo que resulta de cada multiplicación de monomios} ya que nos va a decir si vamos a tener que sumar o restar para reducir.\\

  Vamos a verlo con un \textbf{ejemplo sencillo}:
  \[(2x+1)*(3x - 2)\]
  En este caso vamos a multiplicar dos polinomios que son:
  \begin{itemize}
  \item El primero es $2x + 1$, que tiene de monomios a $+2x$ y a $+1$.
  \item El segundo es $3x -2$, que tiene de monomios a $+3x$ y a $-2$.
  \end{itemize}
  Entonces damos los siguientes pasos:
  \begin{enumerate}
  \item Multiplicamos $+2x$ por $+3x$ y da $+6x^2$.
  \item Multiplicamos $+2x$ por $-2$ obteniendo $-4x$.
  \item Como ya no queda ninguno por el que multiplicar $2x$ pasamos al siguiente monomio del primer polinomio, que es $+1$.
  \item Multiplicamos $+1$ por $+3x$ quedando $+3x$.
  \item Multiplicamos $+1$ por $-2$, que es $-2$.
  \item Lo colocamos todo junto y reducimos:
    \[(2x+1)*(3x - 2) = +6x^2 -4x + 3x -2 = 6x^2 - x - 2\]
  \end{enumerate}

  Ahora un \textbf{ejemplo más difícil} sin explicar los pasos porque son los mismos, solo que tanto en el primer polinomio como en el segundo hay más monomios, únicamente vamos a agruparlos para que sea más fácil ver el origen de cada monomio del resultado:
  \[(x^2 - 3x + 5)*(2x^2 - x - 4) = +2x^4 - x^3 -4x^2 \quad -6x^3 + 3x^2 +12x \quad +10x^2 - 5x -20 =2x^4 -7x^3 + 9x^2  + 7x -20\] 
  \begin{itemize}
  \item La parte $\boldsymbol{+2x^4 - x^3 -4x^2}$ procede de multiplicar $x^2$, que es el primer monomio del primer polinomio, por todos los monomios del segundo.
  \item La parte $\boldsymbol{-6x^3 + 3x^2 +12x}$ viene de multiplicar $-3x$ por todos los monomios del segundo.
  \item Y la parte $\boldsymbol{+10x^2 - 5x -20}$ de multiplicar $+5$ por todos los monomios del segundo.
  \end{itemize}
  \subsubsection{Potencia de un polinomio.}
  Para hacer la potencia de un polinomio solo hay que utilizar la definición de potencia y realizar la operación:
  \[(x^2 + 3x - 1)^2 = (x^2 + 3x - 1)*(x^2 + 3x - 1) = x^4 + 3x^3 - x^2 + 3x^3 + 9x^2 - 3x - x^2 -3x + 1 = x^4 +6x^3 +7x^2 -6x +1\]

  Es largo pero es fácil.

  \subsection{Identidades notables.}
  En matemáticas hay determinadas operaciones que aparecen muchas veces, con lo que a veces es mejor aprenderse el resultado de memoria en lugar de calcularlo cada vez que sale.\\
  Un ejemplo de esto son las tablas de multiplicar. Sabemos que la multiplicación es una suma repetida, pero si cada vez que nos saliese $5*9$ sumásemos $5$ nueve veces en vez de utilizar el resultado que tenemos memorizado perderíamos mucho tiempo en cada operación y no podríamos calcular nada en un tiempo razonable.\\

  En álgebra existen también unas operaciones que se repiten mucho, y son las identidades notables.\\

  Son tres, con lo que no es difícil aprendérselas. En realidad solo son tres multiplicaciones, pero el aprendérselas de memoria nos va a permitir hacer determinadas cosas (resolver ecuaciones, resolver problemas con el teorema de Pitágoras, \dots) en un tiempo razonable.
  \begin{itemize}
  \item La primera que vamos a ver es la suma al cuadrado:
    \[\boldsymbol{(a+b)^2}\]
    Utilizando el cálculo de potencias de polinomios tenemos:
    \[\boldsymbol{(a+b)^2} = (a+b)(a+b) = a^2 + ab + ab + b^2 = \boldsymbol{a^2 + 2ab + b^2}\]
  \item La segunda es la diferencia (resta) al cuadrado, y vamos a hacer lo mismo:
    \[\boldsymbol{(a-b)^2} = (a-b)(a-b) = a^2 - ab - ab + b^2 = \boldsymbol{a^2 - 2ab + b^2}\]
  \item Y la última es la suma por diferencia:
    \[\boldsymbol{(a+b)(a-b)} = a^2 -ab + ab - b^2 = \boldsymbol{a^2 - b^2}\]
  \end{itemize}

  Vamos a analizar cada una de ellas por separado:
  \subsubsection{Suma al cuadrado.}
  Que matemáticamente se escribe $(a + b)^2$ y vemos que dentro del paréntesis hay una suma con dos términos, donde $a$ es el primer término y $b$ el segundo término.\\

  Y el resultado que nos ha salido antes es $(a+b)^2 = a^2 + 2ab + b^2$, y teniendo en cuenta el orden de los términos lo podemos leer así (cambiando ligeramente el orden del resultado, porque $2+3 = 3 +2$):\\
  \textbf{\emph{El cuadrado del primero más el cuadrado del segundo más el doble del primero por el segundo.}}\\
  Esta frase es la que vamos a utilizar para aplicar la identidad notable.\\

  Vamos a por \textbf{un ejemplo}:
  \[(x + 3)^2\]
  Es fácil ver que el primer término es $x$ y el segundo $3$, con lo que lo que tenemos que hacer es:
  \begin{enumerate}
  \item El cuadrado del primero: $\boldsymbol{x^2}$.
  \item El cuadrado del segundo: $\boldsymbol{3^2 = 9}$.
  \item El doble del primero ($\boldsymbol{2*x}$) por el segundo ($\boldsymbol{3}$): $\boldsymbol{2*x*3 = 6x}$
  \end{enumerate}
  Y juntándolo y ordenando de mayor a menor grado nos queda:
  \[(x+3)^2 = x^2 + 6x + 9\]

  Ahora \textbf{un ejemplo más complicado}:
  \[(x^2 + x)\]
  En este caso el primero es $x^2$ y el segundo es $x$, aplicamos la frase:
  \begin{itemize}
  \item El cuadrado del primero: $\left(x^2\right)^2 = x^4$.
  \item El cuadrado del segundo: $x^2$.
  \item El doble del primero ($2x^2$) por el segundo ($x$): $2x^2*x = 2x^3$.
  \end{itemize}
  Y haciendo lo mismo que antes:
  \[(x^2 + x)^2 = x^4 + 2x^3 + x^2\]

  Y para terminar \textbf{un ejemplo mucho más difícil}:
  \[(5x^2 + 3y)^2\]
  En este caso el primer término es $2x^2$ y el segundo $3y$ (hay que recordar que un término es un monomio que está sumado o restando). Siguiendo los pasos:
  \begin{itemize}
  \item El cuadrado del primero: $\left(5x^2\right)^2 = 25x^4$ (hay que recordar las propiedades de las potencias).
  \item El cuadrado del segundo: $(3y)^2 = 9y^2$.
  \item El doble del primero ($10x^2$) por el segundo ($3y$): $10x^2 * 3y = 30x^2 y$.
  \end{itemize}
  Juntando todo y ordenando:
  \[(5x^2 + 3y)^2 = 25x^4 + 30x^2 y + 9y^2\]

  \subsubsection{Diferencia al cuadrado.}
  De manera análoga al caso anterior tenemos una resta con dos términos, donde $a$ es el primer término y $b$ el segundo término.\\

  Y el resultado que nos ha salido antes es $(a-b)^2 = a^2 - 2ab + b^2$ y, al igual que hemos hecho antes, lo podemos escribir en una frase así:\\
  \textbf{\emph{El cuadrado del primero más el cuadrado del segundo menos el doble del primero por el segundo.}}\\

  Vamos a por \textbf{un ejemplo}:
  \[(x - 3)^2\]
  El primer término es $x$ y el segundo $3$, con lo que lo que tenemos que hacer es:
  \begin{enumerate}
  \item El cuadrado del primero: $\boldsymbol{x^2}$.
  \item El cuadrado del segundo: $\boldsymbol{3^2 = 9}$.
  \item El doble del primero ($\boldsymbol{2*x}$) por el segundo ($\boldsymbol{3}$): $\boldsymbol{2*x*3 = 6x}$
  \end{enumerate}
  Y juntándolo y ordenando de mayor a menor grado nos queda:
  \[(x-3)^2 = x^2 - 6x + 9\]

  Ahora \textbf{un ejemplo más complicado}:
  \[(x^2 - x)\]
  En este caso el primero es $x^2$ y el segundo es $x$, aplicamos la frase:
  \begin{itemize}
  \item El cuadrado del primero: $\left(x^2\right)^2 = x^4$.
  \item El cuadrado del segundo: $x^2$.
  \item El doble del primero ($2x^2$) por el segundo ($x$): $2x^2*x = 2x^3$.
  \end{itemize}
  Y haciendo lo mismo que antes:
  \[(x^2 - x)^2 = x^4 - 2x^3 + x^2\]

  Y para terminar \textbf{un ejemplo mucho más difícil}:
  \[(5x^2 - 3y)^2\]
  En este caso el primer término es $2x^2$ y el segundo $3y$ (hay que recordar que un término es un monomio que está sumado o restando). Siguiendo los pasos:
  \begin{itemize}
  \item El cuadrado del primero: $\left(5x^2\right)^2 = 25x^4$ (hay que recordar las propiedades de las potencias).
  \item El cuadrado del segundo: $(3y)^2 = 9y^2$.
  \item El doble del primero ($10x^2$) por el segundo ($3y$): $10x^2 * 3y = 30x^2 y$.
  \end{itemize}
  Juntando todo y ordenando:
  \[(5x^2 - 3y)^2 = 25x^4 - 30x^2 y + 9y^2\]

  \subsubsection{Suma por diferencia.}
  Aunque en la expresión $(a+b)(a-b)$ tengamos dos factores (cada paréntesis es un factor porque está multiplicando) en cada uno de ellos tenemos dos términos, $a$ y $b$, de manera que $a$ es el primero y $b$ es el segundo.\\

  Y el resultado que nos ha salido antes es $(a+b)(a-b) = a^2 - b^2$ y, al igual que hemos hecho antes, lo podemos escribir en una frase así:\\
  \textbf{\emph{El cuadrado del primero menos el cuadrado del segundo.}}\\

  Vamos a por \textbf{un ejemplo}:
  \[(x+3)(x - 3)\]
  El primer término es $x$ y el segundo $3$, con lo que lo que tenemos que hacer es:
  \begin{enumerate}
  \item El cuadrado del primero: $\boldsymbol{x^2}$.
  \item El cuadrado del segundo: $\boldsymbol{3^2 = 9}$.
  \end{enumerate}
  Y juntándolo y ordenando de mayor a menor grado nos queda:
  \[(x+3)(x-3) = x^2 - 9\]

  Ahora \textbf{un ejemplo más complicado}:
  \[(x^2+ x)(x^2 - x)\]
  En este caso el primero es $x^2$ y el segundo es $x$, aplicamos la frase:
  \begin{itemize}
  \item El cuadrado del primero: $\left(x^2\right)^2 = x^4$.
  \item El cuadrado del segundo: $x^2$.
  \end{itemize}
  Y haciendo lo mismo que antes:
  \[(x^2 + x)(x^2 - x) = x^4 - x^2\]

  Y para terminar \textbf{un ejemplo mucho más difícil}:
  \[(5x^2 + 3y)(5x^2 - 3y)\]
  En este caso el primer término es $2x^2$ y el segundo $3y$ (hay que recordar que un término es un monomio que está sumado o restando). Siguiendo los pasos:
  \begin{itemize}
  \item El cuadrado del primero: $\left(5x^2\right)^2 = 25x^4$ (hay que recordar las propiedades de las potencias).
  \item El cuadrado del segundo: $(3y)^2 = 9y^2$.
  \end{itemize}
  Juntando todo y ordenando:
  \[(5x^2 + 3y)(5x^2 - 3y) = 25x^4 - 9y^2\]

  \subsection{Sacar factor común.}
  El origen de esta operación está en dos propiedades de las operaciones que son bien conocidas.\\

  La primera es la \textbf{propiedad distributiva} de la multiplicación respecto a la suma, que nos dice que
  \[a*(b+c) = a*b + a*c\]
  Aunque aquí la vamos a utilizar \textbf{al revés}:
  \[\boldsymbol{a*b + a*c = a*(b+c)} \label{eq:distrib} \tag{*}\] 

  La segunda es que tenemos que tener en cuenta que cuando escribimos monomios estamos abreviando, de manera que si tenemos
  \[4x^2 + 6x\]
  Lo que estamos escribiendo es:
  \[4*x*x + 6*x\]
  Y si además factorizamos los coeficientes nos queda:
  \[2*2*x*x + 2*3*x = 2x*2x + 2x*3\]

  Eso se parece bastante a lo de la propiedad distributiva, vamos a poner uno debajo de otro para verlo:
  \[\begin{array}{ccccccc}
      a&*&b&+&a&*&c\\
      2x&*&x&+&2x&*&3
    \end{array}\]
  Y si lo juntamos con lo que hemos visto en la propiedad distributiva al revés \eqref{eq:distrib} nos queda:
  \[\begin{array}{ccccccccc}
      a&*&b&+&a&*&c&=&a*(b+c)\\
      2x&*&x&+&2x&*&3&=&2x*(2x+3)
    \end{array}\]

  Es decir, hemos escrito $4x^2 + 6x = 2x (2x + 3)$. Hemos sacado las cosas que tenían multiplicando en común (el factor común) y lo hemos puesto multiplicando por un paréntesis en el que queda lo que no es común.\\
  
  Está claro que este método no es nada bueno, hay que reescribir demasiadas cosas y al final queda todo un poco lioso, con lo que vamos a ver un método para hacerlo rápidamente.\\
  Para ello tenemos que tener en cuenta que como el factor común multiplica a todos los términos tiene que ser un divisor común, y conocemos un método para calcular el mayor divisor común posible y es el método de cálculo del \textbf{máximo común divisor}.\\
  Recordemos cual es el método para calcularlo:
  \begin{center}
    \emph{Cogemos los factores comunes con el menor exponente.}
  \end{center}
  Con lo cual tenemos que coger la expresión $4x^2 + 6x$ y dar los siguientes pasos:
  \begin{enumerate}
  \item Factorizar los coeficientes $2^2 x^2 + 2*3 x$ (las $x$ ya están factorizadas).
  \item Coger los factores comunes (en este caso el $2$ y la $x$) con el menor exponente (que en este caso es $1$).\\
    Ese es el factor común.
  \item Lo escribimos fuera del paréntesis y dentro escribimos el resultado de dividir cada término entre el factor común. Lo ponemos con todos los pasos:
    \[\boldsymbol{4x^2 + 6x} = 2^2 x^2 + 2*3*x = 2x\left(\frac{2^2 x^2}{2x} + \frac{2*3*x}{2x}\right) = 
      \boldsymbol{2x*(2x + 3)}\]
  \end{enumerate}

  Vamos a ver \textbf{unos ejemplos} más:\\
  \begin{enumerate}
  \item $\boldsymbol{4x^3 + 5 x}$. En este caso los coeficientes no tienen factores comunes, el único factor común es la $x$ elevada a $1$. Entonces:
    \[4x^3 + 5x = x\left(\frac{4x^3}{x} + \frac{5x}{x}\right) = \boldsymbol{x(4x^2 + 5}\]
    
  \item $\boldsymbol{6x^3 - 3x^2}$. En este caso el factor común de los coeficientes es $3$ y el de las variables es $x^2$ (esta vez el menor exponente es $2$). Con esto nos queda:
    \[6x^3 - 3x^2 = 3x^2 \left(\frac{6x^3}{3x^2} - \frac{3x^2}{3x^2}\right) = \boldsymbol{3x^2 (6x - 1)}\]
    
  \item $\boldsymbol{12x^2y^3 - 8 x^2y + 20x^3 y^2}$. Aquí el factor común de los coeficientes es $4$, la $x$ con menor exponente es $x^2$ mientras que la $y$ con el menor exponente es $y^1$. Por lo tanto:
    \[12x^2y^3 - 8 x^2y + 20x^3 y^2 = \boldsymbol{4x^2 y (3y - 8 + 5x y)}\]
  \end{enumerate}

  \section{Ecuaciones II.}
  En la \textbf{sección \ref{ecs_sencillas}} (página \pageref{ecs_sencillas}) vimos como \textbf{resolver ecuaciones de primer grado sencillas}. Ahora vamos a ver cómo se resuelven ecuaciones más complejas e incluso de mayor grado.

  \subsection{Ecuaciones de primer grado con paréntesis.} \label{ecs_parentesis}
  En este caso lo único que tenemos que utilizar es la jerarquía de las operaciones, que nos dice que \textbf{lo primero que hay que hacer son los paréntesis}. Aunque en este caso casi nunca podremos realizar la operación que hay dentro del paréntesis (recordemos que no podemos sumar monomios con distinta parte literal), con lo que tendremos que utilizar la multiplicación de polinomios o la regla de que cuando hay un menos delante de un paréntesis tenemos que cambiar de signo a todo lo de dentro.\\

  Vamos a ver \textbf{un ejemplo}: resolver
  \[2x - 4(3x - 2) = x + 4 - (2x -5)\]
  En este ejemplo vemos que hay dos paréntesis que tienen una suma que no podemos hacer. El primero (izquierda) está multiplicado por $-4$, y el segundo (derecha) tiene un menos delante. Aplicamos a cada uno el mecanismo que le corresponde:
  \[2x - 12x + 8 = x + 4 -2x + 5\]
  Y ya tenemos una ecuación sencilla que podemos resolver con el método visto en la página \pageref{ecs_sencillas}:
  \begin{enumerate}
  \item Transponemos los términos y reducimos (o al revés, en estos dos el orden no importa):
    \[-9x = 1\]
  \item Despejamos la incógnita:
    \[x = \frac{1}{-9} = -\frac{1}{9}\]
  \end{enumerate}

  \subsection{Ecuaciones con fracciones.} \label{ecs_frac}
  Para entender como resolver ecuaciones con fracciones vamos a hacer primero un par de ejemplos sencillos que podamos resolver intuitivamente e intentar sacar conclusiones de ellos.\\

  \textbf{Primer ejemplo}: resolver la ecuación
  \[\frac{x}{5} + \frac{7}{5} = \frac{10}{5}\]
  En este caso está claro que la solución es $3$, ya que $\frac{3}{5} + \frac{7}{5} = \frac{10}{5}$\\

  \textbf{Segundo ejemplo}: resolver
  \[\frac{x}{153} + \frac{7}{153} = \frac{10}{153}\]
  Y aquí la solución también es $3$, por la misma razón: $\frac{3}{153} + \frac{7}{153} =
  \frac{10}{153}$\\

  ¿Qué conclusión podemos sacar de los anteriores ejemplos?. Fijémonos en las fracciones, en cada ecuación todos los términos tienen el mismo denominador, en la primera $5$ y en la segunda $153$. Con respecto a los numeradores podemos ver que en las dos ecuaciones hay los mismos numeradores en las mismas posiciones.\\
  Entonces \textbf{si todos los términos tienen el mismo denominador podemos ignorarlo y utilizar solo los numeradores}.\\

  Y conocemos un \textbf{método} para hacer que \textbf{todas las fracciones tengan el mismo denominador}, que es \textbf{\emph{reducir a denominador común}}.\\
  Con lo que reduciremos todos los términos de la ecuación a denominador común y nos olvidaremos de los denominadores.\\

  Vamos a verlo con \textbf{un ejemplo}: resolveremos
  \[\frac{x}{3} - 2 = \frac{1}{2} - 2x\]
  Tenemos que hacer el mínimo común múltiplo de $2$ y $3$, que es $6$, con lo que al hacer denominador común queda:
  \[\frac{2x}{6} - \frac{12}{6} = \frac{3}{6} - \frac{12x}{6}\]
  Y, por lo que hemos visto antes, ahora podemos olvidarnos de los denominadores 
  \[2x - 12 = 2 - 12x\]
  \begin{center}\small (\textbf{IMPORTANTE: esto de olvidarse de los denominadores cuando son todos iguales solo lo podemos hacer cuando estamos resolviendo ecuaciones. Nunca en otro caso})\end{center}
  Y ahora continuamos resolviendo la ecuación:
  \[2x - 12x = 2 + 12\]
  \[-10x = 14\]
  \[x = -\frac{10}{14} = -\frac{5}{7}\]

  \subsection{Ecuaciones con paréntesis y fracciones.} \label{ecs_par_frac}
  Al igual que hemos contado en las ecuaciones con paréntesis (sección \ref{ecs_parentesis}, página \pageref{ecs_parentesis}) lo que tenemos que hacer es respetar la prioridad de las operaciones y hacer primero los paréntesis. Pero aquí hay que tener en cuenta que cuando tenemos una operación en el numerador es lo mismo que si estuviese entre paréntesis, porque
  \[\frac{x + 5}{2} = (x + 5):2\]

  Vamos a ver unos cuantos ejemplos de esto:
  \begin{enumerate}
  \item \textbf{Resolver} $2x - \frac{x-1}{2} = 2 + 3*\frac{x +2}{4}$.
    \begin{solution}
      Como hemos dicho lo primero es hacer los paréntesis que en este caso son los numeradores de las fracciones, en el primero hay que cambiar de signo y en el segundo multiplicar (recordemos que $3$ es una fracción con denominador $1$, con lo que solo multiplica al numerador). Con esto nos queda:
      \[2x + \frac{-x+1}{2} = 2 + \frac{3x +6}{4}\]
      Ahora ya podemos hacer denominador común, que es $4$:
      \[\frac{8x}{4} + \frac{-2x + 2}{4} = \frac{8}{4} + \frac{3x+6}{4}\]
      Nos olvidamos de los denominadores:
      \[8x -2x + 2 = 8 + 3x + 6\]
      Y continuamos como en una sencilla:
      \[8x - 2x - 3x = 8+6 - 2\]
      \[3x = 12\]
      \[x = \frac{12}{3} = 4\]
    \end{solution}
    
  \item \textbf{Resolver} $\frac{2x-5}{3} - \frac{5x-1}{6} = \frac{3x-6}{8}$.
    \begin{solution}
      En este caso no tenemos paréntesis que multiplicar, solamente tenemos una fracción con un menos delante (que es como si estuviese delante de un paréntesis). Pero en este caso no vamos a hacerlo lo primero, sino que vamos a hacer primero el denominador común que en este caso se puede:
      \[\frac{16x - 40}{24} - \frac{20x - 4}{24} = \frac{9x - 18}{24}\]
      Ahora, como no hemos hecho primero lo que correspondía en la segunda fracción con el menos, tenemos que tenerlo en cuenta a la hora de olvidarnos de los denominadores y cambiar de signo a todo el denominador que está a su derecha ($20x - 4$):
      \[16x - 40 -20x + 4 = 9x - 18\]
      Y continuamos normalmente:
      \[16x - 20x - 9x = 40 - 4 - 18\]
      \[-13x = -18\]
      \[x = \frac{-13}{18}\]
    \end{solution}
    
  \item \textbf{Resolver} $-\frac{5-2x}{10} - \frac{x+5}{4} = \frac{x-16}{6} - \frac{-7+2x}{12}$.
    \begin{solution}
      A pesar de que parezca más complicada por ser más larga solo tenemos que seguir los mismos pasos.\\
      Hacemos el denominador común:
      \[-\frac{30 - 12x}{60} - \frac{15x + 75}{60} = \frac{10x - 160}{60} - \frac{-35 + 10 x}{60}\]
      Y a la hora de quitar los denominadores tenemos que tener en cuenta lo del menos delante de cada fracción que lo tenga.
      \[-30 + 12x - 15x - 75 = 10x - 160 + 35 - 10x\]
      Y ya es seguir como siempre:
      \[12x - 15x - 10x + 10x = 30+75-160-35\]
      \[-3x = -90\]
      \[x = \frac{-90}{-3}  = 30\]
    \end{solution}

  \end{enumerate}


  \subsection{Situaciones especiales al resolver ecuaciones.}
  Cuando estamos resolviendo ecuaciones es fácil que lleguemos a situaciones como esta:
  \[2x - 1 = 2(x + 3)\]
  \[2x - 1 = 2x + 6\]
  \[2x - 2x = 1+6\]
  \[0x = 7\]
  ¿Y qué significa esto? Pues esto es una de las situaciones especiales que nos podemos encontrar, y en ambas ocurre que en el lado de las incógnitas nos va a aparecer $0x$.\\
  Veamos los casos posibles:
  \begin{itemize}
  \item En el penúltimo paso nos queda $\boldsymbol{0x = 0}$.\\
    En ese caso $x$ tiene que ser un número que multiplicado por cero dé cero. Es decir $x$ puede ser cualquier número, con lo que las soluciones son infinitas. En este caso hablamos de una \textbf{ecuación indeterminada}.
  \item En el penúltimo caso nos queda $\boldsymbol{0x = k}$ \textbf{con} $\boldsymbol{k \neq 0}$.\\
    Aquí $x$ tiene que ser un número que multiplicado por cero dé un resultado distinto de cero, lo cual es imposible, y al ser imposible significa que la ecuación no tiene solución. En este caso hablamos de una \textbf{ecuación incompatible}.
  \end{itemize}

  \subsection{Ecuaciones de 2º grado.} \label{ecs_grado_2}
  Una ecuación de 2º grado es aquella en la que aparece algún monomio de grado 2, y en su forma reducida es como sigue:
  \[ax^2 + bx + c = 0\]
  Esta forma reducida es importante, porque es de la que tenemos que partir para resolverla. Si una ecuación no está en esa forma, por ejemplo
  \[3x^2 - x = 2x + 5\]
  Primero tenemos que ponerla en la forma reducida antes de hacer nada teniendo en cuenta que lo que cambia de lado hace la operación contraria:
  \[3x^2 - x - 2x - 5 = 0\]
  \[3x^2 - 3x - 5 = 0\]

  \subsubsection{Ecuaciones completas e incompletas.} \label{tipos_grado_2}
  Se dice que una ecuación de 2º grado es \textbf{completa si tiene los tres términos en su forma reducida}. Un ejemplo de este tipo es la última que aparece en el punto anterior.\\

  Se dice que una ecuación de 2º grado es \textbf{incompleta si en su forma reducida le falta algún término}. Por ejemplo\\
  \begin{itemize}
  \item $2x^2 - 8 = 0$\quad\quad La falta el término en $x$.
  \item $3x^2 - 5x = 0$\quad\quad La falta el término independiente.
  \end{itemize}

  Más adelante veremos que estas últimas tienen métodos propios para resolverlas de manera muy rápida.
  \subsubsection{Resolución de ecuaciones de 2º grado. Fórmula general.}
  Esta es una fórmula que nos da las soluciones de una ecuación de 2º grado. Es una fórmula fácil de utilizar, solo hay que identificar correctamente cada uno de los valores que hay que poner. Así que vamos a por ella.\\

  Hemos visto que la forma reducida de la ecuación de 2º grado es:
  \[\boldsymbol{ax^2 + bx + c = 0}\]
  Donde $a$ es el coeficiente de $x^2$, $b$ el de $x$ y $c$ el término independiente (término sin $x$). Con esa información podemos resolver la ecuación utilizando la fórmula:
  \begin{large}
    \[x = \frac{-b\pm \sqrt{b^2 - 4*a*c}}{2*a}\]
  \end{large}
  Puede parecer complicada de utilizar, pero vamos a ver algunos detalles sobre ella y algunos ejemplos. Eso sí, tenemos que tener en cuenta que \textbf{esta fórmula hay que aprendérsela de memoria}.\\

  Si nos fijamos en la fórmula vemos que aparece un \textbf{símbolo extraño} ($\boldsymbol{\pm}$) delante de la raíz. Este símbolo significa que en realidad estamos juntando dos fórmulas en una. Es decir, en realidad tenemos dos soluciones:
  \begin{itemize}
  \item $x_1 = \frac{-b+ \sqrt{b^2 - 4*a*c}}{2*a}$
  \item $x_ 2 = \frac{-b- \sqrt{b^2 - 4*a*c}}{2*a}$
  \end{itemize}
  Pero como solo se diferencian en ese signo y queremos escribir lo menos posible se juntan las dos en una poniendo ese símbolo que indica que debemos hacer la operación una vez con el más y otra con el menos.\\

  Vamos a ver \textbf{unos ejemplos} de como utilizar la fórmula para resolver ecuaciones de 2º grado.
  \begin{enumerate}
  \item Resuelve $x^2 - 3x +2 = 0$.
    \begin{solution}
      Lo primero que tenemos que hacer es identificar los coeficientes comparándola con la forma reducida que hemos puesto antes:
      \[\begin{array}{rrrc}
          ax^2&+bx&+c&=0\\x^2&-3x&+2&=0
        \end{array}\]
      Con lo que tenemos que
      \[\begin{cases}
          a=1\\b=-3\\c=2
        \end{cases}
      \]
      ya que $a$ es el coeficiente de $x^2$ (y recordemos que cuando no hay coeficiente es porque vale 1), $b$ es el coeficiente de $x$ (en este caso $-3$, recordemos que los coeficientes incluyen el signo) y $c$ es el que no lleva $x$ (en ese caso $2$).\\
      Ahora solo tenemos que sustituirlos en la fórmula:
      \[x = \frac{-(-3)\pm\sqrt{(-3)^2 - 4*1*2}}{2*1}\]
      Realizamos las operaciones siguiendo la jerarquía:
      \[x = \frac{3\pm\sqrt{9 - 8}}{2} = \frac{3\pm 1}{2}\]
      Y hemos llegado al punto en el que hay que hacer la parte del más-menos:
      \[\def\arraystretch{1.5}
	x = \frac{3\pm 1}{2} = \left\lbrace
          \begin{array}{ll}   
            \frac{3+ 1}{2} &=\frac{4}{2} = 2\\
            \frac{3- 1}{2} &=\frac{2}{2} = 1
          \end{array}
        \right.
      \]
      Es decir, cuando llegamos a ese punto la fórmula se convierte en dos operaciones, una con el más y otra con el menos.\\
      
      Y con esto hemos obtenido que las soluciones de la ecuación son
      \begin{itemize}
      \item $x=1$
      \item $x = 2$
      \end{itemize}
      Solo nos queda comprobarlas con la ecuación original ($x^2 - 3x + 2 = 0$) para ver si la hemos resuelto bien:
      \begin{itemize}
      \item Con $x=1$:\\
        \[1^2 - 3*1 + 2 = 1 - 3 + 2 = 0\]
        Con lo que $x=1$ funciona.
      \item Con $x=2$:
        \[2^2 - 3*2 + 2 = 4 - 6 + 2 = 0\]
        Que también funciona.
      \end{itemize}
      Y si las dos hacen que se cumpla la ecuación es que lo hemos resuelto bien.
      
    \end{solution}
    
  \item Resuelve la ecuación $2x(x-3) = x^2 - 2x-3$.
    \begin{solution}
      Lo primero que observamos es que la ecuación \textbf{no está en forma reducida}, así que tendremos que escribirla de esa forma y para ello usaremos un mecanismo similar al que hemos visto en las ecuaciones de primer grado (sección \ref{ecs_par_frac}, página \pageref{ecs_par_frac}):
      \begin{itemize}
      \item Quitamos los paréntesis:
        \[2x^2 - 6x = x^2 +2x - 3\]
      \item Lo llevamos todo a un lado:
        \[2x^2 - 6x - x^2 + 2x +  3 = 0\]
      \item Reducimos:
        \[x^2 - 4x + 3 = 0\]
      \item Identificamos los coeficientes:
        \[a=1\quad b = - 4\quad c= 3\]
      \item Aplicamos la fórmula para obtener las soluciones:
        \[x = \frac{4\pm \sqrt{16 - 12}}{2} = \frac{4 \pm 2}{2} =
          \begin{cases}
            x = 3\\x=2
          \end{cases}\]
      \end{itemize}
    \end{solution}
    
  \item Resuelve $2x^2 - 3x = 0$.
    \begin{solution}
      En este caso lo que tenemos es que no están todos los términos, con lo que al identificarlos para utilizar la fórmula tendremos que poner cero en los que no estén. Es decir, en la ecuación que nos plantean tenemos que $a=2$, $b=-3$ y $c=0$.\\
      Y con esto ya podemos aplicar la fórmula:
      \[x = \frac{3\pm \sqrt{9 - 0}}{4} = \frac{3\pm 3}{4} =
        \begin{cases}
          x = \frac{3}{2}\\x = 0
        \end{cases}\]
      Y ya tenemos las soluciones.
    \end{solution}
    
  \item Resuelve $2x^2 - 32 = 0$.
    
    \begin{solution}
      Aquí nos ocurre lo mismo que en la anterior, que no están todos los términos, y haciendo cero los que no están identificamos $a= 2$, $b = 0$ y $c = -32$.\\
      
      Aplicando la fórmula:
      \[x = \frac{-0 \pm \sqrt{0^2 - 4 *2*(-32)}}{2*2} = \frac{\pm \sqrt{256}}{4} = \frac{\pm 16}{4} =
        \begin{cases}
          x= 4\\x= -4
        \end{cases}\]
      Así, las soluciones son $=4$ y $x=-4$.
    \end{solution}
  \end{enumerate}
  
  \subsubsection{Ecuaciones de segundo grado incompletas. Métodos propios.}
  Como hemos dicho en la sección \ref{tipos_grado_2} (página \pageref{tipos_grado_2}), hay un tipo de ecuaciones de 2º grado que no tienen todos los términos y que hay maneras de resolverlas más rápidas que utilizando la fórmula. Para hacerlo vamos a dividirlas a su vez en dos tipos y ver qué método hay que utilizar en cada una.

  \paragraph{Ecuaciones a las que las falta el término en $\boldsymbol{x}$}.
    La forma de estas ecuaciones es $\boldsymbol{ax^2 + c = 0}$, y para resolverlas rápidamente vamos a utilizar un método similar al que hemos visto para las ecuaciones de primer grado:
    \begin{enumerate}
    \item Llevamos lo que tiene $x^2$ a un lado y lo que no al otro:
      \[ax^2 = -c\]
    \item Como queremos dejar la $x$ sola, pasamos el coeficiente al otro lado:
      \[x^2 = \frac{-c}{a}\]
    \item Y para quitar el cuadrado tenemos que hacer la raíz en el otro lado, recordando que si nos sale una raíz al resolver una ecuación hay que ponerla el $\pm$:
      \[x= \pm \sqrt{\frac{-c}{a}}\]
      Y ya tenemos las dos soluciones en caso de que el radicando sea positivo. En caso de que que sea negativo ya sabemos que no tiene solución.
    \end{enumerate}
    Vamos a ver \textbf{un ejemplo}: resolver $2x^2 - 18 = 0$.
    \begin{solution}   
      Vemos que es del tipo que estamos estudiando ahora, con lo que seguimos el método indicado:
      \begin{enumerate}
      \item Llevamos el término independiente al otro miembro de la ecuación:
        \[2x^2 = 18\]
      \item Pasamos el coeficiente de $x^2$ dividiendo:
        \[x^2 = \frac{18}{2} = 9\]
      \item Obtenemos las soluciones haciendo la raíz:
        \[x = \pm \sqrt{9} =
          \begin{cases}   
            3\\-3
          \end{cases}\]
        
      \end{enumerate}
      
    \end{solution}
    
  \paragraph{Ecuaciones a las que las falta el término independiente}.
    La forma de estas ecuaciones es $ax^2 + bx = 0$ y para resolverlas vamos a utilizar un método razonando cómo funciona la multiplicación.\\
    Lo primero que hacemos es sacar $x$ factor común:
    \[x*(ax + b) = 0\]
    Y nos queda un producto de dos factores ($x$ es un factor y $ax + b$ el otro) que tiene como resultado cero.\\
    Para que una multiplicación tenga de resultado cero es necesario que uno de los dos factores sea cero, puesto que no hay dos números distintos de cero cuya multiplicación dé cero. Entonces tenemos dos opciones:
    \begin{enumerate}
    \item Que $x$ sea cero, lo cual nos da una primera solución: $x=0$.
    \item Que la operación $ax+b$ de cero, con lo que nos queda la ecuación $ax+b = 0$ que es bastante fácil de resolver, puesto que solo hay que seguir los pasos de la ecuación sencilla de primer grado:
      \[ax = -b\]
      \[x = \frac{-b}{a}\]
    \end{enumerate}
    Y ya tendríamos las dos soluciones.\\
    Veámoslo con un \textbf{ejemplo}: Resolver $3x^2 - 5x = 0$.
    \begin{solution}  
      Siguiendo los pasos lo primero que tenemos que hacer es sacar factor común:
      \[x*(3x - 5) = 0\]
      Y la primera solución es $x=0$.\\
      La segunda la obtenemos al resolver $3x-5=0$:
      \[3x = 5\]
      \[x = \frac{5}{3}\]
      Con lo que las soluciones de la ecuación son:
      \begin{itemize}
      \item $x=0$
      \item $x = \frac{5}{3}$
      \end{itemize}
      
    \end{solution}
    

  \section{Sistemas de ecuaciones I.}\label{sistemasI}
  A veces nos vamos a encontrar con situaciones en las que vamos a desconocer más de una cantidad, con lo cual vamos a tener que utilizar dos incógnitas.\\
  En estas situaciones no nos va a bastar con una ecuación, sino que para poderlas resolver necesitaremos tantas ecuaciones como incógnitas y a esto es a lo que se llama un sistema de ecuaciones. Vamos a definirlo más formalmente:\\
  
  Un sistema de ecuaciones es un conjunto de ecuaciones que se tienen que cumplir a la vez. Es decir, todas tienen que tener las mismas soluciones.\\
  
  En esta situación se ponen las ecuaciones agrupadas con una llave, por ejemplo:
  \[\begin{cases}
      2x - 3y = 0\\4x + 2y = 14
    \end{cases}
  \]
  Y la solución de este ejemplo es $x=3$ e $y=2$. Podemos comprobar que cumple las dos ecuaciones:
  \begin{itemize}
  \item $2*3 - 3*2 = 0$, que es cierto.
  \item $4*3 + 2*2 = 14$, que también es cierto.
  \end{itemize}
  
  El ejemplo que se ha puesto es un sistema de dos ecuaciones con dos incógnitas, que es el tipo de sistema que corresponde estudiar a este nivel.
  
  \subsection{Métodos para la resolución de sistemas de dos incógnitas.}
  Para resolver sistemas de ecuaciones de dos incógnitas existen tres métodos:
  \textbf{
    \begin{itemize}
    \item Método de sustitución.
    \item Método de reducción.
    \item Método de igualación.
    \end{itemize}
  }
  
  Todos los métodos que acabamos de enumerar se basan en reducir e sistema a una ecuación de primer grado sencilla con la que obtener el valor de una incógnita, y una vez obtenida ésta se puede calcular el valor de la otra.\\
  
  Vamos a ver cada uno de ellos en detalle utilizando un ejemplo como guía de los pasos a seguir.
  \subsubsection{Método de sustitución.}
  Utilizaremos el siguiente sistema de ejemplo:
  \[\begin{cases}
      3x-2y&=5 \quad \text{(primera ecuación)}\\
      x-3y &= 4 \quad \text{(segunda ecuación)}
    \end{cases}\]
  Y los pasos que hay que dar son los siguientes:
  \begin{enumerate}
  \item \textbf{Despejamos una de las incógnitas en una de las ecuaciones}, la que nos resulte más fácil.\\
    En el ejemplo que hemos puesto la más fácil de despejar es la $x$ en la segunda ecuación, porque si despejamos la $x$ en la primera o despejamos la $y$ vamos a acabar teniendo que manejar fracciones y siempre es un engorro.\\
    Con esto el sistema queda:
    \[\begin{cases}
        3x-2y&=5 \quad \text{(primera ecuación)}\\
        x&= 4 + 3y \quad \text{(segunda ecuación)}
      \end{cases}  
    \]
  \item \textbf{Sustituimos la incógnita despejada por la expresión obtenida en la otra ecuación.} En este caso, en la primera ecuación sustituimos ($\boldsymbol{x}$) por la expresión que hemos obtenido al despejar en la segunda ($\boldsymbol{4+3y}$, sin olvidarse de poner paréntesis), de manera que solo nos queda una ecuación con una incógnita:
    \[3*(4+3y) - 2y = 5 \quad \text{(primera ecuación)}\]
  \item \textbf{Resolvemos la ecuación obtenida en el apartado anterior.}
    \[12 + 9y - 2y = 5\quad \text{\small {\textit{Desarrollamos la multiplicación por el paréntesis}}}\]
    \[9y - 2y = 5 - 12\quad \text{\small {\textit{Llevamos cada término al lado que le corresponde}}}\]
    \[7y = -7\quad\text{\small{\textit{Reducimos}}}\]
    \[y = \frac{-7}{7} = -1\quad\text{\small{\textit{Despejamos}}}\]
  \item \textbf{Sustituimos el valor obtenido en la expresión que hemos obtenido en el primer paso.} En este caso en la expresión obtenida de la segunda ecuación ($\boldsymbol{x = 4 + 3y}$) cambiaremos la $y$ por $-1$;
    \[x = 4 + 3*(-1) \quad\text{\small{\textit{Como es negativo tenemos que poner paréntesis}}}\]
    \[x = 1\]
  \item \textbf{Comprobamos.} Sustituimos los valores de $x$ e $y$ en ambas ecuaciones y vemos si nos dan el resultado que tienen.
    \[\begin{cases}
        3*1-2*(-1)&=5 \quad \text{(primera ecuación)}\\
        1-3*(-1) &= 4 \quad \text{(segunda ecuación)}
      \end{cases}\]
    \[\begin{cases}
        3+2&=5 \quad \text{(primera ecuación)}\\
        1+3*(-1) &= 4 \quad \text{(segunda ecuación)}
      \end{cases}\]
    Y vemos que en ambas coincide.\\
    Por tanto, \textbf{la solución del sistema es $x=1$, $y=-1$}.
  \end{enumerate}
  Vamos a ver un \textbf{ejemplo un poco más complejo}: resolver el sistema
  $\begin{cases}
     3x+2y = 1\\
     2x+3y = 4
   \end{cases}$
   \begin{solution}
     Vamos por pasos tal y como hemos visto:
     \begin{itemize}
     \item \textbf{Despejamos una incógnita}. En este caso no hay ninguna que nos vaya a resultar más sencilla que las otras. Así que la elegimos al azar, por ejemplo la $y$ en la primera ecuación.
       \[\begin{cases}
           2y = 1-3x\\
           2x +3y= 4
         \end{cases}
       \]
       \[\begin{cases}
           y = \frac{1-3x}{2}\\
           2x+3y = 4
         \end{cases}
       \]
       Y vemos que nos queda una fracción. Esto hará que el resto de la resolución sea un pelín más complicado, pero no mucho.
     \item \textbf{Sustituimos la incógnita despejada en la otra ecuación.}
       \[2x + 3*\frac{1-3x}{2} = 4\]
     \item \textbf{Resolvemos la ecuación obtenida con la sustitución.} En ese caso tendremos que aplicar la resolución de ecuaciones con fracciones (apartado \ref{ecs_frac}, página \pageref{ecs_frac})
       \[2x + \frac{3-9x}{2} = 4 \quad\text{\small{\textit{Hacemos la multiplicación por la fracción}}}\]
       \[\frac{4x}{2} + \frac{3-9x}{2} = \frac{8}{2}
         \quad \text{\small{\textit{Denominador común en toda la ecuación}}}
       \]
       \[
         4x + 3 - 9x = 8\quad\text{\small{\textit{Quitamos los denominadores}}}
       \]
       Y resolvemos normalmente:
       \[4x - 9x = 8 -3\]
       \[5x = -5\]
       \[x = \frac{-5}{5} = -1\]
     \item \textbf{Sustituimos la incógnita obtenida en la expresión de sustitución.} La expresión de sustitución (del primer apartado) es $\boldsymbol{y = \frac{1-3x}{2}}$:
       \[y = \frac{1-3*(-1)}{2} = \frac{1+3}{2} = \frac{4}{2} = 2\]
     \item \textbf{Comprobamos}. Cambiamos $x$ por $-1$ e $y$ por $2$ en el sistema original para ver si coincide.
       \[
         \begin{cases}
           3*(-1)+2*2 = 1\\
           2*(-1)+3*2 = 4
         \end{cases}
       \]

       \[
         \begin{cases}
           -3+4 = 1\\
           -2+6 = 4
         \end{cases}
       \]
       Y como coincide todo significa que lo hemos resuelto bien y la solución del sistema es $x=-1$, $y=2$.
     \end{itemize}
     
   \end{solution}
   \subsubsection{Método de reducción.}
  Utilizaremos el mismo sistema de ejemplo que en el método anterior:
  \[\begin{cases}
      3x-2y&=5 \quad \text{(primera ecuación)}\\
      x-3y &= 4 \quad \text{(segunda ecuación)}
    \end{cases}\]
  En este caso lo que vamos a hacer es sumar, o restar según el caso, las dos ecuaciones término a término, de manera que consigamos hacer desaparecer una de las incógnitas y nos quede una ecuación sencilla. Para esto necesitamos que el coeficiente de una de las incógnitas sea el mismo en las dos ecuaciones, y para ello haremos uso del \textit{mínimo común múltiplo}.\\
  Vamos a ver cómo es este método paso a paso:
  \begin{enumerate}
  \item Elegimos una de las incógnitas, por ejemplo la $x$ y buscamos el mínimo común múltiplo de sus coeficientes (m.c.m.$(1, 3) = 3$) y multiplicamos cada ecuación (toda entera) por el número que corresponda para que la incógnita elegida tenga de coeficiente el m.c.m. calculado.\\
    En este caso el m.c.m. es 3.\\
    La $x$ en la primera ecuación ya tiene coeficiente 3, con lo que no hay que multiplicarla por nada.\\
    La $x$ en la segunda ecuación tiene de coeficiente 1, con lo que hay que multiplicar toda la ecuación por 3: \[3*(x-3y = 4)\ \rightarrow\ 3x - 9y = 12\]
    Con esto el sistema queda:
    \[\begin{cases}
      3x-2y&=5 \quad \text{(primera ecuación)}\\
      3x-9y &= 12 \quad \text{(segunda ecuación)}
      \end{cases}\]
  \item Ahora tenemos que operar las dos ecuaciones para que desaparezca la incógnita elegida.\\
    En este caso hemos elegido la $x$, y la operación que hay que hacer con $3x$ y $3x$ para que el resultado sera cero es la \textit{resta}.\\
    Restamos término a término:
    \[\left\lbrace
        \begin{array}{rrrrr}
          &3x&-2y&=&5\\
          \boldsymbol{-}&&&&\\
          &3x&-9y&=&12\\
          \hline
          &&7y&=&-7
        \end{array}
      \right.
    \]

  \item Resolvemos la ecuación que ha quedado, que suele ser bastante sencilla.
    \[7y = -7\]
    \[y = \frac{-7}{7} = -1\]

  \item Para resolver la otra incógnita sustituimos el valor de la obtenida en una de las dos ecuaciones iniciales.\\
    En el ejemplo que hemos puesto hemos obtenido la $y$, y la ecuación en la que la $x$ aparece más sencilla (con el coeficiente más bajo) es en la segunda ($x - 3y = 4$).\\
    Sustituimos y resolvemos:
    \[x- 3*(-1) = 4\]
    \[x+3 = 4\]
    \[x = 4 - 3 = 1\]
  \item Por último tenemos que comprobar el sistema, pero como ya lo habíamos comprobado en el método anterior nos vamos a saltar este paso.
    
  \end{enumerate}
  Y con este método hemos llegado a la misma solución que con el otro: $x=1$, $y=-1$.

  Vamos a por el mismo ejemplo complejo del caso anterior:
  $\begin{cases}
     3x+2y = 1\\
     2x+3y = 4
   \end{cases}$
   \begin{solution}
     \begin{enumerate}
     \item Elegimos la incógnita y calculamos el m.c.m. de sus coeficientes.\\
       Ahora vamos a elegir la $y$, el mínimo común múltiplo de sus coeficientes es m.c.m.$(2,3)=6$\\
       Para que la $y$ tenga coeficiente 6 en la primera ecuación hay que multiplicarla
       por 3: \[3*(3x + 2y = 1) \rightarrow 9x +6y = 3\]
       Para que la $y$ tenga coeficiente 6 en la segunda ecuación hay que multiplicarla por 2:
       \[2*(2x + 3y = 4) \rightarrow 4x + 6y = 8\]
       Y el sistema nos ha quedado:
       \[\begin{cases}
           9x + 6y = 3\\
           4x+6y = 8
         \end{cases}\]
     \item Sumamos, o restamos, las ecuaciones para que desaparezca la incógnita elegida.\\
       En este caso volveremos a restar:
       \[\left\lbrace
           \begin{array}{rrrrr}
             &9x&+6y&=&3\\
             \boldsymbol{-}&&&&\\
             &4x&+6y&=&8\\
             \hline
             &5x&&=&-5
           \end{array}
         \right.
       \]
     \item Resolvemos la ecuación resultante:
       \[x = \frac{-5}{5} = -1\]
       
     \item Sustituimos en una de las ecuaciones del principio para obtener la otra incógnita.\\
       En este caso vamos a utilizar la primera ($3x + 2y = 1$):
       \[3*(-1) + 2y = 1\]
       \[-3  + 2y = 1\]
       \[2y = 1+3\]
       \[y = \frac{4}{2} = 2\]
     \end{enumerate}

     Que coincide con la solución obtenida con el otro método: $x = -1$, $y=2$.
   \end{solution}
   
   \subsubsection{Método de igualación.}
   Este método consiste en cambiar de lado los términos de cada ecuación de
   manera que en ambas quede la misma expresión en uno de los lados.
   De esta manera se pueden igualar las expresiones resultantes y
   resolver una ecuación relativamente sencilla.\\
   
   Para ver los pasos utilizaremos el mismo sistema de ejemplo que en los métodos anteriores:
   \[
     \begin{cases}
       3x-2y&=5 \quad \text{(primera ecuación)}\\
       x-3y &= 4 \quad \text{(segunda ecuación)}
     \end{cases}
   \]

   Y empezamos con los pasos:
   \begin{enumerate}
   \item Movemos todo para que en ambas ecuaciones quede lo mismo en uno de los lados.\\
     En este caso vamos a hacer que en el lado izquierdo nos quede la $x$ sola en el lado izquierdo.\\
     
     \quad Primero llevamos a la derecha todo lo que no tiene $x$:
     \[
       \begin{cases}
         3x&=5+2y \quad \text{(primera ecuación)}\\
         x&= 4 + 3y \quad \text{(segunda ecuación)}
       \end{cases}
     \]
     \quad Y ahora el coeficiente de la $x$ pasa al otro lado en cada ecuación.
     \[
       \begin{cases}
         x&=\frac{5+2y}{3} \\
         x&= 4 + 3y \quad \text{\small{Ésta queda igual porque el coeficiente
            de $x$ es 1}}
       \end{cases}
     \]
   \item Como tenemos algo que es igual a dos expresiones distintas, estas dos expresiones tienen que ser iguales entre sí.\\
     En el caso en el que estamos $\boldsymbol{x}$ es igual a $\frac{5+2y}{3}$ y $\boldsymbol{x}$ también es igual a $4 + 3y$, con lo que ambas expresiones tienen que ser iguales y nos queda la ecuación:
     \[\frac{5+2y}{3} = 4+3y\]
   \item Resolvemos la ecuación resultante.\\
     En este ejemplo es una ecuación con denominadores, con lo que utilizamos el método que vimos en el apartado \ref{ecs_frac}.
     \[\frac{5+2y}{3} = \frac{12+9y}{3}\quad\text{\small{Hacemos denominador común.}}\]
     \[5+2y = 12+9y\quad\text{\small{Quitamos los denominadores y resolvemos}}\]
     \[5 - 12 = 9y -2y\]
     \[-7 = 7y\]
     \[y = \frac{-7}{7} = 1\]
   \item Obtenemos la otra incógnita sustituyendo el valor de la que ya hemos resuelto en una de las ecuaciones originales o en una de las expresiones obtenidas en el paso 1.\\
     Vamos a utilizar la segunda expresión obtenida en el paso 1 ($x = 4 + 3y$):
     \[x = 4 + 3* (-1)\]
     \[x = 4 -3 = 1\]
   \end{enumerate}
  Y esta solución ($x=1$, $y=-1$) es la misma que hemos obtenido con los métodos anteriores, con lo que no vamos a comprobarla.

  Y ahora resolvemos el mismo ejemplo complejo de los casos anteriores:
  $\begin{cases}
     3x+2y = 1\\
     2x+3y = 4
   \end{cases}$
   \begin{solution}
     \begin{enumerate}
     \item Ahora vamos a dejar la $y$ sola en las dos ecuaciones:
       \[
         \begin{cases}
           2y = 1-3x\\
           3y = 4 - 2x
         \end{cases}
       \]

       \[
         \begin{cases}
           y = \frac{1-3x}{2}\\
           y = \frac{4 - 2x}{3}
         \end{cases}
       \]
     \item Igualamos las dos expresiones a las que es igual $y$:
       \[\frac{1-3x}{2} = \frac{4 - 2x}{3}\]

     \item Resolvemos la ecuación obtenida:
       \[\frac{3-9x}{6} = \frac{8-4x}{9}\]
       \[3-9x = 8-4x\]
       \[3-8 = 9x - 4x\]
       \[-5 = 5x\]
       \[x = \frac{-5}{5} = -1\]

     \item Sustituimos en la primera expresión obtenida en el paso 1 para resolver la $y$:
       \[y = \frac{1 - 3*(-1)}{2}\]
       \[y = \frac{1+3}{2} = 2\]
     \end{enumerate}
     Y volvemos a obtener la misma solución ($x=-1$, $y=2$) que con los métodos anteriores, con lo que no vamos a comprobarla.
   \end{solution}

 \part{3º de ESO}
   \section{Polinomios III.}
   Gran parte del temario de 3º de E.S.O. son los mismos contenidos de 2º, con lo que en estos apuntes solo se van a exponer los contenidos nuevos, que son la división por el método ``estándar'' y por Ruffini.
   \subsection{Operaciones con polinomios III.}
   \subsubsection{División de polinomios.}
   La división de polinomios es exactamente igual que la división de enteros. Está claro que a primera vista parece algo bastante complicado, sobre todo viendo las demás operaciones, pero tenemos que empezar pensando que cualquier entero es en realidad un polinomio:
   \[34\,712 = 3*10^4 + 4* 10^3 + 7* 10^2 + 1*10 + 2\]
   Esto no se diferencia en (casi) nada del polinomio $3x^4 + 4x^3 + 7x^2 + x + 2$.\\

   A la hora de realizar las divisiones de polinomios tenemos que identificar que los grados de los términos se corresponden con las posiciones de las cifras en los enteros y que si nos falta algún grado es porque la cifra correspondiente a esa posición es cero.\\

   Vamos a ver entonces la \textbf{división de polinomios con un ejemplo}. En este ejemplo vamos a realizar la división $\frac{2x^5 + x^3 - 3x^2 + 2x - 1}{x^2 - x}$
   \begin{enumerate}
   \item Empezamos poniendo la división como ponemos la división de enteros, con su cajita:
     \begin{center}
     \begin{tabular}{rrrrrrrr}
 &$2x^5$&$+0x^4$  &$+1x^3$  &$-3x^2$  &$+2x$  & \multicolumn{1}{r|}{$-1$} &  $x^2 - x$\\ \cline{8-8} 
 &  &  &  &  &  & & \phantom{a}
     \end{tabular}
   \end{center}
   Es fácil observar que en el dividendo hemos puesto ceros en los grados que faltan así como, para facilitar la lectura de este primer ejemplo, se han puesto los coeficientes que son 1 y todos los signos.\\
   Una vez hecho esto procedemos con el siguiente paso:
 \item Dividimos el \textbf{monomio} de mayor grado del dividendo entre el monomio de mayor grado del divisor. \label{repite_div}\\
   En el caso que nos ocupa hacemos la división $\frac{2x^5}{x^2}=2x^3$ y eso será lo que pongamos en el cociente \textbf{con su signo}:
   \begin{center}
     \begin{tabular}{rrrrrrrl}
 &$2x^5$&$+0x^4$  &$+1x^3$  &$-3x^2$  &$+2x$  & \multicolumn{1}{r|}{$-1$} &  $x^2 - x$\\ \cline{8-8} 
 &  &  &  &  &  & & $\boldsymbol{+2x^3}$
     \end{tabular}
   \end{center}
 \item Ahora multiplicamos el término que hemos añadido al cociente por todos los términos del divisor:
   \[\begin{cases}
       2x^3* (-x) &= -2x^4\\
       2x^3*x^2 &= 2x^5                                                                              \end{cases}\]
   Y situamos cada uno de los resultados alineado con el grado que le corresponde:
   \begin{center}
     \begin{tabular}{rrrrrrrl}
       &$2x^5$&$+0x^4$  &$+1x^3$  &$-3x^2$  &$+2x$  & \multicolumn{1}{r|}{$-1$} &  $x^2 - x$\\ \cline{8-8} 
       &$\boldsymbol{2x^5}$  &  $\boldsymbol{-2x^4}$&  &  &  & & $+2x^3$
     \end{tabular}
   \end{center}
 \item Y restamos, al igual que en una división de enteros:
   \begin{center}
     \begin{tabular}{rrrrrrrl}
 &$2x^5$&$+0x^4$  &$+1x^3$  &$-3x^2$  &$+2x$  & \multicolumn{1}{r|}{$-1$} &  $x^2 - x$\\ \cline{8-8} 
 $\overline{\phantom{AA}}$&$2x^5$  &  $-2x^4$&  &  &  & & $+2x^3$\\
       \cline{1-7}
       &$\boldsymbol{0}$  &  $\boldsymbol{2x^4}$&$+1x^3$&$-3x^2$&$+2x$&$-1$&
     \end{tabular}
   \end{center}
   En este punto \textbf{tenemos que tener mucho cuidado con los signos}, tal y como se puede ver en el resultado del término de grado 4.
 \item A partir de lo anterior tenemos que repetir desde el paso \ref{repite_div} con el mayor grado que nos quede en el resto, hasta que el grado del resto sea menor que el grado del divisor.\\
   En este ejemplo el mayor grado del resto es 4, con lo que podemos continuar. Hacemos $\frac{2x^4}{x^2} = 2x^2$, lo ponemos en el cociente y repetimos los pasos anteriores (a partir de aquí no vamos a poner los ceros):
   \begin{center}
     \begin{tabular}{rrrrrrrl}
 &$2x^5$&$+0x^4$  &$+1x^3$  &$-3x^2$  &$+2x$  & \multicolumn{1}{r|}{$-1$} &  $x^2 - x$\\ \cline{8-8} 
 $\overline{\phantom{AA}}$&$2x^5$  &  $-2x^4$&  &  &  & & $+2x^3+2x^2$\\
       \cline{1-7}
 &&  $2x^4$&$+1x^3$&$-3x^2$&$+2x$&$-1$&\\
 & $\overline{\phantom{AA}}$ &  $2x^4$&$-2x^3$&&&&\\
       \cline{2-7}
       &&&$3x^3$&$-3x^2$&$+2x$&$-1$&
     \end{tabular}
   \end{center}
   Y como el resto es de mayor grado que el divisor, repetimos.\\
   Dividimos $\frac{3x^3}{x^2} = 3x$, lo añadimos al cociente, multiplicamos y restamos:
   \begin{center}
     \begin{tabular}{rrrrrrrl}
 &$2x^5$&$+0x^4$  &$+1x^3$  &$-3x^2$  &$+2x$  & \multicolumn{1}{r|}{$-1$} &  $x^2 - x$\\ \cline{8-8} 
 $\overline{\phantom{AA}}$&$2x^5$  &  $-2x^4$&  &  &  & & $+2x^3+2x^2 + 3x$\\
       \cline{1-7}
 &&  $2x^4$&$+1x^3$&$-3x^2$&$+2x$&$-1$&\\
 & $\overline{\phantom{AA}}$ &  $2x^4$&$-2x^3$&&&&\\
       \cline{2-7}
 &&&$3x^3$&$-3x^2$&$+2x$&$-1$&\\
 &&$\overline{\phantom{AA}}$&$3x^3$&$-3x^2$&&&\\
       \cline{3-7}
       &&&&&$+2x$&$-1$&
     \end{tabular}
   \end{center}
 \item Ahora el grado del resto es 1, que es menor que el grado del divisor (2) y eso quiere decir que hemos terminado.\\
   Tenemos que colocarla de manera que luego podamos utilizar el resultado. La manera de colocarla es la siguiente:
   \[\frac{\text{Dividendo}}{\text{Divisor}} = \text{Cociente} +
     \frac{\text{Resto}}{\text{Divisor}}\]

   Y con lo obtenido en el ejemplo que hemos hecho, queda:
   \[\frac{\overbrace{2x^5 + x^3 - 3x^2 + 2x - 1}^{\text{Dividendo}}}
     {\underbrace{x^2 - x}_{\text{Divisor}}} \quad=\quad
     \overbrace{2x^3 + 2x^2 + 3x}^{\text{Cociente}} \quad+\quad
     \frac{\overbrace{2x-1}^{\text{Resto}}}{x^2 - x}\]
 \end{enumerate}
 
\subsubsection{División por Ruffini}\label{ruffini}
La división por Ruffini es un procedimiento muy sencillo con el problema de que solo sirve para divisiones en las que el divisor tiene la forma $(\boldsymbol{x\pm a})$ (por ejemplo $(x + 3)$ ó $(x -1)$, pero para $(3x -2)$ no sirve).\\
Parece que está muy limitado, pero eso no quita que se use muchísimo.\\

Antes de empezar a ver cómo funciona tenemos que fijarnos en algunos detalles de las divisiones de polinomios.\\
Si nos fijamos en el procedimiento que hemos visto en el apartado anterior podemos sacar las siguientes conclusiones:
\begin{itemize}\label{props_cociente_resto}
\item El grado del dividendo tiene que ser mayor o igual que el del divisor.
\item El grado del cociente es el grado del dividendo menos el grado del divisor.
\item El grado del resto es menor que el grado del dividendo.
\end{itemize}
Es importante tener esto en cuenta a la hora de hacer una división por Ruffini.\\

Y ahora sí, vamos a ver \textbf{como funciona la división por Ruffini con un ejemplo}:\\
Vamos a realizar la división $\frac{x^3 - 2x + 4}{x-2}$
\begin{enumerate}
\item Colocamos los \textbf{coeficientes} del dividendo de la siguiente manera, poniendo ceros en los grados que faltan:
  \[\begin{NiceArray}[t]{r|rrrr}
      &1&0&-2&4\\
      \phantom{+2}&&&&\\
      \hline
      &&&&
    \end{NiceArray}
  \]

\item Colocamos el \textbf{opuesto del término independiente del divisor} en la parte inferior izquierda, así:
  \begin{center}
    \begin{tabular}{r|rrrr}
      &1&0&-2&4\\
      +2&&&&\\
      \hline
      &&&&
    \end{tabular}
  \end{center}
\item Bajamos directamente el primer coeficiente del dividendo:
  \[
    \begin{NiceArray}{r|rrrr}
      &1&0&-2&4\\
      +2&&&&\\
      \hline
      &\boldsymbol{1}&&&
      \CodeAfter
      \begin{tikzpicture}
        \begin{scope}[->]
          \draw ([yshift=-.3\baselineskip]1-2.center) --
          ([yshift=.3\baselineskip]3-2.center);
        \end{scope}
      \end{tikzpicture}
    \end{NiceArray}
  \]
\item Multiplicamos este resultado por el término independiente (abajo-izquierda y lo ponemos debajo del siguiente coeficiente:
  \[
    \begin{NiceArray}{r|rrrr}
      &1&0&-2&4\\
      2&&\boldsymbol{2}&&\\
      \hline
      &1&&&
      \CodeAfter
      \begin{tikzpicture}
        \begin{scope}[->]
          \draw ([yshift=-.1\baselineskip]2-1.south) to [bend right = 45] (3-2.west);
          \draw (3-2.east) -- ([yshift=-.1\baselineskip]2-3.south);
        \end{scope}
      \end{tikzpicture}
    \end{NiceArray}
  \]
\item Sumamos el coeficiente y el número obtenido:
  \[
    \begin{NiceArray}{r|rrrr}
      &1&0&-2&4\\
      2&&2&&\\
      \hline
      &1&\boldsymbol{2}&&
      \CodeAfter
      \begin{tikzpicture}
        \begin{scope}[->]
          \draw ([xshift=.1\baselineskip]1-3.east) --
          ([xshift=.1\baselineskip]3-3.east) node[midway, right] {\small{+}};
        \end{scope}
      \end{tikzpicture}
    \end{NiceArray}
  \]
\item Y repetimos las operaciones hasta que terminemos con los coeficientes:
  \[
    \begin{NiceArray}{r|rrrr}
      &1&0&-2&4\\
      2&&2&\boldsymbol{4}&\\
      \hline
      &1&2&&
      \CodeAfter
      \begin{tikzpicture}
        \begin{scope}[->]
          \draw ([yshift=-.1\baselineskip]2-1.south) to [bend right = 90]
          ([yshift=-.1\baselineskip]3-3.south);
          \draw (3-3.east) -- ([yshift=-.1\baselineskip]2-4.south);
        \end{scope}
      \end{tikzpicture}
    \end{NiceArray}
  \]

  \[
    \begin{NiceArray}{r|rrrr}
      &1&0&-2&4\\
      2&&2&4&\\
      \hline
      &1&2&\boldsymbol{2}&
      \CodeAfter
      \begin{tikzpicture}
        \begin{scope}[->]
          \draw ([xshift=.1\baselineskip]1-4.east) --
          ([xshift=.1\baselineskip]3-4.east) node[midway, right] {\small{+}};
        \end{scope}
      \end{tikzpicture}
    \end{NiceArray}
  \]

  \[
    \begin{NiceArray}{r|rrrr}
      &1&0&-2&4\\
      2&&2&4&\boldsymbol{4}\\
      \hline
      &1&2&2&\boldsymbol{8}
      \CodeAfter
      \begin{tikzpicture}
        \begin{scope}[->]
          \draw ([yshift=-.1\baselineskip]2-1.south) to [bend right = 90]
          ([yshift=-.1\baselineskip]3-4.south);
          \draw (3-4.east) -- ([yshift=-.1\baselineskip]2-5.south);
          
          \draw ([xshift=.1\baselineskip]1-5.east) --
          ([xshift=.1\baselineskip]3-5.east) node[midway, right] {\small{+}};
        \end{scope}
      \end{tikzpicture}
    \end{NiceArray}
  \]
\item Ahora tenemos que interpretar el resultado obtenido.
  \begin{itemize}
  \item El último número obtenido, el $\boldsymbol{8}$, es el resto de la división.
  \item El resto de los números , $\lbrace \boldsymbol{1,2,2}\rbrace$, son los coeficientes del cociente que , como hemos indicado antes, es un grado menor que el divisor. De esta manera tenemos que el cociente es de grado 2, y con los coeficientes obtenidos, $\lbrace 1,2,2\rbrace$, resulta ser:
    \[1*x^2 + 2*x + 2 = x^2 + 2x + 2\]
  \end{itemize}
\end{enumerate}
\section{Ecuaciones III.}
\subsection{Ecuaciones bicuadradas.}
Las ecuaciones bicuadradas son las que tienen la siguiente forma:
\begin{Large}
  \[\boldsymbol{ax^4 + bx^2 + c = 0}\]
\end{Large}
En ellas solo hay términos con $x^4$, $x^2$ y sin variable. Si aparece una $x$ sola o $x^3$ ya no se puede resolver con este método.\\

La manera de resolver estas ecuaciones es haciendo lo que se llama un \textbf{cambio de variable}, y el cambio que vamos a hacer es:
\[\boldsymbol{t = x^2}\]
De manera que $\boldsymbol{t^2} = \left(x^2\right)^2 = \boldsymbol{x^4}$ y
$\boldsymbol{x = \pm \sqrt{t}}$.

Al hacer el cambio la ecuación $ax^4 + bx^2 +c = 0$ se transforma en:
\[at^2 + bt + c= 0\]
que sabemos resolver con lo visto en el apartado \ref{ecs_grado_2} (página \pageref{ecs_grado_2}), con lo que obtendremos dos soluciones que llamaremos $t_1$ y $t_2$.\\
Una vez que tenemos las soluciones de $t$ podemos obtener las de $x$ deshaciendo el cambio con $x =\pm \sqrt{t}$
\begin{itemize}
\item $x_1 = \sqrt{t_1}$
\item $x_2 = -\sqrt{t_1}$
\item $x_3 = -\sqrt{t_2}$
\item $x_4 = -\sqrt{t_2}$
\end{itemize}

Vamos a ver \textbf{unos ejemplos} de este tipo de ecuaciones:
\begin{questions}
\question Resuelve $x^4 - 5x^2 + 4 = 0$
  \begin{solution}
    Hacemos el cambio $t = x^2$, con lo que nos queda:
    \[t^2 - 5 t + 4 = 0\]
    Como es una ecuación completa la resolvemos utilizando la fórmula:
    \[t = \frac{5 \pm \sqrt{25 - 4*1*4}}{2} = \frac{5 \pm \sqrt{9}}{2}\]
    Y obtenemos las dos soluciones de $t$:
    \begin{itemize}
    \item $t_1 = \frac{5 + 3}{2} = 4$
    \item $t_2 = \frac{5 - 3}{2} = 1$
    \end{itemize}
    Y a partir de estas obtenemos las soluciones para $x$:
    \begin{itemize}
    \item $x_1 = \sqrt{4} = 2$
    \item $x_2 = -\sqrt{4} = -2$
    \item $x_3 = \sqrt{1} = 1$
    \item $x_4 = -\sqrt{1} = -1$
    \end{itemize}
  \end{solution}

\question Resuelve $2x^4 - x^2 - 1 = 0$
  \begin{solution}
    Hacemos el mismo cambio que en la anterior, $t = x^2$:
    \[2t^2 - t - 1= 0\]
    Aplicamos la fórmula:
    \[t = \frac{1 \pm \sqrt{1 - 4*2*(-1)}}{4} = \frac{1 \pm 3}{4}\]
    Con lo que las soluciones de $t$ son $t_1 = 1$ y $t_2 = -\frac{1}{2}$.
    Y a la hora de obtener las soluciones de $x$ nos encontramos con que tenemos que hacer la raíz cuadrada de $-\frac{1}{2}$, cosa que no podemos porque es negativo. Entonces esa solución de $t$ no nos da ninguna solución para $x$, las soluciones negativas de $t$ se desechan.\\
    De esta manera nos queda que las soluciones de $x$ son:
    \begin{itemize}
    \item $x_1 = 1$
      \item $x_2 = -1$
    \end{itemize}
  \end{solution}
\end{questions}
\part{4º de ESO.}
En este apartado no se va a diferenciar entre las distintas clases de matemáticas que se imparten en 4º de ESO, los contenidos de estos apuntes van a ser los máximos y el lector deberá discriminar cuales son los que necesita.
\section{Polinomios IV.}
\subsection{Factorización de polinomios.}
Al igual que ocurre con los números, los polinomios se pueden factorizar de manera que cualquier polinomio de grado $\boldsymbol{n}$ se puede escribir de la forma:
\[P(x) = (x-a_1)^{i_1} * (x-a_2)^{i_2} \cdots (x-a_k)^{i_k}\]
Donde $a_1$, $a_2$, \dots son números reales e $i_1$, $i_2$,\dots son numeros enteros cuya suma tiene que ser el grado del polinomio original.\\

Unos ejemplos de esto pueden ser:
\begin{itemize}
\item $x^2 - x -2 = (x-2)*(x+1)$
\item $2x^2 - 8x + 8 = 2*(x-2)^2$
\item $x^3 - 6x^2 + 5x = x*(x-1)*(x-5)$
\end{itemize}
Que tendrían su equivalente numérico en las siguientes factorizaciones:
\begin{itemize}
\item $6= 2*3$
\item $9 = 3^2$
\item $30 = 2*3*5$
\end{itemize}
\subsubsection{Raíz de un polinomio.}
Si tenemos un polinomio $P(x)$, se dice que $a \in \realset$ ($a$ real) es raíz de $P(x)$ si y solo si $P(a) = 0$.\\

Vamos a ver unos ejemplos:
\begin{itemize}
\item $2$ es raíz de $P(x) = -x^2 + 4$ porque $P(2) = -2^2 + 4= -4+4 = 0$.
\item $-1$ es raíz de $P(x) = x+1$ porque $P(-1) = -1 + 1 = 0$.
\item $-4$ es raíz de $P(x) = x^2 + 6x + 8$ porque $P(-4) =
  (-4)^2 +6*(-4) + 8 = 16-24 + 8 = 0$.
\end{itemize}

En resumen, las raices de un polinomio son las soluciones de la ecuación
\[P(x) = 0\]
y en un polinomio de segundo grado se obtienen fácilmente.\\
Vamos a verlo en \textbf{unos ejemplos que además nos van a servir para definir algunas cosas}:
\begin{questions}
\question Calcula las raíces del polinomio $P(x) = x^2 -2x -3$.
  \begin{solution}
    Tal y como acabamos de ver las raíces son la solución de la ecuación
    \[P(x) = 0\]
    Con lo que tenemos que resolver la ecuación
    \[x^2 - 2x - 3 = 0\]
    Y como es una ecuación completa utilizamos la fórmula:
    \[x = \frac{2 \pm \sqrt{(-2)^2 - 4*1*(-3)}}{2*1} = \frac{2\pm 4}{2}\]
    Con lo que las raíces son:
    \begin{itemize}
    \item $\frac{2+4}{2} = 3$
    \item $\frac{2-4}{2} = -1$
    \end{itemize}
  \end{solution}
\question Calcula las raíces de $Q(x) = x^2 + 2x + 1$.
  \begin{solution}
    Procedemos de la misma manera, hacemos $x^2 + 2x + 1 = 0$ y resolvemos con la fórmula:
    \[x = \frac{-2 \pm \sqrt{2^2 - 4*1*1}}{2*1} = \frac{-2\pm 0}{2}\]
    Con lo que las raíces son:
    \begin{itemize}
    \item $\frac{-2+0}{2} = -1$
    \item $\frac{-2-0}{2} = -1$
    \end{itemize}
    Con lo que en este caso tenemos dos raíces iguales y se dice que $-1$ es raíz doble.
  \end{solution}
\question Escribe un polinomio cuyas raíces sean $-1$, $3$ y $4$.
  \begin{solution}
    Por lo que acabamos de ver tenemos que escribir un polinomio $P(x)$ tal que
    \begin{itemize}
    \item $P(-1) = 0$
    \item $P(3)= 0$
    \item $P(4) = 0$
    \end{itemize}
    Vamos a pensar qué tenemos que hacer para que con cada uno de esos valores el polinomio de
    resultado cero.\\
    Empecemos pensando cual es \textbf{el polinomio más sencillo que conocemos}, y este es
    $\boldsymbol{S(x) = x + k}$ (donde $k$ es un número real).\\
    Si ahora queremos que $S(4) = 0$ tiene que ocurrir que $k = -4$, con lo que $S(x) = x - 4$.\\
    El problema es que para el resto de valores que nos piden este polinomio no daría cero.\\
    
    \emph{¿Y si definimos un polinomio sencillo para cada valor y luego los multiplicamos todos?}.
    En ese caso el resultado si sería cero para cada uno de los valores.\\
    Entonces hacemos $\boldsymbol{R(x) = (x-4)*(x-3)*(x+1) = x^3 -6x^2 + 4x + 12}$, con lo que la
    respuesta sería:
    \[P(x) = x^3 -6x^2 + 4x + 12\]

    Y ese es el polinomio más sencillo que cumple las condiciones que nos piden en el enunciado.
    Si multiplicamos $P(x)$ por un valor distinto de cero o por otro polinomio obtendremos un
    polinomio que también cumple las condiciones del enunciado pero no será tan sencillo
    como $P(x)$.
  \end{solution}
\question Escribe una ecuación que tenga por soluciones los valores $3$, $-2$ y $\frac{1}{3}$.
  \begin{solution}
    Por la definición de raíz sabemos que las soluciones de la ecuación $P(x) = 0$ son las
    raíces del polinomio $P(x)$, con lo que el ejercicio se reduce a escribir un polinomio
    que tenga las raíces indicadas en el enunciado.\\

    Por lo que hemos visto en el ejercicio anterior, el polinomio más sencillo que tiene esas
    raíces es:
    \[P(x) = (x-3)(x+2)\left(x -\frac{1}{3}\right) = x^3 - \frac{4x^2}{3} - \frac{17x}{3} + 2\]
    Con lo que la ecuación más sencilla con esas tres soluciones sería:
    \[x^3 - \frac{4x^2}{3} - \frac{17x}{3} + 2 = 0\]
    Esta ecuación es un poco fea por el tema de las fracciones, pero hemos visto en el ejercicio
    anterior que si multiplicamos el polinomio por un valor distinto de cero las raíces se
    conservan. En este caso multiplicamos todo por $3$ y así desaparece el denominador:
    \[3x^2 - 4x^2 - 17x + 6 = 0\]
    es otra ecuación que tiene también las soluciones indicadas en el enunciado.
  \end{solution}
\end{questions}

\subsubsection{Divisibilidad de polinomios. Teorema del resto.}
Dado un polinomio $P(x)$ y un polinomio $Q(x) = x+a$, el resto de la división $\frac{P(x)}{Q(x)}$ coincide con $P(-a)$ (es importante darse cuenta de que en el dividendo, es decir $P$ hemos puesto el opuesto de $a$)\\

Vamos a verlo con \textbf{un ejemplo} rápido: tenemos los polinomios $P(x) = x^2 - 5$ y $Q(x) = x-1$, así que vamos a hacer la división y comprobar que coincide con lo que nos dice el teorema del resto.\\
Hacemos la división en primer lugar:
\begin{center}
\begin{tabular}{rrrrl}
&$x^2$&$0$& \multicolumn{1}{r|}{$-5$} & $x-1$ \\ \cline{5-5} 
  $-\ $&$x^2$&$-x$& &$x +1$\\ \cline{2-3}
& &$x$&$-5$& \\
  $-\ $&&$x$&$-1$& \\ \cline{3-4}
  &&&$-4$& 
\end{tabular}
\end{center}
Y aquí vemos que el resto de la división es $-4$.
Si utilizamos el teorema del resto tenemos que el divisor es $x-1$, con lo que calculamos el dividendo en $1$ (recordemos que hay que utilizar el opuesto):
\[P(1) = 1^2 - 5 = -4\]

De esta manera tenemos un método rápido para calcular el resto de una división de polinomios sin tener que realizarla. Es decir, tenemos el siguiente \large{\textbf{criterio de divisibilidad para polinomios}}:\\

\textbf{Dado un polinomio $P(x)$, si $\boldsymbol{P(a) = 0}$ entonces $P(x)$ es divisible entre $\boldsymbol{x-a}$.}\label{criterio_divisibilidad}



\subsubsection{Factorización de un polinomio de segundo grado.}\label{factor_segundo_grado}
A partir del criterio de divisibilidad visto al final del apartado anterior (\ref{criterio_divisibilidad}) es fácil deducir que para hayar los factores de un polinomio simplemente tenemos que calcular sus raíces.\\
De esta manera si tenemos un polinomio de 2º grado $P(x) = ax^2 + bx +c$ cuyas raíces sean $x_1$ y $x_2$, el polinomio se factoriza como:
\[\boldsymbol{P(x) = a*(x-x_1)*(x-x_2)}\]
Es importante recordar que hay que utilizar el coeficiente de $x^2$ ($a$), y que las raíces están cambiadas de signo en los paréntesis.\\ 

Vamos a ver \textbf{un par de ejemplos}:
\begin{questions}
\question Factoriza el polinomio $P(x) = 2x^2 - 18$.
  \begin{solution}
    Para calcular las raíces del polinomio tenemos que resolver la ecuación
    \[2x^2 - 18 = 0\]
    que, al ser incompleta se resuelve más rápido por el método propio:
    \[2x^2 = 18\]
    \[x^2 = 9\]
    \[x = \pm 3\]
    Es decir, las raíces son $3$ y $-3$, y teniendo en cuenta que el coeficiente de $x^2$ es $a= 2$ la factorización queda:
    \[P(x) = 2x^2 - 18 = \boldsymbol{2*(x-3)(x+3)}\]
  \end{solution}
\question{Factoriza el polinomio $Q(x) = x^2 - 4x$.}
  \begin{solution}
    En este caso volvemos a tener una ecuación incompleta:
    \[x^2 - 4x = 0\]
    Que se resuelve por su método propio:
    \[x*(x-4) = 0\]
    Entonces las raíces son $0$ y $4$, y teniendo en cuenta que el coeficiente de $x^2$ es $a=1$ la factorización queda:
    \[Q(x) = x^2 - 4x = 1*(x-0)*(x-4) = \boldsymbol{x*(x-4)}\]
  \end{solution}
\question Factoriza el polinomio $R(x) = 2x^2 - 8x + 8$.
  \begin{solution}
    Procedemos de la misma manera, pero esta vez con una ecuación completa. Aplicamos la fórmula y obtenemos que hay una raíz doble que es $2$.\\
    En ese caso lo que hay que hacer es poner un cuadrado, con lo que la factorización queda:
    \[R(x) = 2*(x-2)^2\]
  \end{solution}
\end{questions}
Esto que ha aparecido en el último ejemplo es algo que ocurre a menudo
y hay que tener en cuenta: cuando una raíz se repite varias veces solo se pone una vez indicando en el exponente el número de veces que se repite.\\
Es similar a lo que ocurre al factorizar números:
\begin{tabular}{r|r}
24&2\\12&2\\6&2\\3&3\\1&
\end{tabular}\vspace{2mm}\\
Como el $2$ se repite tres veces la factorización se escribe $24=2^3*3$.

\subsubsection{Raíces múltiples. Polinomios indivisibles.}
Hemos visto en el apartado anterior que si tenemos un polinomio de segundo grado $P(x) = ax^2 + bx + c$ con raíces $x_1$ y $x_2$, significa que el polinomio $P(x)$ es divisible entre $(x-x_1)$ y $(x-x_2)$, que son sus factores.\\

Pero sabemos que al resolver una ecuación de 2º grado nos pueden pasar tres cosas:
\begin{itemize}
\item Que la ecuación tenga dos soluciones, que es el caso contemplado en el aparado anterior.
\item Que la ecuación tenga una solución, que también está en un ejemplo del apartado anterior. En este caso se dice que hay raíces múltiples.
\item Que la ecuación no tenga solución. En este caso decimos que el polinomio es indivisible (aunque más adelante, si seguimos estudiando matemáticas, veremos que se puede factorizar utilizando números complejos).
\end{itemize}


Es decir, si tenemos el polinomio $P(x) = x^2 + x + 1$ y buscamos sus raíces:
\[x = \frac{-1 \pm \sqrt{1^2 - 4*1*1}}{2*1} = \frac{-1 \pm \sqrt{-3}}{2}\]
nos encontramos con una ecuación que no tiene solución, con lo que el polinomio no tiene raíces y no se puede dividir entre ningún otro polinomio. Es un polinomio indivisible.

\subsubsection{Factorización de polinomios de grado mayor que dos por Ruffini.} \label{ruffini_general}
Este caso nos sirve para factorizar polinomios como $P(x) = x^4 - 5x^3 + 5x^2 + 5x -6$ y de cualquier grado.\\
El problema es que aquí no podemos calcular las raíces resolviendo la ecuación ya que en la mayoría de los casos desconocemos el método que se podría utilizar para resolverla si es que existe.\\
En este caso no sabemos cómo resolver la ecuación $x^4 - 5x^3 + 5x^2 + 5x -6 = 0$ y por ello tendremos que utilizar alguna de las otras cosas que hemos visto anteriormente.\\

En el apartado anterior hemos visto que un polinomio de segundo grado se factoriza de la forma $a*(x-x_1)*(x-x_2)$.\\
Es decir, los factores tienen la forma $x\pm c$ y los factores tienen que ser divisores del polinomio con lo que tienen que tener una división exacta.\\

En el apartado \ref{ruffini} (página \pageref{ruffini}) vimos el método de Ruffini que sirve para realizar rápidamente divisiones cuando el divisor tiene la forma $x\pm c$, con lo que haremos lo que se llama \emph{``Tanteo por Ruffini''}, que consiste en probar con diversos números hasta que el resto sea 0.\\
Los números con los que tenemos que probar tienen que ser divisores del término independiente del polinomio (puede que nos encontremos con que no hay término independiente, entonces habrá que extraer primero factor común como veremos más adelante). En el caso que nos ocupa ($P(x) = x^4 - 5x^3 + 5x^2 + 5x -6$) el término independiente es $6$, con lo que los candidatos para probar son $1$, $-1$, $2$, $-2$, $3$, $-3$, $6$ y $-6$.\\

Vamos a empezar probando con $6$:
\begin{center}
  \begin{tabular}{r|rrrrr}
    &1&$-5$&5&5&$-6$\\
    6&&6&6&66&426\\
    \hline
    &1&1&11&71&420
  \end{tabular}
\end{center}
Evidentemente el resto no es 0, con lo que el 6 no vale.\\
Vamos a probar con $1$:
\begin{center}
  \begin{tabular}{r|rrrrr}
    &1&$-5$&5&5&$-6$\\
    1&&1&$-4$&1&6\\
    \hline
    &1&$-4$&1&6&0
  \end{tabular}
\end{center}
Y este sí que nos vale. Por lo visto en el apartado \ref{props_cociente_resto} (página \pageref{props_cociente_resto}) tenemos que $(x-1)$ es un factor (o lo que es lo mismo, 1 es raíz).\\
Y ahora tenemos que continuar factorizando el polinomio $x^3 -4x^2 + x + 6$, que es el cociente que hemos obtenido al dividir el polinomio original entre el factor $(x-1)$. Es decir, es lo mismo que cuando factorizamos $24$, que empezamos haciendo
\begin{tabular}{r|r}
24&2\\12
\end{tabular} y repetimos con el hasta que quede 1. Pues aquí repetiremos hasta que el polinomio cociente sea de grado 2.\\

Entonces vamos a ver si encontramos una raíz de $x^3 -4x^2 + x + 6$, vamos a probar con $-1$ (con los únicos valores que no podemos probar es con los que han fallado antes)\\
\begin{center}
\begin{tabular}{r|rrrr}
  &1&$-4$&1&6\\
  -1&&$-1$&5&$-6$\\
  \hline
  &1&$-5$&6&0
\end{tabular}
\end{center}
Con lo que es divisible entre $(x+1)$ y el cociente es $x^2 -5x+6$.\\
Para factoriza $x^2 - 5x + 6$ utilizamos el método de los polinomios de segundo grado (\ref{factor_segundo_grado}, página \pageref{factor_segundo_grado}): resolvermos la ecuación $x^2 - 5x + 6 = 0$ con la fórmula y obtenemos que las soluciones son $x_1 = 2$ y $x_2 = 3$, con lo que los factores son $(x-2)$ y $(x-3)$.\\

Por último tenemos que juntar todos los factores y el famoso coeficiente $a$ del polinomio de segundo grado (que en este caso es $a=1$) y nos quedará que la factorización es:
\[P(x) = x^4 - 5x^3 + 5x^2 + 5x -6 = \boldsymbol{(x-1)(x+1)(x-2)(x-3)}\]

Vamos a hacer una recopilación paso a paso  de lo que hemos hecho:
\begin{enumerate}
\item Tanteamos con Ruffini con divisores del término independiente. Si el
  resto no es cero probamos con otro divisor.
\item Repetimos el paso anterior hasta que el cociente tenga grado 2.
\item Aplicamos el método para factorizar polinomios de grado 2 (página \pageref{factor_segundo_grado}).
\item Juntamos todos los fractores y el coeficiente $a$ del polinomio de grado 2.
\end{enumerate}

\textbf{Vamos a realizar otro ejemplo} siguiendo los pasos que acabamos de recopilar, vamos a factorizar el polinomio $P(x) = 2x^5+3x^4-11x^3 -14x^2+12x+8$
\begin{solution}
  \begin{enumerate}
  \item Empezamos con el tanteo por Ruffini. Lo más sencillo es empezar a tantear con $1$ ó $-1$, vamos a probar primero con $1$.
    \begin{center}
      \begin{tabular}{r|rrrrrr}
        &2&3&$-11$&$-14$&12&8\\
        1&&2&5&$-6$&$-20$&$-8$\\
        \hline
        &2&5&$-6$&$-20$&$-8$&0
      \end{tabular}
    \end{center}
    Nos ha dado de resto $0$, con lo que $\boldsymbol{1}$ es raíz.\\
  \item Como el polinomio \textbf{cociente es de grado $\boldsymbol{4}$} ($2x^4+5x^3-6x^2-20x-8$, recuerda que con Ruffini el grado del cociente siempre es uno menos que el del dividendo)
    tenemos que seguir tanteando por Ruffini. Si volvemos a probar con $1$ no sale exacta, y con $-1$ tampoco, así que vamos a probar con $2$.
    \begin{center}
      \begin{tabular}{r|rrrrr}
        &2&5&$-6$&$-20$&$-8$\\
        2&&4&18&24&8\\
        \hline
        &2&9&12&4&0\\
      \end{tabular}
    \end{center}
    Con lo cual $\boldsymbol{2}$ es raíz, y el cociente ahora es de grado $3$ con lo que tenemos que seguir con Ruffini.\\
    Si volvemos a probar con $2$ no sale exacta, así que vamos a probar con $-2$ (con $1$ y $-1$ no probamos porque ya vimos antes que no salía exacta)
    \begin{center}
      \begin{tabular}{r|rrrr}
        &2&9&12&4\\
        $-2$&&$-4$&$-10$&$-4$\\
        \hline
        &2&5&2&0
      \end{tabular}
    \end{center}
    Con lo cual $\boldsymbol{-2}$ también es raíz y ahora el polinomio del cociente ($2x^2 +5x +2$) sí que es de grado $2$, con lo que podemos pasar al siguiente paso.
  \item Resolvemos la ecuación $2x^2 + 5x + 2 = 0$, recordando que tenemos que quedarnos con el coeficiente $\boldsymbol{a}$ porque lo tenemos que poner luego en la factorización (en este caso $\boldsymbol{a=2}$ por el término $2x^2$).\\
    Pasamos a resolver la ecuación y tenemos:
    \[x = \frac{-5\pm \sqrt{25-16}}{4} = \frac{-5\pm 3}{4} =
      \begin{cases}\frac{1}{2}\\-2\end{cases}\]
    Con lo que $\boldsymbol{-2}$ y $\boldsymbol{\frac{1}{2}}$ también son raíces del polinomio.
  \item Ahora tenemos que recopilar las raíces y el coeficiente $a$ de la ecuación de segundo grado, y tenemos:
    \begin{itemize}
    \item Raíces: $1$, $2$, $-2$, $-2$ y $\frac{1}{2}$.
    \item Coeficiente $a$: $2$.
    \end{itemize}
    Con todo esto la factorización queda:
    \[\boldsymbol{P(x) = 2*(x-1)(x-2)(x+2)^2\left(x-\frac{1}{2}\right)}\]
  \end{enumerate}

  Para finalizar este ejemplo vamos a ver cómo se suele hacer para que sea más rápido y más eficiente a la hora de recopilar todo para escribir la factorización:
  \begin{small}
  \begin{center}
    \begin{tabular}{r|rrrrrr}
      &2&3&$-11$&$-14$&12&8\\
      1&&2&5&$-6$&$-20$&$-8$\\
      \hline
      &2&5&$-6$&$-20$&$-8$&0\\
      2&&4&18&24&8&\\
        \hline
        &2&9&12&4&0&\\
      $-2$&&$-4$&$-10$&$-4$&&\\
        \hline
        &2&5&2&0&&
    \end{tabular}
  \end{center}
  \end{small}
  \[x = \frac{-5\pm \sqrt{25-16}}{4} = \frac{-5\pm 3}{4} =
    \begin{cases}\frac{1}{2}\\-2\end{cases}\]\\
\vspace*{1cm}Y con todo esto ya podemos escribir la factorización:
  \[P(x) = 2*(x-1)(x-2)(x+2)^2\left(x-\frac{1}{2}\right)\]
\end{solution}

\subsubsection{Factorización general de polinomios.}
Vamos a recopilar todo lo que hemos visto y así poder escribir un método para factorizar cualquier polinomio (siempre que tenga raíces enteras para poder hacer el tanteo por Ruffini), y después haremos un par de ejemplos.\\
Los pasos a dar tienen que ser:
\begin{enumerate}
\item \textbf{Sacar factor común si es posible.}\\
  Este paso es importante, ya que si no lo hacemos es fácil que las cosas acaben saliendo mal. Por ejemplo, si tenemos que factorizar $P(x) = x^3-x^2-2x$ y empezamos haciendo Ruffini tenemos que acordarnos de que el término independiente vale $0$, con lo que a la hora de hacerlo habría que escribirlo de la siguiente forma:
  \begin{center}
    \begin{tabular}{r|rrrr}
      &1&-1&-2&0\\
      &&&&\\
      \hline
      &&&&
    \end{tabular}
  \end{center}
  Y es fácil olvidarse de el último $0$. Además si sacamos factor común queda:
  \[P(x) = x*(x^2 - x -2)\]
  Con lo cual podríamos pasar directamente a factorizar el polinomio de segundo grado.
\item \textbf{Obtener raíces por Ruffini hasta llegar a un polinomio de grado 2.}\\
  Lo mismo que hemos hecho en el apartado \ref{ruffini_general} (página \pageref{ruffini_general})
\item \textbf{Factorizar el polinomio de grado 2 final.}\\
  Aquí utilizaremos el mecanismo visto en el apartado \ref{factor_segundo_grado} (página \pageref{factor_segundo_grado})
\item \textbf{Recopilar lo obtenido y escribir la factorización.}\\
  Aquí tenemos que recopilar todo lo que hemos obtenido en cada paso: factor común, raíces y el coeficiente $a$ del polinomio de 2º grado, y con ello escribir la factorización.\\
  De manera simbólica, si lo obtenido es lo siguiente:
  \begin{itemize}
  \item El factor común es $k*x^n$.
  \item Las raíces obtenidas por Ruffini son $x_1$, $x_2$  dos veces, $x_3$ y $x_4$.
  \item Las raíces del polinomio de 2º grado son $x_2$ y $x_5$ (fíjate que se vuelve a repetir $x_2$, con lo que ya llevamos tres veces), y el coeficiente de $x^2$ es $a$.
  \end{itemize}
  Con todo esto la factorización sería:
  \[P(x) = \overbrace{a}^{\
      \begin{array}{c}
        \text{\scriptsize{Coeficiente}}\\
        \text{\scriptsize{ecuación}}\\
        \text{\scriptsize{2º grado}}
      \end{array}
    }*
    \overbrace{k*x^n}^{\text{\scriptsize Factor común}}*
    \overbrace{(x-x_1)*(x-x_2)^3*(x-x_3)*(x-x_4)*(x-x_5)}^{
      \begin{array}{c}
        \text{\scriptsize Todas las raíces elevadas al número}\\
        \text{\scriptsize de veces que se repiten}
      \end{array}}
  \]
\end{enumerate}

Vamos a ver todo esto con \textbf{un ejemplo complejo}, vamos a factorizar el polinomio
\[P(x) = 6x^8+3x^7-48x^6+12x^5+60x^4-15x^3-18x^2\]
\begin{solution}
  Vamos a ir siguiendo todos los pasos en su orden correspondiente:
  \begin{enumerate}
  \item \textbf{Extraemos factor común.}\\
    Respecto a los coeficientes se ve que su m.c.d. es $3$, y respecto a la
    $x$ está en todos los términos y su menor exponente es $2$, con lo que el
    factor común es $3x^2$, de manera que:
    \[P(x) = 3x^2*(2x^6+x^5-16x^4+4x^3+20x^2-5x - 6)\]
    y continuamos con el polinomio del paréntesis.\\
    \emph{(Para la recopilación: el factor común es $3x^2$)}
  \item \textbf{Tanteamos por Ruffini hasta grado 2.}\\
    Para no hacerlo muy largo lo vamos a escribir todo seguido:
    \begin{center}
      \begin{tabular}{r|rrrrrrr}
        &2&1&$-16$&4&20&$-5$&$-6$\\
        1&&2&3&$-13$&$-9$&11&6\\
        \hline
        &2&3&$-13$&$-9$&11&6&0\\
        $-1$&&$-2$&$-1$&14&$-5$&$-6$&\\
        \cline{1-7}
        &2&1&$-14$&5&6&0&\\
        2&&4&10&$-8$&$-6$&&\\
        \cline{1-6}
        &2&5&$-4$&$-3$&0&&\\
        $-3$&&$-6$&3&3&&&\\
        \cline{1-5}
        &2&$-1$&$-1$&0&&&
      \end{tabular}
    \end{center}
    Al haber llegado a un polinomio de grado 2 paramos y vamos al siguiente
    paso.\\
    \emph{(Para la recopilación: hasta ahora tenemos las raíces $1$, $-1$,
      $2$ y $-3$)}
  \item \textbf{Resolvemos la ecuación de 2º grado}\\
    Si resolvemos la ecuación $2x^2 - x -1 = 0$ obtendremos las soluciones 1 y $-\frac{1}{2}$.
  \item \textbf{Recopilamos todo:}\\
    Vamos a ver lo que tenemos de cada paso:
    \begin{enumerate}
    \item Factor común: $3x^2$
    \item Raíces: $1$, $-1$, $2$ y $-3$
    \item Parámetro $a=2$ y raíces $1$ y $-\frac{1}{2}$
    \end{enumerate}
    Y juntándolo todo nos queda que la factorización de $P(x)$ es (tras simplificar alguna operación):
    \[P(x) = \boldsymbol{6x^2(x+1)(x-1)^2(x-2)(x+3)\left(x-\frac{1}{2}\right)}\]
  \end{enumerate}
\end{solution}

\subsubsection{Mínimo común múltiplo de polinomios.}
En todo este punto de \emph{factorización de polinomios} estamos viendo como hacer con polinomios los mismos procedimientos que hacíamos con números, y una de las cosas que hacíamos con números es calcular el mínimo común múltiplo.\\

Antes de empezar a ver cómo calcular el mínimo común múltiplo de varios polinomios vamos a repasar el concepto de factor y el significado que tiene para cada uno de los casos, números y polinomios.\\

Lo primero que tenemos que recordar es que \textbf{un factor es algo que está multiplicando}, y es lo que acabamos teniendo cuando realizamos una factorización (que de ahí le viene el nombre) una serie de factores cuya multiplicación nos da el original.\\

La principal diferencia entre números y polinomios es que los factores de estos últimos son algo más elaborados.
Veámoslo con unos ejemplos:
\begin{itemize}
\item \textbf{Número}: $60 = 2^2 * 3* 5$\\
  Aquí tenemos tres factores: $2^2$, $3$ y $5$. Aunque en estas situaciones no incluimos los exponentes, de manera que
  hablaríamos de $2$, $3$ y $5$.
\item \textbf{Polinomio}: $3x^4 + 9x^3 - 12x = 3*x*(x-1)*(x+2)^2$\\
  En este caso los factores son $3$, $x$, $(x-1)$ y $(x+2)$. Aquí las cosas cambian con respecto a los números, tenemos los factores $(x-1)$ y $(x+2)$ que son sumas, con lo que pueden llegar a confundirnos.\\
  Pero solo tenemos que recordad la definición de \textbf{factor} y pensar qué esta multiplicando en el caso de $(x-1)$, ¿la $x$, el $1$ o toda la suma?. Y la respuesta es: ``\textbf{toda la suma}'', con lo que \textbf{toda la suma es el factor}.
\end{itemize}

Una vez recordado esto, \textbf{vamos a calcular en paralelo el mínimo común múltiplo de tres números y de tres polinomios}, para ver cuales son las similitudes y las diferencias en cada caso.\\

\begin{tabular}{c}
  \begin{tabular}{p{8cm}|p{8cm}}
    \textbf{En la columna de la izquierda tenemos números.}
    &\textbf{En la columna de la derecha tenemos polinomios.}\\
                                                      
    En el caso de números vamos a calcular el m.c.m. de 4, 10 y 18.
    & En el caso de polinomios vamos a calcular el m.c.m. de $x^2 -1$, $x^2 + 2x + 1$ y $2x^2-2x$.\\
  \end{tabular}\\
  \hline
  \Large{\textbf{Factorizamos todo:}}\\
  \begin{tabular}{p{8cm}|p{8cm}}
    En el caso de números tenemos:
    \begin{itemize}
      \item $4=2^2$
      \item $10 = 2*5$
      \item $18 = 2*3^2$
      \end{itemize}
    De manera que los factores son $2$, $3$ y $5$
    &En el caso de polinomios tenemos:\begin{itemize}
    \item $x^2 - 1= (x+1)*(x-1)$
    \item $x^2 + 2x + 1 = (x+1)^2$
    \item $2x^2 - 2x = 2*x*(x-1)$
      \end{itemize}
      De manera que los factores son $2$, $x$, $(x-1)$ y $(x+1)$.
  \end{tabular}\\
  \hline
  \Large{\textbf{Cogemos los factores comunes y no comunes con el mayor exponente:}}\\
  \begin{tabular}{p{8cm}|p{8cm}}
    Para los números:
    \begin{itemize}
      \item Para el 2 usaremos $2^2$.
      \item Para el 3 usaremos $3^2$.
      \item Para el 5 usaremos solo 5.
      \end{itemize}
    De manera que el mínimo común múltiplo es $\boldsymbol{2^2*3^2*5}$.
    &En el caso de los polinomios tenemos:\begin{itemize}
    \item Para el 2 usaremos 2.
    \item Para la $x$ usaremos $x$.
    \item Para $(x-1)$ usaremos $(x-1)$.
    \item Para $(x+1)$ usaremos $(x+1)^2$
    \end{itemize}
      Con lo que el m.c.m es
     $\boldsymbol{2x(x-1)(x+1)^2}$.
  \end{tabular}
\end{tabular}\vspace{5mm}\\

Vamos a ver un par de \textbf{ejemplos} más:
\begin{questions}
\question Calcula el \textbf{m.c.m} de $\boldsymbol{P(x) = x^2 -2x +1} $ y
  $\boldsymbol{Q(x) = 2x^2 - 2x}$
  \begin{solution}
    Tal y como hemos indicado lo primero es factorizar los polinomios. En este caso lo
    vamos a hacer paso a paso, pero a partir de aquí daremos las factorizaciones hechas y se asumimos que hemos hecho el proceso en sucio ya que si no los apuntes se alargarán en exceso.\\
    
    La factorización de $x^2 -2x+ 1$ se hace con la ecuación de segundo grado (o de manera inmediata si nos acordamos de las identidades notables) y obtenemos que tiene como raíz doble al $-1$, con lo que su factorización es $(x-1)^2$.\\
    Para la factorización de $2x^2 -2x$ se ve que se puede sacar factor común, y eso es lo primero que se debe hacer siempre que se pueda, de manera que nos quedaría $2x(x-1)$ que ya estaría factorizado.\\

    Recopilando tenemos los siguientes factores:
    \begin{itemize}
    \item 2 con exponente 1.
    \item $x$ con exponente 1.
    \item $(x-1)$ con exponentes 1 y 2. El 2 es el mayor.
    \end{itemize}

    Con esto nos queda que:
    \[\text{m.c.m.}(x^2-2x+1,\ 2x^2-2x) = \boldsymbol{2*x*(x-1)^2} = 2x^3 - 2x\]
    \begin{small}
      (Ya veremos más adelante que no siempre vamos a calcular el polinomio resultante)
      \end{small}
  \end{solution}
\question Calcula el m.c.m. de $P(x) = x^2 +x-2$, $Q(x)=x^2 -4x + 4$ y $R(x) = x^2 -3x+2$.
  \begin{solution}
    Hacemos la factorización de cada polinomio, y nos queda:
    \begin{itemize}
    \item $x^2 +x -2 = (x-1)(x+2)$
    \item $x^2-4x+4 = (x-2)^2$
    \item $x^2-3x + 2= (x-1)(x-2)$
    \end{itemize}
    De manera que en este caso el m.c.m es $\boldsymbol{(x-1)(x-2)^2(x+2)}$.
  \end{solution}
\end{questions}

\subsection{Fracciones algebraicas.}
\textbf{Una fracción algebraica se define como aquella cuyo numerador y denominador son polinomios}.\\

Dicho así puede sonar a algo complicado, pero en realidad \emph{no son diferentes a las fracciones numéricas
  y se operan exactamente igual}: para sumar hay que hacer denominador común, se multiplican el línea, se dividen en cruz, \dots\\

La única dificultad que presentan es que la factorización de polinomios no es tan sencilla como la de números, y que tenemos que aprender que lo mejor no es hacer siempre todas las operaciones antes de dar el siguiente paso ya que a veces es mejor echar un vistazo a ver si hay cosas que pueden facilitarnos la vida.\\

Dicho todo esto, vamos a empezar con las operaciones.

\subsubsection{Simplificación de fracciones algebraicas.}
Al igual que con las fracciones numéricas, podremos simplificarlas cuando el numerador y el denominador tengan divisores comunes (aunque aquí vamos a hablar de factores comunes).\\

Vamos a ver esto con \textbf{unos cuantos ejemplos}:\\

\textbf{Primer ejemplo}: Simplifica la fracción $\frac{(x+1)^2 *(x-2)}{(x+1)*(x-3)}$
\begin{solution}
  En este caso es una simplificación sencilla, se ve que $\boldsymbol{(x+1)}$ \textbf{está en el numerador y el denominador} con lo que podemos simplificar utilizando las propiedades de las potencias:
  \[\frac{\boldsymbol{(x+1)^2} *(x-2)}{\boldsymbol{(x+1)}*(x-3)} = \frac{(x+1)(x-2)}{x-3}\]
\end{solution}
\vspace{1cm}
\textbf{Segundo ejemplo}: Simplifica la fracción $\frac{(x+1)(x-1)}{(x+3)^2}$
\begin{solution}
  En este caso hay poco que hacer, el numerador y el denominador \textbf{no tienen factores comunes y no se puede simplificar}.
\end{solution}
\vspace{1cm}
\textbf{Tercer ejemplo}: Simplifica la fracción $\frac{x^2 - 1}{2x^2 - x - 1}$
\begin{solution}
  Ahora ya no es tan sencilla, primero tenemos que factorizar numerador y denominador. Pero como son de segundo grado solo tenemos que resolver las ecuaciones:
  \[\frac{x^2 - 1}{2x^2 - x - 1} = \frac{(x+1)\boldsymbol{(x-1)}}{2\boldsymbol{(x-1)}\left(x+\frac{1}{2}\right)}
  = \frac{x+1}{2\left(x+\frac{1}{2}\right)}\]
\end{solution}
\vspace{1cm}
\textbf{Cuarto ejemplo}: Simplificar $\frac{x^2 - 2x -8}{x^4-3x^3-9x^2+23x-12}$
\begin{solution}
  Ahora ya es complicada, tenemos en el numerador un polinomio de 2º grado, que es sencillo de factorizar,
  mientras que en el denominador hay un polinomio de grado 4, que no es nada sencillo de factorizar.\\

  En estos casos lo mejor es factorizar solo el que sea más sencillo y utilizar el teorema del resto para
  simplificar la fracción si es posible.\\
  Vamos a verlo despacio, y para que sea más sencillo llamaremos $P(x)$ al numerador y $Q(x)$ al denominador:
  \[P(x) = x^2 - 2x - 8\]
  \[Q(x) = x^4-3x^3-9x^2+23x-12\]
  Factorizamos el más sencillo, que en este caso es el numerador:
  \[P(x) = x^2 - 2x - 8 = (x+2)(x-4)\]
  Ahora es cuando utilizamos el teorema del resto: si el denominador tiene alguno de esos factores tendrá que tener alguna de las dos raíces, con lo que al evaluar el polinomio en $-2$ o $4$ tendría que salir
  algún cero.\\
  Así que vamos a ello:
  \begin{flalign*}
    Q(-2) =& (-2)^4 - 3*(-2)^3 - 9*(-2)^2 + 23*(-2) - 12 = -16 -3*(-8) -9*4 -46 -12 \\
    =&-16 +24 -36 -46 -12 \neq 0 \quad\quad\text{\textbf{-2 no es raíz del denominador}}  
  \end{flalign*}

  \begin{flalign*}
    Q(4) =& 4^4 - 3*4^3 -9* 4^2 + 23*4 - 12 = 256 - 3*64 - 9*36 + 23*4 \\
    =&256 - 192 - 144 + 92 - 12 = 0 \quad\quad\text{\textbf{4 es raíz del denominador}}
  \end{flalign*}
  
  Y como $4$ es raíz del denominador hacemos la división por Ruffini, de manera que nos queda:
  \begin{center}
    \begin{tabular}{r|rrrrr}
      &1&$-3$&$-9$&23&$-12$\\
      4&&4&4&$-20$&12\\
      \hline
      &1&1&$-5$&3&0
    \end{tabular}
  \end{center}
  Con lo que la fracción se simplificaría de la siguiente manera:
  \[\frac{x^2 - 2x -8}{x^4-3x^3-9x^2+23x-12} = \frac{(x+2)(x-4)}{(x-4)(x^3 + x^2 - 5x + 3)}=
    \frac{x+2}{x^3 + x^2 - 5x + 3}\]
\end{solution}

\subsubsection{Suma de fracciones algebraicas.}
Sabemos que para poder sumar dos fracciones, estas tienen que tener el mismo denominador. Y esto es independiente de que sean fracciones de enteros, de raíces, de polinomios o de lo que sea.\\
Si no tienen el mismo denominador habrá que hacer denominador común y, al igual que con números, hay que hacerlo con el mínimo común múltiplo.\\

Y lo mejor es que lo veamos con \textbf{unos ejemplos}:
\begin{questions}
\question Realiza la siguiente suma de fracciones algebraicas y simplifica el resultado si es posible.
  \[\frac{x+1}{x} - \frac{1-x}{x}\]
  \begin{solution}
    En este caso tienen el mismo denominador, con lo que la operación es inmediata
    \[\frac{x+1}{x} - \frac{1-x}{x} = \frac{x + 1 - (1-x)}{x} = \frac{2x}{x} = 2\]
  \end{solution}
\question Realiza y simplifica: $x+1 - \frac{2x-1}{x} + \frac{x}{x-1}$
  \begin{solution}
    En este caso no tienen el mismo denominador, con lo que tendremos que hacer denominador común.\\
    Tenemos que los denominadores son $\boldsymbol{x}$ y $\boldsymbol{(x-1)}$, que ya están factorizados, con lo que el m.c.m. de esos dos factores (recuerda que $(x-1)$ es un factor) es $x*(x-1)$.\\

    Para que todo quede más claro vamos a escribir todo en columnas, y así podemos ver bien cómo se transforma cada fracción:
    \begin{center}
      \begin{tabular}{lccc}
        Las fracciones originales&$\frac{x+1}{1}$& $-\frac{2x-1}{x}$&$+\frac{x}{x-1}$\vspace{3mm}\\
        Con el denominador común &$\frac{(x+1)*x*(x-1)}{x*(x-1)}$
                                   &$-\frac{(2x-1)*(x-1)}{x*(x-1)}$&$+\frac{x*x}{x*(x-1)}$
      \end{tabular}\vspace*{5mm}
      \begin{small}
        (Recuerda que al hacer el denominador común hay que dividir el nuevo numerador
        entre el denominador antiguo y multiplicar ese cociente por el numerador antiguo)
      \end{small}
    \end{center}
    De manera que la operación completa será:
    \begin{flalign*}
      &x+1 - \frac{2x-1}{x} + \frac{x}{x-1} = \frac{(x+1)*x*(x-1)}{x*(x-1)}
      -\frac{(2x-1)*(x-1)}{x*(x-1)}+\frac{x*x}{x*(x-1)} =\\
      &\frac{x^3 - x}{x*(x-1)} - \frac{2x^2 - 3x + 1}{x*(x-1)} + \frac{x^2}{x*(x-1)} =
        \frac{x^3 - x^2 +2x  -1}{x(x-1)}
    \end{flalign*}
    
    \emph{Si te fijas bien no hemos hecho la multiplicación del denominador común, ¿por qué?\\
    Pues porque así tenemos las raíces del denominador y podemos saber si alguna de ellas es
    raíz del numerador.}\\
  
  Vamos a ver si se puede simplificar: en este caso el denominador es tiene de factores a $x$ y a $(x-1)$.
  \begin{itemize}
  \item El numerador no es divisible entre $x$, porque tiene un término sin $x$.
  \item La raíz de $(x-1)$ es $1$, vamos a probar si es raíz del numerador:
    \[1^3 - 1^2 +2*1 -1 = 1\]
    Con lo que no es raíz.
  \end{itemize}
  Al no tener nada en común numerador y denominador no se puede simplificar.
\end{solution}
\question Opera y simplifica $\frac{1}{a-1}+\frac{1}{a-3} -\frac{a-1}{a^2 - 4a + 3}$
  \begin{solution}
    La única diferencia con los anteriores es que en vez de $x$ tenemos $a$, pero el procedimiento es el mismo. Al ser \textbf{denominadores más complicados} que los anteriores, primero
    \textbf{reescribimos la operación con los denominadores factorizados}, porque así se ven las cosas más sencillas:
    \[\frac{1}{a-1} + \frac{1}{a-3} - \frac{a-1}{(a-1)(a-3)}\]
    Y vemos que el denominador común es $(a-1)*(a-3)$, con lo que la operación nos queda:
    \[\frac{a-3}{(a-1)(a-3)} + \frac{a-1}{(a-1)(a-3)} -\frac{a-1}{(a-1)(a-3)}\]
    \begin{center}
      \begin{small}
        \emph{(Al haber escrito primero la operación con los denominadores factorizados
        es más fácil hacer denominador común ya que la división entre denominadores es inmediata)}
      \end{small}
    \end{center}
    Realizamos la operación:
    \begin{flalign*}
    \frac{a-3}{(a-1)(a-3)} + \frac{a-1}{(a-1)(a-3)} -\frac{a-1}{(a-1)(a-3)} &=
                                                                              \frac{a-3+a-1 - a + 1}{(a-1)(a-3)} \\
                                                                            &= \frac{a-3}{(a-1)(a-3)} = \frac{1}{a-1}
      \end{flalign*}
    En este caso la simplificación final era evidente.
  \end{solution}
  
\question Y por último: realiza la operación $\frac{x}{x+1} - \frac{3}{x-2} + \frac{9}{x^2 - x - 2}$
  \begin{solution}
    Como hemos visto en el caso anterior, al ser denominadores complicados primero reescribimos la operación con los denominadores factorizados:
    \[\frac{x}{x+1} - \frac{3}{x-2} + \frac{9}{(x+1)(x-2)}\]
    Ahora ya podemos hacer denominador común, y en este caso lo vamos a ver un poco más despacio.\\
    Primero calculamos el m.c.m. de los denominadores, y está claro que es $(x+1)(x-2)$.
    \begin{itemize}
    \item Para la primera fracción el nuevo numerador es el resultado de dividir el m.c.m
      entre el denominador antiguo $\left(\frac{(x+1)(x-2)}{x+1} = (x-2)\right)$ y multiplicarlo
      por el numerador antiguo $(x*(x-2) = x^2 - 2x)$.
    \item Para la segunda fracción hacemos lo mismo, dividimos el m.c.m. entre el denominador
      antiguo $\left(\frac{(x+1)(x-2)}{x-2} = (x+1)\right)$ y lo multiplicamos por el numerador antiguo $(3*(x+1) = 3x + 3)$.
    \item Y para la tercera lo mismo: $\frac{(x+1)(x-2)}{(x+1)(x-2)}* 9 = 1* 9 = 9$.
    \end{itemize}
    La operación con denominador común queda:
    \[\frac{x^2 - 2x}{(x+1)(x-2)} - \frac{3x+3}{(x+1)(x-2)} + \frac{9}{(x+1)(x-2)}\]
    Y ya podemos efectuarla.
    \begin{flalign*}
      \frac{x^2 - 2x}{(x+1)(x-2)} - \frac{3x+3}{(x+1)(x-2)} + \frac{9}{(x+1)(x-2)} &=\\
      \frac{x^2 - 2x -3x - 3 + 9}{(x+1)(x-2)} = \frac{x^2 - 5x + 6}{(x+1)(x-2)}
    \end{flalign*}
    Ahora tenemos que ver si podemos simplificar.\\
    Tenemos que las raíces del denominador son $-1$ y $2$, vamos a probar si alguna es raíz del
    numerador $P(x) = x^2 - 5x + 6$.
    \begin{itemize}
    \item Para $-1$ tenemos $P(-1) = (-1)^2 - 5*(-1) + 6 = 1 + 5 + 6 \neq 0$, con lo que no es raíz.
    \item Con $2$ nos queda $P(2) = 2^2 - 5 *2 + 6 = 4 - 10 + 6 = 0$. Entonces $2$ sí es raíz, con lo que podemos simplificar.
    \end{itemize}
    La manera más fácil de simplificar una vez que sabemos cual es la raíz es utilizar Ruffini:
    \begin{center}
      \begin{tabular}{r|rrr}
        &1&$-5$&6\\
        2&&2&$-6$\\
        \hline
        &1&$-3$&0
      \end{tabular}
    \end{center}
    Con lo que
    \[\frac{x^2 - 5x + 6 }{(x+1)(x-2)} = \frac{x-3}{x+1}\]

    Y todo seguido con un paso en cada línea:
    \begin{gather*}
      \frac{x}{x+1} - \frac{3}{x-2} + \frac{9}{x^2 - x - 2} =\\
      \frac{x^2 - 2x}{(x+1)(x-2)} - \frac{3x+3}{(x+1)(x-2)} + \frac{9}{(x+1)(x-2)} = \\
      \frac{x^2 - 5x + 6}{(x+1)(x-2)} =
      \frac{x-3}{x+1}
    \end{gather*}
  \end{solution}
\end{questions}

\subsubsection{Producto y cociente de fracciones algebraicas.}
Al igual que ocurría con la suma, el producto y el cociente de fracciones algebraicas es
exactamente igual que en fracciones numéricas. La única diferencia es que, por lo general, no calcularemos el resultado del denominador, sino que lo dejaremos indicado porque así nos resultará más sencillo factorizar en caso de que sea necesario.

\subsubsection{Operaciones combinadas con fracciones algebraicas.}
En este apartado vamos a ver algunos ejemplos de operaciones combinadas con fracciones algebraicas en. Sabiendo todo lo anterior estos ejemplos no deberían de resultar difíciles, pero haremos aclaraciones si hay alguna sutileza que sea necesario resaltar.\\

Antes de empezar vamos a recordar algunas reglas importantes a la hora de hacer operaciones combinadas en las que intervienen fracciones:
\begin{itemize}
\item Hay que respetar siempre la prioridad de las operaciones.
\item Nunca se hace denominador común para multiplicar o dividir.
\item Solo se hace denominador común en las fracciones que se van a sumar en ese paso. Por ejemplo, si toca hacer una suma en un paréntesis solo hacemos denominador común de las fracciones que están en ese paréntesis y el resto quedan como están.
\item Todo lo que no se hace en un paso se copia exactamente igual en el siguiente.
\end{itemize}

Y una vez recordado todo esto, podemos empezar a hacer los ejemplos.
\begin{questions}
\question Empezamos con uno sencillo: $\left( \frac{1}{x} - 2 + x \right)*\left(\frac{x^3}{x^2 - 1}
  \right)$
  \begin{solution}
    Empezamos realizando la operación del primer paréntesis, en la que se ve que el denominador común es $x$, con lo que nos queda:
    \begin{gather*}
      \left( \frac{1}{x} - 2 + x \right)*\left(\frac{x^3}{x^2 - 1}\right) = \\
      \left( \frac{1}{x} - \frac{2x}{x} + \frac{x^2}{x} \right)*\left(\frac{x^3}{x^2 - 1}\right) = \\
      \frac{x^2 - 2x + 1}{x}*\frac{x^3}{x^2 - 1}
    \end{gather*}
    Ahora ya podemos realizar la multiplicación:
    \[\frac{x^2 - 2x + 1}{x}*\frac{x^3}{x^2 - 1} = \frac{(x^2 - 2x + 1)x^3}{x(x^2 - 1)}\]
    Solo hemos dejado indicadas las operaciones porque así podemos ver qué se puede simplificar,
    y en un primer vistazo está claro que se puede simplificar la $x$, con lo que queda:
    \[\frac{(x^2 - 2x + 1)x^3}{x(x^2 - 1)} = \frac{(x^2 - 2x + 1)x^2}{x^2 - 1}\]
    Aquí se ve que las raíces del denominador son $\pm 1$, con lo que lo podemos factorizar:
    \[\frac{(x^2 - 2x + 1)x^2}{x^2 - 1} = \frac{(x^2 - 2x + 1)x^2}{(x+1)(x-1)}\]
    Si factorizamos el polinomio del numerador, porque es sencillo, tenemos que $(x^2 - 2x +1 )=(x-1)^2$, con lo que podemos hacer:
    \[\frac{(x^2 - 2x + 1)x^2}{(x+1)(x-1)} = \frac{(x-1)^2x^2}{(x+1)(x-1)} =
      \frac{(x-1)x^2}{x+1} = \frac{x^3 - x}{x+1}\]
    Que es el resultado simplificado.
  \end{solution}
  
\question Realiza y simplifica $\left(\frac{3x}{x-2}:\frac{x^2}{x^2 - 4}\right):\frac{x+2}{2x^3}$
  \begin{solution}
    Empezamos haciendo el paréntesis, que al ser una división tenemos que multiplicar en cruz:
    \[\left(\frac{3x}{x-2}:\frac{x^2}{x^2 - 4}\right):\frac{x+2}{2x^3} =
      \frac{3x*(x^2 - 4)}{x^2*(x-2)}:\frac{x+2}{2x^3}\]
    Observa que no hemos desarrollado completamente las multiplicaciones, así podemos ver si hay algo que se puede simplificar. En este caso es evidente que se pueden simplificar las $x$ que están como factor:
    \[\frac{3x*(x^2 - 4)}{x^2*(x-2)}:\frac{x+2}{2x^3} = \frac{3(x^2 - 4)}{x*(x-2)}:\frac{x+2}{2x^3}\]
    También tenemos que $(x^2 - 4)$ es sencillo de factorizar, con lo que lo factorizamos y simplificamos:
    \[\frac{3(x^2 - 4)}{x*(x-2)}:\frac{x+2}{2x^3} =
      \frac{3(x+2)(x-2)}{x*(x-2)}:\frac{x+2}{2x^3} = \frac{3(x+2)}{x}:\frac{x+2}{2x^3}\]
    Realizamos la última división:
    \[\frac{3(x+2)}{x}:\frac{x+2}{2x^3} = \frac{3(x+2)*2x^3}{x*(x+2)}\]
    Y vemos que podemos simplificar las $x$ que están como factor y el factor $(x+2)$, quedando:
    \[\frac{3(x+2)*2x^3}{x*(x+2)} = 6x^2\]
  \end{solution}
\question Realiza y simplifica $\left(\frac{a-b}{a+b}-\frac{a+b}{a-b}\right)*
  \left(\frac{a}{b} - \frac{b}{a}\right)$
  \begin{solution}
    Tenemos dos paréntesis con una suma en cada uno, con lo que tenemos que hacer el denominador común que corresponda en cada uno.\\
    Al estar factorizados los denominadores es sencillo calcular el denominador común.
    \begin{itemize}
    \item En el primer paréntesis el denominador común es $(a+b)(a-b)$.
    \item En el segundo paréntesis es $a*b$.
    \end{itemize}
    Con esto la operación queda:
    \[\left(\frac{a-b}{a+b}-\frac{a+b}{a-b}\right)*
      \left(\frac{a}{b} - \frac{b}{a}\right) =
      \left(\frac{(a-b)^2}{(a+b)(a-b)}-\frac{(a+b)^2}{(a+b)(a-b)}\right)*
      \left(\frac{a^2}{ab} - \frac{b^2}{ab}\right)\]
    Desarrollamos las identidades notables para realizar las sumas.
    \begin{gather*}
      \left(\frac{(a-b)^2}{(a+b)(a-b)}-\frac{(a+b)^2}{(a+b)(a-b)}\right)*
      \left(\frac{a^2}{ab} - \frac{b^2}{ab}\right) =\\
      \left(\frac{a^2 + b^2 - 2ab}{(a+b)(a-b)}-\frac{a^2+b^2 + 2ab}{(a+b)(a-b)}\right)*
      \left(\frac{a^2}{ab} - \frac{b^2}{ab}\right) =
      \frac{-4ab}{(a+b)(a-b)}*\frac{a^2-b^2}{ab}
    \end{gather*}
    Realizamos el último producto:
    \[\frac{-4ab}{(a+b)(a-b)}*\frac{a^2-b^2}{ab} = \frac{-4a^2b^2(a^2 - b^2)}{ab(a+b)(a-b)}\]
    Y factorizando teniendo en cuenta que el numerador tiene una identidad notable,
    podemos simplificar de la siguiente manera:
    \[\frac{-4a^2b^2(a^2 - b^2)}{ab(a+b)(a-b)} =
      \frac{-4a^2b^2(a+b)(a-b)}{ab(a+b)(a-b)} = -4ab\]
  \end{solution}
\question Realiza y simplifica $\left(\frac{6}{1-x} + \frac{5x}{x-1}\right)*\frac{x^2 - 1}{2x}
  +\frac{3}{x}$
    \begin{solution}
      Tenemos que empezar por la suma del paréntesis, que tiene dos fracciones con distinto denominador. ¿O no?\\
      En realidad \textbf{es el mismo denominador pero cambiado de signo}:
      \[\boldsymbol{-(x-1) = -x + 1 = 1-x}\]
      Con lo que cambiando de signo la primera fracción ya tendremos el mismo denominador en ambas fracciones. \emph{Esto es algo que pasa a menudo en álgebra, que dos sumas que parecen distintas en realidad son la misma pero cambiada de signo.}\\
      \[\left(\frac{6}{1-x} + \frac{5x}{x-1}\right)*\frac{x^2 - 1}{2x} +\frac{3}{x}=
        \left(-\frac{6}{x-1} + \frac{5x}{x-1}\right)*\frac{x^2 - 1}{2x} +\frac{3}{x}=
        \frac{5x-6}{x-1}*\frac{x^2 - 1}{2x} +\frac{3}{x}\]
      Vemos que no hay nada con lo que podamos hacer una simplificación intermedia y vamos
      a por la multiplicación:
      \[\frac{(5x-6)(x^2 - 1)}{2x(x-1)} +\frac{3}{x}\]
      Aquí sí que podemos hacer una simplificación intermedia, ya que $(x^2 - 1) = (x+1)(x-1)$:
      \[\frac{(5x-6)(x^2 - 1)}{2x(x-1)} +\frac{3}{x} = 
        \frac{(5x-6)(x+1)(x-1)}{2x(x-1)} +\frac{3}{x} =
        \frac{(5x-6)(x+1)}{2x} + \frac{3}{x} =
        \frac{5x^2 - x -6}{2x} + \frac{3}{x}\]
      Ahora hemos desarrollado el producto porque lo siguiente que tenemos que hacer es una suma,
      y para poder sumarlo necesitamos que esté desarrollado. Para hacer la resta que queda
      tenemos que hacer denominador común, que en este caso es $2x$.
      \[\frac{5x^2 - x -6}{2x} + \frac{3}{x} =
        \frac{5x^2 - x -6}{2x} + \frac{6}{2x} = \frac{5x^2 - x}{2x}\]
      Si sacamos factor común en el numerador tendremos $5x^2 - x = x(5x-1)$, con lo que se puede
      simplificar de la siguiente manera:
      \[\frac{5x^2 - x}{2x} = \frac{x(5x - 1)}{2x} = \frac{5x - 1}{2}\]
    \end{solution}
  \question Opera y simplifica al máximo: $\cfrac{\cfrac{1}{x-1} - \cfrac{2x}{x^2 -1}}{
      \cfrac{1}{x+1}-1}$
    \begin{solution}
      Como es un castillo primero tenemos que realizar todas las operaciones del numerador y el
      denominador de la fracción grande. Escribimos cada una de las fracciones con el denominador
      factorizado para calcular del denominador común de una manera más cómoda.
      \[\cfrac{\cfrac{1}{x-1} - \cfrac{2x}{x^2 -1}}{\cfrac{1}{x+1}-1} =
        \cfrac{\cfrac{1}{x-1} - \cfrac{2x}{(x+1)(x-1)}}{\cfrac{1}{x+1}-1}\]
      Ahora hacemos el denominador común y hacemos cada operación.
      \[\cfrac{\cfrac{1}{x-1} - \cfrac{2x}{(x+1)(x-1)}}{\cfrac{1}{x+1}-1} =
        \cfrac{\cfrac{x+1}{(x+1)(x-1)} - \cfrac{2x}{(x+1)(x-1)}}{\cfrac{1}{x+1}-\cfrac{x+1}{x+1}} =
        \cfrac{\cfrac{1-x}{(x+1)(x-1)}}{-\cfrac{x}{x+1}}\]
      Cambiamos la línea grande por una división y la hacemos (lo normal es hacer esto en sucio.
      No lo volveremos a hacer aquí)
      \[\cfrac{\cfrac{1-x}{(x+1)(x-1)}}{-\cfrac{x}{x+1}} =
        \frac{1-x}{(x+1)(x-1)}: \frac{-x}{x+1} = \frac{-(1-x)(x+1)}{x*(x+1)(x-1)}\]
      Teniendo en cuenta que $-(1-x) = (x-1)$ se puede simplificar mucho:
      \[\frac{-(1-x)(x+1)}{x*(x+1)(x-1)} = \frac{(x+1)(x+1)}{x*(x+1)(x-1)} = \frac{1}{x}\]
    \end{solution}
\end{questions}

\section{Ecuaciones IV.}
\subsection{La ecuación de 2º grado.}
En estos momentos deberíamos de saber resolver perfectamente cualquier ecuación de 2º grado, completa o incompleta. Esto quiere decir que no vamos a ver cómo se resuelven en este apartado, sino que lo vamos a dedicar a estudiar algunas de sus propiedades así como el origen de la famosa fórmula para resolver las ecuaciones completas. El propósito de estas deducciones y propiedades es meramente didáctico en alguno de los casos, para que la persona que utilice estos apuntes se familiarice con la metodología que se utiliza en álgebra cuando ya se tiene un nivel avanzado.\\
En otros casos, como el significado gráfico de las soluciones, aportan conocimientos que serán utilizados más adelante.

\subsubsection{Deducción de la fórmula.}
En este nivel ya tenemos los suficientes conocimientos como para comprender de dónde procede la
fórmula de la ecuación de 2º grado y porqué es así.\\

\begin{large}
  \textbf{Breve repaso}\labeltext{breve repaso}{ecs_equivalentes}
\end{large}\\
Para poder entender bien todo lo que vamos a contar hay que repasar algunos conceptos de ecuaciones:
\begin{enumerate}
\item \textbf{Ecuaciones equivalentes}: se dice que dos ecuaciones son equivalentes cuando tienen las mismas soluciones, y en ese caso se puede reemplazar una por otra.\\
  Por ejemplo, la ecuación $x-1 = 3x+5$ y la ecuación $x+1 = -2$ son equivalentes, ambas tienen de solución $x= -3$ (esto se puede escribir de las siguiente manera $x-1 = 3x+5 \equiv x+1 = -2$)
\item Si en una ecuación añadimos el mismo término en ambos miembros obtendremos una ecuación equivalente.\\
  Es decir, si tenemos una ecuación $A(x) = B(x)$ la ecuación $A(x) + r= B(x) + r$ es equivalente para cualquier $r$ que sea un número real (esto último se escribe $r \in \realset$).
\item Si en una ecuación multiplicamos ambos miembros por un número distinto de cero obtendremos
  una ecuación equivalente.\\
  En una manera resumida:
  \[A(x) = B(x)\ \equiv\ A(x)*r = B(x)*r,\ \forall r \in \realset\quad
    \text{\small (el símbolo } \forall\ \text{\small significa ``para cualquier'')}\]
\item Con la división pasa lo mismo, si dividimos los dos miembros por un número distinto de cero
  obtendremos una ecuación equivalente.
  \[A(x) = B(x)\ \equiv\ \frac{A(x)}{r} = \frac{B(x)}{r},\ \forall r \in \realset\]
\end{enumerate}
Vamos a ver \textbf{un ejemplo}: tenemos la ecuación $x=-3$, cuya solución es evidente.\\
\begin{itemize}
\item Si sumamos el mismo valor en ambos lados la solución sigue siendo $-3$:
  \[x+5 = -3 + 5\]
\item Si además la multiplicamos por 2 la solución sigue siendo $-3$:
  \[2x+ 10 = -6 + 10\]
\item Si la dividimos entre 3 la solución seguirá siendo $-3$:
  \[\frac{2x + 10}{3} = \frac{-6 + 10}{3}\]
\end{itemize}\vspace*{.8cm}

Una vez repasado esto, podemos empezar a ver la deducción de la ecuación de 2º grado.\\
Empecemos con una sencilla, $x^2 = 4$. Ahí está claro lo que hay que hacer, para quitar el cuadrado
hacemos la raíz y ponemos el más menos delante para tener las dos soluciones:
$x= \pm \sqrt{4} = \pm 2$\\

Podemos complicarla un poco, por ejemplo $(x+3)^2= 5$. Para resolver esta ecuación podemos
desarrollar la identidad notable y luego aplicar la fórmula, pero es mucho más sencillo si damos los siguientes pasos:
\[(x+3)^2 = 5\]
\[x+3 = \pm \sqrt{5}\quad \text{\small quitamos el cuadrado con la raíz}\]
\[x = -3 \pm \sqrt{5}\quad \text{\small dejamos la }x\ \text{\small sola}\]
Es decir, si tenemos una ecuación de la forma
\begin{equation}
  \label{eq:ec2_ref}
  \tag{Ecuación de referencia}
  (x+m)^2 = n
\end{equation}
podemos resolverla de manera sencilla sin necesidad de utilizar la fórmula, simplemente dando los
pasos que hemos visto en el último ejemplo (la hemos llamado ecuación de referencia porque la vamos a utilizar más adelante).\\

\begin{large}
  \textbf{Deducción de la fórmula.}
\end{large}\\
Ahora ya podemos empezar con la deducción propiamente dicha.
Tenemos la \textbf{forma general de la ecuación de 2º grado}, que es:
\[ax^2 + bx + c = 0\]
Y si fuese de la forma de la \ref{eq:ec2_ref} sería sencillo resolverla, así que vamos a utilizar los métodos del \ref{ecs_equivalentes} para convertir la ecuación de 2º grado en la de referencia.\\

En la ecuación de 2º grado tenemos la identidad notable $(x+m)^2 = x^2 + 2mx + m^2$, con lo que vamos a intentar tener el segundo miembro $(x^2 + 2mx + m^2)$ de esta identidad en la ecuación de 2º grado. Para ello partimos de la ecuación general y nos llevamos la $c$ al otro lado porque no forma parte de la identidad notable:
\[ax^2 + bx = -c\]
En la identidad notable la $x^2$ no tiene coeficiente, así que dividimos toda la ecuación entre $a$ y simplificamos:
\[\frac{ax^2}{a} + \frac{bx}{a} = -\frac{c}{a}\]
\[x^2 + \frac{bx}{a} = -\frac{c}{a}\]
En la identidad notable tenemos que el coeficiente de $x$ es $2m$, y como aquí es $\frac{b}{a}$ tenemos que $2m = \frac{b}{a} \Rightarrow m = \frac{b}{2a}$.\\
En este punto tenemos que para la identidad notable completa nos falta $m^2 = \left(\frac{b}{2a}
\right)^2$ y para tenerlo se lo sumamos a ambos miembros (y así queda una ecuación equivalente):
\[\frac{ax^2}{a} + \frac{bx}{a} + \left(\frac{b}{2a}\right)^2 =
  -\frac{c}{a} + \left(\frac{b}{2a}\right)^2\]
Y ahora el primer miembro es la identidad notable $(x+m)^2$, ya que $m = \frac{b}{2a}$, con lo que podemos escribir la ecuación así:
\[\left(x + \frac{b}{2a}\right)^2 = \left(\frac{b}{2a}
  \right)^2  -\frac{c}{a}\]
\[\left(x + \frac{b}{2a}\right)^2 = \frac{b^2}{4a^2}-\frac{c}{a}\]
Vamos a sumar las fracciones del segundo miembro, el denominador común es $4a^2$:
\[\left(x + \frac{b}{2a}\right)^2 = \frac{b^2}{4a^2}-\frac{4ac}{4a^2}\]
\[\left(x + \frac{b}{2a}\right)^2 = \frac{b^2-4ac}{4a^2}\]
Ahora quitamos el cuadrado del primer miembro con la raíz:
\[x + \frac{b}{2a} =\pm \sqrt{\frac{b^2-4ac}{4a^2}}\]
\[x + \frac{b}{2a} =\pm \frac{\sqrt{b^2-4ac}}{\sqrt{4a^2}}\]
\[x + \frac{b}{2a} =\pm \frac{\sqrt{b^2-4ac}}{2a}\]
Y por último, despejamos la $x$ y como las fracciones tienen el mismo denominador ($2a$) las podemos juntar en una
\[\boldsymbol{x = \frac{-b \pm \sqrt{b^2 - 4ac}}{2a}}\]
Que es la fórmula que estábamos buscando.

\subsubsection{Suma y producto de las soluciones.}
En la ecuación de 2º grado $ax^2 + bx + c = 0$ sabemos que las dos soluciones vienen dadas según
el signo del resultado de la raíz que utilicemos:
\begin{itemize}
\item $x_1 = \frac{-b+\sqrt{b^2 - 4ac}}{2a}$
\item $x_2 = \frac{-b-\sqrt{b^2 - 4ac}}{2a}$
\end{itemize}

Pues vamos a sumar estas dos soluciones a ver qué pasa:
\[x_1 + x_2 = \frac{-b+\sqrt{b^2 + 4ac}}{2a} + \frac{-b+\sqrt{b^2 - 4ac}}{2a}
  = \frac{-b - b + \sqrt{b^2 + 4ac} - \sqrt{b^2 + 4ac}}{2a} = \frac{-2b}{2a}= \frac{-b}{a}\]
Es decir, que las \textbf{suma de las dos soluciones} vale $\boldsymbol{x_1 + x_2 = \frac{-b}{a}}$.\\

Vamos a ver qué pasa con el producto:
\[x_1 * x_2 = \frac{-b+\sqrt{b^2 + 4ac}}{2a} * \frac{-b+\sqrt{b^2 - 4ac}}{2a} =
  \frac{\left(-b+\sqrt{b^2 + 4ac}\right) * \left(-b+\sqrt{b^2 - 4ac}\right)}{4a^2}\]
Aquí tenemos un numerador que es una identidad notable, suma por diferencia. La desarrollamos y queda:
\[x_1 * x_2 = \frac{\left(-b+\sqrt{b^2 + 4ac}\right) * \left(-b+\sqrt{b^2 - 4ac}\right)}{4a^2}
  =\frac{b^2 - (b^2 - 4ac)}{4a^2} = \frac{4ac}{4a^2} = \frac{c}{a}\]
Con lo que el \textbf{producto de las dos soluciones vale }$\boldsymbol{x_1 * x_2 = \frac{c}{a}}$.

\subsubsection{Número de soluciones de la ecuación de segundo grado.} \label{discriminante}
A la hora de resolver ecuaciones de 2º grado nos hemos encontrado con que el número de soluciones
no siempre es el mismo, a veces hay dos soluciones, a veces una (que se suele llamar solución doble) y a veces ninguna.\\

En determinadas ocasiones solo nos interesará saber cuantas soluciones tiene la ecuación sin
resolverla completamente, por eso vamos a ver cómo saber el número de soluciones directamente.\\

Si nos fijamos en la fórmula de la ecuación de 2º grado $ax^2 + bx +c = 0$
\[x = \frac{-b \pm \sqrt{b^2 - 4ac}}{2a}\]
es fácil darse cuenta de que lo que nos da las distintas soluciones es el símbolo
$\boldsymbol{\pm}$ y que el hecho de que no haya ninguna solución viene dado por que el radicando
sea negativo.\\
Es decir, parece que todo viene de la parte derecha del $\pm$, la raíz. Y en concreto el
\textbf{radicando} $\boldsymbol{b^2 - 4ac}$. Y es tan importante que tiene nombre propio, se llama
\textbf{discriminante}.
\[\boldsymbol{\Delta = b^2 - 4ac}\quad\quad\text{\small{(Se representa con la letra griega delta
      mayúscula)}}\]

Y este discriminante va a decirnos qué va a pasar con las soluciones:
\begin{itemize}
\item Si el \textbf{discriminante es positivo} ($\boldsymbol{b^2 -4ac > 0}$) la fórmula queda
  de la siguiente manera:
  \[x = \frac{-b \pm k}{2a}\]
  donde $k = \sqrt{b^2 - 4ac}$, con lo que tendremos una solución con el $+$ y otra con el $-$.
  \textbf{Dos soluciones}.
\item \textbf{Si el discriminante es cero} ($\boldsymbol{b^2 - 4ac = 0}$) la fórmula queda:
  \[x = \frac{-b}{2a}\]
  con lo que solo tendremos \textbf{una solución}.
\item \textbf{Si el discriminante es negativo} ($\boldsymbol{b^2 - 4ac < 0}$) no podemos hacer la raíz, con lo que la ecuación \textbf{no tiene solución} real.
\end{itemize}

Vamos a ver un par de \textbf{ejemplos de cómo se utiliza el discriminante}.
\begin{questions}
\question Calcula el discriminante de la siguiente ecuación de segundo grado e indica cuantas soluciones tiene: $x^2 - 3x + 1 = 0$.
  \begin{solution}
    En la ecuación que nos dan tenemos que los coeficientes son $a=1$, $b = -3$ y $c = 1$, con lo que el discriminante esa
    \[\Delta = (-3)^2 - 4*1*1 = 9 - 4 = 5\]
    y como \textbf{5 es positivo, la ecuación tiene dos soluciones}.
  \end{solution}
\question Indica cuantas soluciones tiene la siguiente ecuación sin resolverla: $5x^2 - 3x + 4 = 0$.
  \begin{solution}
    Como en el caso anterior calculamos el discriminante:
    \[\Delta = (-3)^2 - 4*5*4 = 9 - 80 = -71\]
    En este caso \textbf{es negativo, la ecuación no tiene solución} real.
  \end{solution}
\question Calcula el valor del parámetro $k$ para que la ecuación $k*x^2 - 4x + 12 = 0$ tenga dos soluciones iguales.
  \begin{solution}
    Si nos piden que tenga dos soluciones iguales es lo mismo que si nos piden que tenga solución doble, con lo que el discriminante tiene que valer cero.\\
    En este caso los coeficientes son $a=k$, $b= -4$ y $c=12$, con lo que el discriminante es:
    \[\Delta = (-4)^2 - 4*k*12 = 16 + 48k\]
    Y hemos dicho que tiene que valer cero:
    \[16 + 48k = 0\]
    Resolvemos la ecuación y obtenemos
    \[k = -\frac{16}{48} = -\frac{1}{3}\]
    Con lo que la ecuación $-\frac{\ x^2}{3} - 4x + 12 = 0$ tiene una solución doble.
  \end{solution} 
\end{questions}
Veremos más ejercicios de uso del discriminante cuando hayamos visto cómo resolver inecuaciones.
\subsection{Ecuaciones polinómicas de grado mayor que dos.}
La resolución general de ecuaciones de grado mayor que dos es bastante complejo y escapa a los
objetivos de las matemáticas que se imparten en E.S.O. y bachillerato. De hecho ni siquiera existe
un método general para resolver ecuaciones de grado mayor o igual a cinco (ni existe ni puede
existir por el \emph{Teorema de Abel-Ruffini}).\\

Pero el hecho de que no vayamos a estudiar la resolución general no significa que con los
contenidos que hemos aprendido hasta ahora no podamos resolver algunas ecuaciones de grados altos.\\
Y esto podemos hacerlo porque ya hemos visto cómo factorizar un polinomio, lo hacemos obteniendo
las raíces, y las raíces son ¡la solución de la ecuación que iguala el polinomio a cero!.\\
Además hemos visto también cómo resolver ecuaciones bicuadradas, que son un tipo especial de
ecuaciones de cuarto grado y el método es sencillo.

Vamos a ver un par de ejemplos.
\begin{questions}
\question Resuelve la ecuación $x^5+x^4+x^2+2x = 3x^3 +2x^2$
  \begin{solution}
    Claramente no tenemos una ecuación de la forma $P(x) = 0$ para poder buscar las raíces, con lo que lo primero que hacemos es ponerla en esa forma.
    \[x^5 + x^4 -3x^3 - x^2 + 2x = 0\]
    Y ahora utilizamos el mismo método que hemos visto para factorizar polinomios.
    \begin{enumerate}
    \item Sacamos factor común:
      \[x*(x^4 + x^3 - 3x^2 -x + 2)\]
      El hecho de que haya salido una $x$ de factor común significa que
      $\boldsymbol{x= 0}$ es una de las soluciones.
    \item Hacemos Ruffini hasta llegar a grado 2:
      \begin{center}
        \begin{tabular}{r|rrrrr}
          &1&1&$-3$&$-1$&2\\
          1&&1&2&$-1$&$-2$\\
          \hline
          &1&2&$-1$&$-2$&\\
          $-1$&&$-1$&$-1$&2&\\
          \hline
          &1&1&$-2$&&
        \end{tabular}
      \end{center}
      Con esto hemos sacado otras dos soluciones, $\boldsymbol{x=1}$ y $\boldsymbol{x=-1}$.
    \item Resolvemos la ecuación de segundo grado resultante:
      \[x^2 + x - 2 = 0\]
      \[x = \frac{-1 \pm \sqrt{1^2 - 4*1*(-2)}}{2} = \frac{-1 \pm 3}{2}\]
      Con lo que otras dos soluciones son $\boldsymbol{x=-2}$ y $\boldsymbol{x=1}$, que ya la
      teníamos con lo que es solución doble.
    \end{enumerate}
    En resumen, tenemos $x=0$, $x=-1$ y $x= -2$ como soluciones simples y $x=1$ como solución
    doble.\\
    \large{\emph{IMPORTANTE: recuerda que las soluciones (raíces) no cambian de signo como los términos en la factorización.}}
  \end{solution}
\question Resuelve la ecuación $x^6 = x^3$
  \begin{solution}
    Al igual que antes ponemos todo en el mismo lado de la ecuación:
    \[x^6 - x^3 = 0\]
    Y factorizamos:
    \begin{enumerate}
    \item Factor común:
      \[x^3*(x^3 -1)\]
      Luego $x=0$ es solución triple (porque lleva un cubo y es como si apareciese tres veces).
    \item Ruffini:
      \begin{center}
        \begin{tabular}{r|rrrr}
          &1&0&0&$-1$\\
          1&&1&1&1\\
          \hline
          &1&1&1
        \end{tabular}
      \end{center}
      $x=1$ es solución.
    \item Resolvemos la ecuación de segundo grado:
      \[x^2 + x + 1 = 0\]
      \[x = \frac{-1 \pm \sqrt{1^2 - 4*1*1}}{2}\]
      Y no tiene solución porque el radicando es negativo.
    \end{enumerate}
    Entonces las soluciones son: $x=0$ solución triple y $x=1$ solución simple.
  \end{solution}
\end{questions}

\subsection{Ecuaciones racionales.}
Una ecuación racional es aquella que tiene cocientes de polinomios. Es decir, aquella que tiene
polinomios en los denominadores.\\

En el punto \ref{ecs_frac} (página \pageref{ecs_frac}) hemos visto cómo se resuelven ecuaciones
con fracciones, y es lo mismo que vamos a hacer aquí.\\
La única diferencia es que aquí la comprobación de las soluciones forma parte indispensable del proceso de resolución ya que tenemos que comprobar que ninguna de las soluciones anula alguno de los
denominadores de la ecuación, en otro caso esa solución no es válida.\\

Lo mejor es que lo veamos con \textbf{unos ejemplos}.
\begin{questions}
\question Resuelve la ecuación $\frac{x}{2} - \frac{8}{x} = 0$
  \begin{solution}
    Como vimos en el punto \ref{ecs_frac} la manera de resolver ecuaciones con fracciones es
    hacer denominador común de toda la ecuación, quitar los denominadores y resolver normalmente.\\
    Y eso es lo que vamos a hacer aquí.\\

    Tenemos que los denominadores son 2 y $x$, con lo que el denominador común solo puede ser $2x$,
    con lo que la ecuación queda:
    \[\frac{x*x}{2x} - \frac{8*2}{x} = \frac{0}{2x}\]
    \begin{center}
      \begin{small}
        \emph{(Hemos indicado más operaciones de lo normal por si nos hace falta recordar
          el mecanismo para hacer denominador común)}
      \end{small}
    \end{center}
    Es decir: $\frac{x^2}{2x}- \frac{16}{2x} = 0$, con lo que quitamos denominadores y resolvemos.
    \[x^2 - 16 = 0\]
    \[x^2 = 16\]
    \[x = \pm \sqrt{16} = \pm 4\]
    Luego las posibles soluciones son $x=-4$ y $x=4$, ahora tenemos que comprobar que ninguna anula
    el denominador:
    \begin{itemize}
    \item Con $x=4$ tenemos $\frac{4}{2}- \frac{8}{4} = 0$, que no anula ningún denominador y
      se cumple.
    \item Con $x = -4$ tenemos $\frac{-4}{2}- \frac{8}{-4} = 0$, que también se cumple y no anula
      ningún denominador.
    \end{itemize}
  \end{solution}
\question Resuelve la ecuación $\frac{3}{x^2} - \frac{x-2}{x} = 0$
  \begin{solution}
    En este caso tenemos que hacer el mínimo común múltiplo de $x$ y $x^2$, que es $x^2$. De manera que la ecuación queda (ahora no vamos a poner indicadas las operaciones):
    \[\frac{3}{x^2} - \frac{x^2 - 2x}{x^2} = 0\]
    Quitamos denominadores y resolvemos:
    \[3 - x^2 + 2x = 0\]
    Reordenamos y cambiamos de signo para que sea más sencilla de resolver.
    \[x^2 - 2x - 3 = 0\]
    \[x = \frac{2\pm\sqrt{(4 + 12}}{2} = \frac{2 \pm 4}{2}\]
    Con lo que las posibles soluciones son $x = -1$ y $x = 3$. Comprobamos:
    \begin{itemize}
    \item Para $x = -1$ queda $\frac{3}{1} - \frac{-1-2}{-1} = 0$, que se cumple y no se anula ningún
      denominador.
    \item Para $x = 3$ queda $\frac{3}{9} - \frac{3 - 2}{3} = 0$, que también se cumple sin anular algún
      denominador.
    \end{itemize}
  \end{solution}
\question Resolver $\frac{1}{x^2 - x} - \frac{1}{x-1} = 0$
  \begin{solution}
    En este caso el cálculo del denominador común ya no es tan sencillo, con lo que primero vamos
    a factorizar los denominadores y queda:
    \[\frac{1}{x*(x-1)} - \frac{1}{x-1} = 0\]
    De manera que tenemos que hacer el mínimo común múltiplo de $x*(x-1)$ y $(x-1)$ que, obviamente,
    es $x*(x-1)$, así que hacemos denominador común y la ecuación queda:
    \[\frac{1}{x*(x-1)} - \frac{x}{x(x-1)} = 0\]
    Quitamos denominadores:
    \[1 - x = 0\]
    Con lo que la posible solución es $x = 1$. Comprobamos y nos queda:
    \[\frac{1}{0} - \frac{1}{0} = 0\]
    Y vemos que se anulan los dos denominadores, por lo tanto esa solución no es válida.\\
    Y si la única posible solución no es valida la ecuación no tiene solución.
  \end{solution}
\question Resuelve la ecuación $\frac{1}{x-6} + \frac{x}{x-2} = \frac{4}{x^2 - 8x + 12}$.
  \begin{solution}
    Primero factorizamos denominadores:
    \[\frac{1}{x-6} + \frac{x}{x-2} = \frac{4}{(x-6)(x-2)}\]
    Y vemos que el denominador común es $(x-6)(x-2)$. Lo aplicamos y queda:
    \[\frac{x-2}{(x-6)(x-2)} + \frac{x*(x-6)}{(x-6)(x-2)} = \frac{4}{(x-6)(x-2)}\]
    Quitamos denominadores y resolvemos:
    \[x-2 + x^2 - 6x = 4\]
    \[x^2 - 5x - 6 = 0\]
    Y al aplicar la fórmula obtenemos que las posibles soluciones son $x = -1$ y $x = 6$. Las
    comprobamos:
    \begin{itemize}
    \item Para $x = -1$ queda:
      \[\frac{1}{-1-6} + \frac{-1}{-1-2} = \frac{4}{1 +8 + 12}\]
      \[-\frac{1}{7} + \frac{1}{3} = \frac{4}{21}\]
      
      Que es cierto y no anula ningún denominador.
    \item Para $x = 6$ queda $\frac{1}{0} + \frac{6}{4} = \frac{4}{0}$.\\
      
      Y \textbf{al haber denominadores nulos} tenemos que $x = 6$ \textbf{no es una solución
        válida}.
    \end{itemize}
    Por tanto la ecuación del enunciado tiene como única solución $x = -1$.
  \end{solution}
\end{questions}
\subsection{Ecuaciones irracionales.}
Se llama ecuaciones irracionales a aquellas en las que aparece alguna raíz cuyo radicando es una expresión algebraica.\\

En general son ecuaciones complicadas de resolver pero aquí solo vamos a ver los casos más simples,
aquellos en los que solo hay una o dos raíces cuadradas y el radicando es un polinomio de segundo
grado como máximo.\\
Al igual que en el caso anterior la comprobación de las posibles soluciones forma parte del
proceso de resolución ya que no todas son válidas debido a que al elevar al cuadrado se pierden
signos.\\

Para simplificar su comprensión lo vamos a clasificar en dos tipos, un primer tipo cuando hay una raíz cuadrada y un segundo tipo cuando hay dos raíces cuadradas.
\subsubsection{Tipo I. Solo una raíz cuadrada.}
Vamos a ver el mecanismo resolviendo un ejemplo: $x+2\sqrt{x+1} = 2x + 1$
\begin{enumerate}
\item \textbf{Dejamos la raíz sola y pasamos todos lo demás términos al otro miembro.}
  \[2\sqrt{x+1} = 2x + 1 - x\]
  \[2\sqrt{x+1} = x + 1\]
\item \textbf{Elevamos ambos miembros al cuadrado y así desaparece la raíz.}
  \begin{center}
    \small{(al ser una explicación vamos a dar muchos pasos. Más adelante nos saltaremos algunos)}
  \end{center}
  \[\left( 2\sqrt{x+1} \right)^2 = (x+1)^2\]
  \[2^2 * \left(\sqrt{x+1}\right)^2 = x^2 + 2x + 1\quad\quad\text{\small{Identidad notable.}}\]
  \[4*(x+1) = x^2 + 2x + 1\]
\item \textbf{Resolvemos como una ecuación normal.}\\
  En este caso nos queda una ecuación de segundo grado.
  \[4x + 4 = x^2 + 2x +1\]
  \[x^2 -2x -3 = 0\]
  
  Y aplicando la fórmula obtenemos las posibles soluciones $x= -1$ y $x= 3$.
\item \textbf{Comprobamos las soluciones para ver cuales son válidas.}\\
  \begin{itemize}
  \item Para $x = -1$ la ecuación queda:
    \[-1 + 2*\sqrt{-1 + 1} = 2*(-1) + 1\]
    \[-1 + 0 = -2 + 1\]

    Obviamente esto es cierto, con lo que $x= -1$ es una solución válida.
  \item Para $x = 3$ queda:
    \[3 + 2*\sqrt{3+1} = 2*3 + 1\]
    \[3 + 2*2 = 6 + 1\]

    Que también es cierto, con lo que $x = 3$ también es válida.
  \end{itemize}
\end{enumerate}

Por tanto $x = -1$ y $x = 3$ son soluciones de la ecuación propuesta.\\

Vamos a ver algún ejemplo más.\\
\begin{questions}
\question Resuelve la ecuación $\sqrt{2x -1} + 2 = x$.
  \begin{solution}
    Solo tenemos que seguir los pasos indicados en el ejercicio anterior.
    \begin{enumerate}
    \item Dejamos la raíz sola:
      \[\sqrt{2x - 1} = x - 2\]
    \item Elevamos al cuadrado ambos miembros:
      \[2x - 1 = (x- 2)^2 \]
      \[2x - 1 = x^2 - 4x + 4\]
    \item Resolvemos:
      \[x^2 - 6x + 5 = 0\]

      Que nos da como posibles soluciones $x = 1$ y $x = 5$
    \item Comprobamos:
      \begin{itemize}
      \item Para $x = 1$ tenemos:
        \[\sqrt{2*1 - 1} + 2 = 1\]
        \[\boldsymbol{1 + 2 = 1}\]

        Esto \textbf{es claramente falso} con lo que $x = 1$ \textbf{no es una solución válida}.
      \item Para $x = 5$ queda:
        \[\sqrt{2*5 - 1} + 2 = 5\]
        \[3 + 2 = 5\]

        Esto sí que es cierto, con lo que $x = 5$ sí es una solución válida.
      \end{itemize}
    \end{enumerate}

    Con lo que la solución de la ecuación $\sqrt{2x -1} + 2 = x$ es $x = 5$.
  \end{solution}
\question Resuelve la ecuación $2x - \sqrt{x^2 -3} = 5 - 2\sqrt{x^2 - 3}$.
  \begin{solution}
    En este caso \emph{parece que tenemos dos raíces}, pero si nos fijamos bien es \textbf{solo una ya que son radicales semejantes}, con lo que podemos agruparlos:
    \[2x -5 = \sqrt{x^2 - 3} -2\sqrt{x^2 - 3}\]
    \[2x -5 = -\sqrt{x^2 - 3}\]
    Y al agruparlos ya hemos dado el primer paso. Solo queda elevar todo al cuadrado y resolver:
    \[(2x-5)^2 = x^2 - 3\]
    \[4x^2 - 20x +25 = x^2 - 3\]
    \[3x^2 - 20x + 28 = 0\]
    Con la fórmula obtenemos que las posibles soluciones son $x = 2$ y $x = \frac{14}{3}$, ahora
    tenemos que comprobarlas.
    \begin{itemize}
    \item Con $x = 2$ queda:
      \[2*2 - \sqrt{2^2 - 3} = 5 - 2*\sqrt{2^2 - 3}\]
      \[4 - 1 = 5 - 2\]

      Esto es cierto, así que $x = 2$ es una solución válida.
    \item Con $x = \frac{14}{3}$:
      \[2*\frac{14}{3} - \sqrt{\left(\frac{14}{3}\right)^2 - 3} =
        5- 2* \sqrt{\left(\frac{14}{3}\right)^2 - 3}\]
      \[\frac{28}{3} - \frac{13}{3} = 5 - \frac{26}{3}\]
      
      Y esto no es cierto.
    \end{itemize}

    Con lo cual hemos obtenido que la única solución de la ecuación planteada es $x = 2$.
  \end{solution}
\end{questions}

\subsubsection{Tipo II. Ecuaciones con dos raíces cuadradas.}
Se entiende que aquí los radicales que aparecen no son semejantes y hay que tener en cuenta que
este tipo se puede complicar bastante, pero solo vamos a ver las más sencillas. Lo que sí es cierto
es que el método es un poco más largo que en el tipo anterior.\\
Al igual que en el caso anterior vamos a ver los pasos con un ejemplo y luego veremos algunos
ejemplos más. Como ejemplo para ver los pasos vamos a resolver la ecuación $3 - \sqrt{x + 3} =
\sqrt{x - 1} + 1$.
\begin{enumerate}
\item Empezamos por dejar una de las raíces sola en uno de los dos miembros. Da igual cual, por comodidad vamos a intentar dejar las que queden sin signo menos delante para luego no confundirnos. Con lo
  dicho la ecuación queda:
  \[2 - \sqrt{x + 3} =\sqrt{x - 1}\]
\item Elevamos ambos miembros al cuadrado. Aquí hay que tener cuidado, porque queda una identidad notable un poco complicada:
  \[\left(\overbrace{2}^a - \overbrace{\sqrt{x + 3}}^b \right)^2 =\left( \sqrt{x - 1} \right)^2\]
  Es decir, tenemos la identidad $(a-b)^2$ pero el segundo término es la raíz, con lo que:
  \begin{itemize}
  \item $b^2 = x+3$
  \item $2ab = 2*2*\sqrt{x+3}$
  \end{itemize}
  Entonces nos queda:
  \[\overbrace{4}^{a^2} + \overbrace{x + 3}^{b^2} - \overbrace{4\sqrt{x+3}}^{2ab} = x-1\]
  Y esto es una ecuación irracional con solo una raíz cuadrada, con lo que aplicamos lo visto en
  el tipo anterior:
\item Dejamos la raíz sola:
  \[8 = 4\sqrt{x + 3}\]
\item Elevamos ambos miembros al cuadrado:
  \[64 = 16(x+3)\]
\item Resolvemos:
  \[64 = 16 x + 48\]
  \[16 = 16 x\]
  \[x = 1\]
\item Comprobamos:
  \[3 - \sqrt{1 + 3} = \sqrt{1 - 1} + 1\]
  \[3 - 2 = 0 + 1\]
  Y como esto es cierto la solución $x = 1$ es válida.
\end{enumerate}

En resumen, el método es similar al del tipo anterior pero los pasos de dejar una raíz sola y
elevar al cuadrado hay que darlos dos veces.\\

Y vistos los pasos \textbf{vamos a por los ejemplos}.

\begin{questions}
\question Resuelve la ecuación $\sqrt{2x - 3} - \sqrt{x+7} = 4$
  \begin{solution}
    Procedemos a dar los pasos indicados:
    \begin{enumerate}
    \item Dejamos una raíz sola (preferiblemente la que no tenga un menos delante);
      \[\sqrt{2x - 3} = 4 + \sqrt{x + 7}\]
    \item Elevamos al cuadrado (con cuidado por la identidad notable con la raíz):
      \[2x - 3 = \overbrace{16}^{a^2} + \overbrace{x + 7}^{b^2} + \overbrace{8\sqrt{x+7}}^{2ab}\]
    \item Dejamos sola la raíz que queda:
      \[x -26 = 8\sqrt{x+7}\]
    \item Elevamos al cuadrado ambos miembros:
      \[x^2 -52x + 676 = 64 (x +7)\]
    \item Resolvemos la ecuación que queda:
      \[x^2 - 116x +228 = 0\]
      Que aplicando la fórmula nos da como posibles soluciones $x=2$ y $x=114$.
    \item Comprobamos qué soluciones son válidas:
      \begin{itemize}
      \item $x = 2$:
        \[\sqrt{2*2-3} - \sqrt{2+7} = 4\]
        \[1-3 = 4\]
        Que no es cierto, por lo que $x=2$ no es solución.
      \item $x = 114$:
        \[\sqrt{2*114-3} - \sqrt{114 +7} = 4\]
        \[15 - 11 = 4\]
        Que sí es cierto.
      \end{itemize}
    \end{enumerate}
    Por tanto la solución de la ecuación es $x = 114$.
  \end{solution}
\question Resuelve $\sqrt{3x - 5} + \sqrt{2x + 5} = 4$.
  \begin{solution}
    Dejamos un ráiz sola:
    \[\sqrt{3x - 5} = 4 - \sqrt{2x +5}\]
    Elevamos ambos miembros al cuadrado:
    \[3x - 5 = 16 +2x + 5 - 8 \sqrt{2x +5}\]
    \[x -26 = -8\sqrt{2x+5}\]
    Volvemos a elevar y resolvemos la ecuación:
    \[x^2 - 52x + 676 = 64(2x+5)\]
    \[x^2 - 180x + 356 = 0\]
    Y las posibles soluciones son $x = 2$ y $x = 178$.\\
    Comprobamos si son válidas.
    \begin{itemize}
    \item $x = 2$:
      \[\sqrt{3*2 - 5} + \sqrt{2*2 + 5} = 4\]
      \[1 + 3 = 4\]
      Es cierto, con lo que $x = 2$ es solución.
    \item $x = 178$:
      \[\sqrt{3*178 - 5} + \sqrt{2*178+5} = 4\]
      \[23 + 19 = 4\]
      No es cierto, $x = 178$ no es solución.
    \end{itemize}
    La solución de la ecuación es $x = 2$
  \end{solution}
\question Resuelve $\sqrt{\frac{4x-2}{2}} - 1 = \frac{\sqrt{12x + 4}}{4}$
  \begin{solution}
    Esta es un poco más complicada, ya que tenemos fracciones y raíces.\\
    En estos casos lo importante es el orden:
    \begin{itemize}
    \item Si todos los denominadores están fuera de las raíces lo primero es hacer denominador común y quitar denominadores.
    \item Si hay algún denominador dentro de una raíz tenemos que quitar primero las raíces que que tienen el denominador dentro para
      poder hacer el denominador común.
    \end{itemize}
    En este caso tenemos un denominador dentro de una raíz. Si dejamos esta raíz sola al elevar al cuadrado ya no tendremos denominadores
    dentro de ninguna raíz y será más sencillo. Por lo tanto vamos a hacer eso:
    \[\sqrt{\frac{4x - 2}{2}} = 1 + \frac{\sqrt{12x+4}}{4}\]
    \[\sqrt{\frac{4x - 2}{2}} = \frac{4+\sqrt{12x+4}}{4}
      \quad\text{\small{(Así es más fácil hacer la identidad notable).}}\]
    \[\frac{4x - 2}{2} = \frac{16 + 12x + 4 + 8\sqrt{12x+4}}{16}\]
    Ahora ya se puede hacer denominador común y quitar los denominadores, aunque es conveniente simplificar primero y es lo que vamos
    a hacer ya que en el primer miembro todo es divisible entre 2 y en el segundo miembro entre 4.
    \[2x - 1 = \frac{4 + 3x + 1 + 2\sqrt{12x + 4}}{4}\]
    Entonces hacemos el denominador común, que es 4.
    \[\frac{8x - 4}{4} = \frac{4 + 3x + 1 + 2\sqrt{12x + 4}}{4}\]
    Operamos todo para dejar la raíz sola;
    \[8x - 4 = 4 + 3x + 1 + 2\sqrt{12x + 4}\]
    \[5x -9 = 2\sqrt{12x + 4}\]
    Elevamos al cuadrado y resolvemos la ecuación resultante.
    \[25x^2 - 90 x + 81 = 48x + 16\]
    \[25x^2 - 138x + 65 = 0\]
    Y nos da de posibles soluciones son $x = 5$ y $x = \frac{13}{25}$. Comprobamos:
    \begin{itemize}
    \item $x = 5$
      \[\sqrt{\frac{4*5 - 2}{2}} -1 = \frac{\sqrt{12*5 + 4}}{4}\]
      \[3 - 1 = \frac{8}{4}\]
      Lo cual es cierto, $x = 1$ es solución.
    \item $x = \frac{77}{25}$
      \[\sqrt{\dfrac{4*\frac{13}{25} - 2}{2}} - 1 =\dfrac{\sqrt{12*\frac{13}{25} + 4}}{4}\]
      \[\sqrt{\frac{1}{25}} - 1 = \ddfrac{\frac{16}{5}}{4}\]
      \[\frac{1}{5} - 1 = \frac{4}{5}\]
      Y esto no es cierto, con lo que $x = \frac{13}{25}$ no es solución.
    \end{itemize}
  \end{solution}
\end{questions}

\subsection{Ecuaciones logarítmicas.}
Las ecuaciones logarítmicas son \emph{aquellas en las que la incógnita forma parte del argumento del logaritmo}.\\

Por ejemplo, la siguiente ecuación \textbf{es logarítmica}:
\[\log (x+1) = 3-\log x\]

Y las siguiente ecuación \textbf{no es logarítmica}:
\[x^2- 5 + \log 15 = x -1\]

Antes de empezar a ver cómo resolverlas vamos a hacer un breve repaso de logaritmos que nunca
está de más.

\subsubsection{Repaso de logaritmos.}\label{repaso_logaritmos}
\begin{enumerate}
\item \textbf{Definición}:\\
  Se define el logaritmo como el exponente al que hay que elevar una base para obtener un
  resultado determinado.\\
  Es decir:
  \[\boldsymbol{\log_a b = c\ \Leftrightarrow a^c = b}\]

  Esto tiene algunas consecuencias importantes que se utilizarán en la resolución de ecuaciones:
  \begin{itemize}
  \item $n = \log_a a^n$ \emph{(a veces cambiaremos un número por la expresión del logaritmo)}
  \item $b = a^{\log_a b}$ \emph{(esto lo utilizaremos más en ecuaciones exponenciales)}
  \item $0 = \log_a 1$
  \end{itemize}

  Quizá es conveniente también recordar que las bases más utilizadas tienen su propia forma de
  escribir los logaritmos:
  \begin{itemize}
  \item Logaritmo decimal: $\boldsymbol{\log_{10} b = \log b}$
  \item Logaritmo neperiano: $\boldsymbol{\log_{\text{\textbf{\large e}}} b = \ln b}$
  \end{itemize}
\item \textbf{Logaritmo de un producto}:\\
  El logaritmo de un producto se puede transformar en una suma de logaritmos.
  \[\log_a (m*n) = \log_a n + \log_a n\]
  \textbf{No hay que confundir con que el producto de logaritmos se puede transformar en
    una suma ni nada parecido. Eso no se puede hacer.}
\item \textbf{Logaritmo de un cociente}:
  El logaritmo de un cociente se puede transformar en una diferencia de logaritmos.
  \[\log_a \frac{m}{n} = \log_a m - \log_a n\]
\item \textbf{Logaritmo de una potencia}:
  Cuando tenemos el logaritmo de una potencia podemos sacar el exponente multiplicando y dejar
  solo la base en el argumento del logaritmo.
  \[\log_a m^n = n *\log_a m\]
  Esto tiene una consecuencia con las raíces por ser éstas potencias fraccionarias:
  \[\log_a \sqrt[n]{m} = \log_a m^\frac{1}{n} = \frac{1}{n}* \log_a m = \frac{\log_a m}{n}\]
  \textbf{Es importante fijarse en el tamaño de la línea de fracción, no es lo mismo
    $\frac{\log_a m}{n}$ que $\log_a \frac{m}{n}$. Son cosas distintas.}
\item \textbf{Cambio de base}:
  Esta propiedad es especialmente importante cuando tenemos una ecuación en la que aparecen
  logaritmos con distintas bases.
  \[\log_a x = \frac{\log_b x}{\log_b a}\]
  \textbf{De nuevo es importante el tamaño de la línea de fracción.}
\end{enumerate}

Sobra decir que para resolver ecuaciones logarítmicas hay que saberse todo esto que acabamos de
repasar.

\subsubsection{Resolución de ecuaciones logarítmicas.}
La resolución de ecuaciones logarítmicas se basa en la unicidad del logaritmo; es decir, si dos
logaritmos de la misma base son iguales es porque los argumentos son iguales.\\
De forma simbólica:
\[\log_a b = \log_a c \Leftrightarrow b = c\]

\textbf{IMPORTANTE}: Una vez resuelta la ecuación tenemos que comprobar las posibles soluciones, ya que puede que no todas sean válidas.\\

Vamos a ver cómo resolver dos tipos de ecuaciones logarítmicas.
\begin{itemize}
\item \large{\textbf{Tipo I}}\\
  El primer tipo son los ecuaciones en las que podemos aplicar las propiedades de los logaritmos
  para conseguir que haya un único logaritmo con la misma base en cada miembro de la ecuación,
  con lo que si tenemos una ecuación en la que ambos miembros son logaritmos de la misma base
  podemos quitar los logaritmos.\\
  Por ejemplo:
  \[\log (x^2 - 1) = \log \frac{x}{x-1} \Leftrightarrow x^2 - 1 = \frac{x}{x-1}\]

  ¿Y cómo vamos a conseguir tener un único logaritmo en cada miembro de la ecuación? Utilizando la
  definición y las propiedades de los logaritmos.\\

  Vamos a ver esto con \textbf{unos ejemplos}.
  \begin{questions}
  \question Resuelve la ecuación $\log (x^2 + 1) - \log (x^2 - 1) = \log \frac{13}{12}$
    \begin{solution}
      Por lo que hemos contado, tenemos que conseguir que haya un único logaritmo en cada miembro de
      la ecuación. En el caso del miembro derecho ya lo tenemos, y en el izquierdo tenemos una resta
      de logaritmos.\\
      Recordamos las propiedades (\ref{repaso_logaritmos}, página \pageref{repaso_logaritmos}) para
      ver en cual aparece una resta de
      logaritmos, y es en el logaritmo de un cociente.
      \[\log_a \frac{b}{c} = \log_a b - \log_a c \longrightarrow
        \log (x^2 + 1) - \log (x^2 - 1) = \log \frac{x^2 + 1}{x^2 - 1}\]
      Con lo que la ecuación del resultado se
      transforma en:
      \[\log \frac{x^2 + 1}{x^2 - 1} = \log \frac{13}{12}\]
      Ya ahora ya podemos quitar los logaritmos, con lo que nos queda una ecuación racional.
      \[\frac{x^2 + 1}{x^2 - 1} = \frac{13}{12}\]
      Que tras quitar denominadores se queda en:
      \[12(x^2 + 1) = 13(x^2 - 1)\]
      Y se resuelve:
      \[12x^2 + 12 = 13x^2 - 13\]
      \[-x^2 + 25 = 0\]
      \[x^2 = 25\]
      Y las posibles soluciones son $x = -5$ y $x = -5$. Las comprobamos:
      \begin{itemize}
      \item $x= 5$, la ecuación queda:
        \[\log (5^2 + 1) - \log (5^2 - 1) = \log \frac{13}{12}\]
        \[\log 26 - \log 24 = \log \frac{13}{12}\]
        \[\log \frac{26}{24} = \log \frac{13}{12}\]
        Y esto es cierto, $x = 5$ es solución.
      \item $x = -5$:
        \[\log ((-5)^2 + 1) - \log ((-5)^2 - 1) = \log \frac{13}{12}\]
        \[\log 26 - \log 24 = \log \frac{13}{12}\]
        \[\log \frac{26}{24} = \log \frac{13}{12}\]
        Que, lógicamente, también es cierto.
      \end{itemize}
      Por tanto la ecuación tiene de soluciones $x = 5$ y $x = -5$.
    \end{solution}
  \question Resuelve $\log (x + 9) = 2 + \log x$
    \begin{solution}
      Aquí tenemos un problema, que aparece un $2$ que no tiene logaritmo y no podemos aplicar las
      propiedades de los logaritmos para juntar todo el miembro derecho en un solo logaritmo.
      Entonces tenemos que convertir ese 2 en un logaritmo y para ello echamos mano de una de las
      consecuencias de la definición de logaritmo que hemos repasado:
      \[n = \log_a a^n\]
      Aplicándolo al caso que tenemos ahora:
      \[2 = \log 10^2\quad\quad\text{\small{(recuerda que es un logaritmo decimal)}}\]
      Con esto la ecuación queda:
      \[\log (x+9) = \log 10^2 + \log x\]
      Y ahora ya podemos aplicar la propiedad en la que aparece la suma de logaritmos:
      \[\log (x+9) = \log (100*x)\]
      Quitamos los logaritmos y resolvemos:
      \[x + 9 = 100x\]
      \[x = \frac{1}{11}\]
      Comprobamos la solución:
      \[\log (\frac{1}{11} + 9) = 2 + \log \frac{1}{11}\]
      \[\log \frac{100}{11} = 2 + \log \frac{1}{11}\]
      \[\log 100 + \log\frac{1}{11} = 2 + \log \frac{1}{11}\]
      Que es cierto, con lo que es una solución válida.
    \end{solution}
  \question Resolver $2\ln (x - 3) = \ln x - \ln 4$
    \begin{solution}
      En esta ecuación solo tenemos que buscar la propiedad que nos sirve en cada miembro de la
      ecuación. Para ello recordamos el repaso que hemos hecho (\ref{repaso_logaritmos}, pg.
      \pageref{repaso_logaritmos}) y tenemos que para el primer miembro podemos utilizar la propiedad
      del logaritmo de la potencia:
      \[\log_a b^n = n\log_a b\ \longrightarrow 2\ln (x-3) = \ln (x-3)^2\]
      Mientras que para el segundo miembro tenemos la propiedad del logaritmo del cociente:
      \[\log_a \frac{b}{c} = \log_a b - \log_a c \longrightarrow \ln x - \ln 4 = \ln \frac{x}{4}\]

      Y al aplicar esas transformaciones la ecuación queda:
      \[\ln (x-3)^2 = \ln \frac{x}{4}\]
      Con lo que podemos quitar logaritmos y resolver:
      \[(x-3)^2 = \frac{x}{4}\]
      \[x^2 - 6x + 9 = \frac{x}{4}\]
      \[4x^2 - 24 x + 36 = x\]
      \[4x^2 - 25x + 36 = 0\]
      Las posibles soluciones son $x = 4$ y $x = \frac{9}{4}$. Comprobamos:
      \begin{itemize}
      \item $x = 4$:
        \[2*\ln (4 - 3) = \ln 4 - \ln 4\]
        \[2*\ln 1 = 0\]
        Cierto, con lo que $x = 4$ es solución.
      \item $x = \frac{9}{4}$:
        \[2*\ln \left( \frac{9}{4} - 3 \right) = \ln \frac{9}{4} - \ln 4\]
        \[2\ln \left(-\frac{3}{4}\right) = \ln \frac{9}{4} - \ln 4\]
        Y como el argumento de un logaritmo no puede ser negativo, esta posible solución no vale.
      \end{itemize}
      Con lo que la solución es $x = 4$.
    \end{solution}
  \end{questions}

\item \large{\textbf{Tipo II.}}\\
  Hay otro tipo de ecuaciones logarítmicas, que son en las que todos los logaritmos son iguales.
  En ese caso lo que tenemos que hacer es un cambio de variable y luego utilizar la definición de
  logaritmo para obtener el valor de la incógnita.\\
  Esto no quita que tengamos que utilizar las propiedades de los logaritmos para poder resolverlas.

  Vamos con \textbf{unos ejemplos de este otro tipo}.
  \begin{questions}
  \question Resuelve $\log (x - 1) + 3 = \log \sqrt{x-1}$
    \begin{solution}
      En este caso tenemos una raíz dentro del logaritmo que hace que parezca complicado,
      pero recordando el repaso de logaritmos (\ref{repaso_logaritmos}, pg.
      \pageref{repaso_logaritmos}) encontramos que:
      \[\log_a \sqrt[n]{b} = \frac{\log_a b}{n} \longrightarrow \log \sqrt{x-1} =
        \frac{\log (x-1)}{2}\]
      Con lo que la ecuación se transforma en:
      \[\log ( x-1) + 3 = \frac{\log (x-1)}{2}\]
      Ahora todos los logaritmos son iguales, son todos $\log (x-1)$, hacemos el cambio
      $y = \log (x-1)$ y queda:
      \[y + 3 = \frac{y}{2}\]
      Que es una ecuación sencilla de resolver:
      \[2y + 6 = y\]
      \[y = -6\]
      Y ahora tenemos que deshacer el cambio teniendo en cuenta la definición de logaritmo:
      \[\log_a b = c \Leftrightarrow a^c = b \longrightarrow
        \log (x-1) = y \Leftrightarrow 10^y = x-1\]
      Con lo que si $y = -6$:
      \[x -1 = 10^{-6}\]
      \[x = 10^{-6} + 1\]
      \[x = 1.000001\]
    \end{solution}
  \question Resuelve la ecuación $\ln^2 (2x + 1) - 2 = \ln (2x + 1)^2 + 1$
    \begin{solution}
      En este caso tenemos dos términos que parecen iguales, pero no lo son.\\
      El término $\ln^2 (2x + 1)$ significa que primero se hace el logaritmo y luego se eleva
      al cuadrado mientras que $\ln (2x^2  + 1)$ significa que primero elevamos al cuadrado y
      luego hacemos el logaritmo. Entonces en el primero no podemos aplicar ninguna propiedad de
      los logaritmos mientras que en el segundo podemos aplicar el logaritmo de la potencia, de
      manera que queda:
      \[\ln^2 (2x + 1) - 2 = 2 \ln (2x + 1) + 1\]
      \[\left( \ln (2x + 1) \right)^2 - 2 = 2 \ln (2x + 1) + 1\]
      Y tenemos que todos los logaritmos que aparecen son iguales, con lo que hacemos un cambio
      de variable $y = \ln (2x + 1)$:
      \[y^2 - 2 = 2y +1\]
      \[y^2 - 2y - 3 = 0\]
      Las soluciones de esa ecuación son $y = -1$ e $y = 3$, pero para hallar la $x$ tenemos que
      deshacer el cambio ($y = \ln (2x + 1) \Leftrightarrow 2x + 1 = \e^y$):
      \begin{itemize}
      \item $y = -1 \longrightarrow 2x+ 1 = \e^{-1}$
        \[2x = \e^{-1} - 1\]
        \[x = \frac{\e^{-1}-1}{2}\]
      \item $y = 3 \longrightarrow 2x + 1 = \e^3$
        \[2x = \e^3 - 1\]
        \[x = \frac{\e^3 - 1}{2}\]
      \end{itemize}
      Con lo que la ecuación tiene dos soluciones, $x = \frac{\e^{-1}-1}{2}$ y $x = \frac{\e^3-1}{2}$.
    \end{solution}
  \end{questions}
\end{itemize}
\subsection{Ecuaciones exponenciales.}
Las ecuaciones exponenciales son \emph{aquellas en las que la incógnita está en el exponente}.\\
Para resolverlas vamos a tener que utilizar las propiedades de las potencias y en algún caso la definición de logaritmo.\\

En este tipo de ecuaciones la comprobación es obligatoria y todas las soluciones serían válidas,
pero nunca está de más realizar la comprobación para ver si lo hemos hecho bien.\\

Por lo general las ecuaciones exponenciales son complicadas de resolver, pero vamos a ver
un par de tipos que son relativamente sencillos
\begin{itemize}
\item \large{\textbf{Tipo I.}}\\
  El primer tipo que vamos a ver son las ecuaciones en las que no hay sumas entre las bases, solo
  actúan como  factores.\\
  En este caso tenemos que utilizar las propiedades de las potencias para hacer que nos quede una
  potencia en uno de los miembros y un valor en el otro para así poder obtener la incógnita
  haciendo el logaritmo. Algo así:
  \[k^x = n \longrightarrow x = \log_k n\]
  Solo tenemos que conseguir que todas las potencias acaben en $k^n$, y para eso no podemos
  utilizar sumas o restas y hay que saberse bien las propiedades de las potencias.\\

  Suponiendo que nos las sabemos podemos empezar a \textbf{hacer ejemplos}.
  \begin{questions}
  \question Resuelve la ecuación $2^x = 4*3^x$.
    \begin{solution}
      Tal y como hemos visto tenemos que tener una potencia a un lado y un valor al otro, con
      lo que $3^x$ tiene que pasar al lado izquierdo dividiendo:
      $\frac{2^x}{3^x} = 4$
      Ahora utilizamos la propiedad del cociente de potencias del mismo exponente, resultando:
      \[\left(\frac{2}{3}\right)^x = 4\]
      Con lo que ya tenemos algo de la forma $k^x = n$, solo nos queda hacer el logaritmo:
      \[x = \log_{\frac{2}{3}} 4\]
      Y para poder hacer esto en la calculadora con el logaritmo decimal utilizamos el cambio de
      base:
      \[x = \log_{\frac{2}{3}} 4 = \frac{\log 4}{\log \frac{2}{3}} \simeq -3.42\]
      Y comprobamos
      \[2^{-3.42} = 4*3^{-3.42}\]
      \[0.0934 = 4*0.0233\]
      El problema es que no nos sale exactamente lo mismo por el redondeo que hemos hecho al
      calcular el logaritmo, por eso vamos a dejar las soluciones indicadas. O sea, cuando
      lleguemos a
      \[x = \log_{\frac{2}{3}} 4\]
      paramos. De manera que la solución es $\log_{\frac{2}{3}} 4$, sin más.
    \end{solution}
  \question Resuelve $2^{x+1} = 5^{x-2}$
    \begin{solution}
      Si nos llevamos todas las potencias a un lado sin utilizar sumas o restas queda:
      \[\frac{2^{x+1}}{5^{x-2}} = 1\]
      Y se nos plantea el problema de que no tenemos la misma base ni el mismo exponente,
      con lo que no podemos aplicar ninguna propiedad de las potencias para juntarlas, así que
      vamos a utilizar las propiedades de las potencias para hacer que tengan el mismo exponente:
      \[\cfrac{2*2^x}{\cfrac{5^x}{5^2}} = 1\]
      \[5^2*2*\frac{2^x}{5^x} = 1\]
      \[\frac{2^x}{5^x} = \frac{1}{50}\]
      \[\frac{5^x}{2^x} = 50\]
      Y ahora procedemos como en el ejemplo anterior:
      \[\left(\frac{5}{2}\right)^x = 50\]
      \[x = \log_{\frac{5}{2}} 50\]
      Y daríamos por concluido el ejercicio.
    \end{solution}
  \question Resuelve $2^{x +2} = 8^{x-1}$
    \begin{solution}
      Esta es similar a la anterior, pero las bases están relacionadas de manera exponencial,
      de manera que vamos a poder obtener la solución de una manera más sencilla.\\
      Al ser igual que la anterior vamos a realizar el mismo procedimiento, llevando las potencias
      a un miembro y arreglándolo para que nos quede una única potencia:
      \[\frac{2^{x+2}}{8^{x-1}} = 1\]
      \[\cfrac{2^2*2^x}{\cfrac{8^x}{8}} = 1\]
      \[2^2*8*\frac{2^x}{8^x} = 1\]
      \[\frac{2^x}{8^x} = \frac{1}{32}\]
      Ahora, en vez de utilizar directamente la propiedad de la potencia de un cociente,
      vamos a aprovechar el hecho de que $8 = 2^3$ para utilizar la propiedad del producto
      de potencias de la misma base:
      \[\frac{2^x}{2^{3x}} = \frac{1}{32}\]
      \[2^{-2x} = \frac{1}{32}\]
      \[2^{2x} = 32\]
      Como $32 = 2^5$, en vez de hacer el logaritmo podemos hacer:
      \[2^{2x} = 2^5\]
      \[2x = 5\]
      \[x = \frac{5}{2}\]
      Y ya tendríamos resuelta la ecuación.
    \end{solution}
  \question Resolver $5^{x+1} + 5^x + 5^{x-1} = 775$.
    \begin{solution}
      En un principio puede parecer que esta ecuación no encaja con la definición que hemos hecho
      para este tipo de ecuaciones, pero si aplicamos las propiedades para las potencias de la
      misma base tenemos que se convierte en:
      \[5*5^x + 5^x + \frac{5^x}{5} = 775\]
      Utilizamos el procedimiento para fracciones con denominadores:
      \[\frac{25*5^x}{5} + \frac{5*5^x}{5} + \frac{5^x}{5} = \frac{3\,875}{5}\]
      \[25*5^x + 5*5^x + 5^x = 3\,875\]
      Como nos queda que todas las potencias son $5^x$ podemos sumarlas y resolver:
      \[31*5^x = 3\,875\]
      \[5^x = 125\]
      Y al igual que en el anterior vamos a aprovechar que $125 = 5^3$ para no hacer el
      logaritmo.
      \[5^x = 5^x\]
      \[x = 3\]
    \end{solution}
  \end{questions}
\item \large{\textbf{Tipo II.}}\\
  En este caso lo que vamos a tener son ecuaciones en las que sí aparecen sumas de potencias, pero
  tienen la misma base y exponente o se pueden transformar en potencias de la misma base y
  exponente utilizando factorizaciones y las propiedades de las potencias..\\

  En este caso lo que vamos a hacer es un cambio de variable para eliminar las potencias y resolver
  como una ecuación no exponencial para obtener el valor de la incógnita utilizando el
  logaritmo.\\

  Una vez visto el procedimiento vamos a por los \textbf{ejemplos}.
  \begin{questions}
  \question Resuelve la ecuación $2^{2x} - 5*2^x + 4 = 0$.
    \begin{solution}
      Lo primero que vamos a hacer es utilizar la potencia de una potencia para ver las cosas de
      manera más sencilla:
      \[\left( 2^x \right)^2 - 5*2^x + 4 = 0\]
      Aquí ya vemos que todas las potencias son $2^x$, y esa es la potencia que vamos a utilizar
      para el cambio de variable, vamos a hacer $t = 2^x$ y queda:
      \[t^2 - 5*t +4 = 0\]
      Que es una ecuación de segundo grado cuyas soluciones son $t = 4$ y $t = 1$.\\
      Ahora tenemos que deshacer el cambio aplicando la definición de logaritmo: $2^x = t \longrightarrow x = \log_2 t$.
      \begin{itemize}
      \item $t = 1 \longrightarrow x = \log_2 1 = 0$
      \item $t = 4 \longrightarrow x = \log_2 4 = 2$
      \end{itemize}
      Con lo que las soluciones son $x = 0$ y $x = 2$
    \end{solution}
  \question Resuelve $3^{2(x+1)} -18 *3^x +9 = 0$
    \begin{solution}
      Primero desarrollamos el exponente de la primera potencia y aplicamos la propiedad del
      producto de potencias de la misma base, porque $3^{2x}*3^2 = 3^{2x+2}$
      \[3^{2x + 2} - 18*3^x + 9 = 0\]
      \[3^2*3^{2x} - 18*3^x + 9 = 0\]
      \[9*\left(3^x\right)^2 - 18*3^x + 9 = 0\]
      Y ya podemos proceder como en el anterior haciendo el cambio $t = 3^x$:
      \[9t^2 - 18t + 9 = 0\]
      \[t^2 - 2t + 1 = 0\quad\text{\small{(hemos simplificado)}}\]
      Cuya solución es $t = 1$ doble.
      Deshacemos el cambio y tenemos que la solución es $x = \log_3 1= 0$
    \end{solution}
  \question Resuelve la ecuación $2*3^{x+1} - 9^x = 9$
    \begin{solution}
      Ahora las potencias que tenemos no son de la misma base ni el mismo exponente, pero si
      tenemos en cuenta que $9 = 3^2$ y $3^{x+1} = 3*3^x$ la ecuación nos queda:
      \[2*3*3^x - \left(3^2\right)^x = 9\]
      Y esto lo podemos reorganizar quedando:
      \[3^{2x} - 6*3^x + 9 = 0\]
      Hacemos el cambio $t = 3^x$ y se transforma en:
      \[t^2 - 6t + 9 = 0\]
      Que tiene como solución $t=3$, hacemos el logaritmo y obtenemos $x = \log_3 3 = 1$ y
      ya está resuelta la ecuación.
    \end{solution}
  \question Resuelve la ecuación $3*\e^{3x} - 2*\e^x - \e^{-x} = 0$.
    \begin{solution}
      Ahora nos encontramos con que la base es la misma pero los exponentes son distintos,
      entonces vamos a utilizar las propiedades de las potencias para hacer que tengan misma
      base y exponente, con lo que queda:
      \[3*\left(\e^x\right)^3 - 2*\e^x - \frac{1}{\e^x} = 0\]
      Y ahora que todas las potencias son iguales podemos hacer el cambio $t = \e^x$:
      \[3t^3 - 2t - \frac{1}{t} = 0\]
      Hacemos denominador común y quitamos denominadores:
      \[3t^4 - 2t^2 - 1 = 0\]
      Como es una ecuación bicuadrada tenemos que hacer otro cambio, y vamos a hacer $v = t^2$:
      \[3v^2 - 2v - 1 = 0\]
      Que tiene de soluciones $v = 1$ y $v = -\frac{1}{3}$, y tenemos que deshacer dos cambios:\\

      Primero calculamos $t = \pm \sqrt{v}$ y obtenemos:
      \begin{itemize}
      \item $t_1 = \sqrt{1} = 1$
      \item $t_2 = -\sqrt{1} = -1$
      \item Para $v = -\frac{1}{3}$ no podemos hacer la raíz, con lo que solo tenemos $t_1$ y $t_2$.
      \end{itemize}

      Y ya podemos calcular $x = \ln t$ (como la base es $\e$ tenemos que hacer el logaritmo
      neperiano), con lo que:
      \begin{itemize}
      \item $x_1 = \ln 1 = 0$
      \item $x_2 = \ln (-1)$ que no se puede hacer.
      \end{itemize}
      Por tanto la solución de esta ecuación es $x=1$.\\

      La importancia de este ejercicio radica en el hecho de que hay que hacer dos cambios de
      variable, y eso obliga a tener todo escrito y bien ordenado para poder resolverla.
    \end{solution}
  \end{questions}
\end{itemize}

\section{Sistemas de ecuaciones II.}
En 2º de ESO se empiezan a ver los sistemas de ecuaciones y los métodos para resolverlos
(si necesitas recordarlo está en el apartado \ref{sistemasI}, página \pageref{sistemasI}).\\
En aquel momento se explicaron los sistemas lineales y los métodos de resolución (sustitución,
reducción e igualación).\\

En 3º de ESO no se vio nada nuevo de teoría (por eso no aparecen en estos apuntes), pero se
practicó la resolución de sistemas.\\

Y en este curso vamos a ver un poco lo que puede pasar con los sistemas lineales y nos iniciaremos
en los sistemas no lineales.

\subsection{Sistemas de ecuaciones lineales.}
Se dice que un sistema de ecuaciones es lineal cuando todas las incógnitas están en monomios de
primer grado y en cada monomio hay como mucho una incógnita.\\
No vamos a recordar cómo se resuelven, para eso solo tienes que ir a la página \pageref{sistemasI}.\\

Lo que vamos a hacer es ver qué cosas nos pueden pasar con un sistema lineal y el significado
geométrico que tienen estas situaciones, ya que una ecuación lineal con dos incógnitas es
una línea recta (recuerda lo que has visto de funciones).

\subsubsection{Tipos de sistemas lineales.}\label{tipos_sistemas_lineales}
A la hora de resolver un sistema lineal nos vamos a encontrar en alguna de las siguientes
situaciones:
\begin{itemize}
\item \large{\textbf{Sistema compatible.}}\\
  Es el tipo de sistema que tiene una única solución. El que se resuelve sin mayores problemas más
  allá de que sea más o menos complicado o más o menos largo.\\
  Vamos a ver un pequeño ejemplo:
  \[
    \begin{cases}
      2x - \frac{3x-y}{5}& = \frac{22}{5}\\[10pt]
      \frac{y}{3} + \frac{4x-3y}{4}& = \frac{31}{12}
    \end{cases}\]
  \begin{solution}
    Empezamos haciendo denominador común y quitando denominadores:
    \[
      \begin{cases}
        10x - 3x+y& = 22\\
        4y + 12x-9y& = 31
      \end{cases}\]
    Lo recolocamos:
    \[
      \begin{cases}
        7x + y &=22\\
        12x - 5y &= 31
      \end{cases}\]
    Y vamos a utilizar sustitución, ya que de la primera sale que $y = 22-7x$. Sustituimos
    en la segunda:
    \[12x - 110 + 35x = 31\]
    \[47x = 141\]
    Con lo que $x=3$ e $y = 22 - 7*3 = 1$.
  \end{solution}
  Dejando aparte el tema de las fracciones no hemos tenido ningún problema para resolverlo.\\

  Tenido en cuenta que cada ecuación representa una recta (recuerda de los temas de funciones
  de años anteriores que las rectas tienen como ecuación $y = mx +n$, es decir dos incógnitas) la
  solución representa el punto en el que se cortan las dos rectas.
\item \large{\textbf{Sistema compatible indeterminado.}}\\
  Es un tipo de sistema que tiene infinitas soluciones. Tenemos un sistema de este tipo
  cuando al despejar una incógnita nos queda cero en ambos miembros de la ecuación ($0 = 0$),
  lo cual quiere decir que cualquier valor de esa incógnita es solución (ya que cero por
  cualquier número resulta cero).\\
  Veamos otro pequeño ejemplo:
  \[
    \begin{cases}
      -2x + y & = -1\\
      4x - 2y &= 2
    \end{cases}\]
  \begin{solution}
    En este caso vamos a utilizar reducción, multiplicamos la primera ecuación por dos:
    \[\left\lbrace
        \begin{array}{llr}
          -4x + 2y & = &-2\\
          \phantom{-}4x - 2y &=&2\\
          \hline
          \phantom{-}0\phantom{x} + 0\phantom{y}&=&0
        \end{array}
      \right.\]
  \end{solution}
  Como vemos acaba saliendo la igualdad $0 = 0$.\\
  La interpretación geométrica de esto es que las dos ecuaciones representan a la misma recta y
  se ``cortan'' en sus infinitos puntos.
\item \large{\textbf{Sistema incompatible.}}\\
  Este tipo de sistema no tiene ninguna solución. Se caracteriza porque al despejar una de las
  incógnitas queda cero en el miembro de la incógnita y un valor distinto de cero en el otro
  miembro ($0=3$, por ejemplo) y eso es imposible.\\
  Vamos con un ejemplo como en los anteriores:
  \[
    \begin{cases}
      2x + 3y &=6\\
      4x + 6y &= 6
    \end{cases}\]
  \begin{solution}
    Hacemos reducción y queda:
    \[\left\lbrace
        \begin{array}{llr}
          4x + 6y & = &12\\
          4x + 6y &=&6\\
          \hline
          0\phantom{x} + 0\phantom{y}&=&6
        \end{array}
      \right.\]
  \end{solution}
  Nos ha quedado $0=6$ que no tiene ningún sentido. Por tanto no tiene solución.\\
  La interpretación geométrica de este sistema es que las ecuaciones corresponden a dos rectas
  paralelas, que no se cortan en ningún punto y por eso el sistema no tiene solución.
\end{itemize}
\subsection{Sistemas de ecuaciones no lineales.}
Los sistemas de ecuaciones no lineales son aquellos en los que las incógnitas pueden aparecer
con cualquier tipo de operación.\\

El problema que tenemos en este tipo de sistemas es que no tenemos varios métodos para elegir,
sino que tendremos que encontrar el adecuado para acabar teniendo una ecuación con una incógnita
y una vez obtenida una podremos obtener la otra. Además habrá veces que tendremos que hacer
transformaciones previas antes de ver qué método es el que mejor nos viene.\\
El método que más suele funcionar es el de sustitución.\\
Y debemos tener en cuenta a la hora de resolver estos sistemas vamos a tener que echar mano
de todas las técnicas que hemos aprendido hasta ahora, en especial el desarrollo de identidades
notables porque van a aparecer mucho.\\

Otro punto importante a la hora de resolver los sistemas no lineales es la agrupación de soluciones.
En este tipo de sistema es normal que haya varias soluciones pero no podemos mezclarlas como queramos, hay que agrupar la solución de cada incógnita con el valor correspondiente de la otra. Y para
hacer esto hay que ser muy pulcro con el orden en el que lo estamos resolviendo para saber qué
va con qué a la hora de mostrar las soluciones.

Con todo esto no se puede decir otra cosa que la mejor manera de \textbf{aprender a resolver}
este tipo de sistemas es \textbf{viendo ejemplos y practicando}, con lo que vamos a hacer unos
cuantos.
\begin{questions}
\question Resuelve el siguiente sistema:
  \[
    \begin{cases}
      x^2 + y^2 & = 290\quad (1)\\
      x+y &= 24\quad (2)
    \end{cases}
    \]
    \begin{solution}
      Vamos a analizar el sistema para ver qué método deberíamos utilizar y cómo.\\
      La primera ecuación tiene las dos incógnitas al cuadrado, de manera que si conseguimos
      sustituir una de las incógnitas es posible que nos quede una ecuación de segundo grado
      que sabemos resolver.\\
      En la segunda ecuación podemos sacar una sustitución sencilla, con lo que vamos a hacerlo:
      \[\text{De } (2):\quad x =  24 - y\]
      Y sustituimos:
      \[\text{Sustitución en } (1):\quad (24 - y)^2 + y^2 = 290\]
      Ahora solo es desarrollar la identidad notable y resolver.
      \[576 - 48y +y^2 +y^2 = 290\]
      \[2y^2 -48y + 286 = 0\]
      Simplificamos:
      \[y^2 -24y + 143 = 0\]
      \[y = \frac{24 \pm \sqrt{576 - 572}}{2}\]
      Con lo que las soluciones son $y_1 = 13$, $y_2 = 11$.

      Ahora prestamos especial cuidado al orden para no mezclar las soluciones:
      \begin{itemize}
      \item Para $y_1 = 13$ tenemos $x_1 = 24 - 13= 11$, con lo que una solución es $x = 11$ e
        $y=13$.
      \item para $y_2 = 11$ tenemos $x_2 = 24 - 11 = 13$. La segunda solución es $x = 13$, $y = 11$.
      \end{itemize}
      Con lo que, recapitulando, las soluciones del sistema son:
      \begin{itemize}
      \item $x=13$, $y=11$.
      \item $x=11$, $y=13$
      \end{itemize}
      Otra forma de escribirlas es en forma de punto, que luego nos será muy útil para geometría.
      \begin{itemize}
      \item $(x,y)=(13, 11)$.
      \item $(x,y)=(11, 13)$
      \end{itemize}
    \end{solution}
  \question Resuelve $
    \begin{cases}
      x^2 + xy + y^2 &= 19\\
      xy &= 6
    \end{cases}$
    \begin{solution}
      Como en el anterior, primero vamos a analizar el sistema para ir decidiendo qué hacer.\\
      En la primera ecuación tenemos algo con pinta complicada, pero vemos que hay un término
      ($xy$) que está en la segunda y podemos sustituirlo directamente, con lo que el sistema queda:
      \[
        \begin{cases}
          x^2 + 6 + y^2 &=19\\
          xy &=6
        \end{cases}\]
      \[
        \begin{cases}
          x^2 + y^2 &=13\\
          xy &=6
        \end{cases}\]
      Y este sistema se parece mucho al anterior, con lo que vamos a utilizar el mismo método.\\
      De la segunda ecuación obtenemos qué $y = \frac{6}{x}$, con lo que al sustituir en la primera
      queda:
      \[x^2 + \left(\frac{6}{x}\right)^2 = 13\]
      \[x^2 + \frac{36}{x^2} = 13\]
      Hacemos denominador común y quitamos denominadores (todo esto son técnicas que hemos visto):
      \[x^4 + 36 = 13x^2\]
      \[x^4 - 13x^2 + 36 = 0\]
      Resolvemos como una bicuadrada que es, haciendo el cambio $t=x^2$
      \[t^2 - 13t + 36 = 0\]
      Y las soluciones son $t_1 = 4$ y $t_2 = 9$.
      Con lo que las soluciones de $x$ son:
      \begin{itemize}
      \item $x_1 = 2$.
      \item $x_2 = -2$.
      \item $x_3 = 3$.
      \item $x_4 = -3$.
      \end{itemize}
      Y para cada una de ellas tenemos que calcular la $y$:
      \begin{itemize}
      \item Si $x = 2$, $y = \frac{6}{2} = 3$.
      \item Si $x = -2$, $y = \frac{6}{-2} = -3$.
      \item Si $x = 3$, $y = \frac{6}{2} = 2$.
      \item Si $x = -3$, $y = \frac{6}{2} = -2$.
      \end{itemize}
      Y esas son las cuatro soluciones que tiene este sistema.
    \end{solution}
  \question Resuelve el sistema: $
    \begin{cases}
      x^2+xy &= y +1\\
      x^2 - 2xy = y + 7
    \end{cases}$
    \begin{solution}
      Este caso es más complicado que los anteriores, de hecho vamos a tener que aplicar dos
      métodos: uno para obtener la sustitución y después sustitución para obtener las soluciones.\\
      Si nos llevamos las incógnitas al miembro izquierdo queda:
      \[
        \begin{cases}
          x^2 -y + xy &=1\\
          x^2 -y - 2xy &= 7
        \end{cases}
      \]
      y podemos observar como hay una parte que es igual en ambas ecuaciones, $x^2 - y$. Entonces
      la dejamos sola y podemos utilizar igualación:
      \[
        \begin{cases}
          x^2 -y &=1 - xy\\
          x^2 -y &= 7 +2xy
        \end{cases}
      \]
      Con lo que
      \[2xy + 7 = 1-xy\]
      \[3xy = -6\]
      \[xy = -2\]
      \[y = \frac{-2}{x}\]
      Y ya podemos utilizar sustitución, con dos sustituciones porque tenemos la sustitución para
      $y$ y también para $xy$. Así que cogemos la primera ecuación y nos queda:
      \[x^2 +\frac{2}{x} -2 = 1\]
      Quitamos denominadores:
      \[x^3 -3x + 2 = 0\]
      Y es una ecuación cúbica para la cual no sabemos el método, así que vamos a utilizar
      Ruffini:
      \begin{center}
        \begin{tabular}{r|rrrr}
          &1&0&-3&2\\
          1&&1&1&-2\\
          \hline
          &1&1&-2&0
        \end{tabular}
      \end{center}
      Con lo que $x=1$ es una solución y nos queda la ecuación de 2º grado
      \[x^2 + x - 2 = 0\]
      que tiene como soluciones $x=1$ y $x = -2$.\\

      Pasamos a obtener los valores de la $y$ utilizando la sustitución $y = \frac{-2}{x}$ y
      nos queda que las soluciones son:
      \begin{itemize}
      \item Para $x = 1$, $y = -2$.
      \item Para $x = -2$, $y = 1$.
      \end{itemize}
      \large{\textbf{Otro método:}}\\
      Generalmente no suele haber varios métodos para resolver un sistema no lineal, pero
      cuando ocurre un método va a ser más sencillo que el otro.\\
      En el método anterior lo que hemos tenido que hacer es fijarnos en que hay una parte que
      es igual en las dos ecuaciones para hacer igualación, y ahora lo que vamos a hacer es
      obtener la sustitución por fuerza bruta.\\
      Tenemos el sistema:
      \[\begin{cases}
          x^2+xy &= y +1\\
          x^2 - 2xy = y + 7
        \end{cases}
      \]
      Y vamos a despejar la $y$ en la primera ecuación aplicando todas las técnicas que
      conocemos:
      \[xy - y = 1 - x^2\]
      \[y(x-1) = 1-x^2\]
      \[y = \frac{1-x^2}{x-1}\]
      Podemos simplificar esto (no siempre se puede) porque $1-x^2 = -(x^2 - 1) = -(x+1)(x-1)$,
      con lo que:
      \[y = -\frac{(x+1)(x-1)}{x-1} = -(x+1) = -x - 1\]
      Y ahora sustituimos en la segunda ecuación:
      \[x^2 - 2x*(-x -1) = -x-1+7\]
      \[x^2 +2x^2 + 2x = -x + 6\]
      \[3x^2 + 3x - 6 = 0\]
      \[x^2 + x - 2 = 0\]
      Que tiene de soluciones $x=1$ y $x= -2$, con lo que podemos obtener el valor de la otra
      incógnita y obtendríamos las mismas soluciones que de la anterior manera.
    \end{solution}
  \question Resuelve el sistema $
    \begin{cases}
      x-2y &= 1\\
      \sqrt{x+y} - \sqrt{x-y} = 2
    \end{cases}
    $.
    \begin{solution}
      En este caso está claro que el método es sustitución ya que en la primera ecuación obtenemos $x =2y + 1$,
      con lo que sustituimos en la segunda y nos queda:
      \[\sqrt{2y + 1 + y}- \sqrt{2y + 1 - y} = 2\]
      \[\sqrt{3y + 1} - \sqrt{y + 1} = 2\]
      Que resolvemos como una ecuación con dos raíces.
      \[\sqrt{3y + 1} = 2+\sqrt{y + 1}\]
      \[3y + 1 = 4 + y + 1 + 4\sqrt{y + 1}\]
      \[2y -4 = 4\sqrt{y + 1}\]
      \[y - 2 = 2\sqrt{y + 1}\]
      \[y^2 - 4y + 4 = 4(y+1)\]
      \[y^2 - 8y = 0\]
      Que tiene dos soluciones: $y= 0$, $y = 8$.
      Y obtenemos la $x$:
      \begin{itemize}
      \item $y=0 \longrightarrow x =1$
      \item $x = 8 \longrightarrow x = 17$
      \end{itemize}
      \textbf{Pero al haber una raíz} esas son las posibles soluciones, ahora tenemos que comprobarlas para ver si
      valen o no. Y al ser un sistema hay que comprobarlo con las dos ecuaciones:
      \begin{itemize}
      \item Para $x=1$, $y=0$: $
        \begin{cases}
          1 = 2*0 + 1& \text{(Es cierto. Se cumple)}\\
          \sqrt{1 + 0} - \sqrt{1 - 0} = 2& (\text{No es cierto, sale } 0=2)
        \end{cases}$\\
        Con lo que esta solución no es válida.
      \item Para $x = 17$, $y = 8$: $
        \begin{cases}
          17 = 2*8 + 1& \text{(Es cierto. Se cumple)}\\
          \sqrt{17 + 8} - \sqrt{17 - 8} = 2& (\text{También es cierto})
        \end{cases}$\\
        Esta sí es una solución válida.
      \end{itemize}
      Por lo tanto la única solución de este sistema es $x = 17$, $y =8$.
    \end{solution}
  \question Resolver el sistema $
    \begin{cases}
      \frac{1}{x} + \frac{1}{y} &= 1 - \frac{1}{xy}\\
      xy &=6
    \end{cases}
    $
    \begin{solution}
      En este caso podemos hacer denominador común en la primera ecuación, pero vamos a empezar haciendo la sustitución
      porque de la segunda vamos a obtener $x=\frac{6}{y}$, que es lo mismo que decir $\frac{1}{x} =\frac{y}{6}$.
      Y además podemos sustituir $\frac{1}{xy} = \frac{1}{6}$. Con todo esto la primera ecuación queda:
      \[\frac{y}{6} + \frac{1}{y} = 1 - \frac{1}{6}\]
      Resolvemos la ecuación racional teniendo en cuenta que el denominador común es $6y$:
      \[y^2 + 6 = 6y -y\]
      \[y^2 - 5y + 6 = 0\]
      Cuyas soluciones son $y=2$, $y=3$. Con lo que las soluciones del sistema son:
      \begin{itemize}
      \item $x=3$, $y=2$.
      \item $x=2$, $y=3$.
      \end{itemize}
    \end{solution}
  \question Resolver $
    \begin{cases}
      \log (x^2+ y) - \log(x-2y) &= 1\\
      5^{x+1} &= 25^{y+1}
    \end{cases}$
    \begin{solution}
      A simple vista este sistema parece complicadísimo, así que lo que tenemos que hacer es fijarnos muy bien en todo
      lo que hay.\\
      En la primera ecuación tenemos dos logaritmos que están restando, con lo que podemos transformarla en el
      logaritmo de un cociente. Pero con eso conseguimos poco y, aunque a lo mejor nos resulte más cómodo
      visualmente, tampoco nos va a aportar mucho.\\
      En la segunda ecuación tenemos dos potencias que tienen que ser iguales. Si tuviesen las bases iguales los
      exponentes tendrían que ser iguales, y eso lo podemos hacer ya que $25=5^2$, con lo que:
      \[\begin{cases}
      \log (x^2+ y) - \log(x-2y) &= 1\\
      5^{x+1} &= 5^{2y+2}
        \end{cases}\]
      \[\begin{cases}
      \log (x^2+ y) - \log(x-2y) &= 1\\
      x+1 &= 2y + 2
        \end{cases}\]
      \[\begin{cases}
      \log (x^2+ y) - \log(x-2y) &= 1\\
      x &= 2y +1
        \end{cases}\]
      Y ya podemos sustituir en la primera:
      \[\log \left((2y+1)^2 + y\right) - \log (2y + 1 - 2y) = 1\]
      \[\log (4y^2 + 5y + 1 ) - \log 1 = 1\]
      Teniendo en cuenta que $\log 1 = 0$ y que $\log 10 = 1$, nos queda:
      \[\log (4y^2 + 5y + 1 ) = \log 10\]
      Y para que dos logaritmos de la misma base sean iguales los argumentos tienen que ser iguales.
      Quitamos logaritmos, agrupamos y queda:
      \[4y^2 + 5y - 9 = 0\]
      Que tiene de soluciones $y = 1$, $y =-\frac{9}{4}$. Y las soluciones son:
      \begin{itemize}
      \item $x = 3$, $y=1$
      \item $x= -\frac{7}{2}$, $y=-\frac{9}{4}$
      \end{itemize}
      \textbf{Pero al tener logaritmos} hay que comprobarlas:
      \begin{itemize}
      \item Para $x = 3$, $y=1$:
        $
        \begin{cases}
          \log (3^2+ 1) - \log(3-2) = 1 & \text{\small{(Sale $ \log 10 - \log 1 = 1
                                          $ que es cierto)}}\\
          5^{3+1} = 5^{2+2} & (\text{\small{Esto también es cierto}})
        \end{cases}$ \\
        Con lo que esta solución es válida.
      \item Para $x= -\frac{7}{2}$, $y=-\frac{9}{4}$:
        $
        \begin{cases}
          \log \left(\left(-\frac{7}{2}\right)^2-\frac{9}{4}\right) - \log \left( -\frac{7}{2} + \frac{18}{4} \right) = 1 & \text{\small{(Que es cierto)}}\\
          5^{3+1} = 5^{2+2} & \text{\small{(Esto también es cierto)}}
        \end{cases}$\\
        Y ésta también es válida.
      \end{itemize}
      Por tanto las soluciones del sistema son:
      \begin{itemize}
      \item $x = 3$, $y=1$ (ó $(3, 1)$).
      \item $x= -\frac{7}{2}$, $y=-\frac{9}{4}$ $\left(\text{ó }\left(-\frac{7}{2}, -\frac{9}{4}\right)\right)$
      \end{itemize}
    \end{solution}
  \question Resolver $
    \begin{cases}
      2^x - 3^{y-1} &=-5\\
      2^{2x} - 3^y &=-11
    \end{cases}$.
    \begin{solution}
      A este sistema le ocurre lo mismo que al anterior que la primera impresión asusta. Pero vamos
      reescribirlo aplicando las propiedades de las potencias con la intención de que nos queden
      las mismas potencias en ambas ecuaciones:
      \[
        \begin{cases}
          2^x - \frac{3^y}{3} &=-5\\
          \left(2^x\right)^2 - 3^y &=-11
        \end{cases}
      \]
      Y ahora solo tenemos que hacer el cambio $2^x = t$, $3^x = v$:
      \[
        \begin{cases}
          t - \frac{v}{3} &=-5\\
          t^2 - v &=-11
        \end{cases}
      \]
      Hacemos denominador común en la primera y queda:
      \[
        \begin{cases}
          3t - v &=-15\\
          t^2 - v &=-11
        \end{cases}
      \]
      A la segunda la restamos la primera:
      \[t^2 - 3t = 4\]
      Que tiene de soluciones $t = -1$ y $t= 4$. Como $x = \log_2 t$ podemos descartar $t = -1$,
      ya que el argumento de un logaritmo no puede ser negativo.\\
      Con la solución $t = 4$ obtenemos que $v = 27$, con lo que deshaciendo el cambio:
      \begin{itemize}
      \item $x = \log_2 4= 2$
      \item $y = \log_3 27 = 3$
      \end{itemize}
      La solución del sistema es $x=2$, $y = 3$.
    \end{solution}
  \question Resuelve el sistema $
    \begin{cases}
      3x + \frac{y}{2} &= 15\\
      \frac{2}{x} + \frac{3}{y} &=1
    \end{cases}$
    \begin{solution}
      Se ve claramente que hay que hacer sustitución.\\
      
      De la primera ecuación se obtiene $y = 30 - 6x$, y al sustituir en la segunda:
      \[\frac{2}{x} + \frac{3}{30 - 6x} = 1\]
      \[\frac{2}{x} + \frac{1}{10-2x} = 1\quad \text{\small{(simplificando)}}\]
      El denominador común es $x*(10 - 2x)$;
      \[\frac{20 - 4x}{x(10-2x)} + \frac{x}{x(10-2x)} = \frac{x(10-2x)}{x(10-2x)}\]
      Y nos queda la ecuación
      \[20 - 4x + x = x(10 - 2x)\]
      Que al desarrollar y agrupar se acaba resultando ser una ecuación de 2º grado cuyas
      soluciones son $x=4$ y $x = \frac{5}{2}$, y calculando la $y$ tenemos que las soluciones son:
      \begin{itemize}
      \item $x = 4$, $y = 6$.
      \item $x = \frac{5}{2}$, $y = 15$.
      \end{itemize}
    \end{solution}
  \end{questions}
  Resumiendo, para resolver sistemas de ecuaciones no lineales hay que buscar el método que nos
  permita obtener una ecuación de una incógnita que sepamos resolver, para lo cual hay que
  echar mano de todas las técnicas que conocemos. Y una vez obtenida una incógnita no es difícil
  obtener la otra.\\
  Y recuerda que es obligatorio comprobar en los casos en los que haya
  \begin{itemize}
  \item Raíces.
  \item Logaritmos.
  \item Fracciones algebraicas.
  \end{itemize}
  Y esto es cuando tengamos alguna de las operaciones anteriores en la ecuación o al deshacer algún
  cambio de variable.
  \section{Inecuaciones y sistemas de inecuaciones.}
  Hasta ahora \textbf{hemos visto cómo resolver ecuaciones}, que es \textbf{calcular los valores
    de las incógnitas para que se verifiquen una serie de igualdades entre expresiones}.\\

  Pero ¿qué pasa \textbf{si lo que queremos es que se verifiquen desigualdades}, que una
  expresión sea mayor o menor que otra?\\
  \textbf{Por ejemplo}, nos piden que calculemos para que valores de $b$ la ecuación $x^2 - bx +
  2b = 0$ tiene solución. Sabemos que para que la ecuación tenga solución el discriminante (si no
  te acuerdas de lo que es repasa el punto \ref{discriminante} en la página \pageref{discriminante})
  tiene que ser positivo o cero, lo que en este caso se convierte en:
  \[\boldsymbol{x^2 - 8b \geq 0}\]
  Con lo que ya no queremos que sea igual a cero, sino que sea mayor que cero. \textbf{Eso es
    una inecuación}.\\

  Así que vamos a por ellas y empezaremos por algunas reglas que hay que manejar para operar
  cuando hay desigualdades.
  \subsection{Propiedades de las inecuaciones.}\label{propiedades_inecuaciones}
  Para ver claramente porqué estas propiedades son como son en lugar de expresiones algebraicas
  vamos a utilizar expresiones numéricas, y así es más fácil verificar la validez de lo que
  contamos.
  \begin{enumerate}
  \item \emph{Si en una inecuación sumamos, o restamos, el mismo valor en ambos términos
      la desigualdad se conserva}.\\
    Vamos a partir de la desigualdad $ 3 +2 > 4$.
    \begin{itemize}
    \item Si sumamos 5 en ambos miembros se sigue verificando: $5 + 3 + 2 > 5 + 4$.
    \item Si restamos 5 en ambos miembros se sigue verificando: $3 + 2 - 5 > 4 - 5$.
    \end{itemize}
    Esto es equivalente a decir que si cambiamos de miembro un término haciendo la operación
    contraria, la desigualdad se conserva.
    \[3 + 2 > 4\]
    \[3 > 4 -2 \]
  \item \emph{Si en una inecuación multiplicamos o dividimos ambos miembros por un valor
    positivo (no cero) la desigualdad se conserva}.\\
    Vamos a utilizar la misma desigualdad: $3+2 > 4$.
    \begin{itemize}
    \item Si multiplicamos ambos miembros por 6 se conserva: $6*(3+2) > 6*4$.
    \item Si dividimos ambos miembros entre 2 también se conserva: $\frac{3+2}{2} > \frac{4}{2}$
    \end{itemize}
    Esto es equivalente a lo que hacemos al despejar una incógnita. Por ejemplo:
    \[3*2 > 4 \longrightarrow 2>\frac{4}{3}\]
    \[3x > 4 \longrightarrow x > \frac{4}{3}\]
    
    Tenemos que darnos cuenta de que esto es realmente lo que hacemos cuando quitamos los denominadores en una
    ecuación en la que hemos hecho denominador común. Con lo que \textbf{cuando quitamos denominadores en una
      inecuación la desigualdad se conserva}.
    
  \item \textbf{\emph{Si en una inecuación multiplicamos o dividimos ambos miembros por un valor
        negativo (no cero) la desigualdad cambia de sentido.}}\\
    Esta la hemos marcado en negrita porque hay una cierta tendencia a olvidarla y eso hace que salga todo
    mal (al revés, concretamente).\\
    En este caso lo que sucede es esto:
    \[-3x \boldsymbol{>} 4 \longrightarrow x \boldsymbol{<} -\frac{4}{3}\]
    O esto:
    \[-\frac{x}{2} \boldsymbol{\leq} -3 \longrightarrow x \bold{\geq} 6\]
    En ambos casos la desigualdad ha cambiado de sentido cuando el negativo ha pasado al
    otro miembro haciendo la operación contraria.
  \item \emph{Si hacemos denominador común en toda la ecuación la desigualdad se conserva.}\\
    Esto tiene que ver con el significado de hacer denominador común, que solo cambia unas fracciones por otras
    equivalentes. En esta no tiene sentido que pongamos ejemplos numéricos.
  \end{enumerate}
  
  Además de estas propiedades \textbf{hay cosas que son ciertas en ecuaciones y no lo son en inecuaciones}.
  Como muchas veces tendemos a resolver inecuaciones como si fuesen ecuaciones, vamos a citar la que más errores
  causa a este nivel, y la vamos a citar para ecuaciones e inecuaciones viendo un par de ejemplos numéricos en cada
  caso:
  \begin{itemize}
  \item Si en una \textbf{ecuación} elevamos ambos miembros \textbf{al cuadrado} la ecuación \textbf{se conserva}.\\

    Unos ejemplos numéricos:$
    \begin{cases}
      2+3=5&\longrightarrow (2+3)^2 = 5^2 \longrightarrow 25 = 25\\
      1-2 = -1&\longrightarrow (1-2)^2 = (-1)^2 \longrightarrow 1 = 1
    \end{cases}$
  \item Si en una \textbf{inecuación} elevamos ambos miembros al \textbf{cuadrado la relación no se conserva siempre}.\\
    
    Unos ejemplos numéricos:$
    \begin{cases}
      6 > 5&\longrightarrow 6^2 > 5^2 \longrightarrow 36 > 25\quad \text{\small{(se ha conservado)}}\\
      -6 < 5&\longrightarrow (-6)^2 > 5^2 \longrightarrow 36 > 25\quad \text{\small{(NO se ha conservado)}}
    \end{cases}$
  \end{itemize}
  Y esto mismo pasa al tomar raíces y en otros casos. Por esto vamos a ver métodos distintos según la inecuación
  sea resoluble como una ecuación o no.
  \subsection{Soluciones de una inecuación.}
  En todas las ecuaciones que hemos visto hasta ahora las soluciones han sido uno o varios valores para cada variable.\\
  
  En inecuaciones no funciona así.\\
  Por ejemplo, una solución de la inecuación $x < 4$ podría ser $x = 3$, porque $3 <4$.\\
  O también $x = -1$, porque $-1 < 5$, ó $x = 0$, ó $x = 3.75$; \dots\\

  La manera de expresar las soluciones de una ecuación es con intervalos (salvo en algún caso especial que veremos),
  con lo que si no recordamos la teoría y la práctica de los intervalos tendremos que repasar antes de poder
  continuar con las inecuaciones.

  \subsection{Resolución de inecuaciones.}
  \subsubsection{Inecuaciones de primer grado o lineales.}
  Para resolver inecuaciones de primer grado tenemos dos métodos, uno que vamos a llamar ``propio'' y otro al que
  vamos a llamar ``genérico''. Le ponemos estos nombres porque son descriptivos, el propio es propio de las
  inecuaciones de primer grado mientras que el genérico sirve para todas las inecuaciones.\\
  Y sin más, vamos a por ellos.
  
  \paragraph{Método propio.}
  Consiste en resolver la inecuación de primer grado como si fuese una ecuación de primer grado pero teniendo
  en cuenta las propiedades vistas en
  el punto \ref{propiedades_inecuaciones} (página \pageref{propiedades_inecuaciones}).\\
  
  Pero lo mejor es ver esto con \textbf{unos ejemplos}.
  \begin{questions}
  \question Resuelve la inecuación $3x - 2 \geq x + 5$.
    \begin{solution}
      Como hemos indicado vamos a resolverla como si fuese una ecuación de primer grado:
      \begin{enumerate}
      \item Llevamos a un miembro lo que tenga $x$ y al otro lo que no.
        Tal y como hemos visto en las
        propiedades de las inecuaciones esto no cambia la desigualdad.
        \[3x - x \geq 5 + 2\]
        \[2x \geq 7\]
      \item Despejamos la $x$. Y por las propiedades vistas, al ser $2>0$ tampoco cambia nada.
        \[x \geq \frac{7}{2}\]
      \end{enumerate}
      Y con esto ya podríamos darla por resuelta.\\
      Si queremos dar la solución en forma de intervalo sería $x \in \left[\,\frac{7}{2}, \infty\right)$ (cerrado en
      $\frac{7}{2}$ porque no es desigualdad estricta).
    \end{solution}
  \question Resuelve la inecuación $2x - \frac{x + 1}{3} < 3x + \frac{1}{3}$.
    \begin{solution}
      En este caso tenemos una inecuación con fracciones, y al ser de primer grado la vamos a resolver como
      si fuese una ecuación de primer grado.\\
      \begin{enumerate}
      \item Hacemos denominador común y quitamos denominadores. Hemos visto que la desigualdad se conserva.
        \[6x - x - 1 < 9x + 1\]
      \item Hacemos la transposición de términos:
        \[6x -x - 9x < 1+1\]
        \[-4x < 2\]
      \item Y ahora al despejar vemos que va a cambiar de miembro un negativo y la operación es una división,
        con lo que tiene que cambiar el sentido de la desigualdad.
        \[x > -\frac{2}{4}\]
        \[x > - \frac{1}{2}\]
      \end{enumerate}
      Y esa sería la solución, que en forma de intervalo es $x \in \left(\,-\frac{1}{2}, \infty\right)$.
    \end{solution}
  \end{questions}
  Y este sería el método propio. La única dificultad es acordarse de que al despejar tenemos que fijarnos en
  si hay un menos que cambia de miembro y entonces tenemos que cambiar el sentido de la desigualdad.
  \paragraph{Método genérico.}
  Este método sirve para cualquier tipo de inecuación, y se basa en la interpretación geométrica de los distintos
  tipos de ecuaciones.\\

  En los temas de funciones de cursos anteriores se ha visto que la recta tiene una ecuación de la forma
  $y = mx + n$, donde $m,n \in \mathrm{R}$. Es decir, una ecuación de primer grado. Con lo que una ecuación
  de primer grado tiene la forma de una recta siempre.\\

  Y una recta puede ser de una de las dos maneras siguientes:
  \begin{center}
\begin{tikzpicture}
\begin{axis}[axis line style=gray, axis x line=center,
  axis y line=center, xmajorticks=false, ymajorticks=false] %Con xmajorticks=false, ymajorticks=false no pone marcas.
    \addplot[
        domain = -2:4,
        samples = 50,
        smooth,
        thick,
        %blue,
    ] (x, {x-1});
\end{axis}
\end{tikzpicture}
\quad\quad\quad
\begin{tikzpicture}
\begin{axis}[axis line style=gray, axis x line=center,
  axis y line=center, xmajorticks=false, ymajorticks=false] %Con xmajorticks=false, ymajorticks=false no pone marcas.
    \addplot[
        domain = -2:4,
        samples = 50,
        smooth,
        thick,
        %blue,
    ] (x, {1-x});
\end{axis}
\end{tikzpicture}
\end{center}
En ambos casos se observa que hay una parte por encima del eje horizontal (parte positiva) y una
parte por debajo (parte negativa) y ambas rectas solo tocan al eje horizontal en un punto.\\
Y la relación entre esas gráficas con las ecuaciones y las inecuaciones es la siguiente:
\begin{itemize}
\item El punto en el que corta al eje horizontal es la solución de la ecuación.
\item La parte que está por encima del eje horizontal corresponde con una de las
  desigualdades (< ó >).
\item La parte que está por debajo del eje horizontal corresponde con la otra solución.
\end{itemize}
Con lo que si resolvemos la ecuación ya tenemos el punto que divide lo que está por encima de
lo que está por debajo y solo tenemos que probar con un único valor cual de los dos resuelve la
ecuación.\\

Vamos a verlo con \textbf{unos ejemplos}.
\begin{questions}
\question Resuelve la inecuación $2(x-3) - x< 4x + 5$
  \begin{solution}
    Como hemos dicho empezamos por resolver la ecuación $2(x-3) -x = 4x + 5$:
    \[2x - 6 - x = 4x +5\]
    \[-3x = 11\]
    \[x = -\frac{11}{3}\]

    Ese es el punto en el que corta al eje, de manera que nos queda que la solución tiene que
    estar en el intervalo $\left(-\infty, -\frac{11}{3}\right)$ o en el intervalo
    $\left(\,-\frac{11}{3}, \infty\right)$. Y para saber cual de los dos es, cogemos un valor
    de cada uno y lo probamos con la inecuación a ver si lo que sale es cierto o falso.
    \begin{itemize}
    \item De $\left(-\infty, -\frac{11}{3}\right)$ probamos con $x = -5$:
      \testineq{2(-5-3) - (-5) < 4*(-5) + 5}
      \testineq{-16 + 5 < -20 + 5}
      \testineq{-11 < -15}
      Esto no es cierto, con lo que el intervalo $\left(-\infty, -\frac{11}{3}\right)$ NO es
      solución.
    \item De $\left(\,-\frac{11}{3}, \infty\right)$ cogemos el valor $x =0$ y probamos:
      \testineq{2*(0-3) - 0 < 4*0 + 5}
      \testineq{-6 < 5}
      Que sí es cierto, con lo que \textbf{la solución de la inecuación es
        $\boldsymbol{\left(\,-\frac{11}{3}, \infty\right)}$}.
    \end{itemize}
    Y no tenemos que probar con ningún valor más, porque la forma de la recta nos dice que si
    un valor de un intervalo está por encima (o por debajo) el resto van a estar en la misma
    posición ya que la única manera de cambiar de arriba a abajo es pasando por la solución de
    la ecuación, que está en uno de los extremos del intervalo.
  \end{solution}
\question Resuelve la inecuación $\frac{x}{2} - 2x \geq x+10$
  \begin{solution}
    Tal y como hemos dicho cambiamos la desigualdad por una igualdad y resolvemos:
    \[\frac{x}{2} - 2x \geq x+10\]
    \[\frac{x}{2} - 2x = x+10\]
    \[x - 4x = 2x + 10\]
    \[x = -4\]
    Y con esto tenemos que la solución tiene que estar en el intervalo $(-\infty, -4)$ ó en el
    intervalo $(-4, \infty)$, así que probamos:
    \begin{itemize}
    \item Del intervalo $(-\infty, -4)$ escogemos $x = -6$ y al sustituir en la inecuación queda:
      \testineq{\frac{-6}{2} - 2*(-6) \geq -6 + 10}
      \testineq{-3 + 12 \geq 4}
      Que es cierto.
    \end{itemize}
    Y no tenemos que probar con el otro intervalo, ya que la forma de la recta nos garantiza que
    uno de los dos es solución y el otro no.\\
    Es decir, en inecuaciones de primer grado solo tenemos que probar con un intervalo,
    y preferiblemente con el que contenga el cero ya que deja operaciones más sencillas.\\

    Por tanto la solución de la inecuación es $x\in (-\infty, -4]$, \textbf{cerrado} en $-4$
    \textbf{porque la desigualdad de la inecuación no es estricta}.
  \end{solution}
\question Resuelve la inecuación $x - \frac{x - 2}{3} \leq \frac{x}{2} - 2(x-1)$
  \begin{solution}
    Pues lo mismo que en las anteriores, cambiamos la igualdad por desigualdad y resolvemos:
    \[x - \frac{x - 2}{3} = \frac{x}{2} - 2(x-1)\]
    \[x - \frac{x - 2}{3} = \frac{x}{2} - 2x + 2\]
    \[6x - 2(x-2) = 3x -12x + 12\]
    \[13x = 8\]
    Con lo cual la solución es el intervalo $\left(-\infty, \frac{8}{13}\right)$ ó el intervalo
    $\left(\frac{8}{13},\infty\right)$.\\
    Probamos con $x=0$:
    \testineq{0 - \frac{0-2}{3} \leq \frac{0}{2} - 2*(0-1)}
    \testineq{\frac{2}{3} \leq 2}
    Que sí es cierto, con lo que la solución es el intervalo que contenga el cero, y en este caso
    es $x \in \left(-\infty, \frac{8}{13}\right]$.
  \end{solution}
\end{questions}
\subsubsection{Inecuaciones de segundo grado.}\label{inecuaciones_grado_2}
La resolución de inecuaciones de segundo grado es un poco más compleja que la de las de primer
grado, ya que tiene que ver con \textbf{la forma relacionada con un polinomio de
  segundo grado}.\\
Y sabemos que la función $y = ax^2 + bx + c$ \textbf{es una parábola}.

Así que vamos a recordar cómo es una parábola y sus características.
\paragraph{Repaso de la parábola.}
Recordemos que cuando tenemos una función del tipo $y = ax^2 + bx + c$ la gráfica es una parábola,
con los brazos hacia arriba o hacia abajo según el valor del coeficiente $a$.
\begin{center}
  \begin{tikzpicture}
    \begin{axis}[title={$a>0$}, axis line style=gray, axis x line=center,
      axis y line=center, xtick={-1,1}, xticklabels={$x_1$, $x_2$}, ymajorticks=false] %Con xmajorticks=false, ymajorticks=false no pone marcas.
      \addplot[
      domain = -2:2,
      samples = 50,
      smooth,
      thick,
      % blue,
      ] (x, {x^2-1});
    \end{axis}
  \end{tikzpicture}
  \quad\quad\quad
  \begin{tikzpicture}
    \begin{axis}[title={$a<0$}, axis line style=gray, axis x line=center,
      axis y line=center, xtick={-1,1}, xticklabels={$x_1$, $x_2$}, ymajorticks=false] %Con xmajorticks=false, ymajorticks=false no pone marcas.
      \addplot[
      domain = -2:2,
      samples = 50,
      smooth,
      thick,
      % blue,
      ] (x, {1-x^2});
    \end{axis}
  \end{tikzpicture}
\end{center}
Y aquí ya se empiezan a observar cosas relacionadas con lo que estamos haciendo.\\
Vemos que en cada una de las dos parábolas hay \textbf{una parte que está por encima} del eje
horizontal y \textbf{otra que está por debajo}, y lo que \textbf{separa} a cada parte son los
\textbf{puntos de corte} con el eje horizontal.\\
De manera que tenemos tres intervalos:
\begin{itemize}
\item $(-\infty, x_1)$
\item $(x_1, x_2)$
\item $(x_2, \infty)$
\end{itemize}
Y en cada intervalo la parábola estará por encima o por debajo del eje según la posición de la
parábola y el sentido de sus brazos.
Con lo que lo primero que tenemos que calcular son los intervalos y esto se hace resolviendo la
ecuación de segundo grado que se obtiene al cambiar la desigualdad por una igualdad.\\

\textbf{Antes de entrar en el método de resolución que se obtiene de todo esto, hay que recordar que
una ecuación de segundo grado puede tener dos soluciones, una o ninguna. En cada uno de estos
casos tendremos que analizar la situación geométrica y obtendremos las distintas posibilidades y
soluciones.\\
Vamos entonces a por los casos.}
\paragraph{La ecuación de segundo grado tiene dos soluciones.}
En este caso las soluciones de la ecuación nos dan tres intervalos como posibles soluciones, tal
y como se observa en las gráficas anteriores. Estos tres intervalos hemos visto que son:
\begin{itemize}
\item $(-\infty, x_1)$, que llamaremos intervalo izquierdo.
\item $(x_1, x_2)$, que llamaremos intervalo central.
\item $(x_2, \infty)$, que llamaremos intervalo derecho.
\end{itemize}

Y a la vista de la gráfica queda claro que los intervalos derecho e izquierdo van a tener siempre
el mismo comportamiento, mientras que el central va a tener el comportamiento opuesto.\\
Esto quiere decir que cuando la ecuación tiene dos soluciones la inecuación tiene estas dos posibilidades:
\begin{itemize}
\item La solución es la unión de los intervalos izquierdo y derecho: $x \in (\infty, x_1) \cup
  (x_2, \infty)$.\\
  
  Ó
  
\item La solución es el intervalo central: $x \in (x_1, x_2)$
\end{itemize}
\begin{small}\emph{(recordemos que si la desigualdad no es estricta los intervalos tienen que ser
    cerrados en las soluciones de la ecuación).}\end{small}\\

Entendiendo lo anterior, el mecanismo se puede resumir de la siguiente manera:
\begin{enumerate}
\item Resolvemos la ecuación que se obtiene al cambiar la desigualdad por una igualdad y
  comprobamos que se obtienen dos soluciones (si es otro caso hay que aplicar lo que corresponda).
\item Elegimos un valor sencillo (que no puede ser ninguna de las soluciones del paso anterior)
  y comprobamos si es solución de la inecuación original.
\item Dependiendo del resultado del paso anterior la solución será el intervalo central o la
  unión de los dos intervalos laterales, y habrá que abrirlo o cerrarlo según la inecuación sea
  estricta o no.
\end{enumerate}

Vamos a ver cómo se aplica esto con \textbf{unos cuantos ejemplos}:
\begin{questions}
\question Resuelve la inecuación $x^2- 3x > x$.
  \begin{solution}
    Tal y como hemos visto, resolvemos primero la ecuación:
    \[x^2 - 3x = x\]
    \[x^2 - 4x = 0\]
    Nos ha quedado una ecuación incompleta que tiene de soluciones $x_1=0$ y $x_2 = 4$, con lo
    que las posibles soluciones de la inecuación son:
    \begin{itemize}
    \item $x \in (-\infty, 0) \cup (4, \infty)$
    \item $x \in (0, 4)$.
    \end{itemize}
    Probamos con un valor, por ejemplo $x = 1$. Sustituimos en la inecuación y tenemos:
    \testineq{1^2 - 3*1 > 1}
    \testineq{-2 > 1}
    Que no es cierto.\\
    Y como $x = 1$ está en el intervalo central, la solución es la otra posibilidad:
    $x \in (-\infty, 0) \cup (4, \infty)$.\\
    \begin{small}
      \emph{(Hemos puesto que es abierto en $0$ y en $4$ porque la desigualdad de la inecuación
      es estricta).}
    \end{small}
  \end{solution}
\question Resuelve la inecuación $2x^2 - 3x \leq 4x -6$
  \begin{solution}
    Hacemos lo mismo que en el anterior: primero resolvemos la ecuación que se obtiene al
    cambiar la desigualdad por una igualdad.
    \[2x^2 - 3x = 4x - 6\]
    \[2x^2 - 7 x + 6 = 0\]
    Y las soluciones son $x = \frac{3}{2}$ y $x = 2$.
    Entonces la solución será intervalo $\left[\frac{3}{2}, 2\right]$ ó la unión
    $\left(-\infty, \frac{3}{2}\right] \cup [2, \infty)$.\\
    \begin{small}
      \emph{(Ahora los intervalos son cerrados en $\frac{3}{2}$ y $2$ porque la desigualdad
      de la inecuación NO es estricta).}
    \end{small}\\
    Y al igual que en el anterior, probamos con un valor en la inecuación original:
    por ejemplo $x = 1$:
    \[2*1^2 - 3*1 \leq 4*1 - 6\]
    \[2 - 3 \leq 4 -6\]
    Y esto no es cierto, con lo que el intervalo que contiene $x=1$ no es solución, que en este
    caso es la unión $\left(-\infty, \frac{3}{2}\right] \cup [2, \infty)$.\\
    Por lo tanto la solución de este ejercicio es $x \in \left[\frac{3}{2}, 2\right]$
  \end{solution}
\end{questions}
Vistos estos dos ejemplos vamos a por el siguiente caso.
\paragraph{La ecuación de 2º grado solo tiene una solución.}
O una solución doble, que es otra manera de decirlo.\\

En este caso tenemos una de las siguientes situaciones:
\begin{center}
  \begin{tikzpicture}
    \begin{axis}[title={$a>0$}, axis line style=gray, axis x line=center, ymin = -1,
      axis y line=center, xtick={1}, xticklabels={$x_{sol}$}, ymajorticks=false] %Con xmajorticks=false, ymajorticks=false no pone marcas.
      \addplot[
      domain = -2:4,
      samples = 50,
      smooth,
      thick,
      % blue,
      ] (x, {x^2-2*x +1});
    \end{axis}
  \end{tikzpicture}
  \quad\quad\quad
  \begin{tikzpicture}
    \begin{axis}[title={$a<0$}, axis line style=gray, axis x line=center, ymax = 1,
      axis y line=center, xtick={1}, xticklabels={$x_{sol}$}, ymajorticks=false] %Con xmajorticks=false, ymajorticks=false no pone marcas.
      \addplot[
      domain = -2:4,
      samples = 50,
      smooth,
      thick,
      % blue,
      ] (x, {-x^2+2*x - 1});
    \end{axis}
  \end{tikzpicture}
\end{center}

Con lo cual se nos plantean varias situaciones que tenemos que analizar por separado. Vamos a
empezar por el tipo de inecuación que tenemos:
\begin{itemize}
\item \textbf{La inecuación es estricta.} En este caso ya sabemos que la solución de la
  ecuación no puede ser solución de la inecuación (el intervalo va a ser abierto).\\
  Además tenemos el caso de que la parábola no cambia de signo con lo que:
  \begin{itemize}
  \item \emph{La inecuación no tiene solución.}
  \item \emph{La solución es cualquier número excepto la solución de la ecuación.} Esto lo
    escribiremos $ x \in \realset - \lbrace x_{sol}\rbrace$
  \end{itemize}
\item \textbf{La inecuación no es estricta.} En este caso la solución de la ecuación también
  es solución de la inecuación. Por tanto son los mismos casos que antes pero añadiendo la solución
  de la ecuación, con lo que las posibilidades son:
  \begin{itemize}
  \item \emph{La solución es $x= x_{sol}$.}
  \item \emph{Cualquier número es solución ($x \in \realset$).}
  \end{itemize}
\end{itemize}

De esta manera el mecanismo queda así:
\begin{enumerate}
\item Resolvemos la ecuación y comprobamos que tiene solo una solución.
\item Nos quedamos con los dos casos posibles según sea estricta o no.
\item Comprobamos la inecuación original con un valor distinto de la solución de la ecuación, y
  según el resultado la solución sera uno de los dos casos del paso dos.
\end{enumerate}

Y la mejor manera de entenderlo es viendo \textbf{unos ejemplos}:
\begin{questions}
\question Resuelve la inecuación $x^2 -2x + 1 > 0$.
  \begin{solution}
    Resolvemos la ecuación $x^2 - 2x +1 = 0$ y vemos que solo tiene una solución, que es $x = 1$.\\
    Como la inecuación es estricta los casos a tener en cuenta son:
    \begin{itemize}
    \item No tiene solución.
    \item La solución es $x \in \realset - {1}$
    \end{itemize}
    Elegimos un valor sencillo, por ejemplo $x=0$ y comprobamos si resuelve la ecuación:
    \[0^2 - 2*0 + 1 > 0\]
    \[1 > 0\]
    Es cierto, con lo que $\boldsymbol{x = 1}$ \textbf{es solución de la inecuación} y de las
    dos posibilidades que teníamos la única que es compatible con lo que acabamos de obtener es que
    \textbf{la solución de la inecuación es $\boldsymbol{x \in \realset - \{1\}}$}
  \end{solution}
\question Resuelve la inecuación $x^2 -3x \leq 3x - 9$.
  \begin{solution}
    Resolvemos la ecuación $x^2 - 3x = 3x - 9$, y tiene como única solución $x = 3$.\\
    Como la desigualdad NO es estricta los casos posibles son:
    \begin{itemize}
    \item La solución es $x = 3$.
    \item Cualquier valor es solución ($x \in \realset$).
    \end{itemize}
    Elegimos un valor cualquiera (distinto de $3$), por ejemplo $x = 0$,
    y lo comprobamos con la inecuación original.
    \[0^2 - 3*0 \leq 3*0 - 9\]
    \[0 \leq -9\]
    Que no es cierto. Con lo que la única posibilidad es que \textbf{la solución de la inecuación es
    $\boldsymbol{x = 3}$}.
  \end{solution}
\question Resuelve la inecuación $2x^2 - 3 x < 5x -8$.
  \begin{solution}
    Resolvemos la inecuación:
    \[2x^2 - 3x = 5x - 8\]
    \[2x^2 - 8 x - 8 = 0\]
    \[x^2 - 4x + 4 = 0\]
    Que tiene de solución $x = 2$.\\
    Con la inecuación es estricta las posibilidades son:
    \begin{itemize}
    \item La inecuación no tiene solución.
    \item La solución es $x \in \realset - \{2\}$.
    \end{itemize}
    Probamos con $x = 0$ y obtenemos:
    \[2*0^2 - 3*0 < 5*0 - 8\]
    \[0 < -8\]
    Y esto no es cierto, con lo que la posibilidad que nos queda es que \textbf{la inecuación no
      tiene solución}.
  \end{solution}
\end{questions}

Y con esto ya podemos ir al último caso.
\paragraph{La ecuación de 2º grado no tiene solución.}
En este caso las posibles situaciones geométricas son:
\begin{center}
  \begin{tikzpicture}
    \begin{axis}[title={$a>0$}, axis line style=gray, axis x line=center,
      axis y line=center, ymin=-1,xmajorticks=false, ymajorticks=false] %Con xmajorticks=false, ymajorticks=false no pone marcas.
      \addplot[
      domain = -2:3,
      samples = 50,
      smooth,
      thick,
      % blue,
      ] (x, {x^2+1});
    \end{axis}
  \end{tikzpicture}
  \quad\quad\quad
  \begin{tikzpicture}
    \begin{axis}[title={$a<0$}, axis line style=gray, axis x line=center,
      axis y line=center, ymax=1, xmajorticks=false, ymajorticks=false] %Con xmajorticks=false, ymajorticks=false no pone marcas.
      \addplot[
      domain = -2:3,
      samples = 50,
      smooth,
      thick,
      % blue,
      ] (x, {-1-x^2});
    \end{axis}
  \end{tikzpicture}
\end{center}
Es decir \textbf{la parábola no cambia nunca de signo ni toca el cero}.\\

Por esto \textbf{las dos únicas posibilidades son}:
\begin{itemize}
\item \textbf{La inecuación no tiene solución.}
\item \textbf{La solución es $\boldsymbol{x \in \realset}$.}
\end{itemize}

Con lo que los pasos que tenemos que seguir para resolver este caso son:
\begin{enumerate}
\item Resolvemos la ecuación y comprobamos que no tiene solución.
\item Comprobamos la ecuación con un valor cualquiera y dependiendo del resultado la soluciones
  será una de las dos opciones indicadas.
\end{enumerate}

Veamos \textbf{un par de ejemplos}:
\begin{questions}
\question Resolver la inecuación $x^2 - 4x \geq x -8$.
  \begin{solution}
    Como en todas las inecuaciones de segundo grado, empezamos por resolver la ecuación.
    \[x^2 - 4x = x - 8\]
    \[x^2- 5x + 8 = 0\]
    Y esta ecuación no tiene solución ya que su discriminante es negativo:
    \[5^2 - 4*1*8 = 25 -32 = -7\]
    Entonces las posibilidades son que no tenga solución o que la solución sea todos los
    reales. Probamos con $x = 0$:
    \[0^2 - 4*0 \geq 0 -8\]
    \[0 \geq -8\]
    Y esto es cierto, con lo que si $x=0$ es solución de la inecuación no queda otra opción que
    la solución sea $x \in \realset$
  \end{solution}
\question Resuelve la inecuación $x^2 + x < x - 1$.
  \begin{solution}
    Al igual que en todas, resolvemos la ecuación:
    \[x^2 + x = x -1\]
    \[x^2+1 = 0\]
    Y esa ecuación no tiene solución. Recordamos que en este caso las posibilidades son que la
    inecuación no tenga solución o que sea $x \in \realset$.\\
    Probamos con $x = 0$, y obtenemos:
    \[0 ^2 + 0 < 0 -1\]
    \[0 < -1\]
    Que no es cierto, con lo que la inecuación no tiene solución.
  \end{solution}
\end{questions}

\paragraph{Resumen de la resolución de inecuaciones de segundo grado.}
Con todo lo anterior se puede hacer un pequeño resumen, pero para poder utilizarlo hay que
aprenderse cada uno de los tres casos anteriores con todas sus posibilidades.\\

Y el mecanismo es simple:
\begin{enumerate}
\item Convertir la inecuación en ecuación y resolverla.
\item Según el número de soluciones se aplica el método correspondiente.
\end{enumerate}

Y no hay más que hacer.

\subsection{Sistemas de inecuaciones y su resolución.}
Al igual que ocurría con los sistemas de ecuaciones, un sistema de inecuaciones es un conjunto
de inecuaciones que se tienen que verificar a la vez.\\

En el caso de los sistemas de ecuaciones hemos estudiado los que tienen dos ecuaciones con dos
incógnitas y, en general, se suelen estudiar sistemas con el mismo número de ecuaciones que de
incógnitas.\\

En los sistemas de inecuaciones no ocurre así, es bastante habitual resolver sistemas de
inecuaciones que tienen más inecuaciones que incógnitas.\\
Así que vamos a empezar con los de una incógnita y luego iremos a por los de dos.
\subsubsection{Sistemas de inecuaciones con una incógnita.}
Tal y como hemos dicho varias veces, en un sistema se tienen que verificar todas las ecuaciones o
inecuaciones a la vez.\\

Como las soluciones de las inecuaciones suelen ser intervalos, la solución del sistema será donde
coincidan estos intervalos. Es decir la intersección de las soluciones de todas las inecuaciones
que forman el sistema.\\
Habrá veces que la intersección de estas soluciones sea un intervalo, habrá veces que sea un
solo número (es raro, pero puede pasar) y habrá veces que no coincidan y no habrá solución.\\

Y como hemos hecho en los casos anteriores vamos a ver como se resuelven con
\textbf{unos ejemplos}:
\begin{questions}
\question Resuelve el siguiente sistema de inecuaciones:
  $
  \begin{cases}
    2x > 5 - 2x\\
    3x - 1 \leq x + 5
  \end{cases}$
  \begin{solution}
    Tal y como hemos indicado la solución del sistema es la intersección de las soluciones
    de todas las inecuaciones, con lo que vamos a resolverlas una a una:
    \begin{enumerate}
    \item \textbf{La primera} inecuación se resuelve:
      \[2x > 5 - 2x\]
      \[4x > 5\]
      \[x > \frac{5}{4}\]
      Su solución es $x \in \left(\frac{5}{4}, \infty\right)$.
    \item \textbf{La segunda} inecuación:
      \[3x - 2 \leq x + 5\]
      \[2x \leq 7\]
      \[x \leq \frac{7}{2}\]
      La solución de la segunda es $x \in \left(-\infty, \frac{7}{2}\right]$.
    \end{enumerate}
    Y para calcular la intersección podemos hacerlo dibujando, que es la manera más sencilla de
    ver donde coinciden:
    \begin{center}
      \tikzmath{
        \xini = -2;
        \x1 = 1.25;
        \x2 = 3.5;
        \xend = 5;
        \alt = .25;
        \sol1 = .5;
        \sol2 = 1;
        \radius = .15;
      }
      \begin{tikzpicture}
        

        \draw (\xini, 0) -- (\xend,0);
        \draw (\x1, \alt) -- (\x1, -\alt) node[below] {$\frac{5}{4}$};
        \draw (\x2, \alt) -- (\x2, -\alt) node[below] {$\frac{7}{2}$};
        
        \draw[-latex] (\x1, \sol1) -- (\xend, \sol1) node[right]{\scriptsize{(Solución 1ª)}};
        \draw[latex-] (\xini, \sol2) -- (\x2, \sol2);
        \draw[fill=white] (\x1, \sol1) circle (\radius);
        \fill (\x2, \sol2) circle (\radius) node[right]{\scriptsize{(Solución 2ª)}};
      \end{tikzpicture}
    \end{center}
    Aquí se ve que coinciden en todos los puntos que hay entre $\frac{5}{2}$ y $\frac{7}{2}$,
    teniendo en cuenta que $\frac{5}{2}$ no está en las dos soluciones y $\frac{7}{2}$ sí.\\
    Con lo que la solución del sistema es $x \in \left(\frac{5}{2}, \frac{7}{2} \right]$.
  \end{solution}
\question Resolver el siguiente sistema de inecuaciones:
  $
  \begin{cases}
    2x^2 - 3 \leq 6x+5\\
    2x > 3x - 2
  \end{cases}
  $
  \begin{solution}
    Resolvemos cada inecuación por separado.\\

    \textbf{La primera} es de segundo grado:
    \begin{enumerate}
    \item Resolvemos la ecuación $2x^2 - 3 = 6x+5$, y tiene dos soluciones: $x_1 = -1$ y $x_2= 4$.
    \item Aplicamos del método de las inecuaciones de segundo grado y obtenemos que la solución es
      $(-\infty, -1] \cup [4, \infty)$.
    \end{enumerate}

    \textbf{La segunda} es muy sencilla:
    \[2x - 3x > -2\]
    \[-x > -2\]
    \[x < 2\]

    Dibujamos las soluciones para calcular la intersección:
    \begin{center}
      \begin{tikzpicture}
        \tikzmath{
          \xini = -3;
          \x{1} = -1;
          \x{2} = 2;
          \x{3} = 4;
          \xend = 6;
          \alt = .25;
          \sol1 = .5;
          \sol2 = 1;
          \radius = .15;
        }

        \draw (\xini, 0) -- (\xend, 0);
        \foreach \i in {1,2,3}{
          \draw (\x{\i}, \alt) -- (\x{\i}, -\alt) node[below]{$\x{\i}$};
        }

        \draw[latex-] (\xini, \sol1) --(\x{1}, \sol1);
        \draw[fill=white] (\x{1}, \sol1) circle (\radius);
        \draw[-latex] (\x{3}, \sol1) -- (\xend, \sol1) node[right]{\scriptsize{(Solución 1ª)}} ;
        \fill (\x{3},\sol1) circle (\radius);
        
        \draw[latex-] (\xini, \sol2) -- (\x{2}, \sol2) node[right]{\scriptsize{(Solución 2ª)}};
        \draw[fill=white] (\x{2}, \sol2) circle (\radius);
      \end{tikzpicture}
    \end{center}
    Se ve que coinciden entre $-\infty$ y $-1$ pero el $-1$ no está en la solución de la primera,
    con lo que la solución del sistema es $x \in (-\infty, -1)$.
  \end{solution}
\question Resuelve el siguientes sistema de inecuaciones:
  $
  \begin{cases}
    2(x-1) + 3(x-4)< 0\\
    1-(6-2x) > x-2
  \end{cases}
  $
  \begin{solution}
    En este caso son dos inecuaciones de primer grado que se resuelven rápidamente.

    \textbf{La primera}:
    \[2x - 2+3x-12 < 0\]
    \[5x < 14\]
    \[x < \frac{14}{5}\]

    \textbf{La segunda}:
    \[1-6 + 2x > x -2\]
    \[x > 3\]

    Calculamos la intersección:
    \begin{center}
      \begin{tikzpicture}
        \tikzmath {
          \xini = -1;
          \xend = 5;
          \x{1} = 2;
          \x{2} = 3;
          \alt = .25;
          \sol1 = .5;
          \sol2 = 1;
          \radius = .15;
        }
        
        \draw (\xini, 0) -- (\xend, 0);
        \draw (\x{1}, \alt) -- (\x{1}, -\alt) node[below]{$\frac{14}{5}$};
        \draw (\x{2}, \alt) -- (\x{2}, -\alt) node[below]{$3$};

        \draw[-latex] (\x{1}, \sol1) -- (\xini,\sol1) node[left]{\scriptsize{(Solución 1ª)}};
        \draw[fill=white] (\x{1},\sol1) circle (\radius);

        \draw[-latex] (\x{2},\sol2) -- (\xend, \sol2) node[right]{\scriptsize{(Solución 2ª)}};
        \draw[fill=white] (\x{2},\sol2) circle (\radius);
      \end{tikzpicture}
    \end{center}

    En este caso la intersección es el conjunto vacío ($\emptyset$) ya que no coinciden, con
    lo que el sistema no tiene solución.
  \end{solution}
\end{questions}

\subsubsection{Sistemas de inecuaciones lineales con dos incógnitas.}
La principal dificultad de estos sistemas es que únicamente se pueden resolver de modo gráfico.\\

Vamos a verlo con un ejemplo: $
\begin{cases}
  x-y +2 \geq 0\\
  2x-1 < 1-y 
\end{cases}
$\\
Si convertimos cada inecuación en una ecuación y despejamos la $y$ nos queda:
\[
  \begin{cases}
    y =& x+2\\
    y=&-2x + 2
  \end{cases}
\]
Y recordando lo que sabemos de funciones vemos que son dos rectas que sabemos dibujar utilizando una tabla
de valores, además de calcular el punto donde se cortan resolviendo el sistema (y sale $x=0$ e $y = 2$).\\
De manera que tenemos la siguiente situación:
\begin{center}
  \begin{tikzpicture}
    \begin{axis}[axis line style=gray, axis x line=center,
      axis y line=center, xmajorticks=false, ymajorticks=false] %Con xmajorticks=false, ymajorticks=false no pone marcas.
      \addplot[
      domain = -4:4,
      samples = 2,
      smooth,
      thick,
        %blue,
      ] (x, {x+2});
      \addplot[
      domain = -4:4,
      samples = 2,
      smooth,
      thick,
        %blue,
      ] (x, {-2*x+2});
      \fill (0,2) circle (.1) node[right]{$\scriptstyle (0,2)$};
      \node at (2,2){\textbf{I}};
      \node at (1,4){\textbf{II}};
      \node at (-2,2){\textbf{III}};
      \node at (-1,-2){\textbf{IV}};
    \end{axis}
  \end{tikzpicture}
\end{center}
Las dos rectas dividen el plano en cuatro regiones que hemos numerado con números romanos.\\

Una vez que hemos hecho esto tenemos que coger un punto de cada región y comprobar si es solución
del sistema de inecuaciones, y en caso de que lo sea la solución será la región que lo contiene.\\
Es decir, hacemos:
\begin{itemize}
\item De la región I cogemos $(2, 2)$, sustituimos en el sistema y tenemos:
  \[
    \begin{cases}
      2-2 + 2 \geq 0\\
      2*2 - 1 < 2 - 2
    \end{cases}
  \]
  Y la primera es cierta pero la segunda no. La región I no es solución.
\item De la región II cogemos $(0, 4)$:
  \[
    \begin{cases}
      0-4 + 2 \geq 0\\
      2*0 -1 < 1 - 4
    \end{cases}
  \]
  En este caso no es cierta ninguna de las dos. Tampoco es solución.
\item De la región III cogemos $(-2, 2)$:
  \[
    \begin{cases}
      -2 - 2+ 2 \geq 0\\
      2*(-2) - 1 < 1-2
    \end{cases}
  \]
  Ahora es cierta la segunda pero no la primera. Tampoco es solución.
\item De la región 4 vamos a coger $(0,0)$:
  \[
    \begin{cases}
      0 - 0 + 2 \geq 0\\
      2*0 - 1 < 1 - 0
    \end{cases}
  \]
  Y aquí sí son ciertas las dos. \textbf{La solución del sistema es la región IV} incluyendo los puntos
  de la primera recta porque esa desigualdad no es estricta.
\end{itemize}

En forma de pasos, para que no se nos olvide nada:
\begin{enumerate}
\item Resolvemos el sistema que queda al convertir las inecuaciones en ecuaciones para obtener el
  punto de corte de las rectas (no es obligatorio que se corten, lo veremos en alguno de los ejemplos).
\item Dibujamos las rectas y delimitamos las regiones.
\item Comprobamos las regiones hasta que encontremos alguna que sea solución (el sistema de inecuaciones
  no siempre tendrá solución).
\end{enumerate}

Así que vamos a por \textbf{algunos ejemplos} siguiendo los pasos que hemos visto.
\begin{questions}
\question Resuelve el sistema de inecuaciones $
  \begin{cases}
    2x - y < x\\
    x \leq y + 1
  \end{cases}
  $
  \begin{solution}
    En este primero vamos a seguir los pasos de manera explicita.
    \begin{enumerate}
    \item \textbf{Resolvemos el sistema de ecuaciones}:
      \[
        \begin{cases}
          2x - y = x\\
          x = y +1
        \end{cases}
      \]
      Sustituyendo la segunda en la primera nos queda:
      \[2*(y + 1) - y = y + 1\]
      \[2y -y - y = 1- 2\]
      \[ 0 = -2\]
      Y recordando lo que vimos en el punto \ref{tipos_sistemas_lineales}
      (página \pageref{tipos_sistemas_lineales}), sobre los distintos tipos de sistemas lineales, esto
      quiere decir que el sistema no tiene solución. Es decir, vamos a tener dos rectas paralelas.
    \item \textbf{Dibujamos las rectas del sistema y delimitamos las regiones}:
      \begin{center}
        \begin{tikzpicture}
          \begin{axis}[axis line style=gray, axis x line=center,
            axis y line=center, xmajorticks=false, ytick={0.,-1},
            yticklabels={$\scriptstyle 0$,$\scriptstyle -1$} ] %Con xmajorticks=false, ymajorticks=false no pone marcas.
            \addplot[
            domain = -2:2,
            samples = 2,
            smooth,
            thick,
            % blue,
            ] (x, {x});
            \addplot[
            domain = -2:2,
            samples = 2,
            smooth,
            thick,
            % blue,
            ] (x, {x-1});
            \node at (-1,1){\textbf{I}};
            \node at (-1,-1.5){\textbf{II}};
            \node at (1,-1.5){\textbf{III}};
          \end{axis}
        \end{tikzpicture}
      \end{center}
      En este caso tenemos tres regiones.
    \item \textbf{Comprobamos las regiones para ver cual es solución}:
      \begin{itemize}
      \item De I probamos con $(0, 3)$:
        \[
          \begin{cases}
            2*0 - 3 < 0\\
            0 \leq 3 + 1
          \end{cases}
        \]
        Esta región es la solución, con lo que no continuamos más.
      \end{itemize}
      La solución del sistema es la región I incluyendo los puntos de la segunda ecuación.
    \end{enumerate}
  \end{solution}
\question Resolver el sistema $
  \begin{cases}
    x+y < 3\\
    x-y > 3
  \end{cases}
  $
  \begin{solution}
    Resolvemos el sistema
    \[
      \begin{cases}
        x+y < 3\\
        x-y > 3
      \end{cases}
    \]
    Y obtenemos que la solución es $x = 3$ e $y = 0$, con lo que se cortan en el punto $(3, 0)$.\\
    Ya podemos dibujarlo:
    \begin{center}
      \begin{tikzpicture}
        \begin{axis}[axis line style=gray, axis x line=center,
          axis y line=center, xmajorticks=false, ymajorticks=false] %Con xmajorticks=false, ymajorticks=false no pone marcas.
          \addplot[
          domain = -1:5,
          samples = 2,
          smooth,
          thick,
          % blue,
          ] (x, {3-x});
          \addplot[
          domain = -1:5,
          samples = 2,
          smooth,
          thick,
          % blue,
          ] (x, {x-3});
          \fill (3,0) circle (.1) node[above]{$\scriptstyle (3,0)$};
          \node at (4,.5){\textbf{I}};
          \node at (3,2){\textbf{II}};
          \node at (1,1){\textbf{III}};
          \node at (3,-2){\textbf{IV}};
        \end{axis}
      \end{tikzpicture}
    \end{center}
    Y procedemos a comprobar regiones:
    \begin{itemize}
    \item Para la región I elegimos $(4, 0)$:
      \[
        \begin{cases}
          4 + 0 < 3\\
          4 - 0 > 3
        \end{cases}
      \]
      La primera inecuación no se cumple.
    \item Para la región II cogemos $(3, 1)$:
      \[
        \begin{cases}
          3 + 1 < 3\\
          3 - 1 > 3
        \end{cases}
      \]
      No se cumple ninguna de las dos.
    \item De la región III usamos $(0, 0)$:
      \[
        \begin{cases}
          0 + 0 < 3\\
          0 - 0 > 3
        \end{cases}
      \]
      En esta región no se cumple la segunda.
    \item De IV elegimos $(3, -1)$:
      \[
        \begin{cases}
          3 + (-1) < 3\\
          3 - (-1) > 3
        \end{cases}
      \]
      Y vemos que se cumplen las dos.
    \end{itemize}
    La solución es la región IV y no incluye los puntos de ninguna de las dos rectas ya que las
    dos desigualdades son estrictas.
  \end{solution}
\end{questions}
\subsection{Otros tipos de inecuaciones.}
\subsubsection{Inecuaciones racionales.} \label{inecuaciones_racionales}
Vamos a ver las ecuaciones racionales como un método aparte ya que se resuelven de una manera propia
por motivos que ahora vamos a ver.\\

En el punto \ref{propiedades_inecuaciones} (página \pageref{propiedades_inecuaciones}) vimos las propiedades
de las inecuaciones, y una de ellas es que al multiplicar o dividir con negativos la desigualdad cambia
de sentido.\\
El problema es que quitar los denominadores es lo mismo que multiplicar toda la inecuación por un valor del
que desconocemos el signo (ya que el denominador es un polinomio), con lo que no sabemos si deberíamos
de cambiar de sentido la desigualdad al quitar los denominadores y a partir de ahí la solución que nos
salga puede que no tenga nada que ver con la verdadera.\\

Entonces no podemos hacer el mismo método que con las ecuaciones y el método que vamos a utilizar para no quitar los denominadores es el siguiente:
\begin{enumerate}
\item \textbf{Llevamos todo a un lado y dejamos cero en el otro}, esto conserva el sentido de la desigualdad.
\item \textbf{Operamos todo de manera que nos quede solo una fracción.}\\
  En ese momento tendremos algo como $\frac{P(x)}{Q(x)} > 0$ (o con otra desigualdad)
\item \textbf{Obtenemos las soluciones de las ecuaciones que igualan el numerador y el denominador a cero.}\\
  Es decir, resolvemos las ecuaciones $P(x) = 0$ y $Q(x) = 0$ por separado.\\
  O también podemos decir que sacamos las raíces del numerador y el denominador.
\item \textbf{Ordenamos de manera creciente las soluciones} que hayamos obtenido en el paso anterior.\\
  Por ejemplo {$x_1$, $x_2$, $x_3$, \dots, $x_n$}.\\
  Con esas soluciones ordenadas \textbf{tenemos los intervalos que hay que comprobar si son solución de la
  inecuación.}\\
  Tenemos que comprobar los intervalos $(-\infty, x_1)$, $(x_1, x_2)$, \dots, $(x_n, \infty)$.
\item \textbf{Comprobamos cuales de los intervalos anteriores contienen soluciones de la inecuación.}\\
  Al compararlo con cero, que es lo que hemos dejado al otro lado, lo que estamos viendo es cuando la
  fracción es positiva (numerador y denominador del mismo signo) o negativa (numerador y denominador de
  distinto signo), con lo que la operación es bastante sencilla.
\item \textbf{La solución es la unión de los intervalos que verifiquen la inecuación}. Si la inecuación no es
  estricta además tenemos que añadir las soluciones de $P(x) = 0$.
\end{enumerate}

Vamos a ver \textbf{unos cuantos ejemplos} en orden de dificultad creciente.
\begin{questions}
\question Resuelve la siguiente inecuación: $\frac{2x + 4}{3 - x} \geq 0$.
  \begin{solution}
    Vamos a seguir el método paso a paso:
    \begin{enumerate}
    \item Pasamos todo a un lado dejamos cero en el otro.\\
      Este paso no tenemos que darlo en este ejemplo.
    \item Operamos todo para que quede solo una fracción.\\
      Este paso tampoco hay que darlo aquí.
    \item Resolvemos las ecuaciones que igualan el numerador y el denominador a cero.\\
      El numerador igual a cero: $2x + 4 = 0$, la solución $x = -2$.
      El denominador igual a cero: $3 -x = 0$, la solución $x = 3$.
    \item Ordenamos las soluciones de menor a mayor.\\
      Pues son {$-2, 3$}, así que los intervalos que hay que comprobar son:\\
      \[(-\infty, -2),\ (-2, 3),\ (3, \infty)\]
    \item Comprobamos qué intervalos son solución.\\
      Para esto lo mejor es hacerse una tabla como la que sigue:
      \begin{center}
        \begin{tabular}{l|c|c|c|}
          &$(-\infty, -2)$&$(-2,3)$&$(3,\infty)$\\
          \hline
          Valor de prueba&$x = -3$&$x = 0$&$x= 4$\\
          \hline
          $2x+4$&$2*(-3) + 4 = -2$&$2*0 + 4 = 4$&$2*4+4 = 12$\\
          \hline
          $3-x$&$6$&$3$&$-1$\\
          \hline
          ¿Cumple la inecuación? $\scriptstyle (\geq 0)$&$\scriptstyle (\frac{-}{+})$No
                          &$\scriptstyle (\frac{+}{+})$Sí
                                   &$\scriptstyle (\frac{+}{-})$No
        \end{tabular}
      \end{center}
    \item La solución es el intervalo $(-2, 3)$, y como la inecuación no es estricta tenemos que añadir
      $x = -2$.
    \end{enumerate}
    La solución de la inecuación del enunciado es $\boldsymbol{x \in [-2,3)}$.
  \end{solution}
\question Resolver la inecuación $\frac{-3}{x+1} < 1-x$.
  \begin{solution}
    Seguimos el método paso a paso:
    \begin{enumerate}
    \item Pasamos todo a un lado y dejamos cero en el otro.
      \[\frac{-3}{x+1} + x - 1 < 0 \]
    \item Operamos para que quede solo una fracción.\\
      Tenemos que hacer denominador común y sumar:
      \[\frac{-3}{x+1} + \frac{(x-1)(x+1)}{x+1} < 0\]
      \[\frac{-3}{x+1} + \frac{x^2 - 1}{x+1} <0\]
      \[\frac{x^2 - 4}{x+1} < 0 \]
    \item Sacamos las raíces del numerador y del denominador.\\
      El numerador: $x^2 - 4 = 0$ tiene de soluciones $x = -2$ y $x= 2$.
      El denominador: $x + 1 = 0$ tiene de solución $x = -1$
    \item Las soluciones ordenadas son $\{-2, -1, 2\}$.
    \item Comprobamos los intervalos haciendo una tabla como en el ejemplo anterior:
      \begin{center}
        \def\arraystretch{1.5}
        \begin{tabular}{l|c|c|c|c|}
          &$(-\infty, -2)$&$(-2, -1)$&$(-1,2)$&$(2,\infty)$\\
          \hline
          Valor a probar&$x= -3$& $x = -\frac{3}{2}$&$x = 0$&$x = 3$\\
          \hline
          $x^2 - 4$&$5$&$-\frac{7}{4}$&$-4$&$5$\\
          \hline
          $x + 1$&$-2$&$-\frac{1}{2}$&$1$&$4$\\
          \hline
          ¿Cumple la inecuación?&Sí&No&Sí&No
        \end{tabular}
      \end{center}
    \item La solución es $x \in \{(-\infty, -2) \cup (-1,2)\}$, y no tenemos que añadir nada más porque
      la inecuación es estricta.
    \end{enumerate}
  \end{solution}
\question Resuelve la inecuación $\frac{x^2 - x - 2}{x^2 +5x + 6} < 0$
  \begin{solution}
    Tal y como nos presentan el enunciado nos podemos saltar los primeros pasos y vamos directamente
    a calcular las raíces del numerador y del denominador.
    \begin{itemize}
    \item El numerador $x^2 - x - 2$ tiene de raíces $x= -1$ y $x=2$.
    \item El denominador $x^2 + 5x + 6$ tiene de raíces $x = - 3$ y $x = -2$.
    \end{itemize}
    Hacemos la tabla con los intervalos y comprobamos:
    \begin{center}
      \def\arraystretch{1.5}
      \begin{tabular}{l|c|c|c|c|c|}
        &$(-\infty, -3)$&$(-3, -2)$&$(-2,-1)$&$(-1, 2)$&($(2, \infty)$\\
        \hline
        Valor prueba&$x= -4$&$x=-\frac{5}{2}$&$x=-\frac{3}{2}$&$x=0$&$x=3$\\
        \hline
        $x^2 - x - 2$&$18$&$\frac{27}{4}$&$\frac{7}{4}$&$-2$&$4$\\
        \hline
        $x^2 + 5x + 6$&$2$&$-\frac{1}{4}$&$\frac{3}{4}$&$6$&$30$\\
        \hline
        ¿Cumple?&No&Sí&No&Sí&No
      \end{tabular}
    \end{center}
    Como la inecuación es estricta tenemos que coger intervalos abiertos y la solución queda:
    \[x \in (-3,-2) \cup (-1,2)\]
  \end{solution}
\end{questions}
\subsubsection{Inecuaciones de grado mayor que 2.}
Con lo que hemos visto para inecuaciones de 2º grado (\ref{inecuaciones_grado_2}) y para inecuaciones racionales (\ref{inecuaciones_racionales}) podemos llegar a la conclusión de que el método
para resolver casi cualquier tipo de inecuación, que no sea racional, es:
\begin{enumerate}
\item Llevar todo a un lado y dejar cero en el otro.
\item Calcular las raíces de la expresión que nos quede, con lo que tendremos una serie de
  intervalos.
\item Comprobar qué intervalos cumplen la inecuación. La solución es la unión de los intervalos
  que cumplen la inecuación.
\end{enumerate}

Y si es racional hay que obtener las raíces del numerador y el denominador por separado.

En el caso en el que estamos (polinomios de grado mayor que 2) el método es calcular las raíces.
Vamos a ver \textbf{un par de ejemplos}:
\begin{questions}
\question Resuelve la inecuación $x^3 - 4x \geq 0$
  \begin{solution}
    Tenemos que obtener las raíces de $x^3 -4x$, y lo mejor es utilizar el mismo método que para
    factorizar:
    \begin{enumerate}
    \item Sacamos factor común: $x(x^2 - 4)$, esto nos dice que una raíz es $x=0$.
    \item Obtenemos las raíces de lo de dentro del paréntesis ($x^2 - 4$), que en este caso es
      muy sencillo y las raíces son $x= -2$ y $x=2$.
    \end{enumerate}
    Entonces los intervalos son:
    \begin{center}
      \begin{tabular}{l|c|c|c|c|}
        &$(-\infty, -2)$&$(-2,0)$&$(0,2)$&$(2, \infty)$\\
        \hline
        ¿Cumple?&No&Sí&No&Sí
      \end{tabular}
    \end{center}
    La solución es $x \in (-2,0) \cup (2, \infty)$.
  \end{solution}
\question Resuelve la inecuación $x^3 - 4x < 3x - 6$.
  \begin{solution}
    Llevamos todo a un lado y cero al otro:
    \[x^3 - 7x + 6 < 0\]
    Como es grado 3, factorizamos. Y al no haber factor común y ser el grado mayor que dos,
    probamos con Ruffini.
    \begin{center}
      \begin{tabular}{r|rrrr}
        &$1$&$0$&$-7$&$6$\\
        $1$&&$1$&$1$&$-6$\\
        \hline
        &$1$&$1$&$-6$&
      \end{tabular}
    \end{center}
    Ya tenemos que $x=1$ es raíz. Ahora resolvemos $x^2 + x -6 = 0$ y obtenemos otras dos raíces
    que son $x = -3$ y $x = 2$.\\

    Construimos los intervalos y comprobamos:
    \begin{center}
      \begin{tabular}{c|c|c|c}
        $(-\infty, -3)$&$(-3, 1)$&$(1,2)$&$(2, \infty)$\\
        \hline
        No&Sí&No&Sí
      \end{tabular}
    \end{center}
    Por tanto la solución es $x \in (-3,1) \cup (2, \infty)$.
  \end{solution}
\end{questions}

\end{document}
